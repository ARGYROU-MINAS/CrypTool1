\begin{center}
{\bf I.}
\end{center}
Als Martin den Bus verl\"a{\ss}t, hat er sein Ziel schon fast erreicht. Ein
leichter Landregen geht nieder, und unter den Regenwolken verbreitet sich
ein d\"ammeriges Zwielicht in dem engen Tal zwischen den Bergr\"ucken des
Odenwaldes. Der Bus ist bald hinter einem H\"ugel am Rand des kleinen Ortes
verschwunden, und Martin kommt sich gleich ziemlich verlassen vor. Schon
vermi{\ss}t er die stumme Reisegesellschaft der Leute im Bus. Vor ihm, am
Ende einer leicht ansteigenden Kastanienallee, steht das Haus, in dem er als
Zivildienstleistender in den n\"achsten Monaten leben und arbeiten wird. Es
ist eine psychiatrische Privatklinik, ein modernes, im Stil eines gro{\ss}en
Landschlosses erbautes Sanatorium. \"Uber der Mitte des Geb\"audes ragt ein
Turm auf; er wirkt von Martins Standpunkt aus wie ein riesiger W\"achter am
anderen Ende der Allee. An drei Seiten ist die Klinik von dunklen und hohen
Tannenw\"aldern umgeben - wie ein Fremdk\"orper steht das Haus da. Das wilde
Gebirge ringsum unterstreicht nur die Unfreundlichkeit dieses Ortes. Martin
fr\"ostelt, er zieht sich entschlossen den Riemen seiner Reisetasche \"uber
die Schulter und stapft mit festen Schritten die Allee hinauf. Als er endlich
v\"ollig durchn\"a{\ss}t am Haupteingang der Klinik anlangt, wirkt alles in
der fortschreitenden Abendd\"ammerung nur noch bedrohlicher. Martin ahnt,
da{\ss} er jetzt das Tor in eine andere, ihm noch unbekannte Welt passieren
wird. \\
In den ersten Tagen ist Martin damit besch\"aftigt, sich in seiner neuen
Umgebung zurechtzufinden. Er mu{\ss} sich nicht nur bei Vorgesetzten, \"Arzten
und beim Pflegepersonal vorstellen und sich mit den Gepflogenheiten in dem
weitl\"aufigen Geb\"aude vertraut machen, er hat sich vor allem auf die
Allgegenwart der psychisch kranken Menschen einzustellen. Das ist f\"ur
einen Zwanzigj\"ahrigen, der den gr\"o{\ss}ten Teil seines bisherigen
beh\"uteten Lebens die Schulbank gedr\"uckt hat, nicht gerade eine leichte
Aufgabe. Martin m\"ochte nach seiner Zivildienstzeit studieren; was, wei{\ss}
er noch nicht so genau. Aber in eine technische Richtung soll es schon gehen.
Sein neu erworbenes Laptop hat er vorsichtshalber in den Odenwald mitgebracht;
es soll ihm sein fernes Ziel vor Augen halten und ihn in der Abgeschiedenheit
daran erinnern, da{\ss} es noch eine andere Gegenwart gibt. - \\
Die Realit\"at hier im Sanatorium erfordert jedoch einen ganz anderen Zugang.
Hier leben Menschen, die aus nicht ganz unverm\"ogenden Verh\"altnissen
stammen und deren Angeh\"orige ihnen den Aufenthalt in dieser vergoldeten
Abgeschiedenheit finanzieren k\"onnen. Fast alle Heimbewohner sind, sofern
es ihr Zustand erlaubt, in Einzelzimmern untergebracht, denen eine Toilette
mit Na{\ss}zelle angegliedert ist. Martin f\"allt schon bald auf, da{\ss}
sehr schwere F\"alle psychischer Erkrankungen in diesem Haus nicht vorkommen,
obwohl es von der Ausstattung her daf\"ur geeignet w\"are. Auch gibt es
hier nur sehr wenige mongoloide Bewohner. In den meisten F\"allen ist den
Menschen ihre Abweichung von der Norm auf den ersten Blick nicht anzusehen;
nur beim Sprechen oder durch ihre Verhaltensauff\"alligkeiten tritt ihre
Andersartigkeit zutage. Was Martin jedoch gleich recht ungew\"ohnlich
erscheint, ist das Durchschnittsalter der Bewohner. Es d\"urfte kaum \"uber
25~Jahren liegen! \\
Obwohl die Klinik einerseits sehr modern eingerichtet ist und obwohl den
Insassen jede nur denkbare Lebenserleichterung mit einer aufgesetzt
wirkenden Freundlichkeit geboten wird, vermi{\ss}t Martin andererseits eine
zeitgem\"a{\ss}e Personalf\"uhrung. Es herrscht hier eine fast kl\"osterliche
Hierarchie: an der Spitze der Klinik stehen der Chefarzt, ein gewisser
{\em Professor Goldmann\/}, und sein Bruder in der Funktion des Finanzberaters.
Dann kommen die \"Arzte, gefolgt von den hauptamtlichen Pflegern, und an
letzter Stelle stehen die Zivis - das Schlu{\ss}licht ist Martin als der
Neuling im Hause. Gehorsam ohne Gegenfragen wird gro{\ss}geschrieben.
Erschrocken mu{\ss} Martin feststellen, da{\ss} sich die anderen Zivis diesem
System bedingungslos unterworfen haben. Als 'der Neue' ist er ohnehin erst
einmal ein Fremder. Martin h\"alt sich von seinen Kollegen fern und
beschr\"ankt Kontakte auf das Tagesgesch\"aft und einige freundliche Worte bei
den Mahlzeiten. Gott sei Dank ist ihm zum Wohnen in diesem goldenen K\"afig ein
bescheidenes Einzelzimmerchen zugewiesen worden. Hierher zieht er sich abends
gerne allein zur\"uck, wenn von den Zivis nur noch Rufbereitschaft erwartet
wird. \\

\newpage\noindent % Wittwen & Weisen % \[\] 
Nach einer Woche der Einarbeitung wird Martin von der Pflegeleitung sein
zuk\"unftiges Arbeitsfeld zugeteilt. Dies besteht aus drei
nebeneinanderliegenden Appartements, deren Bewohner unter seine besondere
Pflege und Aufsicht gestellt werden. Von nun an, da Martin in n\"aheren Kontakt
zu {\em Rudolf, Anna, Bianca und Doris\/} tritt, beginnt er, seine Eindr\"ucke
seinem elektronischen Tagebuch im Laptop anzuvertrauen. So manches Gespr\"ach
versucht er wiederzugeben, und viele eigene \"Uberlegungen f\"ugt er
hinzu, denn zum Austausch von Gedanken fehlen ihm in diesem Hause die
Gespr\"achspartner. Besonders die Antrittsbesuche mit {\em Dr. Weissenborn\/}
beschreibt er recht ausf\"uhrlich, da er die Pers\"onlichkeiten seiner
Schutzbefohlenen genau studieren will. 