\begin{center}
{\bf II.}
\end{center}
Als ich mit Dr. Weissenborn bei Anna eintrat, stand sie sofort von ihrem Stuhl
auf und begr\"u{\ss}te uns mit den Worten: 'Die Toren sprechen in ihrem
Herzen: 'Es ist kein Gott.' Sie taugen nichts; ihr Freveln ist ein Greuel; da
ist keiner, der Gutes tut.' - Das stammt aus der Bibel, die aufgeschlagen vor
Anna auf dem Tisch lag (nach Annas eigener Aussage: {\em Psalter\/} 53,2; ich
habe das sogar nachgepr\"uft: es stimmt!). Ihr Lebensinhalt besteht nach
Dr.~Weissenborn im Bibelstudium, wobei sie sich ausschlie{\ss}lich auf den
Psalter beschr\"ankt. Sie ist 22~Jahre alt, Autistin und stammt aus einem
streng katholischen Elternhaus. Anna wei{\ss} sicher gar nicht, wie h\"ubsch
sie ist. Ihr \"Au{\ss}eres vernachl\"assigt sie aber v\"ollig zugunsten ihres
Bibelstudiums. Ihren Ehrgeiz richtet sie auf lange Bibelzitate, die sie
auswendig aufsagt. Ob sie wohl versteht, um was es in den Texten geht? Von
mir nahm sie keine weitere Notiz, jedenfalls habe ich keine Reaktion aus
ihren Gesichtsz\"ugen abgelesen, als Dr.~Weissenborn sie auf mich aufmerksam
gemacht hat. Beim Verlassen ihrer Wohnung h\"orten wir zum Abschied noch:
'Gott, sei mir gn\"adig, denn Menschen stellen mir nach; t\"aglich bek\"ampfen
und bedr\"angen sie mich\footnote{{\em Psalter\/} 56,2}.' - \\
Auch Rudolf hat autistische Z\"uge. Als wir sein Zimmer betraten, stand er
in der Mitte des Raumes und schwankte mit dem Oberk\"orper hin und her. Kaum
hatte der Siebzehnj\"ahrige mich erblickt, sprang er auf mich zu, baute sich
mit seinen Eins-Neunzig vor mir auf und br\"ullte: 'Spielst Du Schach?' Dabei
schaute er mich aber nicht einmal an! Ich war erschrocken und stotterte
verlegen: 'Ich kenne wenigstens die Regeln.' 'Dann la{\ss} uns spielen!' Von
irgendwoher hatte er pl\"otzlich ein Taschenschachbrett in den H\"anden. 'Du
mu{\ss}t immer auf Deinen Turm achtgeben, sonst hole ich mir den zuerst!'
'Sp\"ater, Rudolf, sp\"ater' beruhigte ihn Dr.~Weissenborn. 'Der Martin mu{\ss}
jetzt noch arbeiten.' 'Aber er hat versprochen, mit mir zu spielen!' 'Ja,
Rudolf, aber nicht jetzt. Vielleicht heute abend.' 'Aber \dots ' Mit M\"uhe
rissen wir uns los. Drau{\ss}en im Gang meinte Dr.~Weissenborn zu meinem
Erstaunen: 'La{\ss} Dich blo{\ss} nicht auf ein Spiel mit Rudolf ein! Ehe Du
Dich versiehst, bist Du schachmatt. Er ist ein Schachmeister; woher er die
Erfahrung hat, wissen wir nicht. Er kann nicht einen Buchstaben lesen, und
wenn er Dich anschreit, so nimm's ihm nicht krumm. Er hat beim Sprechen keine
Kontrolle \"uber die Lautst\"arke. Und \"ubrigens hat mich Rudolf eben noch
an etwas erinnert, was Du unbedingt wissen mu{\ss}t: Der Turm hier im Hause
ist f\"ur Dich tabu! Das ist der Privatbereich vom Chef und seinem Bruder.
Deshalb ist vor dem Treppenaufgang hinten im Gang auch ein Gitter angebracht.
Wenn es mal offensteht, la{\ss} Dich blo{\ss} nicht verleiten, da hinaufzugehen.
Das gibt \"Arger, und nicht zuwenig!' Weissenborn sprach diese Warnung ziemlich
scharf aus, und ich war wieder mal eingesch\"uchtert, wie so oft schon seit ich
hier bin. \\
Weitere Fragen dazu konnte ich auch gar nicht stellen, denn aus Biancas und
Doris' gemeinschaftlich bewohntem Zimmer drang uns ein seltsamer L\"arm
entgegen. 'Nun kommt erst der pure Wahnsinn!' sagte Weissenborn, und er
sollte Recht behalten. Wir h\"orten das Lachen der Schwestern durch die T\"ur,
und ich glaubte schon, mit denen kann es wohl lustig werden. Aber als wir
eintraten, setzten die beiden Frauen unbeirrt ihre Unterhaltung fort:
'Dreihundertf\"unfundachtzig, achttausendzweihundertsechsundsiebzig, \dots '
Mit zungenbrecherischer Geschwindigkeit warfen sie sich Zahlen zu, die f\"ur
sie einen Sinn ergeben mu{\ss}ten, denn sie h\"orten sich dabei gegenseitig
aufmerksam zu. 'Guten Morgen, Bianca, guten Morgen, Doris. Wie geht es Euch?
Ich m\"ochte Euch Martin vorstellen. Er wird in der n\"achsten Zeit auch bei
uns hier wohnen. Martin, das ist Bianca und das ihre Schwester Doris!'
Gekicher. Ein 'Guten Morgen' bringen beide unter Gel\"achter hervor. Dann
schienen sie sich in ihrer Zahlensprache k\"ostlich \"uber mich zu
am\"usieren. Sp\"ater kl\"arte mich Dr.~Weissenborn auf: 'Jeder hier glaubt,
da{\ss} sie mit ihrem Zahlensalat nur schauspielern und uns allen ein
Gespr\"ach vorgaukeln. Anderenfalls w\"are 
\newpage \noindent % Wittwen & Weisen ....
das ja wohl auch ein ganz ungew\"ohnliches Ph\"anomen.' Ich kann ihm nur 
beipflichten. 