\begin{center}
{\bf V.}
\end{center}
In den folgenden Tagen sieht man Martin nur mi{\ss}gelaunt und \"ubern\"achtigt
bei seiner Arbeit. Die anf\"angliche Euphorie \"uber die Entdeckung des
Verschl\"usselungsmechanismus ist einer ziemlichen Frustration gewichen. Zwar
hat er auf seinem Laptop ein Programm geschrieben, mit dem er nach Eingabe
eines Codezahlenpaares alle Buchstaben verschl\"usseln kann, aber die
Dechiffrierung bereitet ihm heftiges Kopfzerbrechen. Selbst bei Kenntnis eines
Schl\"ussels hat er keinen Plan, wie er aus der Zahl, die einen Buchstaben
repr\"asentiert, die entsprechende Nummer zwischen 1 und 26 gewinnen kann,
mit der er den Buchstaben im Alphabet identifizieren kann. Er beherrscht
bisher nur die Technik der {\em Verschl\"usselung\/}, aber nicht die der
{\em Entschl\"usselung\/}; und f\"ur eine Kommunikation mit Bianca und Doris
ist gerade die Dechiffrierung besonders wichtig. Lediglich in dem Fall, wenn
der Verschl\"usselungsexponent auf 1 gesetzt wird, behalten die Buchstaben
ihre alphabetisch vorgegebene Numerierung von 1 bis 26, und eine einfache
Division mit Rest f\"uhrt dann von den Zahlen zu den Buchstaben zur\"uck. Aber
den Verschl\"usselungsexponenten~1 benutzen die Schwestern nie \dots - \\
Martin macht sich aber auch \"uber die bizarre Welt von Bianca und Doris
seine eigenen Gedanken. Er vertraut sie seinem elektronischen Tagebuch an:
\\ \\
'Wer wei{\ss} schon, nach welchem Schema Bianca und Doris ihre Worte in Zahlen
verwandeln und wieder entschl\"usseln? Selbst wenn dieser Mechanismus durch
Formeln beschrieben werden kann und im Computer in einer g\"angigen
Programmiersprache laufen sollte, hei{\ss}t das noch lange nicht, da{\ss}
Bianca und Doris nach denselben Prinzipien vorgehen. Schon die Geschwindigkeit,
mit der sie dabei ohne technische Hilfsmittel auskommen, ist ein Hinweis
darauf, da{\ss} in ihren K\"opfen wohl einiges {\em v\"ollig anders\/}
abl\"auft als bei anderen Leuten. Wie soll man sonst auch verstehen, da{\ss}
{\em Mozart\/} eine zuf\"allig irgendwo geh\"orte Messe sp\"ater aus der
Erinnerung fehlerfrei und komplett in Partitur niederschreiben konnte?
\"Uberhaupt waren Leute wie {\em Mozart\/} und {\em Einstein\/} Menschen,
die nur das Gl\"uck hatten, da{\ss} sie mit den Resultaten ihrer
ungew\"ohnlichen Vernetzung im Kopf h\"ochste gesellschaftliche Anerkennung
fanden. Zwischen Rudolfs scharfsinnigen Schachstrategien und der F\"ahigkeit
Mozarts, eine ganze Oper nur im Kopf komponieren zu k\"onnen, besteht kein
gradueller Unterschied. Nur die Objekte sind andere und die Einstellungen der
'normalen' Menschen zu ihnen!! Wenn Anna den ganzen Psalter auswendig aufsagt,
geh\"ort sie in diesem Sinne zu den Gradwanderern zwischen Genie und
Wahnsinn, auch wenn ihre Monologe niemand h\"oren will. - \\
In diesem Sanatorium geht etwas Unheimliches vor: ich {\em sp\"ure\/} es, und
Bianca und Doris {\em wissen\/}, was es ist! Vielleicht werde ich noch einmal
dahinterkommen. Wie kann Dr. Weissenborn die Schwestern nur f\"ur wahnsinnig
halten? In einer von ihnen als feindlich empfundenen Umgebung vermittelt
ihnen ihre verschl\"usselte Kommunikation das Gef\"uhl von Sicherheit. Ist es
bei uns sogenannten 'Normalen' in der Welt drau{\ss}en denn anders? Wer
f\"urchtet nicht beim Homebanking oder am Geldautomaten den Zugriff Fremder
auf seine kleinen Geheimnisse? Ohne Geheimnummern kommt heutzutage doch kein
moderner Mensch mehr aus! Verr\"ucktheit ist pure Definitionssache!!' 