\begin{center}
{\bf VII.}
\end{center}
Obwohl Martin an diesem Abend erst sp\"at in das Sanatorium
zur\"uckkehrt, setzt er sich noch an sein Laptop und entwirft ein
Entschl\"usselungsprogramm. Schon auf der Busfahrt hatte er sich dazu einige
Skizzen nach dem Dechiffrierungskonzept von Dr. Praetorius angefertigt. Er
\"uberpr\"uft es mit dem Schl\"ussel {\bf 51,3} an der Kodierung des Satzes
ICH BIN BIANCA. Es l\"auft einwandfrei. Zusammen mit dem
Verschl\"usselungsprogramm, das er bereits vor einigen Tagen fertiggestellt
hatte, verf\"ugt Martin nun \"uber ein komplettes Dolmetscher-System f\"ur
die Geheimsprache der Schwestern. Dr. Praetorius hatte richtig vorausgesagt,
da{\ss} die zugeh\"origen Programme gar nicht so kompliziert sind und die
Routinen sehr schnell laufen. Einzelne W\"orter oder kurze S\"atze kann man
auch mit einem Taschenrechner bew\"altigen, aber keine l\"angeren Gespr\"ache.
Hier hilft nur das Laptop. \"Uberm\"udet, aber mit sich selbst rundum
zufrieden, f\"allt Martin schlie{\ss}lich ins Bett. \\ \\
Zu Martins ersten morgendlichen Pflichten geh\"ort es, nach 'seinen Leuten'
zu sehen. Rudolf und Anna sind Langschl\"afer, aber Bianca und Doris stehen
mit den H\"uhnern auf. Mit Herzklopfen betritt Martin ihr Zimmer;
Diktierger\"at und Laptop hat er schon bei sich. Die Schwestern sitzen
schweigend an ihrem Wohnzimmertisch, auf Martins freundliches 'Guten Morgen'
reagieren sie nicht. Sie bleiben auch stumm, als er die beiden Ger\"ate auf
den Tisch stellt. 'Dieses Verhalten hat doch etwas zu bedeuten!' fragt sich
der Zivi. 'Nat\"urlich!' schie{\ss}t es ihm durch den Kopf. 'Die Losung des
Tages! Am Badezimmerspiegel!' Und richtig, die Schwestern haben absichtlich
f\"ur Martin eine Spielkarte hinter dem Spiegel verborgen. Sie m\"ussen
geahnt haben, da{\ss} er heute auf ihre Weise mit ihnen kommunizieren wird.
Die Losung des Tages lautet:
\begin{center}
{\bf 681,\,151\,\,.}
\end{center}
Martin setzt sich zu den Frauen an den Tisch. Er gibt den Codeschl\"ussel ein
und \"ubersetzt die Worte GUTEN MORGEN. Die Zahlen auf seinem Bildschirm liest
er laut vor. Der Bann ist gebrochen. Die Schwestern antworten verschl\"usselt,
Martin \"ubersetzt zur\"uck. Anfangs benutzt er noch das Diktierger\"at, aber
schlie{\ss}lich merkt er, da{\ss} es auch ohne das Aufnahmeger\"at geht. Bald
hat er \"Ubung und tippt die Zahlen der Frauen direkt w\"ahrend ihrer Rede
ein; sein Laptop \"ubersetzt in Windeseile. Zwar geht es ein wenig langsamer
voran als es die Schwestern unter sich gew\"ohnt sind, weil Martin die
Tastatur bedienen mu{\ss}. Aber die Konversation ist fl\"ussig, wenn auch
\"au{\ss}erst ungew\"ohnlich. \\
Allm\"ahlich leitet Martin das Gespr\"ach von den Belanglosigkeiten auf die
\"Arzte und Pfleger im Hause \"uber. Schlie{\ss}lich ist es soweit: Er tippt
mit zittrigen H\"anden die Frage WAS MACHEN SIE MIT EUCH IM TURM in seinen
Computer. W\"ahrend er die Zahlen der Chiffrierung vorliest, bekommt er einen
trockenen Hals. Es ist Doris, die antwortet:
\begin{center}
172, 1734, 315, 641, 372, 3491, 360, 387, 586, 602, 2358\,\,.
\end{center}
'Eine kurze Antwort', denkt Martin, ehe er die Taste zur Ausf\"uhrung der
\"Ubersetzung bet\"atigt. Es dauert keine Sekunde, bis das Ger\"at die Aufgabe
bew\"altigt hat. Martin starrt fassungslos auf den Bildschirm seines Laptops.
Er kann erst nicht glauben, was er liest. Aber dann handelt er rasch. - \\ \\
Wenige Tage sp\"ater wird ein ungeheurer Skandal in der Odenwald-Klinik
aufgedeckt und landesweit mit Emp\"orung zur Kenntnis genommen.  

\[\]
\[\]
\[\]
\[\]
\hrulefill
\[\]
\[\]
\emph{Anmerkungen des Verfassers:}
\begin{itemize}
\item[(1)] Um den Leser nicht zu verwirren, wurde bei der Beschreibung des
  Verschl"usselungsverfahrens eine notwendige mathematische Voraussetzung
  nicht erw"ahnt. Sie ist f"ur den Verlauf der Geschichte auch nicht
  erforderlich. Es sei aber hier der Vollst"andigkeit halber erw"ahnt, da"s
  \emph{nicht jedes} beliebige Zahlenpaar als Verschl"usselungscode
  tauglich ist, wo der Hauptmodul ein Produkt von zwei Primzahlen
  ist. Vielmehr sollte darauf geachtet werden, da"s der Hilfsmodul und der
  Verschl"usselungsexponent \emph{keinen gemeinsamen Teiler}
  haben. Anderenfalls w"urde man keinen Entschl"usselungsexponenten finden.

  \textbf{Beispiel:} Das Paar 35,15 taugt nicht als Verschl"usselungscode,
  weil der Hilfsmodul $(5-1)\cdot(7-1)=24$ und der
  Verschl"usselungsexponent 15 den gemeinsamen Teiler 3 haben.
\item[(2)] In praktischen Anwendungen des in der Geschichte beschriebenen
  sog.\ RSA-Verfahrens verschl"usselt man allerdings einen Text nicht
  buchstabenweise. Man kodiert einen Text blockweise, indem man die
  Buchstaben des Textes fortlaufend zu Bl"ocken fester L"ange zusammenfa"st
  und diesen Bl"ocken nach einem bestimmten Verfahren jeweils \emph{eine
    einzelne Zahl} zuordnet.
\item[(3)] Des weiteren ist es -- im Gegensatz zu Doris' und Biancas
  Vorgehensweise -- in der Praxis "ublich, die Verschl"usselung nicht in
  Form von Zahlenfolgen, sondern wieder in Buchstabenfolgen anzugeben, die
  als Text nat"urlich keinen Sinn ergeben.
\item[(4)] Details zu diesem RSA-Verfahren findet man in folgendem Buch:\\
  NEAL KOBLITZ, \emph{A Course in Number Theory and Cryptography,} second
  edition, Kapitel IV, Paragraph 2; Springer-Verlag, 1994.
\end{itemize}

\vspace{2cm}

\noindent\emph{Anmerkung zu CrypTool:}
\\
Mit CrypTool, Version 1.3 und h"oher kann man diese Variante der
RSA-Verschl"usselung folgenderma"sen nachvollziehen:
\\
Men"u \textbf{Einzelverfahren / RSA-Demo / RSA-Kryptosystem\dots}.
\\
Im folgenden Dialog \textbf{Das RSA-Kryptosystem} dann die zweite Option w"ahlen --
nur die "offentlichen Parameter sind bekannt -- und dann faktorisieren. Die
zwei gegebenen Code-Zahlen (Beispiel: 51,3) von Bianca und Doris hei"sen hier
$N$ und $e$. Der Algorithmus kann "uber den Knopf \textbf{Optionen f"ur
Alphabet und Zahlensystem\dots} f"ur diese spezielle Variante
parametrisiert werden: \\
Im folgenden Dialog \textbf{Optionen f"ur die
RSA-Verschl"usselung} dann \textbf{Dialog der Schwestern} ausw"ahlen und
das Alphabet auf genau die 26 Buchstaben (A, B, \dots, Y, Z) beschr"anken.


% Local Variables:
% TeX-master: "DialogSchwestern.tex"
% End
