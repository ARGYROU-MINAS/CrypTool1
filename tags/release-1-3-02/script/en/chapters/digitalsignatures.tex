
\newpage
\section{Digital signatures}
\index{Signatures!digital}
(Esslinger B. / Filipovics B. / Schneider J. / Koy H., Update: Juni 2002)

The aim of digital signatures is to guarantee the following two points:
\begin{itemize}
 \item User authenticity: \\
      \index{Authenticity!user} It can be checked whether a message really does
come from a particular person.
 \item Message integrity: \\
      \index{Message integrity} It can be checked whether the message has been
changed (on route).
\end{itemize}


An asymmetric technique is used again (see encryption procedures). Participants
who wish to generate a digital signature for a document must possess a pair of
keys. They use their secret key to generate signatures and the recipient uses
the sender's public key to verify whether the signature is correct. As before,
it must be impossible to use the public key to derive the secret key.

In detail, a \index{Signature procedure} {\em Signature procedure} looks like
this: \\ Senders use their message and secret key to calculate the digital
signature for the message. Compared to hand-written signatures, therefore,
digital signatures have the advantage that they also depend on the document to
be signed. Signatures from one and the same participant are different unless the
signed documents are completely identical. Even inserting a blank in the text
would lead to a different signature. The recipient of the message would
therefore detect any injury to the message integrity as this would mean that the
signature no longer matches the document and is shown to be incorrect when
verified.

The document is sent to the recipient together with the signature. The recipient
can then use the sender's public key, the document and the signature to
establish whether or not the signature is correct. In practice, however, the
procedure we have just described has a decisive disadvantage. The signature is
approximately as long as the document itself. To prevent an unnecessary increase
in data traffic, and also for reasons of performance, we use a cryptographic
hash function.

A cryptographic \index{Hash function} {\em hash function} maps a message of any
length to a string of characters with a constant size (usually 128 or 160 bits),
the \index{Hash value} hash value. It should be practically impossible, for a
given number, to find a message that has precisely this number as hash value.
Furthermore, it should be practically impossible to find two messages with the
same hash value. In both cases the final signature procedure would display weak
points.

So far, no formal proof has been found that perfectly secure cryptographic hash
functions exist. However, there are several good candidates that have not yet
shown any weak points in practice (e.g. \index{SHA-1} SHA-1 or \index{RIPEMD-160} RIPEMD-�160).

The hash function procedure is as follows:\\ Rather than signing the actual
document, the sender now first calculates the hash value of the message and
signs this. The recipient also calculates the hash value of the message (the
algorithm used must be known), then verifies whether the signature sent with the
message is a correct signature of the hash value. If this is the case, the
signature is verified to be correct. This means that the message is authentic,
because we have assumed that knowledge of the public key does not enable you to
derive the secret key. However, you would need this secret key to sign messages
in another name.

Some digtal signature schemes are based on asymmetric \emph{encryption}
procedures, the most prominent example beeing the RSA system, which can be
used for signing by performing the private key operation on the hash value
of the document to be signed.

Other digital signature schemes where developed exclusively for this
purpose, as the DSA (Digtal Signature Algorithm), and are not directly
connected with a corresponding encryption scheme.

Both, RSA and DSA signature are discussed in more detail in the following
two sections. After that we go one step further and show how digital
signatures can be used to create the digital equivalent of ID cards. This
is called Public Key Certification.

\subsection{RSA Signatures}
\index{Signatures!digital}
\index{RSA signatures}

\def\Mod#1{\ (\mbox{mod }#1)}
As mentioned in the comment at the end of \hyperlink{RSAproof}{section
  \ref{RSAproof}} it is possible to perform the RSA private and public key
operation in reverse order, i.~e.\ raising $M$ to the power of $d$ and then
to the power of $e \Mod{N}$ yields $M$ again. Based on this simple fact, RSA
can be used as a signature scheme.

The RSA signature $S$ for a message $M$ is created by performing the private
key operation:
$$ S \equiv M^d \Mod{N} $$
In order to verify, the corresponding public key operation is performed on
the signature $S$ and the result is compared with message $M$:
$$ S^e \equiv (M^d)^e \equiv (M^e)^d \equiv M \Mod{N}$$
If the result matches the message $M$, then the signature is accepted by the
verifyer, otherwise the message has been tampered with, or was never signed
by the holder of $d$.

As explained above, signatures are not performed on the message itself, but
on a cryptographic hash value of the message. To preclude certain attacks
on the signature procdedure (alone or in combination with encryption) it is
necessary to format the hash value before doing the exponentiation, as
described in the PKCS\#1 (Public Key Cryptography Standard \#1
\cite{PKCS1}). The fact that this standard had to be revised recently, after
being in use for several years, can serve as an example how difficult it is
to get the details of cryptography right.

\subsection{DSA Signatures}
\index{Signatures!digital}
\index{DSA signatures}

In August of 1991, the U.S. National Institute of Standards and Technology
(NIST\index{NIST}) proposed a digital signature algorithm (DSA), which was adopted as
a U.S. Federal Information Processing Standard (FIPS 186 \cite{FIPS186})
subsequently. 

The algorithm is a variant of the ElGamal scheme. Its security is based on
the Discrete Logarithm Problem\index{Logarithm problem!discrete}. The DSA public and private key and its
procedures for signature and verification are summarised below.

\paragraph{Public Key}\strut\\
\begin{tabular}{l@{ }l}
$p$ & prime \\
$q$ & 160-bit prime factor of $p - 1$ \\
$g$ & $ = h^{(p-1)/q}  \mbox{ mod } p$, where $h < p - 1$ and
$h^{(p-1)/q} > 1  \Mod{p}$ \\
$y$ & $\strut \equiv  g^x  \mbox{ mod } p$ 
\end{tabular}

\emph{Remark:} Parameters $p,q$ and $g$ can be shared among a group of users.

\paragraph{Private Key}\strut\\
\begin{tabular}{l@{ }l}
$x < q$ (a 160-bit number) 
\end{tabular}

\paragraph{Signing}\strut\\
\begin{tabular}{l@{ }l}
$m$ & the message to be signed\\
$k$ & choose at random, less than $q$\\
$r$ & $= (g^k \mbox{ mod } p) \mbox{ mod } q$\\
$s$ & $= (k^{-1}(\mbox{SHA-1}(m) + xr)) \mbox{ mod } q$
\end{tabular}

\emph{Remark:}
\begin{itemize}
\item $(s,r)$ is the signature.
\item The security of the signature depends not only on the mathematical
  properties, but also on using a good random source  for $k$.
\item SHA-1 \index{SHA-1} is a 160-bit hash function specified in FIPS186.
\end{itemize}
\paragraph{Verifying}\strut\\
\begin{tabular}{l@{ }l}
$w$ & $= s^{-1}  \mbox{ mod } q$\\
$u_1$ & $= (\mbox{SHA-1}(m)w) \mbox{ mod } q$\\
$u_2$ & $= (rw)  \mbox{ mod } q$\\
$v$ & $= (g^{u_1}y^{u_2}) \mbox{ mod } p)  \mbox{ mod } q$\\

\end{tabular}

\emph{Remark:} If $v = r$, then the signature is verified.

While DSA was specifically designed, such that it can be exported from
countries regulating export of encryption soft and hardware (like the U.S.\ 
at the time when it was specified), it has been noted
\cite[p.~490]{5Schneier1996}, that the operations involved in DSA can be
used to emulate RSA and ElGamal encryption.

\subsection{Public key certification}
\index{Certification!public key}
The aim of public key certification is to guarantee the connection between a
public key and a user and to make it traceable for external parties. In cases in
which it is impossible to ensure that a public key really belongs to a
particular person, many protocols are no longer secure, even if the individual
cryptographic modules cannot be broken.

\subsubsection{Impersonation attacks}
\index{Impersonation attack}
Assume Charlie has two pairs of keys (PK1, SK1) and (PK2, SK2), where SK denotes
the secret key and PK the public key. Further assume that he manages to palm off
PK1 on Alice as Bob's public key and PK2 on Bob as Alice's public key (by
falsifying a public key directory).

Then he can attack as follows:
\begin{itemize}
    \item Alice wants to send a message to Bob. She encrypts it using PK1
because she thinks that this is Bob's public key. She then signs the message
using her secret key and sends it.
    \item Charlie intercepts the message, removes the signature and decrypts the
message using SK1. If he wants to, he can then change the message in any way he
likes. He then encrypts the message again, but this time using Bob's genuine
public key, which he has taken from a public key directory, signs the message
using SK2 and forwards it to Bob.
    \item Bob verifies the signature using PK2 and will reach the conclusion
that the signature is correct. He then decrypts the message using his secret
key.
\end{itemize}

Charlie can thus listen in on communication between Alice and Bob and change the
exchanged messages without them noticing. The attack also works if Charlie only
has one pair of keys.

Another name for this type of attack is \index{Man-in-the-middle
attack} ``man-�in-�the-�middle attack''. Users are promised
protection against this type of attack by public�key
certification, which is intended to guarantee the
\index{Authenticity} authenticity of public keys. The most common
certification method is the X.509 standard.

\subsubsection{X.509}

All participants who want to \index{X.509}  have X.509 verify that their public
key belongs to a real person consult what is known as a \index{Certification
authority (CA)} certification authority (CA). They prove their identity to this
CA (for example by showing their ID). The CA then issues them an electronic
document (certificate) which essentially contains the names of the certificate-holder
and the CA, the certificate-holder's public key and the validity period
of the certificate. The CA then signs the certificate using its secret key.


Anyone can now use the CA's public key to verify whether a certificate is
falsified. The CA therefore guarantees that a public key belongs to a particular
user.

This procedure is only secure as long as it can be guaranteed that the CA's
public key is correct. For this reason, each CA has its public key certified by
another CA that is superior in the hierarchy. In the upper hierarchy level there
is usually only one CA, which can of course then have its key certified by
another CA. It must therefore transfer its key securely in another way. In the
case of many software products that work with certificates (such as the
Microsoft and Netscape Web browsers), the certificates of these root CAs are
permanently embedded in the program right from the start and cannot be changed
by users at a later stage. However, (public) CA keys, in particularly those of
the root entity, can also be secured by means of making them available publicly.

\begin{thebibliography}{99999}
\addcontentsline{toc}{subsection}{Bibliography}
\bibitem[Schneier1996]{5Schneier1996} \index{Schneier 1996} 
    Bruce Schneier, \\
    Applied Cryptography, Protocols, Algorithms,
    and Source Code in C, Wiley, 2nd edition, 1996.

        \bibitem[PKCS1]{PKCS1} RSA Laboratories\\ \index{PKCS\#1}
          PKCS \#1 v2.1 Draft 3: RSA Cryptography Standard. April 19, 2002.

        \bibitem[FIPS186]{FIPS186} U.S. Departement of
          Commerce/N.I.S.T.\\\index{FIPS186}
          Entity authentication using public key cryptography. February 18, 1997.
\end{thebibliography}


