\begin{center}
{\bf VI.}
\end{center}
Trotz intensiven Nachdenkens scheitert Martin am Problem der Dechiffrierung
von Biancas und Doris' Geheimsprache. Aber er gibt trotzdem nicht auf: Er
nutzt einen freien Tag, f\"ahrt nach Heidelberg und trifft sich dort mit
{\em Dr. Praetorius\/} im mathematischen Institut der Universit\"at. Martin
hatte sich mit diesem Herrn, dessen Namen er im Personenverzeichnis
der Hochschule ausfindig gemacht hatte, telefonisch in Verbindung gesetzt
und diesen Termin vereinbart. Der akademische Rat empf\"angt ihn freundlich
und h\"ort geduldig zu, als Martin ihm von seiner Arbeit, den Schwestern
und ihrer Zahlensprache erz\"ahlt. Auch Dr. Praetorius ist von den
F\"ahigkeiten der beiden Frauen fasziniert, hat aber auch keine Erkl\"arung
f\"ur dieses eigenartige Ph\"anomen. Martin setzt ihm dann die
Verschl\"usselungstechnik auseinander, soweit er sie bereits analysieren
konnte. Sein eigentliches Ziel, von den Schwestern etwas \"uber gewisse
Vorkommnisse im Sanatorium erfahren zu wollen, erw\"ahnt der Zivi dabei
nat\"urlich nicht. \\ \\
'Haben Sie eine solche Karte mit den Verschl\"usselungscodes mitgebracht?'
fragt Dr. Praetorius. Martin reicht ihm die Spielkarte, die er im Bad der
Schwestern hinter dem Spiegel gefunden hatte, und der Rat tippt einige Zahlen
aus den Verschl\"usselungscodes in seinen Computer ein. Er l\"a{\ss}t die
Maschine rechnen, nickt zufrieden und wendet sich wieder Martin zu. 'Sie
wissen doch, was Primzahlen sind?' Martin ist um die Antwort nicht verlegen,
solche Dinge hat er in der Schule bei Dr. Bruckner gelernt. 'Das sind Zahlen
ohne Teiler' erwidert er. 'Richtig, genauer gesagt, ohne {\em echte\/} Teiler,
denn {\em jede\/} Zahl ist immer durch 1 und sich selbst teilbar. Die ersten
Primzahlen sind:
\begin{center}
2,\,3,\,5,\,7,\,11,\,13,\,17,\,19,\,23,\,29, \quad \dots
\end{center}
Ist ihnen schon aufgefallen, da{\ss} die erste Zahl eines jeden Codepaares
stets aus {\em zwei\/} Primzahlen zusammengesetzt ist?' Martin sch\"uttelt
den Kopf. 'Schauen Sie, hier!' Der Rat weist auf den Bildschirm des Computers:
\[\begin{array}{rcc}
4\,267 & = & 17\cdot 251 \\
17\,819 & = & 103\cdot 173
\end{array} \]
'Die Zahlen 17, 251, 103 und 173 sind alle Primzahlen. Und gerade diese
Primfaktorzerlegung der jeweils ersten Codezahl mu{\ss} man kennen, um
dechiffrieren zu k\"onnen!' Martin sieht den Mathematiker bewundernd an. 'Sie
fragten mich eben nach der M\"oglichkeit der Entschl\"usselung. Ich will sie
Ihnen erkl\"aren. In der Kryptographie, der mathematischen Teildisziplin, in
der man Verschl\"usselungstechniken untersucht und entwickelt, kennt man das
Verfahren Ihrer beiden Schwestern aus dem Heim schon lange; es ist ein
einfacheres Verfahren. Erschrecken Sie nicht! Es ist trotzdem sehr wirkungsvoll.
Das Geheimnis bei der Entschl\"usselung beruht auf einer Erkenntnis, die wir
dem alten {\em Fermat\/}\footnote{{\em Pierre de Fermat\/}, franz\"osischer
Mathematiker, 1601 - 1665.} verdanken. Und so geht es!' Dr. Praetorius
greift zu den Aufzeichnungen, die ihm Martin mitgebracht hat. 'Sie haben den
Satz ICH BIN BIANCA mit dem Schl\"ussel {\bf 51,3} kodieren lassen, und
beispielsweise f\"ur den Buchstaben {\bf H} von Bianca die Zahl 665 erhalten.
Die erste Zahl aus dem Schl\"ussel, die 51, wollen wir den {\em Hauptmodul\/}
nennen, die 3 haben Sie als {\em Verschl\"usselungsexponent\/} bezeichnet;
das soll mir recht sein. Sicher wird es Sie jetzt \"uberraschen, da{\ss} man
einen {\em Entschl\"usselungsexponenten\/} bestimmen kann. Und genau hierf\"ur
ben\"otigen Sie die Primfaktorzerlegung des Hauptmoduls. Er ist hier 51. Sie
haben die 51 damals willk\"urlich gew\"ahlt, und zuf\"allig ist das auch ein
Produkt von zwei Primzahlen.' Dr. Praetorius schreibt auf ein Blatt:
\begin{center}
\(51\,\,=\,\,3\cdot 17 \)
\end{center}
'Jetzt verringern Sie die beiden Primfaktoren jeweils um 1 und multiplizieren
die Resultate miteinander. Was dabei herauskommt, wollen wir den
{\em Nebenmodul\/} nennen.'
\begin{center}
Nebenmodul \,=\,\((3-1)\cdot (17-1)\,=\,2\cdot 16 \,=\,32 \)
\end{center}
'K\"onnen Sie mir nun eine Zahl angeben, die folgende Eigenschaft hat: Wenn man
sie mit dem Verschl\"usselungsexponenten 3 multipliziert und anschlie{\ss}end
das Produkt durch den Nebenmodul 32 teilt, bleibt bei dieser Division mit 32 der
Rest 1 \"ubrig \dots ?' Martin \"uberlegt kurz, dann hat er das Resultat: 'Elf.'
'Sehr gut!' lobt ihn der Rat und schreibt:
\[\frac{11\cdot 3}{32}\,=\,1 \,\,\,\mbox{Rest}\,\,1\,\,,\quad \mbox{oder}\,\,
11\cdot 3\,=\,33\,=\,1\cdot 32 + \mbox{1}\,\,.\]
'Woher kommt diese Eins, die bei der Division mit dem Nebenmodul als Rest
\"ubrigbleiben soll?' fragt Martin. 'Egal, mit welchen Zahlen Sie auch
hantieren, an dieser Stelle mu{\ss} bei der Division mit dem Nebenmodul
{\em immer\/} der Rest~1 \"ubrigbleiben. Das hat mit der Erkenntnis des alten
Fermat zu tun, die dieser Rechnung zugrundeliegt. Ja, und die Zahl 11, die Sie
eben ermittelt haben, ist der gesuchte Entschl\"usselungsexponent.' Martin
staunt. 'Soll ich jetzt die 665, die mir Bianca f\"ur den Buchstaben {\bf H}
genannt hat, elfmal mit sich selbst multiplizieren?' fragt er ungl\"aubig.
'Sie machen dabei nichts falsch. Aber diese Zahl wird ziemlich gro{\ss}. Da
wir anschlie{\ss}end das Ergebnis wieder durch den {\em Hauptmodul\/} 51
teilen und dabei den Rest bestimmen wollen, bietet es sich an, von der
Verschl\"usselungszahl 665 bereits {\em vor\/} dem Potenzieren mit 11 den
Rest bei der Division durch den Hauptmodul 51 zu ermitteln. Sie haben das in
Ihren Aufzeichnungen ja schon getan:'
\[665\,\,=\,\,13\cdot 51\,+\,{\bf 2} \]
'Anstelle der 665 d\"urfen Sie auch die 2 potenzieren. Das ist viel bequemer.'
Martin kommt aus dem Staunen nicht mehr heraus:
\[2^{11}\,=\,2048\,=\,40\cdot 51\,+\,{\bf 8} \]
Jetzt ist Martin richtig verbl\"ufft: Die 8 steht in seiner Tabelle f\"ur den
Buchstaben {\bf H}!!! 'Und durch dieses Potenzieren mit 11 k\"onnen Sie die
ganze von Bianca genannte Zahlenkolonne wieder in die Buchstaben
zur\"uckverwandeln. \"Ubrigens: Dieses Kodierungsverfahren l\"a{\ss}t sich
auch f\"ur Ihren Laptop leicht programmieren.' 'Darauf soll erst 'mal einer
kommen!', meint Martin, 'kein Wunder, da{\ss} ich diese Nu{\ss} nicht geknackt
habe!' Dr. Praetorius hat aber noch einen weiteren Trumph in der Hinterhand:
'In der Praxis k\"onnen zwei Personen, die nach diesem Verfahren geheime
Nachrichten austauschen, den Codeschl\"ussel - also den Hauptmodul samt
Verschl\"usselungsexponent - sogar {\em \"offentlich\/} bekanntgeben,
vorausgesetzt, der Hauptmodul ist sehr, sehr gro{\ss}.' Der junge Mann sieht
den akademischen Rat ungl\"aubig an. Der aber l\"achelt und f\"ahrt fort: 'Das
Problem bei Kodierungen sind weniger die Verfahren. Die sind vielen bekannt
und beruhen meist auf recht elementaren mathematischen Prinzipien. Die Kunst
der Dechiffrierung besteht n\"amlich zumeist in der Zerlegung sehr gro{\ss}er
Zahlen in ein Produkt von Primzahlen, wie wir es hier bei unserem Hauptmodul
\(51=3\cdot 17 \) auch getan haben. Bei so kleinen Zahlen, auch solchen, wie
sie Bianca und Doris verwenden, ist das kein Problem. Mit Tabellen oder mit der
Hilfe eines Taschenrechners findet man die Zerlegung rasch. Aber stellen Sie
sich einen hundertstelligen Hauptmodul vor, dessen beide Primfaktoren etwa
f\"unfzigstellig sind! Diese Primfaktoren bleiben nat\"urlich geheim. Auch mit
gr\"o{\ss}eren Computern dauert es dann schon seine Zeit, bis ein Dritter diese
Primfaktoren gefunden hat!' -  \\ \\
Als Martin wieder den Bus besteigt, der ihn in den Odenwald zur\"uckbringt,
brummt ihm der Sch\"adel. Aber er ist nun voller Zuversicht, mit Bianca und
Doris k\"unftig unbegrenzt kommunizieren zu k\"onnen. Mit Hilfe des
Diktierger\"ates, dem Laptop und seinem Wissen sollte das nun doch m\"oglich
sein! - Auf den Einfall, hundertstellige Codeschl\"ussel zu verwenden, wird
er die Schwestern aber nicht bringen. 