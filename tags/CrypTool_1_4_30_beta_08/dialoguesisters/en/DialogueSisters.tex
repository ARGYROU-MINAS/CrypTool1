\documentstyle[12pt]{article}
\begin{document}

\begin {center} 
{\bf THE DIALOGUE OF THE SISTERS\footnote{last update: \today}} 
\[\] 
({\em by Carsten Elsner\/}) 
({\em translated from the German by Klaus Esslinger\/}) 
\end {center} 

%%%%%%%%%%%%%%%%%%%%%%%%%%%%%%%%%%%%%%%%%%%%%%%%%%%%%%%%%%%%%%%%%%%%%%%%%%%%%%%%%%%%%%%%%%%%%%%%%

\begin {center} 
{\bf I.} 
\end {center}
When Martin gets off the bus, he has almost reached his destination. 
A light continuous rainfall is coming down, and under the rain clouds a dim
twilight spreads in the narrow valley in between the ridges of the 
{\em Odenwald\/}\footnote {{\em Odenwald}, a nice and quiet area 
within Germany with lots of forests.}. 
The bus soon disappears behind a hill on the edge of the small village, 
and Martin feels pretty quickly abandoned. He already misses the silent company of the people on the bus. 
Before him, at the end of an gently rising avenue of chestnut trees, there
is the house in which he will be carrying out alternative service as a {\em Zivi\/}\footnote {{\em Zivi} 
= abbr.\ of `Zivildienstleistender' (literally someone performing civil services in Germany).
This refers to people who perform a year's service working in a hospital, a school 
or another social institution as an alternative to doing military service. All young men 
who don't have a handicap have to complete a period of either `civil' or military 
service.} for the next months.
It is a private psychiatric clinic, a modern sanatorium, built in the style of 
a large country castle. 
In the middle of the building a tower rises up; from where Martin is standing 
it looks like an enormous guard at the other end of the avenue. 
On three sides the clinic is surrounded by dark and tall fir forests -- the 
house looks like a foreign body. 
The wild mountains all around just emphasize the unfriendliness of this place. 
Martin shivers, he picks up his suitcase with an air of real determination and 
trudges down the avenue with firm steps. 
When he arrives at last, completely soaked in the main entrance of the clinic,
everything still looks more threatening in the dusk which is getting ever darker. 
Martin anticipates that he will pass now the gate into another world which is 
still unknown to him. \\ 
In the first days Martin is busy finding his way about in his new surroundings. 
Not only does he have to introduce himself to superiors, doctors and the nursing staff as well as 
familiarising himself with the customs in the extensive building, but he has to adjust
himself first of all to the omnipresence of the psychologically ill people. 
This is not an easy thing for a twenty-year-old who has spent the main part of his 
previously sheltered life at school.
Martin would like to study after his time of alternative service; but he does not 
know yet exactly which subject.
But a technical direction it should be. 
He has brought his newly acquired laptop with him to the Odenwald, just as a 
precaution; it should remind him in his seclusion of his long term goal and of the 
fact that there is still another present world. -- \\ 
The reality here in the sanatorium requires another approach. 
The people who live here certainly don't come from backgrounds without money. Their 
relatives can finance their stay in this gilded seclusion. 
Almost all the inhabitants are accommodated in single rooms, provided that their 
condition allows it, to which a toilet with a bathroom unit is attached. 
It soon strikes Martin that very difficult cases of psychological illnesses are not 
found in this house, although the equipment would be suitable here for it. 
There are also only very few mongoloid inhabitants here.
In most cases their abnormality of the standard cannot to be looked at the people at 
first sight; only while speaking or by the way in which they behave their unusual
personalities come to light. 
What appears immediately rather strange to Martin, however, is the average age of 
the inhabitants. 
It can hardly be more than 25! \\ 
Although the clinic on the one hand has very modern furnishings and although every only
possible life relief combined with a semblance of friendliness is offered to the inmates, 
Martin misses on the other hand a contemporary personnel management. 
An almost monastic hierarchy rules in here: at the head of the clinic the chief 
doctor, a certain {\em Professor Goldmann\/}, and his brother in the position of
the financial consultant.
Then there come the doctors, followed by the full-time nurses, and in the latter
place there come the Zivis -- the tail light is Martin as the newcomer in the house. 
Obedience without question is stressed to him. 
Martin finds out to his shock that the other Zivis have submitted themselves to this
system unconditionally. 
As the `New Boy' he is anyway first of all a stranger. 
Martin keeps away from his colleagues and limits contact to the day-to-day 
operations and some polite words during meals. 
Thank heavens a modest single room has been assigned to him while he's living in 
this gold cage. 
In the evening here he withdraws gladly alone if from the Zivis only call readiness 
is expected. \\ 
\[\] 
After a week of training Martin is assigned to his future field of work by the 
nursing staff. 
This consists of three apartments lying side by side whose inhabitants are placed 
under his special care and surveillance.
From now on, because Martin has closer contact to {\em Rudolf, Anna, Bianca and 
Doris\/}, he begins to confide his impressions to electronic diary on his laptop. 
He tries to reproduce a lot of discussions, and he adds many thoughts of his own, 
because there is no one in this building with whom he can discuss his thoughts. 
He describes particularly the first visits with {\em Dr.~Weissenborn\/} 
in detail, because he wants to study the personalities of his clients carefully.

%%%%%%%%%%%%%%%%%%%%%%%%%%%%%%%%%%%%%%%%%%%%%%%%%%%%%%%%%%%%%%%%%%%%%%%%%%%%%%%%%%%%%%%%%%%%%%%%%%%%%%%

\begin {center} 
{\bf II.} 
\end {center} 
When I entered Anna's room with Dr.~Weissenborn, she got up immediately from her 
chair and greeted us with the words: `The fool has said in his heart: ``There is no 
God in heaven above.'' Their deeds are corrupt, depraved; not a good man is left.'
-- This comes from the Bible which lay open in front of Anna on the table (after her 
own statement: {\em Psalter\/} 53,2; I have even checked this: it's correct!). 
Her purpose in life consists according to Dr.~Weissenborn of studying the Bible 
and she limits herself exclusively to the psalter. 
She is 22~years old, autistic and comes from a home with an orthodox Roman Catholic parents. 
Anna certainly does not know at all how beautiful she is. 
She neglects her appearance, however, completely in favour of her Bible study. 
She directs her ambition upon long Bible quotations which she knows by heart. 
But does she understand what the texts are about? 
She takes no other notice of me, anyhow, I read no reaction from her features, as Dr.~Weissenborn has pointed me out to her. 
When we leave her flat we heard her crying out: `Have mercy on me, God, men crush me; they 
fight me all day long and oppress me.\footnote {{\em Psalter\/} 56,2}' -- \\ 
Rudolf also has some autistic features. 
As we entered his room, he was standing in the middle of it and swaying with the 
upper part of the body back and forth. 
The 17-year-old had hardly seen me, he jumped at me, faced up to me with all of 
his height of one metre ninety and roared: `Do you play chess?' moreover, he didn't even look at me! 
I had got a fright and stuttered in an embarrassed way: 
`Well, I mean, at least I know the rules.' `Then let us play!' 
From somewhere he suddenly produced a pocket chessboard. 
`You must always pay attention to your castle, otherwise I'll get it first!' 
`Later, Rudolf, later' said Dr.~Weissenborn reassuringly. `Martin has got to work now.'
`However, he has promised to play with me!' Yes, Rudolf, but not now. 
Maybe tonight. `However, \dots' With some effort we tore ourselves away. 
Outside in the corridor Dr.~Weissenborn exclaimed to my surprise: `Don't ever get 
involved in a game with Rudolf! Even before you make a mistake, you'll be check-mate. 
He is a chess champion; we do not know where his experience comes from. 
He can't read a single letter, and if he screams at you, do not take offence. 
He has no control about the volume while he's speaking. 
And, by the way, Rudolf has reminded me of something which you need to 
know: just the tower is a taboo for you here in the house! 
That is the private area of the boss and his brother. 
Therefore iron bars are also attached on the stairway behind in the passage. 
If it is open, do not be tempted to go up there. 
You won't half get into trouble!'
Weissenborn gave this warning quite sharply, and I was intimidated once again, as it 
had already happened so often since I had been here. \\ 
I could not put any further questions at all, because from Bianca's and Doris' room 
which they lived in together, a strange noise reached our ears. 
`Now you'll experience absolute madness!' Weissenborn said, and he was absolutely right. 
We heard the laughter of the sisters by the door, and I already believed, that we 
were going to have some fun.
But when we entered, both women continued their entertainment unflustered: `three 
hundred and eighty five, eight thousand two hundred and seventy six, \dots' 
At a tongue-twisting speed they threw to themselves numbers which seemed to make 
sense to them, because, they listened to each other attentively. 
`Good morning, Bianca, good morning, Doris. How are you? 
I would like to introduce you to Martin. He will also be living with us here for a while. 
Martin, this is Bianca and her sister Doris!'
Giggling. 
A `Good morning' they both managed to say despite their laughters. 
Then they seemed to amuse themselves a great deal with their language of numbers.
Later Dr.~Weissenborn explained all this to me: 
`Everybody believes that they only deceive us with their jumble of numbers in order 
to make us believe it's a conversation. 
Otherwise it would probably be a quite unusual phenomenon.'
I can only agree with him. 

%%%%%%%%%%%%%%%%%%%%%%%%%%%%%%%%%%%%%%%%%%%%%%%%%%%%%%%%%%%%%%%%%%%%%%%%%%%%%%%%%%%%%%%%%%%%%%%%%%%%%%%%%

\begin {center} 
{\bf III.} 
\end {center}
It does not last long at all, until Martin has won the confidence of Anna, Rudolf, 
Bianca and Doris. 
A certain skill soon adjusts itself: one must simply try not to listen to the 
quotations from the Bible uttered by Anna. 
Martin has already tried a game of chess with Rudolf: he never had a chance.
When he puts down his king on the board at the check position, Rudolf jumps up, 
runs around the room and is delighted every time about his victory. 
Martin has become especially fond of Bianca and Doris.
Both are older than thirty and have been living for a long time the clinic. 
He can talk to them in a normal way about some unimportant incidents in the house, 
in any case. 
They are always good tempered and usually a little bit silly. 
But any longer discussion with Bianca and Doris is impossible! 
Suddenly the sisters continue their dialogue using number hieroglyphs, and Martin is 
sent out. 
It is soon obvious: behind the columns of figures one doesn't simply find a chaos, 
but there is in fact a secret language.
Every time Martin would like to ask the sisters something personal, e.g. what both 
of them think about Dr.~Weissenborn or Professor Goldmann, they immediately leave 
the level of the conventional language. 
Martin feels that this is the reason for their uninterrupted amusement: They know 
that no one can follow their dialogue, and this gives them total privacy. 
They probably know a lot about the sanatorium, but they would never tell anybody 
else, especially because no one beyond the gates of the clinic would listen. 
Because Martin likes the sisters, this fact hurts to him. 
He would gladly chat to them about various matters, naturally only in the context of 
their capabilities -- however in the colloquial language, not in numbers.
In the course of time he got accustomed to the severity, especially on how the Zivis 
are treated here, but it remains nevertheless without justification to him. 
Quietly Martin attentively observes the doctors and nurses, and increasingly he 
gets an odd feeling. 
He does not know why, but this uncanny and rejecting coldness, which was present in 
the hospital on his arrival in the rainy summer evening, returns again and again. 
Then he gets a goose bump on his back. -- \\
You do not often come face to face with Professor Goldmann and his brother. 
They only seldom leave their rooms in the tower; Martin is mostly reminded of their 
existence only if the two brothers roar with their Porsche up or down the avenue 
lined with chestnut trees. 
They always leave and return to the sanatorium together.
Martin does not understand this behaviour. 
Professor Goldmann seems nevertheless to have a special interest in each new 
acquisition in the hospital. 
He alone decides on the accommodation or the rejection of a patient, and he does not 
feel bound to the opinion of physicians and psychologists. 
Martin once accidentally overheard to a discussion between two hospital physicians 
by a door which was not totally closed. 
Thereby he received a view of the customs with the accommodation of a patient. 
This convinced him more and more that something unusual was going on in this house. 
\[\] 
On a long and dark autumn evening Rudolf persuades `his Zivi' to play chess. 
Since Martin does not have anything better to do, he begins a series of several 
games with Rudolf. 
Even though he loses every game and causes Rudolf several times to be happy, he 
then sharpens his analytic skills and improves appreciably his strategy. 
He is again and again surprised, how much he can learn from Rudolf in the game. 
The clock already shows one hour after midnight, when Martin -- contrary to Rudolf -- 
feels an urgent need for his bed.
On the way along the long passages to his room he suddenly hears the steps of 
several people in his proximity in the quiet house. 
He stops instinctively on the halfdark stairs and lets the people below pass by. 
He sees, how three nurses are bringing two patients, a young man and a woman, to 
Professor Goldmann into the tower. 
Nobody notices him on the stairway, his tiredness suddenly vanishes. 
Surprised by this unusual process and with dark thoughts, Martin comes out from the 
shadows a long time after the steps have faded away. 
He creeps into his room and closes with soft knees the door behind himself.

%%%%%%%%%%%%%%%%%%%%%%%%%%%%%%%%%%%%%%%%%%%%%%%%%%%%%%%%%%%%%%%%%%%%%%%%%%%%%%%%%%%%%%%%%%%%%%%%%%%%%%%

\begin{center} 
{\bf IV.} 
\end{center} 
Some days later after this incident Martin finds a single playing card from a 
{\em Skat\/}\footnote {{\em Skat\/}, a very popular cards game in middle Europe.} 
game in the bath while clearing up Bianca's and Doris' apartment. 
It is half covered and wedged behind the mirror. 
He is a bit surprised and brings the card into the living room, where he wants to 
put it with the remaining cards in a drawer, in which the sisters keep some party games. 
Martin has long given up trying to understand the actions of his patients. 
When he wants to arrange however the wildly scattered Skat cards in the drawer, he 
hesitates: all cards are written upon the rear side with numbers. 
They are always arranged in pairs, and each pair of numbers was deleted on each card 
again except for one. 
Martin examines the card, which he had just found in the bath. 
The pair of numbers which haven't been deleted read:
\begin{center} 
{\bf 14\,857\,,\,3\,\,.} 
\end{center} 
Suddenly the laughter of the two sisters comes from outside; they are just returning 
from breakfast in the common room to their flat. 
Martin quckly closes the drawer and puts the card from the bath in his trouser 
pocket. \\
Bianca and Doris briefly exchange a few words with him, fool around and return to 
their secret language. 
The young man considers whether he should dare it: he still tidies up the room for 
a bit longer, then he turns to leave. 
However before he closes the door behind him, he calls loud and clearly into the 
room: 
\begin{center} 
`fourteen thousand eight hundred and fifty seven, three.' 
\end{center} 
Both sisters immediately let out a short cry. 
Bianca rushes into the bath, however returns right away and looks questioningly at 
Martin. 
He holds out the found card, however neither of them wants it back. 
Then he closes the door, his heart pounds up to the neck. 
For the first time he sees the two sisters looking embarrassed. 
This pair of numbers 14\,857,\,\,3 is obviously a code to understanding their language. 
Martin is determined to solve this mystery. 
Perhaps he can participate sometime in the discussions of the sisters. 
If they accept him, perhaps he will find out more about this sanatorium.
Martin is so deep in thought that he almost collides in the corridor with Anna. 
Completely carried away without noticing anything around her, she soliloquizes: 
`Save me, O God, for the waters have risen to my neck.
I have sunk into the mud of the deep and there is no foothold.
I have entered the waters of the deep and the waves overwhelm me.
I am wearied with all my crying, my throat is parched. My eyes are wasted away from 
looking for my God\footnote{{\em Psalter\/} 69, 2-4}.' 
Martin is at the moment not prepared to argue with Anna.
He leaves her behind and is once again buried deep in thought: `These deleted pair of 
numbers on the cards -- that must be daily codes, on which the sisters agree in the
bath in the morning. 
Yes, certainly, that must be it! 
But: Bianca and Doris surely won't explain anything to me; I can only interfere by 
way of trial into their dialogues, in order to find out whether I am on the right 
track!' 
Thereby the first problem is already that the sisters utter their numbers very 
fast. Therefore Martin gets himself a spare dictating machine in the hospital 
administration during his lunch break. 
He pretends to make notes while reading a medical text book in order to prepare it 
in writing later. 
The secretary looks at him askance, but finally she gives it to him. 
Back in his room Martin fetches the playing card he had found by the sisters. He 
studies the pair of numbers on the back: 
\begin{center}
{\bf \dots \,\,4\,267\,,\,7\,\,;\quad 17\,819\,,\,5\,\,;\quad 59\,989\,,\,33
\,\,\dots}
\end{center}
It is remarkable that the first number of a pair is always considerably larger than 
the second. 
Lacking any better inspiration, Martin starts with the obvious basic assumption 
that in a secret language the 26~letters of the alphabet are assigned to the first 
26 numbers. 
Therefore Martin creates a spreadsheet: 
\begin{center}
\begin{tabular}{c|r||c|r}
A & 1 & N & 14 \\
B & 2 & O & 15 \\
C & 3 & P & 16 \\
D & 4 & Q & 17 \\
E & 5 & R & 18 \\
F & 6 & S & 19 \\
G & 7 & T & 20 \\
H & 8 & U & 21 \\
I & 9 & V & 22 \\
J & 10 & W & 23 \\
K & 11 & X & 24 \\
L & 12 & Y & 25 \\
M & 13 & Z & 26
\end{tabular}
\end{center}
\[\]
But it is not as simple as that with the sisters: they operate with much larger 
numbers, and for this simple spreadsheet you don't need any code in the form of a 
pair of numbers. 
`If this approach is correct', considers Martin, `then, the numbers from 1 to 26, 
corresponding with the 26 letters in the alphabet, are used to create even larger 
numbers with the help of the pair of numbers, in such a way so that one can also 
reckon back again clearly on the numbers 1 to 26; given that one knows the code.
But how does it work? According to which rules?' 
Martin is frustrated: there are an infinite number of possibilities for such a coding. 
`I must proceed differently, and try something different', he thinks to himself. 
He gets a memo pad out and thinks for a long time about a short sentence, in which 
as many letters as possible occur several times. 
Finally he writes in large type characters over the whole page: 
\begin{center} 
I AM BIANCA 
\end{center} 
So far, so good. Now to the code. 
`To have any chance to discover a regularity with coding of these words later, I 
must as carefully as possible select the pair of numbers. 
The second number of the pair is chosen always quite small by the sisters; what 
will happen if I set it to 1? 
Then perhaps its influence could be rather small in the beginning with the coding.'
Martin is satisfied with this working hypothesis. 
`And how do I select the first number of the code?' 
Martin hasn't got the faintest idea.
Finally he thinks it would be the best to equate it with the number of letters of the alphabet. 
Thus he notes in the left upper corner of the page: 
\begin{center}
{\bf 26\,,\,1\,\,.}
\end{center}
Martin looks sceptically at the result. 
`Hopefully I am not wrong!' he murmurs to himself. 
`Well, let's see.' -- \\ 
He knocks on the door of Bianca and Doris. `Come in!' 
The sisters look at Martin expectantly. They are now much calmer than when earlier 
in the morning he had found the card.
Looking guilty the Zivi comes to the living room desk, at which the two women are sitting. 
In the middle on the desk lies a playing card -- a pair of numbers is written in large 
digits upon it. 
The sisters don't attempt to hide the card from Martin. 
Just the opposite: since they look at him silent and watch each of his movements, 
Martin feels invited into the world of their secrets. 
He is unbelievably relieved about this silent request, and has the impression, that 
they have been sitting here at the desk for a long time waiting for him. \\
Somehow he feels very honoured that they trust him. 
`Can you translate something for me?\footnote{{\em Cryptanalysis\/}, Martin instinctively tried 
the typical approach of a cryptanalyst (often popularly referred to as a codebreaker): He tries to achieve more information by getting pairs of 
cleartext and ciphertext, fitting together (here especially chosen-plaintext). 
In World War 2 the British applied this approach very successfully to break the Enigma: some 
single objects had been attacked to provoke the according radio message.
In this story we have the very special case that the sisters are cooperative and deliver the 
fitting text pairs by will.}'
He places the dictaphone on the desk and switches it on. 
Doris with almost exaggerated friendliness answers gladly, `Of course'.
Martin submits his prepared sheet of paper to Bianca, points to the code numbers 26, 1 and 
observes the womens' faces. 
Bianca smiles, as if the required encoding appears to her much too simple. 
After a few moments however she bursts out with 9 numbers: 139, 235, 325, 184, 503, 365, 378, 81, 53. 
`What good luck that I had the idea of having a dictaphone!' Martin thinks about how quickly Bianca speaks. 
He takes the dictaphone and the paper again, asks casually, whether he can still do something 
for them, and disappears quickly into his room. 
There he listens to the dictating machine and writes the numbers under the letters: 
\[
\begin{array}{cccccccccccc}
I & A & M & B & I & A & N & C & A \\
139 & 235 & 325 & 184 & 503 & 365 & 378 & 81 & 53
\end{array} 
\]
Afterwards the young man is quite desperate.
The letter A was coded once with 53, with 235 and with 365. 
According to my spreadsheet however I would assign the 1 to it. 
Martin stares at the sheet of paper. 
`And somehow the 26 seems to have a role as an encoding number. \dots' 
He carries out the simplest arithmetic operation with the 26, addition, in order to get from the 1 to round about 53: 
\[
\begin{array}{l}
1\,+\,26\,=\,27 \\
1\,+\,26\,+\,26\,=\,{\bf 53}
\end{array} 
\]
`Oh! Is that coincidence?' 
Probably not, because Martin rapidly checks it: 
\[
1\, +\, 14\cdot 26\, =\, {\bf 365}\,\,.
\]
There we have the solution: Each number can represent the letter A, which leaves the remainder of 1 while divided by 26. 
Thus Bianca could also have selected for A 79, because \(1+3\cdot 26=79 \). 
And therefore a letter can be represented by very many different numbers. 
`Very smart!' Martin checks the remaining letters from Biancas translation: 
\[
\begin{array}{cccclccccl}
I & \longrightarrow & 9 & \longrightarrow  & 9  & + & 5\cdot 26 & \,=\, &  139 \\
A & \longrightarrow & 1 & \longrightarrow  & 1  & + & 9\cdot 26 & \,=\, &  235 \\
M & \longrightarrow & 13 & \longrightarrow & 13 & + & 12\cdot 26 & \,=\, & 325 & \mbox{etc.}
\end{array} 
\]
`That can also easily be translated back again. 
One must only look, which remainder between 1 and 26 a number leaves when it is divided by 26:' 
\[
139\,:\,26\,=\,5\,\,\mbox{remainder} \,\,9\,; \,\mbox{therefore}\,\,9\,\longrightarrow
\,\mbox{I}\,\,.
\]
Martin leans back satisfied. 
He has revealed the first bit of the secret. 
But: what influence does the second number in the code have, which he had set so far to 1?
Martin erases the code numbers of 26\,,\,1 on the white sheet of paper and adds instead {\bf 51\,,\,3} in; 
he selected the 51 arbitrarily, and he intentionally uses with the 3 still another particularly small number. \\ 
After dinner he visits the sisters again and has them encode the text once again with a new code.
Now Bianca supplies him, again with the help of the dictating machine: 
\[
\begin{array}{cccccccccccc}
I & A & M & B & I & A & N & C & A \\
1647 & 103 & 871 & 620 & 525 & 2500 & 2336 & 486 & 1174
\end{array} 
\]
Because the first code number is 51, Martin speculates on account of his previous experience that 
it depends here only on the remainders after a division by 51;
and therefore numbers, which code {\em the same letter \/}, would have to result in {\em the same remainder \/}, 
when dividing them by 51. 
First of all Martin checks this hypothesis and works out the remainders with a calculator: 
\[\begin{array}{crccrc}
{\bf I}: & 1647 & = 32\cdot 51 + {\bf 15} \qquad & {\bf A}: & 103  & = 2\cdot 51 + {\bf 1} \\
         & 525  & = 10\cdot 51 + {\bf 15} \qquad &          & 1174 & = 23\cdot 51 + {\bf 1} \\ 
         &      &                                &          & 2500 & = 40\cdot 51 + {\bf 1} \\*[+6pt]
{\bf M}: & 871 & = 17\cdot 51 + {\bf 4} \qquad & {\bf B}: & 620 & = 12\cdot 51 + {\bf 8}  \\*[+6pt] 
{\bf N}: & 2336 & = 45\cdot 51 + {\bf 41} \qquad & {\bf C}: & 486 & = 9\cdot 51 + {\bf 27} 
\end{array} \]
The representation of a letter with the same remainder when divided by 51 has proved to be correct. 
Now the question about the influence of the second code number~3 still remains. 
Except to the letter {\bf A}, the remainders determined above however no longer correspond to the
natural numbering of the letters in the alphabet with the numbers of 1 to 26; well, and with {\bf C} 
and {\bf N} the numbers 27 and 41 occur, which are even larger than 26.
That must be the effect of the code number~3! 
In the case of the division by 51 altogether 51 possible remainders can result, and somehow the code 
number 3 selects from them 26 and assigns to these remainder the letters of the alphabet.
Martin jumps up from the desk and runs back and forth around in his close, little room like a tiger in the cage.
`I must find out, how the 3 changes the numbers of 1,2,\dots,26. 
Otherwise I'm not going to get to the bottom of this!' He talks loudly with himself. 
The simplest arithmetic operations have already led him once to the target. 
`Let's see': \\ 
{\em addition with 3:} \quad {\bf A} \(\longrightarrow \) 1 \(\longrightarrow \)
1+{\bf 3} = 4\,, \quad {\bf B} \(\longrightarrow \) 2 \(\longrightarrow \)
2+{\bf 3} = 5\,\,. \\
`Rubbish! In Biancas encoding the {\bf A} yields again 1 and the {\bf B} produces the 8. 
That would have been really too simple!' \\ 
{\em multiplication with 3:} \quad {\bf A} \(\longrightarrow \) 1
\(\longrightarrow \) {\bf 3}\,\(\cdot \)\,1 = 3\,, \quad {\bf B}
\(\longrightarrow \) 2 \(\longrightarrow \) {\bf 3}\,\(\cdot \)\,2 = 6\,\,. \\
`That isn't it either!' Martin has drops of sweat on his forehead. 
He closes his eyes and sees himself sitting in his maths course at school.
`With {\em Dr.~Bruckner \/} we did actually learn a lot! 
If it would only help me now!' 
In fact, I was not the worst. 
Even the formulae for powers and logarithms caused no difficulties for me, unlike my friend {\em Carsten\/} \dots. 
`Hold on! What was that, right now?' 
Martin talks again loudly to himself. 
`Powers \dots -- the continued multiplication!!' 
He rushes to the desk and calculates: 
\[\begin{array}{ccrcrcl}
{\bf A} & \longrightarrow & 1 & \longrightarrow & 1^{{\bf 3}} & = & 1\cdot 1
\cdot 1 = {\bf 1} \\
{\bf B} & \longrightarrow & 2 & \longrightarrow & 2^{{\bf 3}} & = & 2\cdot 2
\cdot 2 = {\bf 8} \\
{\bf C} & \longrightarrow & 3 & \longrightarrow & 3^{{\bf 3}} & = & 3\cdot 3
\cdot 3 = {\bf 27} \\
{\bf I} & \longrightarrow & 9 & \longrightarrow & 9^{{\bf 3}} & = & 9\cdot 9
\cdot 9 = 729 = 14\cdot 51 + {\bf 15} \\
{\bf M} & \longrightarrow & 13 & \longrightarrow & 13^{{\bf 3}} & = & 13\cdot 13 \cdot 13 = 2197 = 43\cdot 51 + {\bf 4} \\
{\bf N} & \longrightarrow & 14 & \longrightarrow & 14^{{\bf 3}} & = & 14\cdot
14\cdot 14 = 2744 = 53\cdot 51 + {\bf 41}
\end{array} \]
`Wow! I've got it. I am the best! 
3, the second number in the code is an {\em encoding exponent \/}!'
Martin jumps off his bed with joy. 
`I have seen through Bianca and Doris. I have now cracked all the secrets!' 
Martin returns to the desk and encodes the word THANKS with the key {\bf 51,3} in his joy: 
\[
\!\!\!
\begin{array}{ccrcrcrcrcrcr}
{\bf T} & \! \longrightarrow \! & 20 & \! \longrightarrow \! & 20^3 & = \! & 8000 & = \! & 156\cdot 51 + {\bf 44} & \! \longrightarrow \! & 6\cdot 51 + {\bf 44} & = \! & {\bf 350} \\
{\bf H} & \! \longrightarrow \! & 8 & \! \longrightarrow \! & 8^3 & = \! & 512 & = \! & 10\cdot 51 + {\bf 2} & \! \longrightarrow \! & 1\cdot 51 + {\bf 2} & = \! & {\bf 53} \\
{\bf A} & \! \longrightarrow \! & 1 & \! \longrightarrow \! & 1^3 & = \! & 1 & = \! &  {\bf 1} &
\! \longrightarrow \! & 21\cdot 51 + {\bf 1} & = \! & {\bf 1072} \\
{\bf N} & \! \longrightarrow \! & 14 & \! \longrightarrow \! & 14^3 & = \! & 2744 & = \! &
53\cdot 51 + {\bf 41} & \! \longrightarrow \! & 2\cdot 51 + {\bf 41} & = \! & {\bf 143}
\\
{\bf K} & \! \longrightarrow \! & 11 & \! \longrightarrow \! & 11^3 & = \! & 1331 & = \! &
26\cdot 51 + {\bf 5} & \! \longrightarrow \! & 17\cdot 51 + {\bf 5} & = \! & {\bf 872}
\\
{\bf S} & \! \longrightarrow \! & 19 & \! \longrightarrow \! & 19^3 & = \! & 6859 & = \! & 134\cdot 51 + {\bf 25} & \! \longrightarrow \! & 3\cdot 51 + {\bf 25} & = \! & {\bf 178}
\end{array} \]
Martin has arbitrarily chosen the factors 6, 1, 21, 2, 17 and 3 to multiply by 51. 
He writes the result on a small piece of paper: 
\begin{center} 
{\bf 51,3}: \qquad 350, 53, 1072, 143, 872, 178. 
\end{center} 
He sprints down the passages of the sanatorium and after several moments he reaches Bianca's and Doris' room. 
He knocks at the door and pushes the piece of paper through under the door. 
He hears the sound of somebody coming to the door. 
Martin waits and listens closely.
Biancas loudly expressed `thank you' gives him the confirmation for the correctness of his considerations 
and calculations. 
As he softly returns to his room, he hears the two sisters giggling.

%%%%%%%%%%%%%%%%%%%%%%%%%%%%%%%%%%%%%%%%%%%%%%%%%%%%%%%%%%%%%%%%%%%%%%%%%%%%%%%%%%%%%%%%%%%%%%%%%%%%%%%%%%%%%%%%%%

\begin{center} 
{\bf V.} 
\end{center} 
In the course of the next days Martin feels bad-tempered and tired at work. 
The initial euphoria about the discovery of the encoding mechanism has given way to a sheer frustration. 
Certainly, he has written a program with his laptop with which he can encode all letters after input of a 
pair of code numbers, but the decoding gives him a hell of a puzzle. 
Even with the knowledge of a key he has no plan of how he can win the corresponding number between 1 and 26 
from the number which represents a letter. The numbers between 1 and 26 then identify the letter in the alphabet.
Up to now he controls only the mechanism of {\em encoding\/}, but not the one of {\em decoding\/}; and for 
communication with Bianca and Doris just the decoding is particularly important. 
Only in the case if the coding exponent is set to 1, the letters keep their alphabetically given numbering from 
1 to 26, and then a simple division with a remainder reduces the numbers to the letters. 
But the sisters never use the encoding exponent~1. \dots -- \\ 
Martin also has his own feelings about the bizarre world of Bianca and Doris. 
He puts them into his electronic diary: \\ \\ 
`Who knows according to which plan Bianca and Doris transform their words into numbers and decipher them again?
Even if this mechanism can be described by formulae and should run in a usual programming language in the 
computer, it doesn't mean that Bianca and Doris adhere to the same principles. 
Already the rate, with which they manage without technical aids, is a note on the fact that in their
brains probably something runs {\em completely differently \/} to other people. 
How can it otherwise be understood that {\em Mozart \/}, who has heard coincidentally a mass somewhere, could 
write it down later from memory without an error and completely in score? 
Most of all {\em Mozart \/} and {\em Einstein \/} were humans, who had only the luck that they found highest 
social acknowledgment with the results of their unusual networking in the brain. 
Between Rudolf's astute chess strategies and the ability of Mozart to be able to compose a whole opera only 
in his head no gradual difference exists. 
Only the objects are others and the attitude of `normal' humans to them!! 
If Anna says the whole psalter by heart, she belongs in this sense to the degree wavers between genius and 
madness, even if nobody wants to hear her monologues. -- \\ 
In this sanatorium something is rather lugubrious: I can {\em sense\/} it, and Bianca and Doris {\em know\/} 
what it is! 
Perhaps I will finally get to the bottom of it. 
How on earth can Dr.~Weissenborn consider the sisters to be  insane? 
In a surroundings felt hostile by them their encoded communication provides to them a feeling of security. 
Is it different with us so-called `normal' beings in the world outside?
Who is not afraid when using homebanking or with cash dispensers that strangers can access our accounts? 
Nevertheless, nowadays a modern person can no longer manage without secret numbers! 
Madness is a pure definition thing!!'

%%%%%%%%%%%%%%%%%%%%%%%%%%%%%%%%%%%%%%%%%%%%%%%%%%%%%%%%%%%%%%%%%%%%%%%%%%%%%%%%%%%%%%%%%%%%%%%%%%%%%%%%%%%%%%%%%

\begin{center} 
{\bf VI.} 
\end{center} 
In spite of thinking hard Martin fails to solve the problem of decoding Bianca's and Doris's secret language. 
Nevertheless he does not give up: He uses a free day to go to Heidelberg and meets {\em Dr.~Praetorius\/} 
from the university Maths Department. 
Martin had got in touch with this man whose name he had traced through the personal catalogue of the university, 
and he rang up to arrange an appointment. 
The academic receives him in a friendly way and listens patiently, when Martin tells him about his work, the sisters and 
their numbers language.
Dr.~Praetorius is also fascinated by the abilities of both women, although he has also no explanation for this 
peculiar phenomenon either.
Then Martin explains to him the encoding technique, as far as he could already analyse it.
His real aim to find out something about what is going on in the sanatorium from the sisters, although of course 
the Zivi does not mention this. \\ \\ 
`Have you brought such a card with the coding codes?' asks Dr.~Praetorius. 
Martin hands him the playing card which he had found behind the bathroom mirror, and the academic 
types some numbers from the encryption codes into his computer. 
He lets the machine work, nods triumphantly and turns again towards Martin.
`I'm sure, you know what prime numbers are?' 
Martin has no doubt about the answer, as he has learned all this with Dr.~Bruckner at school. 
`These are numbers without a divisor' he answers. `That's it, more exactly said, without a {\em true\/} divisor, 
because {\em every\/} number is always divisible by 1 and itself. 
The first prime numbers are:
\begin{center} 
2, \, 3, \, 5, \, 7, \, 11, \, 13, \, 17, \, 19, \, 23, \, 29, \quad 
\dots 
\end{center} 
Has it already struck you that the first number of every code pair is always composed from {\em two\/} prime numbers?'
Martin shakes the head. 
`Look, here!' 
The mathematician points to the computer screen: 
\[\begin{array}{rcc}
4267 & = & 17\cdot 251 \\
17\,819 & = & 103\cdot 173
\end{array} \]
`The numbers 17, 251, 103 and 173 are all prime numbers. 
And one must just know how to disassemble these prime factors of the first code number in each case to be able to decode!' 
Martin looks at the mathematician admiringly.
`You asked me how to decode this.
I want to explain it to you. 
The science of cryptography is the mathematical partial discipline in which one examines and developes encryption 
techniques. The procedure of both of your sisters from the hospital has been known for a long time; it is a more 
simple procedure. 
Do not be frightened! It is very effective nonetheless. 
The secret with the decoding is based on a knowledge, we owe old {\em Fermat\/}\footnote{{\em Pierre de Fermat\/}, 
French mathematician, 1601$-$1665.}. And so it works!' 
Dr.~Praetorius reaches for the notes which Martin has brought to him. 
`You have had encoded the sentence I AM BIANCA with the key {\bf 51\,,\,3}, and, for example, you got the number 871 
from Bianca for the letter {\bf M}.
We want to name the first number from the key, the 51, {\em main module\/} and the 3 you have called 
as {\em encoding exponent\/}; this should be right to me. 
You'll certainly be surprised to learn that a {\em decoding exponent\/} can be determined. 
And right for this you need to find the prime numbers of the main module; this procedure is called factorization.
Here it is 51. You have chosen 51 arbitrarily at that time, and by chance it is also a product of two prime numbers.' 
Dr.~Praetorius writes on a sheet of paper: 
\begin{center} 
\(51 \, \, = \, \, 3\cdot 17 \) 
\end{center} 
`Now please reduce both prime factors in each case with 1 and multiply the results by each other. 
The result, besides, we want to call the {\em auxiliary module\/}.' 
\begin{center} 
auxiliary module\,=\,\((3-1)\cdot (17-1)\,=\,2\cdot 16 \,=\,32 \)
\end{center}
`Can you give me a number which has the following property: 
If one multiplies it by the encoding exponent 3 and divides afterwards the product by the auxiliary module 32, 
the remainder 1 remains with this division by 32\dots?'
Martin considers it shortly, then he has the result: `Eleven.' 
`Very good!' The academic praises him and writes: 
\[
\frac{11\cdot 3}{32}\,=\,1 \,\,\,\mbox{remainder}\,\,1\,\,,\quad \mbox{or}\,\,
11\cdot 3\,=\,33\,=\,1\cdot 32 + \mbox{1}\,\,.
\]
`Where does the number one come from which should be left with the division by the auxiliary module as a 
remainder?' asks Martin. 
`All the same, with which numbers you ever work, at this point the remainder~1 must {\em always\/} be left 
with the division by the auxiliary module. 
This has something to do with the knowledge of old Fermat which underlies this calculation. 
Yes, and the number 11 which you have just determined, is the searched for decoding exponent.' marvels Martin. 
`And now I should multiply 871, which Bianca has named for the letter {\bf M} to me, eleven times by 
itself?' he asks unbelievingly.
`You can do nothing wrong. 
But this number becomes rather large. 
Since we want to divide afterwards the result again by the {\em main module \/} 51 and want to determine 
the remainder, it offers itself to determine from the encoding number of 871 the remainder with the division 
by the main module 51 {\em before \/} it is even powered by 11. 
You have already done this in your recordings:' 
\[
871\, \, =\, \, 17\cdot 51\, +\, {\bf 4} 
\] 
`In place of the 871 the 4 may also be powered. 
That is much more comfortable.' Martin is completely amazed: 
\[
4^{11}\, =\, 4\,194\,304\, =\, 82\,241\cdot 51\, +\, {\bf 13} 
\] 
Now Martin is absolutely astonished: In his spreadsheet the 13 is located for the letter {\bf M}! 
`And by this powering with 11 you can reconvert the whole line of numbers specified by Bianca again into the letters. 
By the way: This coding procedure can easily be programmed for your laptop.' 
`Well, one must hit upon this first!', means Martin, `no wonder I have not cracked this problem!' 
However, Dr.~Praetorius still holds another trump in his hand. 
`In practice two people who exchange secret messages with this procedure can announce the code key even in 
{\em public\/}. 
Thereby the code key is the main module with the encoding exponent, but with the assumption that the main 
module is very, very big.'
The young man gives the academic a disbelieving look. 
However, he smiles and continues: 
`The problem with codings are not actually the procedures. 
They are known to many and are based mostly on rather elementary mathematic principles. 
The art of decoding mostly consists in disassembling (factorizing) of very large numbers in a product of 
the prime numbers, as we have also done it here with our main module \(51=3\cdot 17 \). 
With such small numbers, even the ones applied by Bianca and Doris, this does not present a problem.
You can quickly find the disassembling with arithmetic tables or with the help of a calculator. 
However, imagine a 100-decimal digit long main module where both prime factors are approximately 50-digits long! 
These prime factors stay naturally secret. 
It also takes a long while with larger computers, until a third person finds these prime factors!' -- \\ \\ 
As Martin enters the bus which takes him back to the Odenwald, his brain is humming.
But now he is full of confidence that he can communicate with Bianca and Doris in the future without limits. 
Indeed, with the help of the dictaphone, the laptop and his knowledge this should now be possible! 
-- However, he will not point out to the sisters the idea of applying 100-digit code keys.

%%%%%%%%%%%%%%%%%%%%%%%%%%%%%%%%%%%%%%%%%%%%%%%%%%%%%%%%%%%%%%%%%%%%%%%%%%%%%%%%%%%%%%%%%%%%%%%%%%%%%%%%%%%%%%%

\begin{center} 
{\bf VII.} 
\end{center} 
Although Martin returns late in the evening to the sanatorium, he still sits down to his laptop and 
maps out a decoding program. 
Already on the bus ride he had made some sketches, describing the decoding concept of Dr.~Praetorius. 
He checks it with the key {\bf 51\,,\,3} for the coding of the sentence I AM BIANCA. 
It runs perfectly. 
Now together with the coding program he already had finished some days ago, Martin disposes of a complete 
interpreter's system for the secret language of the sisters. 
Dr.~Praetorius had forecast correctly that the associated programs are not complicated at all and the 
routines run very quickly. 
One can master separate words or short sentences also with a pocket calculator, but not longer conversations.
Here only the laptop can help. 
Finally, extremely tired, but altogether satisfied with himself, Martin falls into the bed. \\ \\ 
Martin's first task in the morning is to look after `his charges'. 
Rudolf and Anna are late risers, but Bianca and Doris get up very early in the morning. 
Martin nervously enters their room; he has brought the dictaphone and the laptop with him. 
The sisters sit silently at their living room table, they do not react to Martin's friendly `Good morning'. 
They also remain quiet, when he puts both devices on the table.
`This behavior must indicate something!' the Zivi asks himself. `Of course!' it suddenly occours to him. 

`The password of the day! At the bathroom mirror!' 
And he's absolutely right, the sisters have hidden deliberately a playing card behind the mirror for him. 
They must have suspected that he will communicate today in their way with them. 
The password of the day reads: 
\begin{center} 
{\bf 681\,,\,151\,\,.}
\end{center} 
Martin turns to the women at the desk. 
He puts in the code key and translates the words GOOD MORNING. 
He reads out the numbers on his display loudly. 
The spell is broken. 
The sisters answer in the code, Martin translates back. 
At first he still uses the dictaphone, but, finally, he notices that it works fine without recording equipment. 
Soon he has the exercise and types the numbers of the women directly during their conversation; his laptop 
translates in lightning speed. 
Certainly, it is a little bit slower than the sisters are accustomed to when talking between themselves because Martin  
needs to use a keyboard. 
But the conversation is fluent, even if rather unusual. \\ 
Gradually Martin passes over from the talk of the insignificance to the doctors and nurses in the house. 
Finally: He types the question with shaky hands in his computer WHAT DO THEY DO WITH YOU IN THE TOWER.
While he is reading out the numbers of the encoding, his throat goes dry. 
It is Doris who answers: 
\begin{center} 
4258, 7182, 9168, 6089, 1734, 9391, 8487, 1363, 853, 10898, 12618, 10512, 172, 7493, 1734, 10229, 7182, 13105 \,\,. 
\end{center} 
`A short answer', thinks Martin, before he operates the key for the execution of the translation. 
It takes less than a second for the machine to work. 
Martin stares aghastly at the screen of his laptop. 
He cannot believe what he is reading. 
But then he acts rapidly. -- \\ \\ 
A few days later a huge scandal is uncovered in the Odenwald clinic and is condemned throughout the nation.

\[\]
\[\]
\[\]
\[\]
\hrulefill
\[\]
\[\]

\noindent
{\em Note of the author:\/}
\begin{itemize}
\item[(1)] In order not to confuse the reader, a necessary mathematic
  presupposition was not mentioned with the description of the coding
  procedure.  However it is not required for the purposes of this story.
  Nevertheless, it should be mentioned here on behalf of the completeness
  that not {\em every\/} arbitrary pair of numbers is suitable as a coding
  code where the main module is a product of two prime numbers.  Rather
  the fact that the auxiliary module and the encoding
  exponent have {\em no common devisor\/} should be respected.
  Otherwise no decoding exponent would be found. \\*[+2pt]
  {\bf Example}: The pair 35\,,\,15 is not good as a coding code
  because the auxiliary module \((5-1) \cdot (7-1) =24\) and the coding
  exponent \(15\) have the common divisor \(3\).
\item[(2)] However, in practical applications one does not encode a text literally
as it is described in the story as a so-called RSA-procedure.  One
block-likely codes a text, while one combines the letters of the text
sequentially to blocks of firm length and assigns with a certain procedure
{\em a separate number\/} to these blocks.

\item[(3)] Besides it is usual in the practice -- in contrast to Doris' and
Bianca's method of approach -- to present the coding not in a form of
numerical orders, but in letters which prove naturally nonsense as a text.

\item[(4)] One finds details to this RSA-procedure in the following book: {\small
  NEAL KOBLITZ}, {\em A Course in Number Theory and Cryptography\/},
second edition, chapter IV, paragraph 2; Springer, 1994. \\
\end{itemize}
\[\]
\hrulefill
\[\]
\[\]
{\em Note of the translator:} \\
I want to thank Mr Richard Christensen for his review of this translation.
\\[1ex]
With CrypTool version 1.3 and higher it is easy to comprehend this variant of the RSA encryption scheme by using 
the following steps: \\
menu {\bf Indiv. Procedures / RSA Demonstration / RSA Cryptosystem\dots}. \\
Within the following dialog box {\bf The RSA Cryptosystem} choose the second option -- knowing only public parameters -- and factorize N. The given 
two code numbers (e.g. 51, 3) of Bianca and Doris are N and e. The algorithms behind can be switched to this 
special variant by hitting the button \textbf{Options for alphabet and number system\dots}. \\
In the following dialog 
box {\bf Options for RSA Encryption} select \textbf{Dialogue of the Sisters} and limit the alphabet to exactly 26 
letters (A, B, ..., Y, Z).



\end{document}
