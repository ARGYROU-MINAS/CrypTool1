% $Id$
% ............................................................................
%      V O R W O R T  und  E I N F � H R U N G (Zusammenspiel Skript-CT) 
% ~~~~~~~~~~~~~~~~~~~~~~~~~~~~~~~~~~~~~~~~~~~~~~~~~~~~~~~~~~~~~~~~~~~~~~~~~~~~


% --------------------------------------------------------------------------
\section*{Preface to the 8th Edition of the CrypTool Script\footnote{For the 9th edition no extra preface was written.}}
\addcontentsline{toc}{section}{Preface to the 8th Edition of the CrypTool Script}

Starting in the year 2000 this script became a part of the 
CrypTool\index{CrypTool} package. It is designed to accompany the program 
CrypTool by explaining some mathematical topics in more detail, 
but still in a way which is easy to understand.

In order to also enable developers/authors to work together independently 
the topics have been split up and for each topic an extra chapter has been 
written which can be read on its own. The later editorial work in TeX added 
cross linkages between different sections and footnotes describing where you
can find the according functions within the CrypTool\index{CrypTool} 
program \hyperlink{appendix-menutree}{(see menu tree in appendix A).}
% in \ref{s:appendix-menutree}   AUFPASSEN, DASS nichts doppelt !!
% \hypertarget{appendix-menutree}{}\label{s:appendix-menutree}
Naturally there are many more interesting topics in mathematics and
cryptography which could be discussed in greater depth -- therefore this
is only one of many ways to do it.

\vspace{12pt}
The rapid spread of the Internet has also lead to intensified research in the
technologies involved, especially within the area of cryptography where a good
deal of new knowledge has arisen.

%This edition of the script adds some topics, but mainly updates areas (e.g. the
%summaries of topical research areas):
Again this edition of the script amended and updated some topics, e.g.:
\vspace{-7pt}
\begin{itemize}
  \item the search for the largest prime numbers
        (Mersenne and Fermat primes, ``M44'' or the biggest non-Mersenne prime
         found in March 2007 when investigating Sierpinski numbers) \\ 
	(chap. \ref{spezialzahlentypen}, \ref{RecordPrimes};
         table \ref{L_n_Largest_Known-Primes}, 
         page \pageref{L_n_Largest_Known-Primes}), 
  \item the factorisation of big numbers like ``C307''\index{C307} 
        (chap. \ref{C307}),
  \item the current record (April 2007) calculating the discrete logarithm 
        (The used sample power consisted of the first 160 digits of Euler's
        number e and the calculation was done modulo a 160 digit prime)
        (chap. \ref{L_Discrete_Logarithm}),
%  \item progress in cryptanalysis of hash algorithms 
%        (chap. \ref{collision-attacks-against-sha-1}) and
%  \item progress in ideas for new crypto methods (RSA successor) 
%        (chap. \ref{xxxxxxxxxBrute-force-gegen-Symmetr})\index{xxxxxxxxxxxxxxxx} and
  \item the list of movies or novels, in which cryptography or number theory 
        played major role (see appendix \ref{s:appendix-movies});
        and where primes are used as hangers  
        (see curiouses in \ref{HT-Quaint-curious-Primes-usage}).
\end{itemize}
Newly added is chapter \ref{Chapter_Crypto2020} {\bf Crypto 2020}
by Prof. Buchman et al. and the appendix \ref{s:appendix-Learn-NT} 
about the {\bf learning tool for number theory}.

\vspace{8pt}
The first time the document was delivered with CrypTool\index{CrypTool} 
was in version 1.2.01. Since then it has been expanded and revised in almost
every new version of CrypTool (1.2.02, 1.3.00, 1.3.02, 1.3.03, 1.3.04, 1.4.00
and now 1.4.10).

I'd be more than happy if this also continues in the further open-source
versions of CrypTool\index{CrypTool}.

I am deeply grateful to all the people helping with their impressive
commitment who have made this global project so successful.
Thanks also to the readers who sent us feedback.
% Especially I would like to acknowledge the English language proof-reading
% of this script version done by Richard Christensen and Lowell Montgomery.

I hope that many readers have fun with this script and that they get 
out of it more interest and greater understanding of this modern but 
also very ancient topic.
\\[1.5\baselineskip]
\enlargethispage*{2\baselineskip}
\nopagebreak
Bernhard Esslinger
\\[\baselineskip]
Frankfurt (Germany), June 2007
\vspace{-12pt}

% --------------------------------------------------------------------------
\newpage
\section*{Introduction -- How do the Script and the Program Play together?}  \addcontentsline{toc}{section}{Introduction -- How do the Script and the Program Play together?}


\textbf{This script}

This document is delivered together with the program CrypTool\index{CrypTool}.
\par \vskip + 3pt

The articles in this script are largely self-contained and
can also be read independently of CrypTool\index{CrypTool}.

Chapters  \ref{Chapter_ModernCryptography} (Modern Cryptography) and 
\ref{Chapter_EllipticCurves} (Elliptic Curves) require a deeper knowledge
in mathematics, while the other chapters should be understandable with a 
school leaving certificate.

The authors
% \hyperlink{appendix-authors}{(authors)}
% in \ref{s:appendix-authors}
% \hypertarget{appendix-authors}{}\label{s:appendix-authors}
have attempted to describe cryptography for a broad 
audience -- without being mathematically incorrect. We believe that this
didactical pretension is the best way to promote the awareness for IT
security and the readiness to use standardised modern cryptography.
\par \vskip + 15pt


\textbf{The program CrypTool\index{CrypTool}}

CrypTool\index{CrypTool} is a program with an extremely comprehensive online
help enabling you to use and analyse cryptographic procedures within a
unified graphical user interface.\par \vskip + 3pt

CrypTool\index{CrypTool} was developed during the end-user awareness program
at Deutsche Bank in order to increase employee awareness of IT security and provide them with
a deeper understanding of the term security.
A further aim has been to enable users to understand the cryptographic
procedures. In this way, using CrypTool
as a reliable reference implementation of the various encryption procedures,
you can test the encryption implemented in other programs. \par \vskip + 3pt

CrypTool\index{CrypTool} is currently been used for 
training in companies and teaching at school and universities, and
moreover several universities are helping to further develop the project.
\par \vskip + 15pt


\textbf{Acknowledgment}

At this point I'd like to thank explicitly the following people who
particularly contributed to CrypTool\index{CrypTool}. Without their special
talents and engagement CrypTool would not be what it is today:
\vspace{-7pt}
%\begin{itemize}
\begin{list}{\textbullet}{\addtolength{\itemsep}{-0.5\baselineskip}}
   \item Mr.\ Henrik Koy
   \item Mr.\ J\"org-Cornelius Schneider
   \item Mr.\ Florian Marchal
   \item Dr.\ Peer Wichmann
   \item Staff of Prof.\ Claudia Eckert, Prof.\ Johannes Buchmann and Prof.\ Torben Weis.
\end{list}
%\end{itemize}
Also I want to thank all the many people not mentioned here for their 
hard work (mostly carried out in their spare time).
\\

Bernhard Esslinger

Frankfurt (Germany), July 2008

% Local Variables:
% TeX-master: "../script-en.tex"
% End:
