% .........................................................................................
%                                      E I N F � H R U N G
% ~~~~~~~~~~~~~~~~~~~~~~~~~~~~~~~~~~~~~~~~~~~~~~~~~~~~~~~~~~~~~~~~~~~~~~~~~~~~~~~~~~~~~~~~~
\section*{Einf"uhrung}  \addcontentsline{toc}{section}{Einf"uhrung}

Dieses Skript wird zusammen mit CrypTool ausgeliefert.

CrypTool ist ein Programm mit sehr umfangreicher Online-Hilfe, mit dessen Hilfe Sie unter einer einheitlichen
Oberfl"ache kryptographische Verfahren anwenden und analysieren k"onnen.\par \vskip + 3pt

CrypTool wurde im Zuge des End-User Awareness-Programmes entwickelt, um die Sensibilit"at der
Mitarbeiter f"ur IT-Sicherheit zu erh"ohen und um ein tieferes Verst"andnis f"ur den Begriff Sicherheit
zu erm"oglichen. \par \vskip +3pt 

Ein weiteres Anliegen war die Nachvollziehbarkeit der in der Deutschen Bank eingesetzten
kryptographischen Verfahren. So ist es mit CrypTool als verl"asslicher Referenzimplementierung
der verschiedenen Verschl"usselungsverfahren (aufgrund der Nutzung der Industrie-bew"ahrten 
\index{Secude GmbH} Secude-Bibliothek) auch m"og\-lich, die in anderen Programmen eingesetzte
Verschl"usselung zu testen.\par \vskip + 3pt

Inzwischen wird CrypTool in Ausbildung und Lehre eingesetzt und von mehreren Universit"aten
mit weiterentwickelt. \par \vskip + 3pt

Da die Artikel dieses Skripts weitgehend in sich abgeschlossen sind, kann dieses Skript auch unabh"angig
von CrypTool gelesen werden.\par \vskip + 3pt

Die {\em Autoren} haben sich bem"uht, Kryptographie f"ur eine m"oglichst breite Leserschaft darzustellen
-- ohne mathematisch unkorrekt zu werden. Dieser didaktische Anspruch ist am besten geeignet, die Awareness
f"ur die IT-Sicherheit und die Bereitschaft f"ur den Einsatz standardisierter, moderner
Kryptographie zu f"ordern.

Au"serdem wurde versucht, die jeweils neuesten Erkenntnisse bez"uglich Primzahlen und Faktorisierung
darzustellen, um top-aktuell zu sein.


% Local Variables:
% TeX-master: "../script-de.tex"
% End:
