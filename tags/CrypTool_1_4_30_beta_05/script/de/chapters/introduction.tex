% $Id$
% ............................................................................
%      V O R W O R T  und  E I N F U E H R U N G (Zusammenspiel Skript-CT) 
% ~~~~~~~~~~~~~~~~~~~~~~~~~~~~~~~~~~~~~~~~~~~~~~~~~~~~~~~~~~~~~~~~~~~~~~~~~~~~


% --------------------------------------------------------------------------
\clearpage\phantomsection
% \vspace{-10pt}  wirkungslos!
\addcontentsline{toc}{chapter}{Vorwort zur 10. Auflage des CrypTool-Skripts}
\chapter*{Vorwort zur 10. Auflage des CrypTool-Skripts}

\vspace{-12pt}
Ab dem Jahr 2000 wurde mit dem CrypTool-Paket\index{CrypTool} auch ein
Skript ausgeliefert, das die Mathematik einzelner Themen genauer, aber doch 
m�glichst verst�ndlich erl�utern sollte.

Um auch hier die M�glichkeit zu schaffen, dass getrennte Entwickler/Autoren
mitarbeiten k�nnen, wurden die Themen sinnvoll unterteilt und daf�r jeweils
eigenst�ndig lesbare Kapitel geschrieben. In der anschlie�enden
redaktionellen Arbeit wurden in TeX Querverweise erg�nzt und Fu�noten
hinzugef�gt, die zeigen, an welchen Stellen man die entsprechenden
Funktionen in CrypTool v1\index{CrypTool 1.x} aufrufen kann 
\hyperlink{appendix-menutree}{(vgl. den Men�baum} im Anhang \ref{s:appendix-menutree}).
%\hypertarget{appendix-menutree}{}\label{s:appendix-menutree}
Nat�rlich g�be es viel mehr Themen in Mathematik und Kryptographie, die
man vertiefen k�nnte -- deshalb ist diese Auswahl auch nur eine von
vielen m�glichen.

Der Erfolg des Internets hat zu einer verst�rkten 
Forschung der damit verbundenen Technologien gef�hrt, was auch im
Bereich Kryptographie viele neue Erkenntnisse schaffte.

%In dieser Ausgabe des Skripts wurden einige Themen erg�nzt und andere auf
%den aktuellen Stand gebracht (z.B. die Zusammenfassungen zu aktuellen 
%Forschungsergebnissen):                   
In dieser Ausgabe des Skripts wurden die TeX-Sourcen des Dokuments komplett
�berarbei"-tet, und etliche Themen korrigiert,
erg�nzt und auf den aktuellen Stand gebracht, z.B.:
\vspace{-5pt}
\begin{itemize}
  \item die gr��ten Primzahlen (Kap. \ref{search_for_very_big_primes}),
        neue Faktorisierungsrekorde (Kap. \ref{RSA-768}),
  \item Fortschritte bei der Kryptoanalyse von AES 
        (Kap. \ref{NeueAES-Analyse}) und
%  \item Fortschritte bei der Kryptoanalyse von Hashverfahren 
%        (Kap. \ref{collision-attacks-against-sha-1}) und
%  \item Fortschritte bei den Ideen f�r neue Kryptoverfahren (RSA-Nachfolger)
%	(Kap. \ref{xxxxBrute-force-gegen-Symmetr})\index{xxxxxxxxxxxxxxxxxxxxx} und
  \item die Auflistung, in welchen Filmen und Romanen Kryptographie eine
        wesentliche Rolle spielt (siehe Anhang \ref{s:appendix-movies}),
        und Kurioses zu Primzahlen (siehe \ref{HT-Quaint-curious-Primes-usage}).
  \item Neu hinzu gekommen sind z.B. der Absatz zu Benfords Gesetz und Primzahlen
        (Kap. \ref{primes:Benford-Law}), und der Anhang \ref{s:appendix-using-sage} �ber
        das Computer-Algebra-System Sage. Sage wird mehr und mehr zum
        Standard-Open-Source CAS. Entsprechend
	wurden alle Codebeispiele, die vorher in PARI-GP und Mathematica
	geschrieben waren, durch Sage-Code ersetzt.
	Au�erdem kamen dank Nguyen und Massierer etliche neue Code-Beispiele hinzu
        (vgl. auch Kap.~\ref{ec:Sage_Massierer}).
\end{itemize}

Seit das Skript dem CrypTool-Paket in Version 1.2.01 das erste Mal beigelegt
wurde, ist es mit fast jeder neuen Version von CrypTool ebenfalls erweitert
und aktualisiert worden.

Herzlichen Dank an alle, die mit Ihrem gro�em Einsatz zum 
Erfolg und zur weiten Verbreitung dieses Projekts beigetragen haben.
Danke auch an die Leser, die uns Feedback sandten. 

Ich hoffe, dass viele Leser mit diesem Skript mehr Interesse an und 
Verst�ndnis f�r dieses moderne und zugleich uralte Thema finden.
\\
\\
% [1.5\baselineskip]
% \enlargethispage*{2\baselineskip}
% \nopagebreak
Bernhard Esslinger
\\
\\
% [\baselineskip]
% \nopagebreak
Frankfurt, Januar 2010



% --------------------------------------------------------------------------
\clearpage\phantomsection
\addcontentsline{toc}{chapter}{Einf�hrung -- Zusammenspiel von Skript
                               und CrypTool}
\chapter*{Einf�hrung -- Zusammenspiel von Skript und CrypTool}

\textbf{Das Skript}

Dieses Skript wird zusammen mit dem Open-Source-Programm CrypTool\index{CrypTool}
ausgeliefert.

Die Artikel dieses Skripts sind weitgehend in sich abgeschlossen und k�nnen
auch unabh�ngig von CrypTool\index{CrypTool} gelesen werden.

W�hrend f�r das Verst�ndnis der meisten Kapitel Abiturwissen ausreicht,
erfordern Kapitel \ref{Chapter_ModernCryptography} (Moderne Kryptografie)
und \ref{Chapter_EllipticCurves} (Elliptische Kurven) tiefere mathematische
Kenntnisse.

Die \hyperlink{appendix-authors}{Autoren}
% \hyperlink{appendix-authors}{(Autoren)}
% in \ref{s:appendix-authors}
% \hypertarget{appendix-authors}{}\label{s:appendix-authors}
haben sich bem�ht, Kryptographie f�r eine m�glichst 
breite Leserschaft darzustellen -- ohne mathematisch unkorrekt zu werden. 
Dieser didaktische Anspruch ist am besten geeignet, die Awareness
f�r die IT-Sicherheit und die Bereitschaft f�r den Einsatz 
standardisierter, moderner Kryptographie zu f�rdern.
\par \vskip + 15pt


\noindent \textbf{Das Programm CrypTool\index{CrypTool}}

CrypTool\index{CrypTool} ist ein Lernprogramm mit umfangreicher Online-Hilfe,
mit dem Sie unter einer einheitlichen Oberfl�che
kryptographische Verfahren anwenden und analysieren k�nnen.

Inzwischen wird CrypTool\index{CrypTool} weltweit
in Ausbildung und Lehre eingesetzt und von mehreren Hochschulen
mit weiterentwickelt.
\par \vskip + 15pt


\noindent \textbf{Dank}

An dieser Stelle m�chte ich explizit folgenden Personen danken, 
die bisher in ganz besonderer Weise zu CrypTool\index{CrypTool} 
beigetragen haben.
Ohne ihre besonderen F�higkeiten und ihr gro�es Engagement w�re CrypTool
nicht, was es heute ist:
\vspace{-7pt}
\begin{list}{\textbullet}{\addtolength{\itemsep}{-0.5\baselineskip}}
   \item Hr.\ Henrik Koy
   \item Hr.\ J�rg-Cornelius Schneider
   \item Hr.\ Florian Marchal
   \item Dr.\ Peer Wichmann
   \item Mitarbeiter von Prof.\ Claudia Eckert, Prof.\ Johannes Buchmann und Prof.\ Torben Weis.
\end{list}
Auch allen hier nicht namentlich Genannten ganz herzlichen Dank f�r das 
(meist in der Freizeit) geleistete Engagement.
\\
\\
Bernhard Esslinger
\\
\\
Frankfurt, November 2009

% Local Variables:
% TeX-master: "../script-de.tex"
% End:
