\begin{center}
{\bf III.}
\end{center}
Es dauert gar nicht so lange, bis Martin das Vertrauen von Anna, Rudolf,
Bianca und Doris gewonnen hat. Eine gewisse Routine stellt sich bald ein: Bei
den Bibelspr\"uchen Annas mu{\ss} man einfach wegh\"oren. Mit Rudolf hat
Martin schon so manche Schachpartie versucht: er ist immer der hoffnungslos
Unterlegene. Wenn er dann seinen mattgesetzten K\"onig auf dem Brett umlegt,
springt Rudolf auf, l\"auft durch das Zimmer und freut sich jedesmal kindisch
\"uber seinen Sieg. Bianca und Doris hat Martin besonders ins Herz geschlossen.
Beide sind Ende Drei{\ss}ig und leben schon lange in dieser Klinik. Mit ihnen
kann er sich auf normale Weise durchaus \"uber manchen belanglosen Vorfall im
Hause unterhalten. Sie sind immer gut gelaunt und meistens etwas albern. Nur:
Ein l\"angeres Gespr\"ach mit Bianca und Doris ist unm\"oglich! Schlagartig
setzen die Schwestern ihren Dialog in Zahlenhieroglyphen fort, und Martin ist
ausgeschlossen. Eines wird ihm bald klar: Hinter den Zahlenkolonnen verbirgt
sich kein Chaos, sondern eine Geheimsprache. Immer wenn Martin die Schwestern
etwas Pers\"onliches fragen m\"ochte, etwa was sie beide von Dr. Weissenborn
oder Professor Goldmann halten, verlassen sie sofort die Ebene der
herk\"ommlichen Sprache. Martin sp\"urt hierin auch den Grund ihrer
unausgesetzten Heiterkeit: Sie wissen, da{\ss} ihrem Dialog keiner folgen
kann, und das verleiht ihnen absolute Sicherheit. Wahrscheinlich wissen sie
\"uber vieles im Sanatorium genau Bescheid, nur w\"urden sie es einem Dritten
in Worten nie mitteilen, zumal ihre Welt an den T\"uren der Klinik ihre Grenzen
hat. Weil Martin die Schwestern mag, tut ihm das ein wenig weh. Er w\"urde sich
mit ihnen gerne \"uber manches unterhalten, nat\"urlich nur im Rahmen ihrer
Ausdrucksm\"oglichkeiten - aber eben in der Umgangssprache, nicht in Zahlen.
Im Laufe der Zeit hat er sich an die Strenge gew\"ohnt, mit der die Zivis hier
behandelt werden, aber sie bleibt f\"ur ihn dennoch grundlos. Im Stillen
beobachtet Martin aufmerksam die \"Arzte und Pfleger, und zunehmend beschleicht
ihn ein eigenartiges Gef\"uhl. Er wei{\ss} nicht, wieso, aber diese unheimliche
und abweisende K\"alte, die das Haus bei seiner Ankunft an dem verregneten
Sommerabend auf ihn ausgestrahlt hat, kehrt immer wieder zur\"uck. Dann bekommt
er eine G\"ansehaut auf dem R\"ucken. - \\
\newpage \noindent % Wittwen & Weisen ....
Professor Goldmann und seinen Bruder bekommt man nicht oft zu Gesicht. Sie
verlassen ihre R\"aume im Turm nur selten; meistens wird Martin an ihre
Existenz nur dann erinnert, wenn die beiden Br\"uder mit ihrem Porsche die
Kastanienallee hinauf- oder hinunterbrausen. Immer verlassen sie gemeinsam das
Sanatorium und kehren ebenso dorthin zur\"uck. Martin wird aus diesem Verhalten
nicht schlau. Dennoch scheint Professor Goldmann ein besonderes Interesse an
jedem Neuzugang in der Klinik zu haben. Er allein entscheidet \"uber die
Aufnahme oder die Zur\"uckweisung eines Patienten, und dabei f\"uhlt er sich
nicht an die Begutachtung durch \"Arzte und Psychologen gebunden. Martin hat
einmal ungewollt ein Gespr\"ach zwischen zwei Klinik\"arzten durch eine nicht
ganz fest geschlossene T\"ur mitverfolgen k\"onnen. Dabei hat er Einblick
in die Gepflogenheiten bei der Aufnahme eines Patienten erhalten. Das
best\"arkte ihn in seinen Gef\"uhlen, da{\ss} in diesem Haus etwas
Ungew\"ohnliches vorgeht. \\
\[\]
An einem langen und dunklen Herbstabend hat Rudolf 'seinen Zivi' zum
Schachspielen \"uberredet. Da Martin nichts Besseres zu tun hat, beginnt er
mit Rudolf eine Partie nach der anderen. Zwar verliert er jedes Spiel und
gibt Rudolf so mehrfach Anla{\ss} zu hemmungsloser Freude, aber er sch\"arft
dabei seine analytischen F\"ahigkeiten und verbessert zusehends seine
Spielstrategien. Er ist immer wieder erstaunt, was er von Rudolf alles beim
Spiel lernen kann. Die Uhr zeigt schon eine Stunde nach Mitternacht, als
Martin - im Gegensatz zu Rudolf - das dringende Bed\"urfnis nach seinem Bett
versp\"urt. Auf dem Weg durch die langen Korridore zu seinem Zimmer h\"ort er
in dem stillen Haus pl\"otzlich die Schritte von mehreren Personen in seiner
N\"ahe. Instinktiv bleibt er auf der halbdunklen Treppe stehen und l\"a{\ss}t
die Leute unten im Gang vor\"ubergehen. Er sieht, wie drei Pfleger zwei
Patienten, einen jungen Mann und eine Frau, zu Professor Goldmann in den Turm
bringen. Niemand bemerkt ihn auf dem Treppenabsatz, seine M\"udigkeit ist
schlagartig verflogen. Von diesem ungew\"ohnlichen Vorgang \"uberrascht und mit
d\"usteren Gedanken kommt Martin erst nach einiger Zeit aus dem Schatten heraus,
als die Schritte l\"angst verhallt sind. Er schleicht in sein Zimmer und
schlie{\ss}t mit weichen Knien die T\"ur hinter sich.
