% $Id$
% ............................................................................
%             V E R S C H L U E S S E L U N G S V E R F A H R E N
% ~~~~~~~~~~~~~~~~~~~~~~~~~~~~~~~~~~~~~~~~~~~~~~~~~~~~~~~~~~~~~~~~~~~~~~~~~~~~

\begin{center}
\fbox{\parbox{15cm}{%
{\em Indisches Sprichwort}:\\
Erkl"are es mir, ich werde es vergessen.\\
Zeige es mir, ich werde es vielleicht behalten.\\
Lass es mich tun, und ich werde es k"onnen.}}
\end{center}

\hypertarget{Kapitel_1}{}
\section{Verschl"usselungsverfahren}
(Bernhard Esslinger, Mai 1999; Updates: Dez. 2001, Feb. 2003, June 2005)


Dieses Kapitel soll einen eher beschreibenden Einstieg bieten und die
Grundlagen ohne Verwendung von allzuviel Mathematik vermitteln.


% --------------------------------------------------------------------------
% \subsection{Verschl"usselung}

Sinn der Verschl"usselung \index{Verschl""usselung} ist es, Daten so zu
ver"andern, dass nur ein autorisierter Empf"an\-ger in der Lage ist,
den Klartext zu rekonstruieren. Das hat den Vorteil, dass verschl"usselte
Daten offen "ubertragen werden k"onnen und trotzdem keine Gefahr besteht,
dass ein Angreifer die Daten unberechtigterweise lesen kann. Der
autorisierte Empf"anger ist im Besitz einer geheimen Information, des
sogenannten Schl"ussels, die es ihm erlaubt, die Daten zu entschl"usseln,
w"ahrend sie jedem anderen verborgen bleiben.\par \vskip + 3pt

Es gibt ein beweisbar sicheres Verschl"usselungsverfahren, 
das \index{One-Time-Pad} {\em One-Time-Pad}. Dieses weist allerdings
einige praktische Nachteile auf (der verwendete Schl"ussel muss zuf"allig
gew"ahlt werden und mindestens so lang wie die zu sch"utzende Nachricht sein),
so dass es au"ser in geschlossenen Umgebungen, zum Beispiel beim hei"sen
Draht zwischen Moskau und Washington, kaum eine Rolle spielt.\par \vskip + 3pt

F"ur alle anderen Verfahren gibt es (theoretische) M"oglichkeiten, sie
zu brechen. Bei guten Verfahren sind diese jedoch so aufw"andig, dass sie
praktisch nicht durchf"uhrbar sind und diese Verfahren als (praktisch)
sicher angesehen werden k"onnen.\par \vskip + 3pt

Einen sehr guten "Uberblick zu den Verfahren bietet das Buch von Bruce
Schneier \cite{Schneier1996cm}. Grunds"atzlich unterscheidet man zwischen
symmetrischen und asymmetrischen Verfahren zur Verschl"usselung.

% --------------------------------------------------------------------------
\subsection[Symmetrische Verschl"usselung]
{Symmetrische Verschl"usselung\footnotemark}
   \footnotetext{% XXX dieses Prozent ist n�tig, sonst startet die Fu�note mit einem Blank !
     Mit CrypTool\index{CrypTool} k"onnen Sie "uber das Men"u
     {\bf Ver-/Entschl"usseln \textbackslash{} Symmetrisch (modern)} folgende 
     modernen symmetrischen Verschl"usselungsverfahren ausf"uhren: \\
     IDEA, RC2, RC4, DES (ECB), DES (CBC), Triple-DES (ECB), Triple-DES (CBC),
     MARS (AES-Kandidat), RC6 (AES-Kandidat), Serpent (AES-Kandidat), 
     Twofish (AES-Kandidat), Rijndael (offizielles AES-Verfahren).
     }

Bei der {\em symmetrischen} Verschl"usselung 
\index{Verschl""usselung!symmetrisch} m"ussen Sender und Empf"anger
"uber einen gemeinsamen (geheimen) Schl"ussel verf"ugen, den sie vor
Beginn der eigentlichen Kommunikation ausgetauscht haben. Der Sender
benutzt diesen Schl"ussel, um die Nachricht zu verschl"usseln und der
Empf"anger, um diese zu entschl"usseln.\par \vskip + 3pt

Alle klassischen Verfahren sind von diesem Typ. Beispiele dazu finden Sie im
Programm CrypTool, im Kapitel  \ref{Kapitel_PaperandPencil} 
(Papier- und Bleistift-Verschl"usselungsverfahren) in diesem Skript oder 
in \cite{Nichols1996}. 
Im folgenden wollen wir jedoch nur die moderneren Verfahren betrachten.

Vorteile von symmetrischen Algorithmen sind die hohe
Geschwindigkeit, mit denen Daten ver- und entschl"usselt werden.
Ein Nachteil ist das Schl"usselmanagement. Um miteinander
vertraulich kommunizieren zu k"onnen, m"ussen Sender und
Empf"anger vor Beginn der eigentlichen Kommunikation "uber einen
sicheren Kanal einen Schl"ussel ausgetauscht haben. Spontane
Kommunikation zwischen Personen, die sich vorher noch nie begegnet
sind, scheint so nahezu unm"oglich. Soll in einem Netz mit $ n $
Teilnehmern jeder mit jedem zu jeder Zeit spontan kommunizieren
k"onnen, so muss jeder Teilnehmer vorher mit jedem anderen der
$n - 1$ Teilnehmer einen Schl"ussel ausgetauscht haben. Insgesamt
m"ussen also $n(n - 1)/2$ Schl"ussel ausgetauscht werden.\par \vskip + 3pt

Das bekannteste symmetrische Verschl"usselungsverfahren ist der \index{DES} 
DES-Algorithmus. Der DES-Algorithmus ist eine Entwicklung von IBM 
in Zusammenarbeit mit der National Security Agency \index{NSA} (NSA). 
Er wurde 1975 als Standard ver"offentlicht. Trotz seines relativ hohen
Alters ist jedoch bis heute kein  ~\glqq effektiver\grqq~ Angriff auf
ihn gefunden worden. Der effektivste Angriff besteht aus dem Durchprobieren
(fast) aller m"oglichen Schl"ussel, bis der richtige gefunden wird 
({\em brute-force-attack})\index{Angriff!Brute-force}. Aufgrund der relativ
kurzen Schl"ussell"ange von effektiv 56 Bits (64 Bits, die allerdings 8
Parit"atsbits enthalten), sind in der Vergangenheit schon mehrfach mit dem
DES verschl"usselte Nachrichten gebrochen worden, so dass er heute als nur
noch bedingt sicher anzusehen ist. Symmetrische Alternativen zum DES sind
zum Beispiel die Algorithmen IDEA \index{IDEA} oder Triple-DES.
\par \vskip + 3pt

Hohe Aktualit"at besitzt das symmetrische AES-Verfahren: \index{AES} 
Der dazu geh"orende Rijndael-Algo"-rithmus wurde am 2. Oktober 2000 zum
Gewinner der AES-Ausschreibung erkl"art und ist damit Nachfolger
des DES-Verfahrens.

N"aheres zum AES-Algorithmus und den AES-Kandidaten der letzten Runde finden
Sie in der CrypTool-Online-Hilfe\index{CrypTool}%
\footnote{%
      CrypTool-Online-Hilfe\index{CrypTool}: das Stichwort {\bf AES} im Index
      f"uhrt auf die 3 Hilfeseiten: {\bf AES-Kandidaten}, 
      {\bf Der AES-Gewinner Rijndael} und {\bf Der AES-Algorithmus Rijndael}.
  }.


% --------------------------------------------------------------------------
\subsubsection{Neue Ergebnisse zur Kryptoanalyse von AES}
\index{Kryptoanalyse} \label{NeueAES-Analyse}

Im Anschluss finden Sie einige Informationen, die das AES-Verfahren in
letzter Zeit in Zweifel zogen -- unserer Ansicht nach aber (noch)
unbegr"undet. Die folgenden Informationen beruhen auf den unten
angegebenen Originalarbeiten und auf \cite{Wobst-iX2002} und
\cite{Lucks-DuD2002}.

Der AES bietet mit einer Mindestschl"ussell"ange von 128 Bit gegen
Brute-force-Angriffe auch auf l"angere Sicht gen"ugend Sicherheit -- es sei
denn, es st"unden entsprechend leistungsf"ahige Quantencomputer zur
Verf"ugung. Der AES war immun gegen alle bis dahin bekannten
Kryptoanalyse-Verfahren, die vor allem auf statistischen
"Uberlegungen beruhen und auf DES angewandt wurden: man konstruiert aus
Geheim- und Klartextpaaren Ausdr"ucke, die sich nicht rein zuf"allig
verhalten, sondern R"uckschl"usse auf den verwendeten Schl"ussel zulassen.
Dazu waren aber unrealistische Mengen an abgeh"orten Daten n"otig.

Bei Kryptoverfahren bezeichnet man solche Kryptoanalyseverfahren als
''akademischen Erfolg'' oder als ''kryptoanalytischen Angriff'', da sie
theoretisch schneller sind als das Durchprobieren aller Schl"ussel, das
beim Brute-force-Angriff\index{Angriff!Brute-force} verwendet wird. Im
Fall des AES mit maximaler Schl"ussell"ange (256 Bit) braucht die
ersch"opfende Schl"usselsuche im Durchschnitt $2^{255}$
Verschl"usselungsoperationen. Ein kryptoanalytischer Angriff muss diesen
Wert unterbieten. Als aktuell gerade noch praktikabel (z.B. f"ur einen
Geheimdienst) gilt ein Aufwand von $2^{75}$ bis $2^{90}$
Verschl"usselungsoperationen.

Eine neue Vorgehensweise wurde in der Arbeit von Ferguson, Schroeppel
und Whiting im Jahre 2001 \cite{Ferguson2001} beschrieben: sie stellten
AES als geschlossene Formel (in der Form einer Art Kettenbruch) dar,
was aufgrund seiner ''relativ'' "ubersichtlichen Struktur gelang. Da
diese Formel aus rund 1 Billiarde Summanden besteht, taugt sie zun"achst
nicht f"ur die Kryptoanalyse. Dennoch war die wissenschaftliche Neugier
geweckt. Bereits bekannt war, dass sich der 128-Bit-AES als ein
"uberbestimmtes System von rund 8000 quadratischen Gleichungen
("uber algebraischen Zahlk"orpern) mit rund 1600 Variablen (einige
Unbekannte sind die Schl"usselbits) darstellen l"asst -- solch gro"se
Gleichungssysteme sind praktisch nicht l"osbar. Dieses Gleichungssystem
ist relativ d"unn besetzt (''sparse''), d.h. von den insgesamt etwa
1.280.000 m"oglichen quadratischen Termen tauchen nur relativ wenige
"uberhaupt im Gleichungssystem auf.

Die Mathematiker Courois und Pieprzyk \cite{Courtois2002} ver"offentlichten 
2002 eine Arbeit, die in der Krypto-Szene stark diskutiert wird: Sie 
entwickelten das auf der Eurocrypt 2000 von Shamir et al. vorgestellte 
XL-Verfahren (eXtended Linearization) weiter zum XSL-Verfahren 
(eXtended Sparse Linearization). Das XL-Verfahren ist eine heuristische 
Technik, mit der es manchmal gelingt, gro"se nicht-lineare Gleichungssysteme 
zu l"osen und die bei der Analyse eines asymmetrischen Algorithmus (HFE) 
angewandt wurde.  Die Innovation von Courois und Pieprzyk war, das 
XL-Verfahren auf symmetrische Verfahren anzuwenden: das XSL-Verfahren kann
auf spezielle Gleichungssysteme angewandt werden. Damit k"onnte ein 
256-Bit-AES-Verfahren in rund $2^{230}$ Schritten geknackt werden. Dies ist 
zwar immer noch ein rein akademischer Angriff, aber er ist richtungsweisend
f"ur eine ganze Klasse von Blockchiffren. Das generelle Problem mit diesem
Angriff besteht darin, dass man bisher nicht angeben kann, unter welchen
Umst"anden er zum Erfolg f"uhrt: die Autoren geben in ihrer Arbeit notwendige
Bedingungen an; es ist nicht bekannt, welche Bedingungen hinreichend sind.
Neu an diesem Angriff war erstens, dass dieser Angriff nicht auf Statistik,
sondern auf Algebra beruhte. Dadurch erscheinen Angriffe m"oglich, die nur
geringe Mengen von Geheimtext brauchen. Zweitens steigt damit die Sicherheit
eines Produktalgorithmus\index{Produktalgorithmus}%
\footnote{%
Ein Geheimtext kann selbst wieder Eingabe f"ur eine Chiffrierung sein. Eine
Mehrfachverschl"usselung (cascade cipher)\index{Mehrfachverschl\""usselung}
entsteht aus einer Komposition von mehreren Verschl"usselungstransformationen.
Die Gesamtchiffrierung wird Produktalgorithmus oder Kaskadenalgorithmus
genannt (manchmal ist die Namensgebung abh"angig davon, ob die verwendeten
Schl"ussel statistisch abh"angig oder unabh"angig sind).\\
Nicht immer wird die Sicherheit eines Verfahrens durch Produktbildung 
erh"oht.\\
Diese Vorgehen wird auch {\em innerhalb} moderner Algorithmen angewandt:
Sie kombinieren in der Regel einfache und, f"ur sich genommen, kryptologisch
relativ unsichere Einzelschritte in mehreren Runden zu einem leistungsf"ahigen
Gesamtverfahren. Die meisten modernen Blockchiffrierungen (z.B. DES, IDEA)
sind Produktalgorithmen.\\
Als Mehrfachverschl"usselung wird auch das Hintereinanderschalten desselben
Verfahrens mit verschiedenen Schl"usseln wie bei Triple-DES bezeichnet. 
}
nicht mehr exponentiell mit der Rundenzahl.

Aktuell wird sehr intensiv auf diesem Gebiet geforscht: z.B. stellten
Murphy und Robshaw auf der Crypto 2002 ein Papier vor \cite{Robshaw2002a},
das die Kryptoanalyse drastisch verbessern k"onnte: der Aufwand f"ur
128-Bit-Schl"ussel wurde auf $2^{100}$ gesch"atzt, indem sie AES als
Spezialfall eines Algorithmus BES (Big Encryption System) darstellten,
der eine besonders ''runde'' Struktur hat. Aber auch $2^{100}$
Rechenschritte liegen jenseits dessen, was praktisch in absehbarer Zeit
realisierbar ist. Bei 256 Bit Schl"ussell"ange sch"atzen die Autoren
den Aufwand f"ur den XSL-Angriff auf $2^{200}$ Operationen.

Weitere Details dazu stehen unter:
% nur \href r"uckt nicht ein, deshalb \item dazu !
\vspace{-10pt}
\begin{itemize}
  \item[] \href{http://www.cryptosystem.net/aes}
               {\texttt{http://www.cryptosystem.net/aes}} \\
          \href{http://www.minrank.org/aes/}
               {\texttt{http://www.minrank.org/aes/}}
\end{itemize}

F"ur 256-AES w"are der Angriff ebenfalls viel besser als Brute-force
\index{Angriff!Brute-force}, aber auch noch weiter au"serhalb der
Reichweite realisierbarer Rechenpower.

\begin{sloppypar}
Die Diskussion ist z.Zt. sehr kontrovers: Don Coppersmith (einer der DES-Erfinder) 
z.B. bezweifelt die generelle Durchf"uhrbarkeit des Angriffs: XLS liefere f"ur 
AES gar keine L"osung \cite{Coppersmith2002}. Dann w"urde auch die Optimierung von 
Murphy und Robshaw \cite{Robshaw2002b} nicht greifen.
\end{sloppypar}

% --------------------------------------------------------------------------
\subsubsection{Aktueller Stand der Brute-force-Angriffe auf 
symmetrische Verfahren (RC5)}
\index{Angriff!Brute-force}
\index{RC5}
\label{Brute-force-gegen-Symmetr}

Anhand der Blockchiffre RC5 kann der aktuelle Stand von Brute-force-Angriffen
auf symmetrische Verschl"usselungsverfahren gut erl"autert werden.

\begin{sloppypar}
Als Brute-force-Angriff (exhaustive search, trial-and-error) bezeichnet
man das vollst"andige Durchsuchen des Schl"usselraumes: dazu m"ussen
keine besonderen Analysetechniken eingesetzt werden. Stattdessen wird
der Ciphertext mit allen m"oglichen Schl"usseln des Schl"usselraums
entschl"usselt und gepr"uft, ob der resultierende Text einen sinnvollen
Klartext ergibt. Bei einer Schl"ussell"ange von 64 Bit sind dies maximal
$2^{64}$ = 18.446.744.073.709.551.616 oder rund 18 Trillionen zu
"uberpr"ufende Schl"ussel\index{CrypTool}%
\footnote{%
Mit CrypTool\index{CrypTool} k"onnen Sie "uber das Men"u
{\bf Analyse \textbackslash{} Symmetrische Verschl"usselung (modern)}
ebenfalls Brute-force-Analysen von modernen symmetrischen Verfahren
durchf"uhren (unter der schw"achsten aller m"oglichen Annahmen: der
Angreifer kennt nur den Geheimtext, er f"uhrt also einen Ciphertext
only-Angriff durch). 
Um in einer sinnvollen Zeit auf einem Einzel-PC ein Ergebnis zu
erhalten, sollten Sie nicht mehr als 20 Bit des Schl"ussels als
unbekannt kennzeichnen.
}.
\end{sloppypar}

Um zu zeigen, welche Sicherheit bekannte symmetrische Verfahren wie
DES, Triple-DES oder RC5 haben, veranstaltet z.B. RSA Security
sogenannte Cipher-Challenges\footnote{%
\href{http://www.rsasecurity.com/rsalabs/challenges/secretkey/index.html}
{\tt http://www.rsasecurity.com/rsalabs/challenges/secretkey/index.html}}.
Unter kontrollierten Bedingungen werden Preise ausgelobt, um Ciphertexte
(verschl"usselt mit verschiedenen Verfahren und verschiedenen
Schl"ussell"angen) zu entschl"usseln und den symmetrischen Schl"ussel
zu ermitteln. Damit werden theoretische Resultate praktisch best"atigt.

Dass das "`alte"' Standard-Verfahren DES mit der fixen Schl"ussell"ange
von 56 Bit nicht mehr sicher ist, wurde schon im Januar 1999 von der
Electronic Frontier Foundation (EFF) demonstriert, als sie mit Deep Crack
eine DES-verschl"usselte Nachricht in weniger als einem Tag knackten\footnote{%
\href{http://www.rsasecurity.com/rsalabs/challenges/des3/index.html}
{\tt http://www.rsasecurity.com/rsalabs/challenges/des3/index.html}}.

Der aktuelle Rekord f"ur starke symmetrische Verfahren liegt bei 64 Bit
langen Schl"usseln. Dazu wurde das Verfahren RC5 benutzt, eine
Blockchiffre mit variabler Schl"ussell"ange. 

Die RC5-64 Challenge wurde im September 2002 nach 5 Jahren vom
distributed.net-Team gel"ost\footnote{%
\href{http://distributed.net/pressroom/news-20020926.html}
{\tt http://distributed.net/pressroom/news-20020926.html}}.
Insgesamt arbeiteten 331.252 Personen gemeinsam "uber das Internet
zusammen. Getestet wurden rund 15 Trillionen Schl"ussel, bis der
richtige Schl"ussel gefunden wurde.

Damit ist gezeigt, dass symmetrische Verfahren (auch wenn sie keinerlei
kryptographische Schw"a"-chen haben) mit 64 Bit langen Schl"usseln
keine geeigneten Verfahren mehr sind, um sensible Daten l"anger geheim
zu halten.

"Ahnliche Cipher-Challenges gibt es auch f"ur asymmetrische Verfahren
(siehe Kapitel \ref{NoteFactorisation}).




% --------------------------------------------------------------------------
\subsection[Asymmetrische Verschl"usselung]
{Asymmetrische Verschl"usselung\footnotemark}
        \footnotetext{%
        Mit CrypTool\index{CrypTool} k"onnen Sie "uber das Men"u
        {\bf Ver-/Entschl"usseln \textbackslash{} Asymmetrisch} mit RSA 
        ver- und entschl"usseln. In beiden F"allen m"ussen Sie ein 
        RSA-Schl"usselpaar ausw"ahlen. Nur bei der Entschl"usselung wird der 
        geheime RSA-Schl"ussel ben"otigt: deshalb wird nur hier die PIN
        abgefragt. 
        }

Bei der {\em asymmetrischen} Verschl"usselung
\index{Verschl""usselung!asymmetrisch} hat jeder Teilnehmer 
ein pers"onliches Schl"usselpaar, % Schl"us\-selpaar, be_24.5.03: so vorher und kam auch richtig raus! 
das aus einem {\em geheimen} \index{Schl""ussel!geheim} und einem 
{\em "offentlichen} Schl"ussel \index{Schl""ussel!""offentlich} besteht.
Der "offentliche Schl"ussel wird, der Name deutet es an,
"offentlich bekanntgemacht, zum Beispiel in einem Schl"usselverzeichnis
im Internet.\par \vskip + 3pt

M"ochte Alice\index{Alice}%
\footnote{%
  Zur Beschreibung kryptographischer Protokolle werden den Teilnehmern
  oft Namen gegeben (vergleiche \cite[S. 23]{Schneier1996cm}). 
  Alice und Bob f"uhren alle allgemeinen 2-Personen-Protokolle durch,
  wobei Alice dies initiiert und Bob antwortet.
  Die Angreifer werden als Eve (eavesdropper = passiver Lauscher) und
  Mallory (malicious active attacker = b"oswilliger, aktiver Abgreifer)
  bezeichnet.
  }
mit Bob kommunizieren, so sucht sie Bobs "offentlichen Schl"ussel aus
dem Verzeichnis und benutzt ihn, um ihre Nachricht an ihn zu
verschl"usseln. Diesen verschl"usselten Text schickt sie dann an Bob,
der mit Hilfe seines geheimen Schl"ussels den Text wieder entschl"usseln
kann. Da einzig Bob Kenntnis von seinem geheimen Schl"ussel hat, ist
auch nur er in der Lage, an ihn adressierte Nachrichten zu
entschl"usseln.
Selbst Alice als Absenderin der Nachricht kann aus der von ihr
versandten (verschl"usselten) Nachricht den Klartext nicht wieder
herstellen. Nat"urlich muss sichergestellt sein, dass man aus dem
"offentlichen Schl"ussel nicht auf den geheimen Schl"ussel schlie"sen
kann.\par \vskip + 3pt

Veranschaulichen kann man sich ein solches Verfahren mit einer
Reihe von einbruchssicheren Briefk"asten. Wenn ich eine Nachricht
verfasst habe, so suche ich den Briefkasten mit dem Namensschild
des Empf"angers und werfe den Brief dort ein. Danach kann ich die
Nachricht selbst nicht mehr lesen oder ver"andern, da nur der
legitime Empf"anger im Besitz des Schl"ussels f"ur den Briefkasten
ist.\par \vskip + 3pt

Vorteil von asymmetrischen Verfahren ist das einfache
\index{Schl""usselmanagement} Schl"usselmanagement. Betrachten wir
wieder ein Netz mit $n$ Teilnehmern. Um sicherzustellen, dass
jeder Teilnehmer jederzeit eine verschl"usselte Verbindung zu
jedem anderen Teilnehmer aufbauen kann, muss jeder Teilnehmer ein
Schl"usselpaar besitzen. Man braucht also $2n$ Schl"ussel oder $n$
Schl"usselpaare. Ferner ist im Vorfeld einer "Ubertragung kein
sicherer Kanal notwendig, da alle Informationen, die zur Aufnahme
einer vertraulichen Kommunikation notwendig sind, offen
"ubertragen werden k"onnen. Hier ist lediglich auf die
Unverf"alschtheit (Integrit"at und Authentizit"at)
\index{Authentizit""at} des "offentlichen Schl"ussels zu achten.
Nachteil: Im Vergleich zu symmetrischen Verfahren sind reine
asymmetrische Verfahren jedoch um ein Vielfaches langsamer.\par \vskip + 3pt

Das bekannteste asymmetrische Verfahren ist der \index{RSA} 
RSA-Algorithmus\index{CrypTool}%
\footnote{%
RSA wird in diesem Skript ab Kapitel \ref{rsabeweis} ausf"uhrlich beschrieben.
Das RSA-Kryptosystem kann mit CrypTool\index{CrypTool} "uber das Men"u 
{\bf Einzelverfahren \textbackslash{} RSA-Kryptosystem \textbackslash{} 
RSA-Demo} in vielen Variationen nachvollzogen werden. Die 
aktuellen Forschungsergebnisse im Umfeld von RSA werden in Kapitel
\ref{SecurityRSA} beschrieben.
}%
,
der nach seinen Entwicklern Ronald \index{Rivest, Ronald} Rivest, 
Adi \index{Shamir, Adi} Shamir und Leonard \index{Adleman, Leonard} Adleman
benannt wurde. Der RSA-Algorithmus wurde 1978 ver"offentlicht. 
Das Konzept der asymmetrischen Verschl"usselung wurde erstmals
von Whitfield Diffie \index{Diffie, Whitfield}  und 
Martin \index{Hellman, Martin} Hellman in Jahre 1976 vorgestellt. 
Heute spielen auch die Verfahren nach
ElGamal \index{ElGamal, Tahir} eine bedeutende Rolle, vor allem die
\index{Schnorr, C.P.} Schnorr-Varianten im \index{DSA} DSA (Digital
\index{Signatur!digitale}\index{DSA-Signatur}\index{Signatur!DSA}
Signature Algorithm).


% --------------------------------------------------------------------------
\subsection[Hybridverfahren]
{Hybridverfahren\footnotemark} 
\footnotetext{%
In CrypTool\index{CrypTool} erhalten Sie "uber das Men"u
{\bf Ver-/Entschl"usseln \textbackslash{} Hybrid} eine Visualisierung
dieses Verfahrens: darin werden die einzelnen Schritte und ihre
Abh"angigkeiten mit konkreten Zahlen gezeigt. Benutzt werden als
asymmetrisches Verfahren RSA und als symmetrisches Verfahren AES.
}\index{Hybridverfahren}

Um die Vorteile von symmetrischen und asymmetrischen Techniken gemeinsam 
nutzen zu k"onnen, werden (zur Verschl"usselung) in der Praxis meist 
Hybridverfahren \index{Verschl""usselung!hybrid} verwendet.\par \vskip + 3pt

Hier werden die Daten mittels symmetrischer Verfahren
verschl"usselt: der Schl"ussel ist ein vom Absender zuf"allig\footnote{%
  Die Erzeugung zuf"alliger Zahlen ist ein wichtiger Bestandteil 
  kryptographisch sicherer Verfahren. Mit CrypTool\index{CrypTool} 
  k"onnen Sie "uber das Men"u {\bf Einzelverfahren \textbackslash{}
  Zufallsdaten erzeugen} verschiedene Zufallszahlengeneratoren ausprobieren.
  "Uber das Men"u {\bf Analyse \textbackslash{} Zufallsanalyse} k"onnen Sie
  verschiedene Testverfahren f"ur Zufallsdaten auf bin"are Dokumente
  anwenden. \\
  Bisher konzentriert sich CrypTool auf kryptographisch starke 
  Pseudozufallszahlengeneratoren. Nur im Secude-Generator
  \index{SECUDE IT Security GmbH} wird eine \glqq echte\grqq~ 
  Zufallsquelle einbezogen. 
}\index{Zufall}
generierter Sitzungsschl"ussel (session key)\index{Session Key}, der
nur f"ur diese Nachricht verwendet wird. Anschlie"send wird dieser
Sitzungsschl"ussel mit Hilfe des asymmetrischen Verfahrens
verschl"usselt und zusammen mit der Nachricht an den Empf"anger
"ubertragen. Der Empf"anger kann den Sitzungsschl"ussel mit Hilfe
seines geheimen Schl"ussels bestimmen und mit diesem dann die
Nachricht entschl"usseln. Auf diese Weise nutzt man das bequeme
Schl"usselmanagement \index{Schl""usselmanagement} asymmetrischer
Verfahren und kann trotzdem gro"se Datenmengen schnell und effektiv mit
symmetrischen Verfahren verschl"usseln.


% --------------------------------------------------------------------------
\subsection{Weitere Details}

Neben den anderen Kapiteln in diesem Skript, der umfangreichen Fachliteratur
und vielen Stellen im Internet enth"alt auch die Online-Hilfe von 
CrypTool\index{CrypTool} sehr viele weitere Informationen zu den 
einzelnen symmetrischen und asymmetrischen Verschl"usselungsverfahren.


% --------------------------------------------------------------------------
%xxxxxxxxxxx
%\bibitem[x]{x}  \index{x}
%        x, \\
%        {\em x}, 
%       x x.
%xxxxxxxxxxx
\newpage
\begin{thebibliography}{99999}
\addcontentsline{toc}{subsection}{Literaturverzeichnis}

\bibitem[Coppersmith2002]{Coppersmith2002}  \index{Coppersmith 2002}
        Don Coppersmith, \\
        {\em Re: Impact of Courtois and Pieprzyk results}, \\ 
        19.09.2002, in den \glqq AES Discussion Groups\grqq~ unter \\
        \href{http://aes.nist.gov/aes/}
        {\texttt{http://aes.nist.gov/aes/}}

\bibitem[Courtois2002]{Courtois2002}  \index{Courtois 2002}
        Nicolas Courtois, Josef Pieprzyk, \\
        {\em Cryptanalysis of Block Ciphers with Overdefined Systems
             of Equations}, \\
        received 10 Apr 2002, last revised 9 Nov 2002.\\
        Eine davon verschiedene ~\glqq compact version\grqq~ des 
	ersten XSL-Angriffs wurde 
        auf der Asiacrypt im Dezember 2002 ver"offentlicht. \\
        \href{http://eprint.iacr.org/2002/044}
        {\texttt{http://eprint.iacr.org/2002/044}}

\bibitem[Ferguson2001]{Ferguson2001}  \index{Ferguson 2001}
        Niels Ferguson, Richard Schroeppel, Doug Whiting, \\
        {\em A simple algebraic representation of Rijndael}, 
        Draft 1. Mai 2001, \\
        \href{http://www.xs4all.nl/~vorpal/pubs/rdalgeq.html}
        {\texttt{http://www.xs4all.nl/\~{}vorpal/pubs/rdalgeq.html}}

\bibitem[Lucks-DuD2002]{Lucks-DuD2002}  \index{Lucks 2002}
        Stefan Lucks, R"udiger Weis, \\
        {\em Neue Ergebnisse zur Sicherheit des Verschl"usselungsstandards
             AES}, 
        in DuD 12/2002.

\bibitem[Nichols1996]{Nichols1996} \index{Nichols 1996} 
       Randall K. Nichols, \\
       {\em Classical Cryptography Course, Volume 1 and 2}, \\
       Aegean Park Press 1996; \\
       oder in 12 Lektionen online unter \\
       \href{http://www.fortunecity.com/skyscraper/coding/379/lesson1.htm}
       {\texttt{http://www.fortunecity.com/skyscraper/coding/379/lesson1.htm}}

\bibitem[Robshaw2002a]{Robshaw2002a}  \index{Robshaw 2002}
       S.P. Murphy, M.J.B. Robshaw, \\
       {\em Essential Algebraic Structure within the AES}, 
       5. Juni 2002, auf der Crypto 2002,  \\
       \href{http://www.isg.rhul.ac.uk/~mrobshaw/rijndael/rijndael.html}
       {\texttt{http://www.isg.rhul.ac.uk/\~{}mrobshaw/rijndael/rijndael.html}}

\bibitem[Robshaw2002b]{Robshaw2002b}  \index{Robshaw 2002}
       S.P. Murphy, M.J.B. Robshaw, \\
       {\em Comments on the Security of the AES and the XSL Technique}, 
       26. September 2002, \\
       \href{http://www.isg.rhul.ac.uk/~mrobshaw/rijndael/rijndael.html}
       {\texttt{http://www.isg.rhul.ac.uk/\~{}mrobshaw/rijndael/rijndael.html}}

\bibitem[Schmeh2001]{Schmeh2001}  \index{Schmeh 2001}
        Klaus Schmeh, \\
        {\em Kryptografie und Public-Key Infrastrukturen im Internet},\\ 
        dpunkt.verlag, 2. akt. und erw. Auflage 2001. \\
        Sehr gut lesbares, hoch aktuelles und umfangreiches Buch "uber 
	Kryptographie. 
        Geht auch auf praktische Probleme, wie Standardisierung oder 
        real existierende Software, ein.

\bibitem[Schneier1996]{Schneier1996cm} \index{Schneier 1996} 
    Bruce Schneier, \\
    {\em Applied Cryptography, Protocols, Algorithms, and Source Code in C}, \\
    Wiley 1994, 2nd edition 1996.

\bibitem[Wobst-iX2002]{Wobst-iX2002}  \index{Wobst 2002}
        Reinhard Wobst, \\
        {\em Angekratzt - Kryptoanalyse von AES schreitet voran}, 
        in iX 12/2002, \\
        und der Leserbrief dazu von Johannes Merkle in der iX 2/2003.

\end{thebibliography}


% --------------------------------------------------------------------------
\newpage
\section*{Web-Links}\addcontentsline{toc}{subsection}{Web-Links}

\begin{enumerate}

%  \item[] \href{http://www.cryptosystem.net/aes}
%       {\texttt{http://www.cryptosystem.net/aes}} 
%  per item[] werden sie NICHT durchnummeriert !
  \item Das AES-/Rijndael-Kryptosystem \\
        \href{http://www.cryptosystem.net/aes}
         {\tt http://www.cryptosystem.net/aes} \\
	\href{http://www.minrank.org/aes/}
         {\tt http://www.minrank.org/aes/}

  \item \glqq AES Discussion Groups'' beim NIST \\
	\href{http://aes.nist.gov/aes}{\tt http://aes.nist.gov/aes}

  \item distributed.net:~\glqq RC5-64 has been solved'' \\
        \href{http://distributed.net/pressroom/news-20020926.html}
         {\tt http://distributed.net/pressroom/news-20020926.html}

  \item RSA: ~\glqq The RSA Secret Key Challenge'' \\
      \href{http://www.rsasecurity.com/rsalabs/challenges/secretkey/index.html}
       {\tt http://www.rsasecurity.com/rsalabs/challenges/secretkey/index.html}

  \item RSA: ~\glqq DES Challenge'' \\
        \href{http://www.rsasecurity.com/rsalabs/challenges/des3/index.html}
         {\tt http://www.rsasecurity.com/rsalabs/challenges/des3/index.html}

  \item Weiterf"uhrende Links auf der CrypTool-Homepage \\
        \href{http://www.cryptool.de}
         {\tt http://www.cryptool.de}
	       
\end{enumerate}




% Local Variables:
% TeX-master: "../script-de.tex"
% End:
