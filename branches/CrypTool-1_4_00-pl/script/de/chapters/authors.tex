
% ++++++++++++++++++++++++++++++++++++++++++++++++++++++++++++++++++++++++++
\subsection{Autoren des CrypTool-Skripts\index{CrypTool}}
\hypertarget{appendix-authors}{}\label{s:appendix-authors}

Dieser Anhang f"uhrt die Autoren\index{Autoren} dieses Dokuments auf.\\
Die Autoren sind namentlich am Anfang jedes Kapitels aufgef"uhrt,
zu dem sie beigetragen haben.

\begin{description}

\item[Bernhard Esslinger,] \mbox{}\\  % be_2005: Das \\ allein wird ignoriert (brauche beide)!  
Initiator des CrypTool-Projekts, Hauptautor dieses Skripts, Leiter IT-Security
in der Deutschen Bank und Dozent an der Universit"at Siegen. E-Mail:
besslinger@web.de.

\item[Matthias B"uger,] \mbox{}\\
Mitautor des Kapitels ``Elliptische Kurven'', Research
Analyst bei der Deutschen Bank.

\item[Bartol Filipovic,] \mbox{\\}
Urspr"unglicher Autor der Elliptische-Kurven-Implementierung
in CrypTool und des entsprechenden Kapitels in diesem Skript.

\item[Henrik Koy, ] \mbox{}\\
Hauptentwickler und Koordinator der CrypTool-Entwicklung
seit Version 1.3, Reviewer des Skripts und \TeX{}-Guru, 
Projektleiter IT und Kryptologe bei der Deutschen Bank.

\item[Roger Oyono, ] \mbox{}\\
Implementierer des Faktorisierungs-Dialogs in CrypTool und urspr"unglicher
Autor des Kapitels "`Die mathematischen Ideen hinter der modernen Kryptographie"'.

\item[J"org Cornelius Schneider,] \mbox{}\\
Design und Support von CrypTool, Kryptographie-Enthusiast und 
Senior-Projektleiter IT bei der Deutschen Bank.

\item[Christine St"otzel,] \mbox{}\\
Diplom Wirtschaftsinformatikerin an der Universit"at Siegen.


\end{description}






% ++++++++++++++++++++++++++++++++++++++++++++++++++++++++++++++++++++++++++
\newpage
\subsection{Filme und Literatur mit Bezug zur Kryptographie}
\hypertarget{appendix-movies}{}
\label{s:appendix-movies}  
% {\bf Filme und Literatur mit Bezug zur Kryptographie} (siehe Anhang \ref{s:appendix-movies})
\index{Filme}
\index{Literatur}


Kryptographische Verfahren -- sowohl klassische wie moderne -- fanden auch
Eingang in die Literatur und in Filme. In manchen Medien werden diese nur erw"ahnt
und sind reine Beigabe, in anderen sind sie tragend und werden genau erl"autert,
und manchmal ist die Rahmenhandlung nur dazu da, dieses Wissen motivierend zu
transportieren.

Anbei der Beginn eines "Uberblicks.

%be_2005: Hatte zuerst \begin{thebibliography}{99999} und \bibitem[... ,
%         aber dann wurde immer der feste Titel "Literatur" bzw. "References"
%         geschrieben und wir fanden keine M�glichkeit, ihn weg zu bekommen.
%         L�sung: Stattdessen \begin{description} \item[...


\begin{description}

\item[\textrm{[Doyle1905]}] \index{Doyle 1905}
    Arthur Conan Doyle \index{Doyle, Sir Arthur Conan}, \\
    {\em Die tanzenden M"annchen}, 1905. \\
    In der Sherlock-Holmes-Erz"ahlung {\em Die tanzenden M"annchen} 
    (erschienen erstmals 1903 im "`Strand Magazine"', und dann 1905 im 
    Sammelband "`Die R"uckkehr des Sherlock Holmes"' erstmals in Buchform)
    wird Sherlock Holmes mit einer Geheimschrift konfrontiert, die zun"achst
    wie eine harmlose Kinderzeichnung aussieht. \\
    Sie erweist sich als
    monoalphabetische Substitutions-Chiffre des Verbrechers Abe Slaney.
    Sherlock Holmes knackt die Geheimschrift mittels H"aufigkeitsanalyse.


\item[\textrm{[Sayer1932]}] \index{Sayer 1932}
    Dorothy L. Sayer, \\
    {\em Zur fraglichen Stunde und Der Fund in den Teufelsklippen 
    (Orginaltitel: Have his carcase)}, Harper, 1932 \\
    (1. dt. "Ubersetzung {\em Mein Hobby: Mord} bei A. Scherz, 1964; \\
    dann {\em Der Fund in den Teufelsklippen} bei Rainer Wunderlich-Verlag, 
    1974;\\
    Neu"ubersetzung 1980 im Rowohlt-Verlag). \\
    In diesem Roman findet die Schriftstellerin Harriet Vane eine Leiche 
    am Strand und die Polizei h"alt den Tod f"ur einen Selbstmord. 
    Doch Harriet Vane und der elegante Amateurdetektiv Lord Peter Wimsey
    kl"aren in diesem zweiten von Sayers's ber"uhmten Harriet Vane's
    Geschichten den widerlichen Mord auf. \\
    Dazu ist ein Chiffrat zu l"osen. Erstaunlicherweise beschreibt der
    Roman nicht nur die Playfair-Chiffre, sondern auch die Kryptoanalyse
    dieser Chiffre recht genau. 


\item[\textrm{[Seed1990]}] \index{Seed 1990}
    Regie Paul Seed (Paul Lessac), \\
    {\em Das Kartenhaus (Orginaltitel: HOUSE OF CARDS)}, 1990 (dt. 1992). \\
    In diesem Film versucht Ruth, hinter das Geheimnis zu kommen, das ihre
    Tochter verstummen lie"s. Hierin unterhalten sich Autisten mit Hilfe von
    5- und 6-stelligen Primzahlen. Nach "uber eine Stunde kommen im Film
    die folgenden beiden (nicht entschl"usselten) Primzahlfolgen vor:
%    \vskip -30pt  %be_2005 Bewirkt anscheinend nichts -- Abstand etwas zu gro�.
    \begin{center}
    $21.383, \;\;176.081, \;\;18.199, \;\;113.933, \;\;150.377, \;\;304.523, \;\;113.933$\\
    $193.877, \;\;737.683, \;\;117.881, \;\;193.877$
    \end{center}


\item[\textrm{[Robinson1992]}] \index{Robinson 1992}
    Regie Phil Alden Robinson, \\
    {\em Sneakers - Die Lautlosen}, Universal Pictures Film, 1992. \\
    In diesem Film versuchen die "`Sneakers"', Computerfreaks um ihren Boss Martin
    Bishop, den "`B"osen"' das Dechiffrierungsprogramm SETEC abzujagen.
    SETEC wurde von einem genialen Mathematiker vor seinem gewaltsamen Tod
    erfunden und kann alle Geheimcodes dieser Welt entschl"usseln.\\
    Das Verfahren wird nicht beschrieben.


\item[\textrm{[Elsner1999]}] \index{Elsner 1999}
    Dr.~C.~Elsner, \\
    {\em Der Dialog der Schwestern}, c't, 1999. \\
    In dieser Geschichte, die als PDF-Datei dem CrypTool-Paket\index{CrypTool}
    beigelegt ist, unterhalten sich die Heldinnen vertraulich mit einer
    Variante des RSA-Verfahrens.
    

\item[\textrm{[Stephenson1999]}] \index{Stephenson 1999}
    Neal Stephenson, \\
    {\em Cryptonomicon}, Harper, 1999. \\
    Der sehr dicke Roman besch"aftigt sich mit Kryptographie sowohl im 
    zweiten Weltkrieg als auch in der Gegenwart.
    Die zwei Helden aus den 40er-Jahren sind der gl"anzende Mathematiker und
    Kryptoanalytiker Lawrence Waterhouse, und der "ubereifrige, morphiums"uchtige
    Bobby Shaftoe von den US-Marines. 
    Sie geh"oren zum Sonderkommando 2702, einer Alliiertengruppe, die versucht,
    die gegnerischen Kommunikationscodes zu knacken und dabei ihre eigene
    Existenz geheim zu halten. \\
    Diese Geheimnistuerei spiegelt sich in der Gegenwartshandlung wider,
    in der sich die Enkel der Weltkriegshelden -- der Programmierfreak 
    Randy Waterhouse und die sch"one Amy Shaftoe -- zusammentun. \\
    Cryptonomicon ist f"ur nicht-technische Leser teilweise schwierig zu
    lesen. Mehrere Seiten erkl"aren detailliert Konzepte der Kryptographie.
    Stephenson legt eine ausf"uhrliche Beschreibung der Solitaire-Chiffre
    bei, ein Papier- und Bleistiftverfahren\index{Papier- und Bleistiftverfahren},
    das von Bruce Schneier 
    entwickelt wurde und im Roman "`Pontifex"' genannt wird. Der benutzte 
    moderne Algorithmus wird nicht offengelegt.


\item[\textrm{[Elsner2001]}] \index{Elsner 2001}
    Dr.~C.~Elsner, \\
    {\em Das chinesische Labyrinth}, c't, 2001. \\
    In dieser Geschichte, die als PDF-Datei dem CrypTool-Paket\index{CrypTool}
    beigelegt ist, muss Marco Polo in einem Wettbewerb Probleme aus der
    Zahlentheorie l"osen, um Berater des gro"sen Khan zu werden.


\item[\textrm{[Colfer2001]}] \index{Colfer 2001}
    Eoin Colfer, \\
    {\em Artemis Fowl}, List-Verlag, 2001. \\  
    In diesem Jugendbuch gelangt der junge Artemis, ein Genie und Meisterdieb, 
    an eine Kopie des streng geheimen "`Buches der Elfen"'. Nachdem er es mit 
    Computerhilfe entschl"usselt hat, erf"ahrt er Dinge, die kein Mensch erfahren
    d"urfte. \\
    Der Code wird in dem Buch nicht genauer beschrieben.


\item[\textrm{[Howard2001]}] \index{Howard 2001}
    Ron Howard, \\
    {\em A Beautiful Mind}, 2001. \\
    Verfilmung der von Sylvia Nasar verfa"sten Biographie des Spieltheoretikers
    John Nash. 
    Nachdem der brillante, aber unsoziale Mathematiker geheime kryptografische
    Arbeiten annimmt, verwandelt sich sein Leben in einen Alptraum. Sein 
    unwiderstehlicher Drang, Probleme zu l"osen, gef"ahrden ihn und sein 
    Privatleben.
    Nash ist in seiner Vorstellungswelt ein staatstragender Codeknacker. \\
    Konkrete Angaben zur seinen Analyseverfahren werden nicht beschrieben.


\item[\textrm{[Apted2001]}] \index{Apted 2001}
    Regie Michael Apted, \\
    {\em Enigma -- Das Geheimnis}, 2001. \\
    Verfilmung des von Robert Harris verfa"sten "`historischen Romans"' 
    {\em Enigma} (Hutchinson, London, 1995) "uber die ber"uhmteste 
    Verschl"usselungsmaschine in der Geschichte, die in
    Bletchley Park nach polnischen Vorarbeiten gebrochen wurde. 
    Die Geschichte spielt 1943, als der eigentliche Erfinder Alan Turing
    schon in Amerika war. So kann der Mathematiker Tom Jericho als Hauptperson
    in einem spannenden Spionagethriller brillieren.\\
    Konkrete Angaben zu dem Analyseverfahren werden nicht gemacht.


\item[\textrm{[Kippenhahn2002]}] \index{Kippenhahn 2002}
    Rudolf Kippenhahn, \\
    {\em Streng geheim! -- Wie man Botschaften verschl"usselt und 
    Zahlencodes knackt}, rororo, 2002. \\
    In dieser Geschichte bringt ein Gro"svater, ein Geheimschriftexperte, seinen
    4 Enkeln und deren Freunden bei, wie man Botschaften verschl"usselt, die 
    niemand lesen soll. Da es jemand gibt, der die Geheimnisse knackt, muss der 
    Gro"svater den Kindern immer komplizierte Verfahren beibringen. \\
    In dieser puren Rahmenhandlung werden die wichtigsten klassischen Verfahren 
    und ihre Analyse kindgerecht und spannend erl"autert.


\item[\textrm{[Isau2003]}] \index{Isau 2003}
    Ralf Isau, \\
    {\em Das Museum der gestohlenen Erinnerungen}, Thienemann-Verlag, 2003. \\
    In diesem Roman kann der letzte Teil des Spruches nur durch die 
    vereinten Kr"afte der Computergemeinschaft gel"ost werden.


\item[\textrm{[Brown2003]}] \index{Brown 2003}
    Dan Brown, \\
    {\em Sakrileg (Orginaltitel: The Da Vinci Code)}, L"ubbe, 2004. \\
    Der Direktor des Louvre wird in seinem Museum vor einem Gem"alde Leonardos
    ermordet aufgefunden, und der Symbolforscher Robert Langdon ger"at in eine
    Verschw"orung.\\
    Innerhalb der Handlung werden verschiedene klassische Codes (Substitution
    wie z.B. Caesar oder Vigenere, sowie Transposition und Zahlencodes) 
    angesprochen. Au�erdem klingen interessante Nebenbemerkungen "uber 
    Schneier oder die Sonnenblume an.
    Der zweite Teil des Buches ist sehr von theologischen Betrachtungen 
    gepr�gt. \\
    Das Buch ist einer der erfolgreichsten Romane der Welt.

% Rezensionen aus der Amazon.de-Redaktion:
% Bestsellerautor Dan Brown bietet mit Sakrileg erneut spannende und intelligente Unterhaltung vom Feinsten. Der Direktor des Louvre wird in seinem Museum vor einem Gem�lde Leonardos ermordet aufgefunden, und der Symbolforscher Robert Langdon ger�t ins Fadenkreuz der Polizei, war er doch mit dem Opfer just zur Tatzeit verabredet. Eine Verschw�rung ist immer noch das Sch�nste. Stimmt, wenn sie schriftstellerisch so �berzeugend und raffiniert inszeniert ist, wie es dem Amerikaner Dan Brown in diesem Thriller gelingt. Genaue Recherchen an den Schaupl�tzen und penible historische Studien in Zusammenarbeit mit seiner Frau Blythe, einer Kunsthistorikerin, machen das umfangreiche Werk nicht nur f�r Historiker und Religionswissenschaftler, sondern gerade auch f�r ein gro�es Publikum zu einem echten Vergn�gen. Der Symbolologe Robert Langdon sitzt in der Klemme. Er gilt als Hauptverd�chtiger im Fall Jacques Sauni�re, des ermordeten Direktors des Louvre, und ger�t als solcher in die F�nge von Capitaine Bezu Fache, der als �beraus gerissener Ermittler gilt. Sauni�re hatte im Todeskampf einen Hinweis auf Langdon gegeben. Mithilfe von Sophie Neveu, der Enkelin des Ermordeten, gelingt Langdon die Flucht. Beide sind der �berzeugung, dass Sauni�re vielmehr Informationen �ber eine Verschw�rung des Opus Dei und der katholischen Kirche liefern wollte. Im Verlauf einer atemlosen Flucht von Frankreich nach England haben Langdon und Neveu knifflige Codes zu knacken, um Sauni�res Geheimnis zu l�ften, der sich als Gro�meister der Geheimorganisation Prieur� de Sion entpuppt. Auf ihren Fersen befindet sich nicht nur die Polizei. Die Handlung einer Nacht und eines Tages auf 600 fesselnden Seiten, die �berdies Lust machen auf mehr Informationen zu Templern, Prieur� de Sion, Opus Dei sowie auf mehr historische Fakten -- was will man mehr. Und wer das Ganze nicht allzu ernst nimmt, wird die Lekt�re sehr genie�en -- am besten innerhalb einer Nacht und eines Tages.
% --Ulrich Deurer



% Matrix, Tron, ...
% Mastercode, ...


\end{description}



%    \href{http://www....}
%    {\texttt{http://www....}}


