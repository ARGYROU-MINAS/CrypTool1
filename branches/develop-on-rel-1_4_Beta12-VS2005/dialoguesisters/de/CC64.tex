\begin{center}
{\bf IV.}
\end{center}
Einige Tage nach diesem Vorfall findet Martin beim Aufr\"aumen von Biancas
und Doris' Appartement im Bad eine einzelne Karte aus einem Skatspiel. Sie
ist halb verdeckt hinter den Spiegel geklemmt. Er wundert sich ein wenig
und bringt die Karte ins Wohnzimmer, wo er sie zu den \"ubrigen Karten in
eine Schublade legen will, in der die Schwestern einige Gesellschaftsspiele
aufbewahren. Martin hat es l\"angst aufgegeben, bei seinen Schutzbefohlenen
nach einem Sinn jeder ihrer Handlungen zu fragen. Als er jedoch die wild in
der Schublade verstreuten Skatkarten ordnen will, stutzt er: alle Karten sind
auf der R\"uckseite mit Zahlen beschrieben. Sie sind stets in Paaren
angeordnet, und jedes Zahlenpaar ist bis auf ein einziges auf jeder Karte mit
einem Bleistift wieder ausgestrichen worden. Martin untersucht die Karte,
die er eben im Bad gefunden hat. Das auf ihr nicht gel\"oschte Zahlenpaar
lautet:
\begin{center}
{\bf 14\,857\,\,,\quad 3\,\,.}
\end{center}
Da dringt pl\"otzlich das Gel\"achter der beiden Schwestern von drau{\ss}en
herein; sie kommen gerade vom Fr\"uhst\"uck aus dem Gemeinschaftsraum in ihr
Zimmer zur\"uck. Martin st\"o{\ss}t rasch die Schublade zu und l\"a{\ss}t
die Karte aus dem Bad in seiner Hosentasche verschwinden. \\
Bianca und Doris wechseln kurz ein paar Worte mit ihm, albern herum und
verfallen dann wieder in ihre Geheimsprache. Der junge Mann \"uberlegt, ob er
es wagen soll: er werkelt erst noch ein wenig im Zimmer herum, dann wendet er
sich zum Gehen. Ehe er aber die T\"ur hinter sich schlie{\ss}t, ruft er laut
und vernehmlich ins Zimmer zur\"uck:
\begin{center}
'Vierzehntausendachthundertsiebenundf\"unfzig,  Drei.'
\end{center}
Augenblicklich entf\"ahrt beiden Schwestern ein kurzer Schrei. Bianca st\"urmt
ins Bad, kommt aber gleich wieder heraus und schaut Martin fragend an. Der
h\"alt ihr die gefundene Karte entgegen, die aber keine von beiden
zur\"ucknehmen will. Dann schlie{\ss}t er die T\"ur, sein Herz pocht bis zum
Halse. Zum ersten Mal hat er die beiden Schwestern in Verlegenheit gesehen.
Offensichtlich ist dieses Zahlenpaar 14\,857,\,\,3 ein Schl\"ussel zum
Verst\"andnis ihrer Sprache. Martin nimmt sich fest vor, dieses R\"atsel zu
l\"osen. Vielleicht, ja vielleicht kann er irgendwann an den Gespr\"achen der
Schwestern teilnehmen. Wenn sie ihn dabei akzeptieren, erf\"ahrt er
vielleicht mehr \"uber dieses Sanatorium. Martin ist so in Gedanken vertieft,
da{\ss} er auf dem Gang beinahe mit Anna zusammenst\"o{\ss}t. V\"ollig
entr\"uckt, ohne irgendetwas um sich herum wahrzunehmen, monologisiert sie:
'Gott, hilf mir! Denn das Wasser geht mir bis an die Kehle. Ich versinke in
tiefem Schlamm, wo kein Grund ist; ich bin in tiefe Wasser geraten, und die
Flut will mich ers\"aufen. Ich habe mich m\"ude geschrien, mein Hals ist
heiser. Meine Augen sind tr\"ube geworden, weil ich so lange harren mu{\ss}
auf meinen Gott!\footnote{{\em Psalter\/} 69,2-4}' Martin ist im Augenblick
nicht bereit, sich mit Anna auseinanderzusetzen. Er l\"a{\ss}t sie stehen
und geht weiter seinen Gedanken nach: 'Diese ausgestrichenen
Zahlenpaare auf den Karten - das m\"ussen Tagescodes sein, auf die sich die
Schwestern morgens im Bad einigen. Ja, nat\"urlich, so wird es sein! Aber:
Bianca und Doris werden mir sicher nichts erkl\"aren k\"onnen; ich kann mich
nur versuchsweise in ihre Dialoge einmischen, um herauszubekommen, ob ich auf
dem richtigen Weg bin!' \\
Da das erste Problem schon damit anf\"angt, da{\ss} die Schwestern ihre Zahlen
sehr schnell sprechen, besorgt sich Martin in der Mittagspause in der
Klinikverwaltung ein ausrangiertes Diktierger\"at. Er gibt vor, beim Lesen
eines medizinischen Lehrbuches Notizen auf Band festhalten zu wollen, um sie
sp\"ater schriftlich auszuarbeiten. Die Sekret\"arin sieht ihn zwar scheel von
der Seite an, aber sie r\"uckt das Ger\"at schlie{\ss}lich heraus. Zur\"uck
in seinem Zimmer holt Martin die bei den Schwestern gefundene Spielkarte hervor
und studiert die Zahlenpaare auf der R\"uckseite:
\begin{center}
{\bf \dots \,\,4\,267\,,\,7\,\,;\quad 17\,819\,,\,5\,\,;\quad 59\,989\,,\,33
\,\,\dots}
\end{center}
Auffallend ist, da{\ss} die erste Zahl eines Paares immer bedeutend
gr\"o{\ss}er ist als die zweite. Mangels eines besseren Einfalls geht Martin
von der naheliegenden Grundvoraussetzung aus, da{\ss} in einer Geheimsprache
die 26~Buchstaben des Alphabets den ersten 26 Zahlen zugeordnet sind. Martin
fertigt sich hierzu eine Tabelle an:
\begin{center}
\begin{tabular}{c|r||c|r}
A & 1 & N & 14 \\
B & 2 & O & 15 \\
C & 3 & P & 16 \\
D & 4 & Q & 17 \\
E & 5 & R & 18 \\
F & 6 & S & 19 \\
G & 7 & T & 20 \\
H & 8 & U & 21 \\
I & 9 & V & 22 \\
J & 10 & W & 23 \\
K & 11 & X & 24 \\
L & 12 & Y & 25 \\
M & 13 & Z & 26
\end{tabular}
\end{center}
\[\]
Aber so einfach ist es bei den Schwestern nicht: sie operieren mit viel
gr\"o{\ss}eren Zahlen, und f\"ur diese einfache Tabelle ben\"otigte man auch
keinen Schl\"ussel in Form eines Zahlenpaares. 'Wenn dieser Ansatz richtig
ist', \"uberlegt Martin, 'so werden mit Hilfe des Zahlenpaares die Zahlen von
1 bis 26, die f\"ur die jeweiligen Buchstaben stehen, auf andere, gr\"o{\ss}ere
Zahlen abgebildet, und zwar so, da{\ss} man das auch wieder eindeutig auf die
Zahlen 1 bis 26 zur\"uckrechnen kann; vorausgesetzt, man kennt den Schl\"ussel.
Aber wie geht das hier, nach welchen Regeln?' Martin ist frustriert: es gibt
beliebig viele M\"oglichkeiten f\"ur eine solche Kodierung. 'Hier mu{\ss} ich
anders vorgehen, einen Versuchsballon starten', denkt er. Er holt einen
Schreibblock hervor und denkt lange \"uber einen kurzen Satz nach, in dem
m\"oglichst viele Buchstaben mehrfach vorkommen. Schlie{\ss}lich schreibt er
in gro{\ss}en Lettern \"uber das ganze Blatt:
\begin{center}
ICH BIN BIANCA
\end{center}
So. Und nun zum Schl\"ussel. 'Damit ich eine Chance habe, bei der Kodierung
dieser Worte sp\"ater eine Gesetzm\"a{\ss}igkeit zu entdecken, mu{\ss} ich das
Zahlenpaar m\"oglichst geschickt w\"ahlen. Die zweite Zahl des Paares nehmen
die Schwestern immer recht klein; was geschieht wohl, wenn ich sie auf 1
setze? Dann k\"onnte ihr Einflu{\ss} bei der Kodierung vielleicht erst einmal
ziemlich gering sein.' Martin ist mit dieser Arbeitshypothese zufrieden. 'Und
wie w\"ahle ich die erste Zahl des Schl\"ussels?' Hierzu will Martin nichts
Rechtes einfallen. Schlie{\ss}lich h\"alt er es f\"ur das Einfachste, sie mit
der Anzahl der Buchstaben des Alphabets gleichzusetzen. So notiert er noch in
der linken oberen Ecke des Blattes:
\begin{center}
{\bf 26\,,\,\,1\,\,.}
\end{center}
Martin blickt skeptisch auf das Ergebnis. 'Hoffentlich bin ich nicht auf dem
falschen Dampfer!' murmelt er vor sich hin. 'Nun denn, mal schauen.' - \\
Er klopft bei Bianca und Doris an die T\"ur. 'Herein!' Die Schwestern blicken
Martin erwartungsvoll an; von der Unruhe, in die sie heute morgen der
Kartenfund versetzt hatte, ist nichts mehr zu sp\"uren. Etwas verlegen tritt
der Zivi an den Wohnzimmertisch heran, an dem die beiden Frauen sitzen. Mitten
auf dem Tisch liegt eine Spielkarte; in gro{\ss}en Ziffern steht ein
Zahlenpaar darauf. Die Schwestern machen keine Anstalten, die Karte vor Martin
zu verbergen. Im Gegenteil: da sie ihn immer noch stumm anblicken und jede
seiner Bewegungen verfolgen, f\"uhlt Martin sich eingeladen, in die Welt ihrer
Geheimnisse einzutreten. \"Uber diese stumme Aufforderung ist er unglaublich
erleichtert, und er hat den Eindruck, sie sitzen hier schon lange am Tisch
und haben auf ihn gewartet. Irgendwie f\"uhlt er sich durch ihr Vertrauen
sehr geehrt. 'K\"onnt ihr mir mal etwas \"ubersetzen?'\footnote{ % Ab hier folgt eine zus"atzliche Fu�note
{\em Kryptoanalyse\/},  
Martin w\"ahlt instinktiv den typischen Ansatz eines Kryptoanalysten. Er versucht, mehr 
Informationen zu gewinnen, indem er sich zueinander passende St\"ucke von Klar- und 
Geheimtext besorgt (hier sogar ausgew\"ahlte Klartexte). Im 2. Weltkrieg wurde dies z.B. 
von den Briten sehr erfolgreich angewandt, um die Enigma zu knacken: bestimmte 
Einzelziele wurden angegriffen, um die damit verbundenen Funkspr\"uche zu provozieren.
Hier in dieser Geschichte liegt ein Sonderfall vor, weil die Schwestern kooperieren 
und freiwillig die zusammengeh\"orenden Texte liefern.} % Ende der Fu�note 
Er stellt das
Diktierger\"at auf den Tisch und schaltet es ein. 'Aber gerne doch' erwidert
Doris mit geradezu \"ubertriebener Freundlichkeit. Martin legt Bianca seinen
vorbereiteten Bogen vor, weist auf die Codezahlen 26,1 und beobachtet die
Gesichter der Frauen. Bianca l\"achelt, als ob ihr die geforderte
Verschl\"usselung viel zu einfach erscheint. Nach einigen wenigen Augenblicken
sprudeln jedoch 12 Zahlen aus ihr hervor: 139, 289, 112, 496, 1335, 612, 80,
2063, 365, 508, 133, 53. 'Ein Gl\"uck, da{\ss} ich den Einfall mit dem
Diktierger\"at hatte!' denkt Martin bei Biancas Zungenschlag. Er nimmt das
Ger\"at und den Bogen wieder an sich, fragt beil\"aufig, ob er f\"ur beide
noch etwas tun k\"onne, und verzieht sich rasch in sein Zimmer. Dort h\"ort er
das Diktierger\"at ab und schreibt die Zahlen unter die Buchstaben:
\[\begin{array}{cccccccccccc}
I & C & H & B & I & N & B & I & A & N & C & A \\
139 & 289 & 112 & 496 & 1335 & 612 & 80 & 2063 & 365 & 508 & 133 & 53
\end{array} \]
Danach ist der junge Mann ziemlich ratlos. 'Der Buchstabe A ist also einmal
mit 53 und einmal mit 365 chiffriert worden. Nach meiner Tabelle w\"urde ich
ihm aber die 1 zuordnen.' Martin starrt auf das Papier. 'Und irgendwie spielt
die 26 als Verschl\"usselungszahl dabei eine Rolle \dots ' Er f\"uhrt die
einfachste Rechenoperation mit der 26, die Addition, aus, um von der 1 in die
N\"ahe der 53 zu gelangen:
\[\begin{array}{l}
1\,+\,26\,=\,27 \\
1\,+\,26\,+\,26\,=\,{\bf 53}
\end{array} \]
'Ups. Ist das ein Zufall?' Wohl nicht, denn rasch rechnet Martin nach:
\[1\,+\,14\cdot 26\,=\,{\bf 365}\,\,.\]
'Da haben wir ja das Geheimnis: Jede Zahl kann den Buchstaben A
repr\"asentieren, die bei der Division mit 26 den Rest 1 l\"a{\ss}t. Also
h\"atte Bianca f\"ur das A auch 79 w\"ahlen k\"onnen, denn \(1+3\cdot 26=79 \).
Und demnach kann ein Buchstabe durch ganz viele Zahlen repr\"asentiert werden.
Raffiniert!' Martin \"uberpr\"uft die \"ubrigen Buchstaben aus Biancas
\"Ubersetzung:
\[\begin{array}{cccclccc}
I & \longrightarrow & 9 & \longrightarrow & 9+5\cdot 26 & \,=\, & 139 & \\
C & \longrightarrow & 3 & \longrightarrow & 3+11\cdot 26 & \,=\, & 289 & \\
H & \longrightarrow & 8 & \longrightarrow & 8+4\cdot 26 & \,=\, & 112 &
\mbox{u.s.w.}
\end{array} \]
'Das l\"a{\ss}t sich auch wieder leicht zur\"uck\"ubersetzen. Man mu{\ss} nur
schauen, welchen Rest zwischen 1 und 26 eine Zahl bei der Division mit 26
l\"a{\ss}t:
\[139\,:\,26\,=\,5\,\,\mbox{Rest} \,\,9\,; \,\mbox{also}\,\,9\,\longrightarrow
\,\mbox{I}\,\,.\]
Martin lehnt sich zufrieden zur\"uck. Einen ersten Zipfel des Geheimnisses
hat er gel\"uftet. Doch welchen Einflu{\ss} hat nun dabei noch die zweite
Zahl aus dem Schl\"ussel, die er bisher auf 1 gesetzt hatte? Martin radiert
auf dem wei{\ss}en Bogen die Codezahlen 26,1 aus und f\"ugt stattdessen
{\bf 51,3} ein; die 51 hat er willk\"urlich gew\"ahlt, mit der 3 verwendet er
absichtlich noch eine besonders kleine Zahl. \\
\newpage \noindent % Wittwen & Weisen ....
Nach dem Abendessen sucht er die Schwestern wieder auf und l\"a{\ss}t
sich den Text mit dem neuen Schl\"ussel noch einmal chiffrieren. Jetzt
liefert ihm Bianca, wieder unter Zuhilfenahme des Diktierger\"ates:
\[\begin{array}{cccccccccccc}
I & C & H & B & I & N & B & I & A & N & C & A \\
1647 & 741 & 665 & 1487 & 780 & 2897 & 620 & 525 & 2500 & 2336 & 486 & 1174
\end{array} \]
Da die erste Codezahl 51 ist, mutma{\ss}t Martin aufgrund seiner bisherigen
Erfahrung, da{\ss} es hier nur auf Reste bei der Division mit 51 ankommt;
und so m\"u{\ss}ten Zahlen, die {\em denselben Buchstaben\/} chiffrieren,
{\em denselben Rest\/} bei der Division mit 51 ergeben. Martin \"uberpr\"uft
diese These zuerst und ermittelt mit einem Taschenrechner die Reste:
\[
\begin{array}{crccrc}
{\bf I}: & 1647 & = 32\cdot 51 + {\bf 15} \qquad & {\bf C}: & 741 &
= 14\cdot 51 + {\bf 27} \\
& 780 & = 15\cdot 51 + {\bf 15} \qquad & & 486 & = \,\,\,9\cdot 51 + {\bf 27} \\
& 525 & = 10\cdot 51 + {\bf 15} \qquad & & & \\ \\
{\bf H}: & 665 & = 13\cdot 51 + {\bf 2} \qquad & {\bf B}: & 1487 &
= 29\cdot 51 + {\bf 8} \\
& & & & 620 & = 12\cdot 51 + {\bf 8} \\ \\
{\bf N}: & 2897 & = 56\cdot 51 + {\bf 41} \qquad & {\bf A}: & 2500 &
= 49\cdot 51 + {\bf 1} \\
& 2336 & = 45\cdot 51 + {\bf 41} \qquad & & 1174 & = 23\cdot 51 + {\bf 1}
\end{array} \]
Die Darstellung eines Buchstaben mit demselben Rest bei der Division mit
51 hat sich somit best\"atigt. Nun bleibt noch die Frage nach dem Einflu{\ss}
der zweiten Codeziffer~3. Bis auf den Buchstaben {\bf A} entsprechen die oben
ermittelten Reste aber nicht mehr der nat\"urlichen Numerierung der Buchstaben
des Alphabets mit den Zahlen 1 bis 26; ja, bei {\bf C} und {\bf N} kommen mit
27 und 41 sogar Zahlen oberhalb von 26 vor. Das mu{\ss} die Einwirkung der
Codeziffer~3 sein! Bei der Division mit 51 k\"onnen sich insgesamt 51
m\"ogliche Reste ergeben, und irgendwie trifft die 3 hieraus eine Auswahl von
26 St\"uck und ordnet diesen Resten die Buchstaben des Alphabets zu. Martin
springt vom Tisch auf und l\"auft wie ein Tiger im K\"afig in seinem engen
Zimmerchen auf und ab. 'Ich mu{\ss} herausfinden, wie die 3 die Zahlen
1,2,\dots ,26 ver\"andert. Sonst komme ich nicht weiter!' redet er laut mit
sich selbst. 'Die einfachsten Rechenoperationen haben schon einmal zum Ziel
gef\"uhrt. Mal schauen': \\
\newpage\noindent % Wittwen & Weisen ...
{\em Addition mit 3:} \quad {\bf A} \(\longrightarrow \) 1 \(\longrightarrow \)
1+{\bf 3} = 4\,, \quad {\bf B} \(\longrightarrow \) 2 \(\longrightarrow \)
2+{\bf 3} = 5\,\,. \\
'Quatsch mit So{\ss}e! In Biancas Verschl\"usselung kommt bei {\bf A} wieder
1 und bei {\bf B} die 8 heraus. Das hier w\"are auch zu simpel!' \\
{\em Multiplikation mit 3:} \quad {\bf A} \(\longrightarrow \) 1
\(\longrightarrow \) {\bf 3}\,\(\cdot \)\,1 = 3\,, \quad {\bf B}
\(\longrightarrow \) 2 \(\longrightarrow \) {\bf 3}\,\(\cdot \)\,2 = 6\,\,. \\
'Ist es auch nicht!' Martin stehen Schwei{\ss}perlen auf der Stirn. Er
schlie{\ss}t die Augen und sieht sich in seinem Mathe-Kurs in der Schule
sitzen. 'Bei {\em Dr. Bruckner\/} haben wir doch so viel gelernt! Wenn es mir
jetzt nur helfen w\"urde! Ich war doch nicht der Schlechteste. Mir haben doch
selbst die Formeln f\"ur Potenzen und Logarithmen keine Schwierigkeiten
bereitet, im Gegensatz zu meinem Freund {\em Carsten\/} \dots . Moment! Was
war das eben?' Martin spricht wieder laut mit sich selbst. 'Potenzen \dots -
die fortgesetzte Multiplikation!!' Er st\"urzt an den Tisch und rechnet:
\[  \!\!
\begin{array}{ccrcrcl}
{\bf A} & \longrightarrow  & 1 & \longrightarrow  & 1^{{\bf 3}} & = & 1\cdot 1
\cdot 1 = {\bf 1} \\
{\bf B} & \longrightarrow  & 2 & \longrightarrow  & 2^{{\bf 3}} & = & 2\cdot 2
\cdot 2 = {\bf 8} \\
{\bf C} & \longrightarrow  & 3 & \longrightarrow  & 3^{{\bf 3}} & = & 3\cdot 3
\cdot 3 = {\bf 27} \\
{\bf H} & \longrightarrow  & 8 & \longrightarrow  & 8^{{\bf 3}} & = & 8\cdot 8
\cdot 8 = 512 = 10\cdot 51 + {\bf 2} \\
{\bf I} & \longrightarrow  & 9 & \longrightarrow  & 9^{{\bf 3}} & = & 9\cdot 9
\cdot 9 = 729 = 14\cdot 51 + {\bf 15} \\
{\bf N} & \longrightarrow  & 14 & \longrightarrow & 14^{{\bf 3}} & = & 14\cdot
14\cdot 14 = 2744 = 53\cdot 51 + {\bf 41}
\end{array} \]
'Wou! Ich hab's. Ich bin der Gr\"o{\ss}te! Die zweite Zahl 3 im Code ist ein
{\em Verschl\"usselungsexponent\/}!' Martin springt vor Freude auf sein Bett.
'Ich habe Bianca und Doris durchschaut. Jetzt gibt's keine Geheimnisse mehr
f\"ur mich!' Martin kehrt an den Tisch zur\"uck und verschl\"usselt im
Rausch seiner Freude das Wort DANKE mit dem Schl\"ussel {\bf 51,3}:
\[ \!\!
\begin{array}{ccrcrcrcrcrcr}
{\bf D} & \! \longrightarrow \! & 4 & \! \longrightarrow \! & 4^3 & = \! & 64 & = \! & 51 +
{\bf 13} & \! \longrightarrow \! & 9\cdot 51 + {\bf 13} & = \! & {\bf 472} \\
{\bf A} & \! \longrightarrow \! & 1 & \! \longrightarrow \! & 1^3 & = \! & 1 & = \! &  {\bf 1} &
\! \longrightarrow \! & 21\cdot 51 + {\bf 1} & = \! & {\bf 1072} \\
{\bf N} & \! \longrightarrow \! & 14 & \! \longrightarrow \! & 14^3 & = \! & 2744 & = \! &
53\cdot 51 + {\bf 41} & \! \longrightarrow \! & 2\cdot 51 + {\bf 41} & = \! & {\bf 143}
\\
{\bf K} & \! \longrightarrow \! & 11 & \! \longrightarrow \! & 11^3 & = \! & 1331 & = \! &
26\cdot 51 + {\bf 5} & \! \longrightarrow \! & 17\cdot 51 + {\bf 5} & = \! & {\bf 872}
\\
{\bf E} & \! \longrightarrow \! & 5 & \! \longrightarrow \! & 5^3 & = \! & 125 & = \! &
2\cdot 51 + {\bf 23} & \! \longrightarrow \! & 7\cdot 51 + {\bf 23} & = \! & {\bf 380}
\end{array} \]
Die Faktoren 9, 21, 2, 17 und 7 bei 51 hat Martin dabei willk\"urlich gew\"ahlt.
Er schreibt das Ergebnis auf einen kleinen Zettel:
\begin{center}
{\bf 51,3}: \qquad 472, 1072, 143, 872, 380.
\end{center}
Im Laufschritt spurtet er durch die G\"ange des Sanatoriums und steht nach
wenigen Augenblicken vor Biancas und Doris' Zimmer. Er klopft an und
schiebt den Zettel unter der T\"ur durch. Man h\"ort, wie sich drinnen jemand
der T\"ur n\"ahert. Martin wartet und horcht angespannt. Ein von Bianca laut
gesprochenes 'Bitte sehr' gibt ihm die Best\"atigung f\"ur die Korrektheit
seiner \"Uberlegungen und Rechnungen. W\"ahrend er mit langsamen Schritten in
sein Zimmer zur\"uckkehrt, h\"ort er im Weggehen noch das Gekicher der beiden
Schwestern. 