% ....................................................................................
%                             P R I M Z A H L E N
% ~~~~~~~~~~~~~~~~~~~~~~~~~~~~~~~~~~~~~~~~~~~~~~~~~~~~~~~~~~~~~~~~~~~~~~~~~~~~~~~~~~~~
\newpage
\section{Primzahlen}
(Bernhard Esslinger, besslinger@web.de, Mai 1999, Updates: Nov. 2000, Dezember 2001)
\hypertarget{Kapitel_2}{}

\begin{center}
\fbox{\parbox{15cm}{
    \emph{Albert Einstein\footnotemark:}\\
    Der Fortschritt lebt vom Austausch des Wissens.
}}
\end{center}
\addtocounter{footnote}{0}\footnotetext{%
  deutscher Physiker und Nobelpreistr"ager, 14.03.1879$-$14.04.1955
}

\subsection{Was sind Primzahlen?}
\index{Primzahlen} \index{Zahlen!Primzahl} Primzahlen sind ganze,
positive Zahlen gr"o"ser gleich $2$, die man nur durch 1 und durch
sich selbst teilen kann. Alle anderen nat"urlichen Zahlen gr"o"ser
gleich $2$ lassen sich durch Multiplikation von Primzahlen
bilden.

Somit bestehen die {\em nat"urlichen} \index{Zahlen} Zahlen $ \mathbb{N} = \{1, 2,
3, 4,\cdots \} $ aus
\begin{itemize}
   \item der Zahl $1$ (dem Einheitswert)
   \item den Primzahlen (primes) und
   \item den zusammengesetzten Zahlen (composite numbers).
\end{itemize}

Primzahlen haben aus 3 Gr"unden besondere Bedeutung erlangt:
\begin{itemize}
  \item Sie werden in der Zahlentheorie als die Grundbausteine der nat"urlichen Zahlen betrachtet,
        anhand derer eine Menge genialer mathematischer "Uberlegungen gef"uhrt wurden.
  \item Sie haben in der modernen \index{Kryptographie!moderne} Kryptographie (Public Key \index{Kryptographie!Public Key} Kryptographie) gro"se praktische
        Bedeutung erlangt. Das verbreitetste Public Key Verfahren ist die Ende der siebziger Jahre
        erfundene \index{RSA} RSA-Verschl"usselung. Nur die Verwendung (gro"ser) Primzahlen f"ur bestimmte
        Parameter garantiert die Sicherheit des Algorithmus sowohl beim RSA-Verfahren als auch bei
        noch moderneren Verfahren (digitale \index{Signatur!digitale} Signatur, Elliptische Kurven).
  \item Die Suche nach den gr"o"sten bekannten Primzahlen hat wohl bisher keine praktische
        Verwendung, erfordert aber die besten Rechner, gilt als hervorragender Benchmark (M"oglichkeit
        zur Leistungsbestimmung von Computern) und f"uhrt zu neuen Formen der Berechnungen auf
        mehreren Computern \\
        (siehe auch: \href{http://www.mersenne.org/prime.htm}{\tt http://www.mersenne.org/prime.htm}).
\end{itemize}
Von Primzahlen lie"sen sich im Laufe der letzten zwei Jahrtausende sehr viele Menschen faszinieren.
Der Ehrgeiz, zu neuen Erkenntnissen "uber Primzahlen zu gelangen, f"uhrte dabei oft zu genialen Ideen
und Schlu"sfolgerungen.
Im folgenden wird in einer leicht verst"andlichen Art in die mathematischen Grundlagen der Primzahlen
eingef"uhrt. Dabei kl"aren wir auch, was "uber die Verteilung (Dichte, Anzahl von Primzahl in einem
bestimmten Intervall) der Primzahlen bekannt ist oder wie Primzahltests funktionieren.


% -----------------------------------------------------------------------------------------------------------------------
\subsection{Primzahlen in der Mathematik}\label{primesinmath}

Jede ganze Zahl hat Teiler. Die Zahl 1 hat nur einen, n"amlich
sich selbst. Die Zahl $12$ hat die sechs Teiler $1, 2, 3, 4, 6,
12$. Viele Zahlen sind nur teilbar durch sich selbst und durch
$1$. Bez"uglich der Multiplikation sind dies die \glqq Atome\grqq
~ im Bereich der Zahlen. Diese Zahlen nennt man Primzahlen.

In der Mathematik ist eine etwas andere (aber "aquivalente) Definition "ublich.

\begin{definition}\label{def-pz-prime}
Eine ganze Zahl $p \in {\bf N}$ hei"st Primzahl \index{Zahlen!Primzahl}, wenn $p > 1$ und $p$ nur die trivialen
Teiler $\pm 1$ und $\pm p$ besitzt.
\end{definition}


Per definitionem ist die Zahl $1$ keine Primzahl. Im weiteren bezeichnet der Buchstabe $p$ stets eine Primzahl.

Die Primzahlenfolge startet mit
$$ 2,~ 3,~ 5,~ 7, ~ 11, ~ 13, ~ 17, ~ 19, ~ 23, ~ 29, ~ 31, ~ 37, ~ 41, ~ 43, ~ 47, ~ 53, ~ 59, ~ 61, ~ 67, ~ 71, ~ 73, ~ 79, ~ 83, ~ 89, ~ 97, \cdots .$$
Unter den ersten 100 Zahlen gibt es genau 25 Primzahlen. Danach nimmt ihr prozentualer Anteil stets
ab. Primzahlen k"onnen nur auf eine einzige {\em triviale} Weise zerlegt werden:
$$5 = 1 \cdot 5,\quad  17 = 1 \cdot 17, \quad 1013 = 1 \cdot 1.013,  \quad 1.296.409 = 1 \cdot 1.296.409.$$
Alle Zahlen, die $2$ und mehr Faktoren haben, nennt man \index{Zahlen!zusammengesetzte} {\em zusammengesetzte}. Dazu geh"oren
$$ 4 = 2 \cdot 2, \quad 6 = 2\cdot 3 $$
aber auch Zahlen, die {\em wie Primzahlen aussehen}, aber doch keine sind:
$$ 91 = 7 \cdot 13, \quad 161=7 \cdot 23, \quad 767 =13 \cdot 59. $$

\begin{satz}\label{thm-pz-sqr}
Jede ganze Zahl $m$ gr"o"ser als $1$ besitzt einen kleinsten Teiler gr"o"ser als $1$.
Dieser ist eine Primzahl $p$. Sofern $m$ nicht selbst eine Primzahl ist, gilt:
$p$ ist kleiner oder gleich der Quadratwurzel aus $m$.
\end{satz}

Aus den Primzahlen lassen sich alle ganzen Zahlen gr"o"ser als $1$ zusammensetzen --- und das sogar in
einer eindeutigen Weise. Dies besagt der 1. Hauptsatz der Zahlentheorie (= Hauptsatz der elementaren Zahlentheorie =
fundamental theorem of arithmetic = fundamental building block of all positive integers).\index{Zahlentheorie!Hauptsatz}

\begin{satz}\label{thm-pz-prod}
Jedes Element $n$ gr"o"ser als $1$ der nat"urlichen Zahlen l"asst sich als Produkt
$n = p_1 \cdot p_2 \dots p_m$ von Primzahlen schreiben.
Sind zwei solche Zerlegungen
$$n =  p_1 \cdot p_2 \cdots p_m = p'_1 \cdot p'_2 \cdots p'_{m'}$$
gegeben, dann gilt nach eventuellem Umsortieren $\;m = m'\;$ und  f"ur alle $i$:  $\;p_i = p'_i$.
\end{satz}

In anderen Worten: Jede nat"urliche Zahl au"ser der $1$ l"asst sich auf genau eine Weise als Produkt von
Primzahlen schreiben, wenn man von der Reihenfolge der Faktoren absieht. Die Faktoren sind also
eindeutig (die {\em Expansion in Faktoren} ist eindeutig)!
Zum Beispiel ist
$$ 60 = 2 \cdot 2 \cdot 3 \cdot 5 = 2^2\cdot 3^1 \cdot 5^1 $$
Und das ist --- bis auf eine ver"anderte Reihenfolge der Faktoren
--- die einzige M"oglichkeit, die Zahl $60$ in Primfaktoren zu
zerlegen. Wenn man nicht nur Primzahlen als Faktoren zul"asst,
gibt es mehrere M"oglichkeiten der Zerlegung in Faktoren und die
Eindeutigkeit (uniqueness) geht verloren: \hypertarget{uniqueness}{}
$$ 60 = 1 \cdot 60 = 2 \cdot 30 = 4 \cdot 15 = 5 \cdot 12 =6 \cdot 10 = 2 \cdot 3 \cdot 10 =
        2 \cdot 5 \cdot 6 = 3 \cdot 4 \cdot 5 = \cdots . $$

Der folgende Absatz wendet sich eher an die mit der mathematischen Logik vertrauteren Menschen:
Der 1. Hauptsatz ist nur scheinbar selbstverst"andlich\label{remFundTheoOfArithm}. Man kann viele andere Zahlenmengen
(ungleich der positiven ganzen Zahlen gr"o"ser als 1) konstruieren, bei denen selbst eine Zerlegung in
die Primfaktoren dieser Mengen nicht eindeutig ist:
In der Menge $M = \{1, 5, 10, 15, 20, \cdots\}$ gibt es unter der Multiplikation kein Analogon zum Hauptsatz.
Die ersten f"unf Primzahlen dieser Folge sind $5, 10, 15, 20, 30$ (beachte: $10$ ist prim, da innerhalb
dieser Menge $5$ kein Teiler von $10$ ist --- das Ergebnis $2$ ist kein Element der gegebenen Grundmenge
$M$). Da in $M$ gilt:
$$ 100 = 5 \cdot 20 = 10 \cdot 10 $$
und sowohl $5, 10, 20$ Primzahlen dieser Menge sind, ist hier die Zerlegung in Primfaktoren nicht
eindeutig.

% ---------------------------------------------------------------------------------------------------------------------------
\subsection{Wie viele Primzahlen gibt es?}

F"ur die nat"urlichen Zahlen sind die Primzahlen vergleichbar mit den Elementen in der Chemie oder
den Elementarteilchen in der Physik (vgl. \cite[S. 22]{Blum1999}).

W"ahrend es nur $92$ nat"urliche chemische Elemente gibt, ist die Anzahl der Primzahlen unbegrenzt.
Das wu"ste schon der Grieche \index{Euklid} Euklid\footnote{Euklid, griechischer Mathematiker des 4./3. Jahrhunderts vor Christus.
Wirkte an der Akademie in Alexandria und verfa"ste mit den \glqq Elementen\grqq~ das bekannteste systematische Lehrbuch
der griechischen Mathematik.} im dritten vorchristlichen Jahrhundert.
\begin{satz}[Euklid]\label{thm-pz-euklid}\hypertarget{thm-pz-euklid}{}\footnote{Die "ublich gewordene Benennung soll nicht sagen,
dass Euklid der Entdecker des Satzes ist, da dieser
nicht bekannt ist. Der Satz wird bereits in Eulkids \glqq Elementen\grqq ~(Buch IX, Satz 20) 
formuliert und bewiesen. Die dortige Formulierung ist insofern bemerkenswert,
als sie das Wort \glqq unendlich\grqq~ nicht verwendet; sie lautet
$$
O\acute{\iota}~\pi\varrho\tilde{\omega}\tau o \iota~\grave{\alpha}\varrho\iota\vartheta\mu o\grave{\iota}~
\pi\lambda\varepsilon\acute{\iota}o \upsilon\varsigma~\varepsilon\grave{\iota}\sigma\grave{\iota}~
\pi\alpha\nu\tau\grave{o}\varsigma~\tau o \tilde{\upsilon}~
\pi\varrho o \tau\varepsilon\vartheta\acute{\varepsilon}\nu\tau o \varsigma~
\pi\lambda\acute{\eta}\vartheta\ o \upsilon\varsigma~
\pi\varrho\acute{\omega}\tau\omega\nu~
\grave{\alpha}\varrho\iota\vartheta\mu\tilde{\omega}\nu,
$$
zu deutsch: Die Primzahlen sind mehr als jede vorgegebene Menge von Primzahlen.}
Die Folge der Primzahlen bricht nicht ab, es gibt also unendlich viele Primzahlen.
\end{satz}

Sein Beweis, dass es
unendlich viele Primzahlen gibt, gilt bis heute als ein Glanzst"uck mathematischer "Uberlegung und
Schlu"sfolgerung (Widerspruchsbeweis). Er nahm an, es gebe nur endlich viele Primzahlen und damit
eine gr"o"ste Primzahl. Daraus zog er solange logische Schl"usse, bis er auf einen offensichtlichen
Widerspruch stie"s. Damit mu"ste etwas falsch sein. Da sich in die Schlu"skette kein Lapsus
eingeschlichen hatte, konnte es nur die Annahme sein. Demnach mu"ste es unendlich viele
Primzahlen geben!

\hypertarget{euklid}{}
\paragraph{Euklid's Widerspruchsbeweis}\index{Euklid's Widerspruchsbeweis}
f"uhrt die Argumentation wie folgt:

{\bf Annahme:} \quad Es gibt {\em endlich} viele Primzahlen.
\\*[4pt] {\bf Schlu"s:} \quad Dann lassen sie sich auflisten $p_1
< p_2 < p_3 < \dots < p_n$, wobei $n$ f"ur die (endliche) Anzahl
der Primzahlen steht. $p_n$ w"are also die gr"o"ste Primzahl. Nun
betrachtet Euklid die Zahl $a = p_1 \cdot p_2 \cdots p_n +1$.
Diese Zahl kann keine Primzahl sein, da sie in unserer
Primzahlenliste nicht auftaucht. Also mu"s sie durch eine Primzahl
teilbar sein. D.h. es gibt eine nat"urliche Zahl $i$ zwischen $1$
und $n$, so dass $p_i$ die Zahl $a$ teilt. Nat"urlich teilt $p_i$
auch das Produkt $a-1 = p_1 \cdot p_2 \cdots p_n$, da $p_i$ ja ein
Faktor von $a-1$ ist. Da $ p_i $ die Zahlen $ a $ und $ a-1 $
teilt, teilt sie auch die Differenz dieser Zahlen. Daraus folgt:
$p_i$ teilt  $a - (a-1) = 1$. $p_i$ m"u"ste also $1$ teilen und
das ist unm"oglich.\\*[4pt] 
{\bf Wiederspruch}: \quad Unsere Annahme war falsch.\par
Also gibt es {\em unendlich} viele Primzahlen
\hyperlink{primhfk}{(Verweis: "Ubersicht unter \ref{s:primhfk} "uber die
Anzahl von Primzahlen in verschiedenen Intervallen).}\par \vskip + 3pt

Wir erw"ahnen hier auch noch eine andere, auf den ersten Blick "uberraschende Tatsache, dass n"amlich in der
Folge aller Primzahlen $p_1, p_2, \cdots $ L"ucken von beliebig gro"ser L"ange $n$ auftreten. Unter den $n$ 
aufeinanderfolgenden nat"urlichen Zahlen
$$ 
    (n+1)!+2, \cdots, (n+1)!+(n+1),
$$
ist keine eine Primzahl, da ja in ihnen der Reihe nach die Zahlen $2,\cdots, n+1$ als echte Teiler enthalten sind
(Dabei bedeutet $n!$ das Produkt der ersten $n$ nat"urlichen Zahlen, also 
$n!=n*(n-1)* \cdots *3*2*1$). 
% ---------------------------------------------------------------------------------------------------------------------
\subsection{Die Suche nach sehr gro"sen Primzahlen}

Die gr"o"sten heute bekannten Primzahlen haben mehrere
hunderttausend Stellen. Das ist unvorstellbar gro"s. Die Anzahl
der Elementarteilchen im Universum wird auf \glqq nur\grqq\ eine
$80$-stellige Zahl gesch"atzt \hyperlink{grosord}{(Verweis:
"Ubersicht unter \ref{s:grosord} "uber verschiedene Gr"o"senordnungen /
Dimensionen)}.

Nahezu alle bekannten riesig gro"sen Primzahlen sind spezielle Kandidaten, sogenannte \index{Mersenne Marine}
{\em Mersennezahlen} der Form $2^p -1,$
wobei $p$ eine Primzahl ist. Marin Mersenne (1588-1648) war ein franz"osicher Priester und
Mathematiker. Nicht alle Mersennezahlen sind prim:
$$
\begin{array}{cl}
2^2 - 1 = 3 & \Rightarrow {\rm prim} \\
2^3 - 1 = 7 & \Rightarrow {\rm prim} \\
2^5 - 1 = 31    & \Rightarrow {\rm prim} \\
   \vdots        \\
2^{11} - 1 = 2.047 = 23 \cdot 89    & \Rightarrow  {\rm NICHT~prim} !
\end{array}
$$

\index{Zahlen!Mersennezahlen}\index{Mersenne Marine!Mersennezahlen} \index{Mersenne Marine!Satz} 
Dass Mersennezahlen nicht immer Primzahlen (Mersenne-Primzahlen) sind, wu"ste auch schon Mersenne (siehe Exponent $p = 11$).
Dennoch ist ihm der interessante Zusammenhang zu verdanken, dass eine Zahl der Form $2^n-1$
keine Primzahl ist, wenn $n$ eine zusammengesetzte Zahl ist:

\begin{satz}[Mersenne]\label{thm-pz-mersenne}
Wenn $2^n - 1$ eine Primzahl ist, dann folgt, $n$ ist ebenfalls
eine Primzahl.
\end{satz}

\begin{Beweis}{}
Der Beweis des Satzes von Mersenne kann durch Widerspruch
durchgef"uhrt werden. Wir nehmen also an, dass es eine
zusammengesetzte nat"urliche Zahl $ n $ (mit echter Zerlegung)  $
n=n_1 n_2 $ gibt, mit der Eigenschaft, dass $ 2^n -1 $ eine
Primzahl ist.

Wegen
\begin{eqnarray*}
(x^r-1)((x^r)^{s-1} + (x^r)^{s-2} + \cdots + x^r +1) & = &  ((x^r)^s + (x^r)^{s-1} + (x^r)^{s-2} + \cdots + x^r) \\
&  & -((x^r)^{s-1} + (x^r)^{s-2} + \cdots + x^r +1)  \\
& = & (x^r)^s -1 = x^{rs } -1,
\end{eqnarray*}
folgt
\[ 2^{n_1 n_2} - 1 = (2^{n_1} -1)((2^{n_1})^{n_2 -1} + (2^{n_1})^{n_2 -2} + \cdots + 2^{n_1} + 1). \]
Da $ 2^n - 1 $ eine Primzahl ist, mu"s einer der obigen beiden
Faktoren auf der rechte Seite gleich 1 sein. Dies kann nur dann
der Fall sein, wenn $ n_1 =1 $ oder $ n_2 =1$ ist. Dies ist aber
ein Widerspruch unserer Annahme. Deshalb ist unsere Annahme
falsch. Also gibt es keine zusammengesetzte Zahl $ n, $ so dass $
2^n -1 $ eine Primzahl ist.
\end{Beweis} \vskip + 5pt
Leider gilt dieser Satz nur in einer Richtung (die Umkehrung gilt
nicht, keine "Aquivalenz): siehe das obige Beispiel $2^{11}-1, $
wo $11$ prim ist.

Mersenne behauptete, dass $2^{67}-1$ eine Primzahl ist. Auch zu
dieser Behauptung gibt es mathematisch eine interessante Historie:
Zuerst dauerte es "uber 200 Jahre, bis \index{Lucas Edouard}
Edouard Lucas (1842-1891) bewies, dass diese Zahl zusammengesetzt
ist. Er argumentierte aber indirekt und kannte keinen der
Faktoren. Dann zeigte Frank Nelson 1903, aus welchen Faktoren,
diese Primzahl besteht:
$$ 2^{67} -1 =147. 573. 952. 588. 676. 412. 927 = 193. 707. 721 \cdot 761. 838. 257. 287. $$
Er gestand, 20 Jahre an der Faktorisierung (Zerlegung in Faktoren)
dieser 21-stelligen Dezimalzahl gearbeitet zu haben!

Dadurch, dass man bei den Exponenten der Mersennezahlen nicht alle nat"urlichen Zahlen verwendet,
sondern nur die Primzahlen, engt man den {\em Versuchsraum} deutlich ein.
Die derzeit bekannten Mersenne-Primzahlen \index{Mersenne Marine!Mersenne-Primzahlen} gibt es f"ur die Exponenten
$$
\begin{array}{c}
2, ~ 3, ~ 5, ~ 7, ~ 13, ~ 17, ~ 19, ~ 31, ~ 61, ~ 89, ~ 107, ~ 127, ~ 521, ~ 607, ~ 1.279, ~ 2.203, ~ 2.281, ~ 3.217, ~ 4.253, \\
4.423, ~9.689, ~ 9.941, ~ 11.213, ~ 19.937, ~ 21.701, ~ 23.207, ~ 44.497, ~ 86.243, ~ 110.503, ~ 132.049,\\
216.091, ~ 756.839, ~ 859.433, ~ 1.257.787, ~ 1.398.269, ~ 2.976.221, ~ 3.021.377, ~ 6.972.593,\\
13.466.917.
\end{array}
$$
Damit sind heute $39$ Mersenne-Primzahlen\index{Primzahlen!Mersenne}\index{Mersenne Marine!Mersenne-Primzahlen} bekannt.

Die $19$. Zahl
mit dem Exponenten $4.253$ war die erste mit mindestens $1.000$
Stellen im Zehnersystem (der Mathematiker Samual \index{Yates
Samual} Yates pr"agte daf"ur den Ausdruck {\em titanische}
\index{Primzahlen!titanische} Primzahl; sie wurde 1961 von Hurwitz
gefunden); die $27$. Zahl mit dem Exponenten $44.497$ war die
erste mit mindestens $10.000$ Stellen im Zehnersystem (Yates
pr"agte daf"ur den Ausdruck \index{Primzahlen!gigantische}  {\em
gigantische} Primzahl. Diese Bezeichnungen sind heute l"angst
veraltet).

Die 37. Zahl wurde im Januar 1998 gefunden und hat
909.526 Stellen im Zehnersystem, was 33 Seiten in der FAZ
entspricht!

Man findet diese Zahlen unter den URLs

{\href{http://reality.sgi.com/chongo/prime/prime_press.html}{\tt
http://reality.sgi.com/chongo/prime/prime\_press.html}}
\vskip - 4pt \noindent
\begin{quote}
(Der Supercomputerhersteller SGI Cray Research besch"aftigte nicht
nur hervorragende Mathematiker, sondern benutzte die Primzahltests
auch als Benchmarks f"ur seine Maschinen.)
\end{quote}
{\href{http://www.utm.edu/}{\tt http://www.utm.edu/}}
\vskip - 4pt \noindent
\begin{quote}
(An der Universit"at von Tennessee findet man umfangreiche Forschungsergebnisse "uber Primzahlen.)
\end{quote}

\subsubsection{M-38 -- Juni 1999} \index{Mersenne Marine!M-38}
Die 38. Mersenne-Primzahl, genannt M-38, $$ 2^{6.972.593} - 1 $$
wurde im Juni 1999 gefunden und hat $2.098.960$ Stellen im
Zehnersystem (das entspricht rund 77 Seiten in der FAZ).
\subsubsection{M-39 -- Dezember 2001}\index{Mersenne Marine!M-39}
Die 39. Mersenne-Primzahl, genannt M-39, $$2^{13.466.917}-1$$  wurde am 6.12.2001 gefunden -- genau
genommen war am 6.12.2001 die Verifikation der am 14.11.2001 von dem kanadischen Studenten Michael Cameron 
gefundenen Primzahl abgeschlossen.
Diese Zahl hat rund 4 Millionen Stellen (genau 4.053.946 Stellen). Allein zu ihrer Darstellung
($924947738006701322247758 \cdots 1130073855470256259071$) br"auchte man in der FAZ knapp 200 Seiten.
\subsubsection{GIMPS}\index{GIMPS}
Mit der 39. Mersenne-Primzahl hat das 1996 gegr"undete GIMPS-Projekt (Great Internet Mer"-senne-Prime Search) bereits
zum 5. Mal die gr"o"ste Mersenne-Zahl entdeckt, die erwiesenerma"sen prim ist.

Am GIMPS-Projekt beteiligen sich z.Zt. rund 130.000 freiwillige Amateure und Experten, die ihre Rechner in das
von der Firma entropia organisierte \glqq primenet\grqq~ einbinden, um mit verteilten Computerprogrammen solche Zahlen zu finden.
\subsubsection{EFF}\index{EFF}

Angefacht wird diese Suche noch zus"atzlich durch einen Wettbewerb, den die
Non\-profit-Orga"-nisa"-tion
EFF (Electronic Frontier Foundation) mit den Mitteln eines unbekannten Spenders gestartet hat. Den
Teilnehmern winken Gewinne im Gesamtwert von 500.000 USD, wenn sie die l"angste Primzahl
finden. Dabei sucht der unbekannte Spender nicht nach dem schnellsten Rechner, sondern er will auf
die M"oglichkeiten des {\em cooperative networking} aufmerksam machen

{\href{http://www.eff.org/coopawards/prime-release1.html}{\tt http://www.eff.org/coopawards/prime-release1.html}}.

Der Entdecker von M-38 erhielt f"ur die Entdeckung der ersten Primzahl mit "uber 1 Million Dezimalstellen von der EFF eine
Pr"amie von 50.000 USD. Die n"achste Pr"amie von 100.000 USD gibt es von der EFF f"ur eine Primzahl mit mehr als 10 Millionen
Dezimalstellen.

Edouard Lucas (1842-1891) hielt "uber 70 Jahre den Rekord der gr"o"sten bekannten Primzahl, indem
er nachwies, dass $2^{127}-1$ prim ist. Solange wird wohl kein neuer Rekord mehr Bestand haben.

% ----------------------------------------------------------------------------------------------------------------------------
\subsubsection{Primzahltests}
\index{Primzahltest}
F"ur die Anwendung sicherer Verschl"usselungsverfahren braucht man sehr gro"se Primzahlen (im
Bereich von $2^{2.048}$, das sind Zahlen im Zehnersystem mit "uber $600$ Stellen!).

Bisher haben wir immer nach den Primfaktoren gesucht, um zu entscheiden, ob eine Zahl prim ist.
Wenn aber auch der kleinste Primfaktor riesig ist, dauert die Suche zu lange. Die Zerlegung in
Faktoren mittels rechnerischer systematischer Teilung oder mit dem Sieb des Eratosthenes (siehe
unten) ist mit heutigen Computern anwendbar f"ur Zahlen mit bis zu circa $20$ Stellen im Zehnersystem.

Ist aber etwas "uber die {\em Bauart} der fraglichen Zahl bekannt, gibt es sehr hochentwickelte Verfahren,
die deutlich schneller sind.
Fermat\index{Fermat} hatte im 17. Jahrhundert an Mersenne geschrieben, dass er vermute, dass alle Zahlen der Form
$$ F(n) = 2^{2^n} + 1 $$
f"ur alle ganzen Zahlen $n$ gr"o"ser oder gleich $0$ prim seien (siehe unten).

Schon im 19. Jahrhundert wu"ste man, dass die $29$-stellige Zahl
$$ F(7) = 2^{2^7} + 1 $$
keine Primzahl ist. Aber erst 1970 fanden Morrison/Billhart ihre Zerlegung.
\begin{eqnarray*}
F(7) & = & 34.279.974.696.877.740.253.374.607.431.768.211.457 \\
& = & 59. 649. 589. 127. 497. 217 \cdot  574. 689. 200. 685.
129.054. 721.
\end{eqnarray*}
Der Ausgangspunkt vieler schneller Primzahltests ist der (kleine)
Fermatsche Satz, den Fermat im Jahr 1640 aufstellte.
\index{Fermat!kleiner Fermat}
\begin{satz}[\glqq kleiner\grqq\ Fermat]\label{thm-pz-fermat1}
Sei $p$ eine Primzahl und $a$ eine beliebige ganze Zahl, dann gilt f"ur alle $a$
$$a^p \equiv a \; {\rm mod} \; p.$$
Eine alternative Formulierung lautet: \\
Sei $p$ eine Primzahl und $a$ eine beliebige ganze Zahl, die kein
Vielfaches von $p$ ist (also $a \not\equiv 0 \; {\rm mod} \; p$),
dann gilt $a^{p-1} \equiv 1 \; {\rm mod} \; p$.
\end{satz}

Wer mit dem Rechnen mit Resten (Modulo-Rechnung) nicht so vertraut ist, m"oge den Satz einfach so
hinnehmen oder \hyperlink{Kapitel_3}{Kapitel 3 \glqq Einf"uhrung in die
elementare Zahlentheorie mit Beispielen\grqq} lesen. Wichtig ist, dass aus diesem Satz folgt, dass wenn diese Gleichheit f"ur irgendein ganzes $a$
nicht erf"ullt ist, dann ist $p$ keine Primzahl! Die Tests lassen sich (zum Beispiel f"ur die erste
Formulierung) leicht mit der {\em Testbasis} $a = 2$ durchf"uhren.

Damit hat man ein Kriterium f"ur Nicht-Primzahlen, also einen negativen Test, aber noch keinen
Beweis, dass eine Zahl $a$ prim ist.
Leider gilt die Umkehrung zum Fermatschen Satz nicht, sonst h"atten wir einen einfachen Beweis f"ur
die Primzahleigenschaft (man sagt auch, man h"atte dann ein einfaches Primzahlkriterium).

Anmerkung: Zahlen, die die Eigenschaft
$$ 2^n \equiv 2 \;{\rm mod}\; n $$
erf"ullen, aber nicht prim sind, bezeichnet man als \index{Pseudoprimzahlen}\index{Zahlen!Pseudoprimzahl} {\em Pseudoprimzahlen}.
Die erste Pseudoprimzahl (also keine Primzahl) ist
$$ 341 = 11 \cdot 31. $$
Es gibt Zahlen, die den Fermat-Test mit allen Basen bestehen und doch nicht prim sind: diese Zahlen
hei"sen \index{Zahlen!Carmichael Zahlen} {\em Carmichael-Zahlen}. Die erste ist
$$ 561 = 3 \cdot 11 \cdot 17. $$
Ein st"arkerer Test stammt von \index{Miller} \index{Rabin}
Miller/Rabin: er wird nur von sogenannten {\em starken
Pseudoprimzahlen} bestanden. Wiederum gibt es starke
Pseudoprimzahlen, die keine Primzahlen sind, aber das passiert
deutlich seltener als bei den (einfachen) Pseudoprimzahlen. Die
kleinste starke Pseudoprimzahl zur Basis $2$ ist
$$ 15.841 = 7 \cdot 31 \cdot 73. $$
Testet man alle 4 Basen $2, 3, 5$ und $7$ findet man bis $25 \cdot 10^9$
nur eine starke Pseudoprimzahl, also eine Zahl, die den Test besteht und doch keine Primzahl ist.

Weiterf"uhrende Mathematik hinter dem Rabin-Test gibt dann die Wahrscheinlichkeit an, mit der die
untersuchte Zahl prim ist (solche Wahrscheinlichkeiten liegen heutzutage bei circa  $10^{-60}$).

Ausf"uhrliche Beschreibungen zu Tests, um herauszufinden, ob eine Zahl prim ist, finden sich zum
Beispiel unter:

{\href{http://www.utm.edu/research/primes/mersenne.shtml}{\tt http://www.utm.edu/research/primes/mersenne.shtml}} und

{\href{http://www.utm.edu/research/primes/prove/index.html}{\tt
http://www.utm.edu/research/primes/prove/index.html}}.



% --------------------------------------------------------------------------------------------------------
\subsection{Die Suche nach einer Formel f"ur Primzahlen}
\index{Primzahlen!Formel}
Derzeit sind keine brauchbaren, offenen (also nicht rekursiven) Formeln bekannt, die nur Primzahlen
liefern (rekursiv bedeutet, dass zur Berechnung der Funktion auf dieselbe Funktion in Abh"angigkeit
einer kleineren Variablen zugegriffen wird). Die Mathematiker w"aren schon zufrieden, wenn sie eine
Formel f"anden, die wohl L"ucken l"asst (also nicht alle Primzahlen liefert), aber sonst keine
zusammengesetzten Zahlen (Nichtprimzahlen) liefert.

Optimal w"are, man w"urde f"ur das Argument $n$ sofort die $n$-te Primzahl bekommen, also f"ur
$f(8) = 19\,$ oder f"ur  $f(52) = 239$.

Ideen dazu finden sich in

{\href{http://www.utm.edu/research/primes/notes/faq/p_n.html}{\tt http://www.utm.edu/research/primes/notes/faq/p\_n.html}}.

Verweis: \hyperlink{ntePrimzahl}{Die Tabelle unter \ref{s:ntePrimzahl}} enth"alt  die exakten Werte f"ur
die $n$-ten Primzahlen f"ur ausgew"ahlte $ n.$

Im folgenden einige der verbreitetsten Ans"atze f"ur
Primzahlformeln:
\begin{enumerate}
  \item Mersennezahlen  $f(n) = 2^n - 1$: \\
    Wie oben gesehen, liefert diese Formel wohl relativ viele gro"se Primzahlen, aber es kommt -- wie
    f"ur $n=11$ [$f(n)=2.047$] -- immer wieder vor, dass das Ergebnis nicht prim ist.
  \item $F(k,n) = k \cdot 2^n \pm 1$  f"ur $ n $ prim und $ k $ kleine Primzahlen:\\
   F"ur diese Verallgemeinerung der Mersennezahlen gibt es (f"ur kleine $k$) ebenfalls sehr schnelle
    Primzahltests (vgl. \cite{Knuth1981}).
    Praktisch ausf"uhren l"asst sich das zum Beispiel mit der Software PROTHS von Yves Gallot

    ({\href{http://www.prothsearch.net/index.html}
          {\tt http://www.prothsearch.net/index.html}}).
  \item {\em Fermatzahlen}\footnote{Die Fermatschen Primzahlen spielen unter anderem eine wichtige Rolle in der Kreisteilung. Wie Gauss
\index{Gauss} bewiesen hat, ist das regul"are $p$-Eck f"ur eine Primzahl $p>2$ dann und nur dann mit Zirkel und
Lineal konstruierbar, wenn $p$ eine Fermatsche Primzahl ist.}
\index{Zahlen!Fermatzahlen}\index{Fermat!Fermatzahl}  $F(n) = 2^{2^n} + 1$:\\
    Wie oben erw"ahnt, schrieb Fermat diese Vermutung an Mersenne. Erstaunlicherweise w"are es f"ur
    ihn m"oglich gewesen, mit dem auf seinem kleinen Satz beruhenden negativen Primzahltest f"ur
    $n=5$  ein positives Ergebnis zu erhalten.
$$
\begin{array}{lll}
F(0) = 2^{2^0} + 1  = 2^1 + 1 & = 3 &   \mapsto {\rm ~prim}  \\
F(1) = 2^{2^1} + 1  = 2^2 + 1 & = 5 &   \mapsto {\rm ~prim}  \\
F(2) = 2^{2^2} + 1  = 2^4 + 1 & = 17 &  \mapsto {\rm ~prim}  \\
F(3) = 2^{2^3} + 1  = 2^8 + 1 & = 257 & \mapsto {\rm ~prim}  \\
F(4) = 2^{2^4} + 1  = 2^{16} + 1 &  = 65.537 &  \mapsto {\rm ~prim}  \\
F(5) = 2^{2^5} + 1  = 2^{32} + 1 &  = 4 .294. 967. 297 = 641 \cdot
6.700.417 &  \mapsto {\rm ~NICHT~prim} !
\end{array}
$$
  \item Carmichaelzahlen: s.o.
  \item Pseudoprimzahlen: s.o.
  \item starke Pseudoprimzahlen: s.o.
  \item Idee aufgrund von \hyperlink{thm-pz-euklid}{Euklids Beweis} (unendlich viele Primzahlen)
        $p_1 \cdot p_2 \cdots p_n +1$:
$$
\begin{array}{lll}
2{\cdot}3 +1 &      = 7 &          \mapsto {\rm ~prim} \\
2{\cdot}3{\cdot}5 +1 &      = 31    &      \mapsto {\rm ~prim} \\
2{\cdot}3{\cdot}5{\cdot}7 +1 &      = 211   &      \mapsto {\rm ~prim} \\
2{\cdot}3{\cdots}11 +1 &        = 2311  &      \mapsto {\rm ~prim} \\
2\cdot3 \cdots 13 +1 &  = 59 \cdot 509 &    \mapsto {\rm ~NICHT~prim} ! \\
2\cdot3 \cdots 17 +1 &  = 19 \cdot 97 \cdot 277 &   \mapsto {\rm ~NICHT~prim} ! \\
\end{array} 
$$
  \item Wie oben, nur statt $+1$: $p_1 \cdot p_2 \cdots p_n -1$
$$
 \begin{array}{lll}
2\cdot 3 -1     &   = 5 &   \mapsto {\rm ~prim} \\
2\cdot 3 \cdot  5  -1   &   = 29 &  \mapsto {\rm ~prim} \\
2\cdot 3 \cdots 7  -1   &   = 11 \cdot 19 & \mapsto {\rm ~NICHT~prim} ! \\
2\cdot 3 \cdots 11 -1   &   = 2309 &    \mapsto {\rm ~prim} \\
2\cdot 3 \cdots 13 -1  &    = 30029 &   \mapsto {\rm ~prim} \\
2\cdot 3 \cdots 17 -1    &  = 61 \cdot 8369 &   \mapsto {\rm ~NICHT~prim!}
\end{array}
$$
  \item \index{Euklidzahlen} {\em Euklidzahlen} $e_n = e_0 \cdot e_1 \cdots e_{n-1} + 1$  mit $n$ gr"o"ser oder gleich $1$ und $e_0 := 1$.
        $e_{n-1}$ ist nicht die $(n-1)$-te Primzahl, sondern die zuvor hier gefundene Zahl.
    Diese Formel ist leider nicht offen, sondern rekursiv.
    Die Folge startet mit
$$
\small\begin{array}{lll}
e_1 = 1 + 1 &   = 2 &   \mapsto {\rm ~prim} \\
e_2 = e_1 + 1   &   = 3 &   \mapsto {\rm ~prim} \\
e_3 = e_1 \cdot e_2 + 1 &   = 7 &   \mapsto {\rm ~prim} \\
e_4 = e_1 \cdot e_2 \cdot e_3 + 1 & = 43 &  \mapsto {\rm ~prim} \\
e_5 = e_1 \cdot e_2 \cdots e_4 + 1 &    = 13 \cdot 139 &    \mapsto {\rm ~NICHT~prim} ! \\
e_6 = e_1 \cdot e_2 \cdots e_5 + 1 &    = 3.263.443 &   \mapsto {\rm ~prim} \\
e_7 = e_1 \cdot e_2 \cdots e_6 + 1 &    = 547 \cdot 607 \cdot 1.033 \cdot 31.051 & \mapsto {\rm ~NICHT~prim} ! \\
e_8 = e_1 \cdot e_2 \cdots e_7 + 1 &    = 29.881\cdot 67.003 \cdot 9.119.521 \cdot 6.212.157.481 & \mapsto {\rm ~NICHT~prim} !
\end{array}
$$
Auch $e_9, \cdots, e_{17}$ sind zusammengesetzt, so dass dies auch
keine brauchbare Formel ist. Bemerkung: Das Besondere an allen
diesen Zahlen ist aber, dass sie jeweils paarweise keinen
gemeinsamen Teiler au"ser $1$ haben, sie sind also
\index{Primzahlen!relative} {\em relativ zueinander prim}.
\pagebreak %%%%%%%%%%%%%%%%%%%%%%%%%%%%%%%%%%%%%%%%
  \item $f(n) = n^2 + n + 41$:\\
  Diese Folge hat einen sehr {\em erfolgversprechenden} Anfang,
  aber das ist noch lange kein Beweis.
$$
\begin{array}{ll}
f(0) = 41 & \mapsto {\rm ~prim} \\
f(1) = 43 & \mapsto {\rm ~prim} \\
f(2) = 47 & \mapsto {\rm ~prim} \\
f(3) = 53 & \mapsto {\rm ~prim} \\
f(4) = 61 & \mapsto {\rm ~prim} \\
f(5) = 71 & \mapsto {\rm ~prim} \\
f(6) = 83 & \mapsto {\rm ~prim} \\
f(7) = 97 & \mapsto {\rm ~prim} \\
\vdots \\
f(39) = 1.601 & \mapsto {\rm ~prim} \\
f(40) = 11 \cdot 151 &  \mapsto {\rm ~NICHT~prim}! \\
f(41) = 41 \cdot 43 &   \mapsto {\rm ~NICHT~prim}! \\
\end{array}
$$
        Die ersten $40$ Werte sind Primzahlen (diese haben die auffallende Regelm"a"sigkeit, dass ihr
    Abstand beginnend mit dem Abstand $2$ jeweils um $2$ w"achst), aber der $41$. und der $42$. Wert
    sind keine Primzahlen.
    Da"s $f(41)$ keine Primzahl sein kann, l"asst sich leicht "uberlegen:
    $f(41) = 41^2 + 41 + 41 = 41 (41 + 1 + 1) = 41 \cdot 43$.
\pagebreak %%%%%%%%%%%%%%%%%%%%%%%%%%%%%%%%%%%%%%%%
  \item $f(n) = n^2 - 79 \cdot n + 1.601$: \\
    Diese Funktion liefert f"ur die Werte $n=0$ bis $n=79$ stets Primzahlwerte. Leider ergibt
        $f(80) = 1.681 = 11 \cdot 151$ keine Primzahl. Bis heute kennt man keine Funktion, die mehr aufeinanderfolgende
    Primzahlen annimmt. Andererseits kommt jede Primzahl doppelt vor (erst in der absteigenden,
    dann in der aufsteigenden Folge), so dass sie insgesamt 40 verschiedene Primzahlwerte liefert (dieselben wie die Funktion aus Punkt 10).
$$
\begin{array}{|ll||ll|}
\hline
f(0) = 1.601    & \mapsto {\rm ~prim} &  f(28) = 173    & \mapsto {\rm ~prim} \\
f(1) = 1.523    & \mapsto {\rm ~prim} &  f(29) = 151    & \mapsto {\rm ~prim} \\
f(2) = 1.447    & \mapsto {\rm ~prim} &  f(30) = 131 & \mapsto {\rm ~prim} \\
f(3) = 1.373    & \mapsto {\rm ~prim} &  f(31) = 113 & \mapsto {\rm ~prim} \\
f(4) = 1.301    & \mapsto {\rm ~prim} &  f(32) = 97 & \mapsto {\rm ~prim} \\
f(5) = 1.231    & \mapsto {\rm ~prim} &  f(33) = 83 & \mapsto {\rm ~prim} \\
f(6) = 1.163    & \mapsto {\rm ~prim} &  f(34) = 71 & \mapsto {\rm ~prim} \\
f(7) = 1.097    & \mapsto {\rm ~prim} &  f(35) = 61 & \mapsto {\rm ~prim} \\
f(8) = 1.033    & \mapsto {\rm ~prim} &  f(36) = 53 & \mapsto {\rm ~prim} \\
f(9) = 971  & \mapsto {\rm ~prim} &  f(37) = 47 & \mapsto {\rm ~prim} \\
f(10) = 911 & \mapsto {\rm ~prim} &  f(38) = 43 & \mapsto {\rm ~prim} \\
f(11) = 853 & \mapsto {\rm ~prim} &  f(39) = 41 & \mapsto {\rm ~prim} \\
f(12) = 797 & \mapsto {\rm ~prim} &             &                     \\
f(13) = 743 & \mapsto {\rm ~prim} &  f(40) = 41 & \mapsto {\rm ~prim} \\
f(14) = 691 & \mapsto {\rm ~prim} &  f(41) = 43 & \mapsto {\rm ~prim} \\
f(15) = 641 & \mapsto {\rm ~prim} &  f(42) = 47 & \mapsto {\rm ~prim} \\
f(16) = 593 & \mapsto {\rm ~prim} &  f(43) = 53 & \mapsto {\rm ~prim} \\
f(17) = 547 & \mapsto {\rm ~prim} &  \cdots  &  \\
f(18) = 503 & \mapsto {\rm ~prim} &  f(77) = 1.447  & \mapsto {\rm ~prim} \\
f(19) = 461 & \mapsto {\rm ~prim} &  f(78) = 1.523  & \mapsto {\rm ~prim} \\
f(20) = 421 & \mapsto {\rm ~prim} &  f(79) = 1.601  & \mapsto {\rm ~prim} \\
f(21) = 383 & \mapsto {\rm ~prim} &  f(80) = 11 \cdot 151 & \mapsto {\rm ~NICHT~prim!} \\
f(22) = 347 & \mapsto {\rm ~prim} &  f(81) = 41 \cdot 43 & \mapsto {\rm ~NICHT~prim!} \\
f(21) = 383 & \mapsto {\rm ~prim} &  f(82) = 1.847  & \mapsto {\rm ~prim} \\
f(22) = 347 & \mapsto {\rm ~prim} &  f(83) = 1.933  & \mapsto {\rm ~prim} \\
f(23) = 313 & \mapsto {\rm ~prim} &  f(84) = 43 \cdot 47 &  \mapsto {\rm ~NICHT~prim!} \\
f(24) = 281 & \mapsto {\rm ~prim} & & \\
f(25) = 251 & \mapsto {\rm ~prim} & & \\
f(26) = 223 & \mapsto {\rm ~prim} & & \\
f(27) = 197 & \mapsto {\rm ~prim} & & \\
\hline
\end{array}
$$
  \item Polynomfunktionen $f(x) = a_n x^n + a_{n-1}x^{n-1} + \cdots + a_1 x^1 + a_0$  ($a_i$ aus ${\mathbb Z}$, $n \geq 1$):

    Es existiert kein solches Polynom, das f"ur alle $x$ aus ${\mathbb Z}$ ausschlie"slich Primzahlwerte annimmt.
    Zum Beweis sei auf \cite[S. 83 f.]{Padberg1996}, verwiesen, wo sich auch weitere Details zu
    Primzahlformeln finden.
  \item \index{Catalan} \index{Catalanzahlen} Catalan, nach dem auch die sog. {\em Catalanzahlen} $A(n) = (1 / (n+1) ) * (2n)! / (n!)^2$ benannt sind,
    "au"serte die Vermutung, dass $C_4$ eine Primzahl ist:
$$
\begin{array}{l}
C_0 = 2, \\
C_1 = 2^{C_0} - 1,  \\
C_2 = 2^{C_1} - 1,  \\
C_3 = 2^{C_2} - 1, \\
C_4 = 2^{C_3} - 1, \cdots \\
\end{array}
$$
\begin{sloppypar}
    (siehe
        {\href{http://www.utm.edu/research/primes/mersenne.shtml}{\tt http://www.utm.edu/research/primes/mersenne.shtml}}
        unter Conjectures and Unsolved Problems).
\end{sloppypar}

    Diese Folge ist ebenfalls rekursiv definiert und w"achst sehr schnell. Besteht sie nur
    aus Primzahlen?
$$
\begin{array}{lll}
C(0) = 2 & & \mapsto {\rm ~prim}\\
C(1) = 2^2 - 1 &    = 3 & \mapsto {\rm ~prim}\\
C(2) = 2^3 - 1 &    = 7 & \mapsto {\rm ~prim} \\
C(3) = 2^7 - 1 &    = 127& \mapsto {\rm ~prim} \\
C(4) = 2^{127} - 1 &      = 170. 141. 183. 460. 469. 231. 731. 687. 303. 715. 884. 105. 727 & \mapsto {\rm ~prim} \\
\end{array}
$$
Ob $C_5$ bzw.\ h"ohere Elemente prim sind, ist (noch) nicht bekannt, aber auch nicht wahrscheinlich.
Bewiesen ist jedenfalls nicht, dass diese Formel nur Primzahlen liefert.
\end{enumerate}


% ------------------------------------------------------------------------------------------------------------
\subsection{Dichte und Verteilung der Primzahlen}

Wie Euklid herausfand, gibt es unendlich viele Primzahlen. Einige unendliche Mengen sind aber
{\em dichter} \index{Primzahlen!Dichte} als andere. Innerhalb der nat"urlichen Zahlen gibt es unendlich viele gerade, ungerade und
quadratische Zahlen.

Nach folgenden Gesichtspunkten gibt es mehr gerade Zahlen als quadratische:
\begin{itemize}
  \item die Gr"o"se des $n$-ten Elements: \\
    Das $n$-te Element der geraden Zahlen ist $2n$; das $n$-te Element der Quadratzahlen ist $n^2$. Weil f"ur
    alle $n>2$ gilt: $2n < n^2$, kommt die $n$-te gerade Zahl viel fr"uher als die $n$-te quadratische Zahl.
    Daher sind die geraden Zahlen dichter verteilt, und wir k"onnen sagen, es gibt mehr gerade als
    quadratische Zahlen.
  \item die Anzahl der Werte, die kleiner oder gleich einem bestimmten {\em Dachwert} $x$ aus ${\mathbb R}$ ist: \\
    Es gibt $[x/2]$ solcher gerader Zahlen und $[\sqrt{x}]$ Quadratzahlen. Da f"ur gro"se $x$ der
    Wert $x/2$ viel gr"o"ser ist als die Quadratwurzel aus $2$, k"onnen wir wiederum sagen, es gibt mehr
    gerade Zahlen.
\end{itemize}
\vskip +3 pt
\begin{satz}\label{thm-pz-density}
F"ur gro"se $n$ gilt: Der Wert der $n$-ten Primzahl $P(n)$ ist
asymptotisch zu $n \cdot ln(n)$, d.h. der Grenzwert des
Verh"altnisses  $P(n)/(n\cdot \ln n)$ ist gleich $1$, wenn $n$
gegen unendlich geht.
\end{satz}


"Ahnlich wird die Anzahl der Primzahlen $PI(x)$ definiert, die den
Dachwert $x$ nicht "ubersteigen:

\begin{satz}\label{thm-pz-pi-x}
$PI(x)$  ist asymptotisch zu  $x / ln(x)$.
\end{satz}


Dies ist der \index{Primzahlsatz} \textbf{Primzahlsatz} (prime
number theorem). Er wurde von Legendre (1752-1833) \index{Legendre} und Gauss\index{Gauss}
(1777-1855) aufgestellt und erst "uber 100 Jahre sp"ater bewiesen.

\hyperlink{primhfk}{(Verweis: "Ubersicht unter \ref{s:primhfk} "uber die
Anzahl von Primzahlen in verschiedenen Intervallen).}

Es gilt f"ur gro"se $n$, dass  $P(n)$  zwischen  $2n$  und  $n^2$  liegt.
Somit gibt es also weniger Primzahlen als gerade nat"urliche Zahlen, aber es gibt mehr Primzahlen als
Quadratzahlen.

Diese Formeln, die nur f"ur $n$ gegen unendlich gelten, k"onnen durch pr"azisere Formeln ersetzt werden.
F"ur $x \geq 67$ gilt:
$$ ln(x) - 1,5 < x / PI(x) < ln(x) - 0,5 $$
Im Bewu"stsein, dass $PI(x)  =  x / \ln x$ nur f"ur sehr gro"se $x$ ($x$ gegen unendlich) gilt, kann man
folgende "Ubersicht erstellen:
$$
\begin{array}{ccccc}
x     &  ln(x)  &  x / ln(x) & PI(x)(gez"ahlt) &       PI(x) / (x/ln(x)) \\
10^3  &  6,908  &   144      &  168        &       1,160 \\
10^6  &  13,816 &   72.386    &  78.498          &       1,085 \\
10^9  & 20,723  &   48.254.942 &  50.847.534       &       1,054
\end{array}
$$

F"ur eine Bin"arzahl (Zahl im Zweiersystem) der L"ange $250$ Bit ($2^{250}$ ist ungef"ahr = $1,809 251 * 10^{75}$)
gilt:
$$ PI(250) = 2^{250} / (250 \cdot \ln 2) \; {\rm ist~ungef"ahr} \; =
 2^{250} /173,28677 = 1,045 810 \cdot 10^{73}. $$
Es ist also zu erwarten, dass sich innerhalb der Zahlen der
Bitl"ange kleiner als 250 Zahlen ungef"ahr $10^{73}$ Primzahlen
befinden (ein beruhigendes Ergebnis?!).

Man kann das auch so formulieren: Betrachtet man eine {\em zuf"allige} nat"urliche Zahl $n$, so sind die
Chancen, dass diese Zahl prim ist, circa $1 / \ln(n)$. Nehmen wir zum Beispiel Zahlen in der Gegend von
$10^{16}$, so m"ussen wir ungef"ahr (durchschnittlich) $ 16 \cdot \ln 10 = 36,8 $
Zahlen betrachten, bis wir eine Primzahl finden.
Ein genaue Untersuchung zeigt: Zwischen $10^{16}-370$ und $10^{16}-1$ gibt es $10$
Primzahlen.

Unter der "Uberschrift {\em How Many Primes Are There} finden sich unter

{\href{http://www.utm.edu/research/primes/howmany.shtml}{\tt http://www.utm.edu/research/primes/howmany.shtml}}

viele weitere Details.

$PI(x)$ l"asst sich leicht per

{\href{http://www.math.Princeton.EDU/~arbooker/nthprime.html}{\tt
http://www.math.Princeton.EDU/\~{}arbooker/nthprime.html}}

bestimmen.


Die \textbf{Verteilung} der Primzahlen weist viele
Unregelm"a"sigkeiten auf, f"ur die bis heute kein �System�
gefunden wurde (einerseits liegen viele eng benachbart wie $2$ und
$3,$ $ 11$ und $13, $ $ 809$ und $811$, andererseits tauchen auch
l"angere Primzahll"ucken auf. So liegen zum Beispiel zwischen
$113$ und $127, $ $ 293$ und $307, $ $317$ und $331, $ $ 523$ und
$541, $ $ 773$ und $787, $ $ 839$ und $853$ sowie zwischen $887$
und $907$ keine Primzahlen) (Details siehe.

{\href{http://www.utm.edu/research/primes/notes/gaps.html}{\tt http://www.utm.edu/research/primes/notes/gaps.html}}).

Gerade dies macht einen Teil des Ehrgeizes aus, ihre Geheimnisse herauszufinden.

\paragraph{Sieb des Eratosthenes\index{Eratosthenes!Sieb}}
Ein einfacher Weg, alle $PI(x)$ Primzahlen kleiner oder gleich $x$
zu berechnen, ist das Sieb des Erathostenes. Er fand schon im 3.
Jahrhundert vor Christus einen sehr einfach automatisierbaren Weg,
das herauszufinden. Zuerst werden ab 2 alle Zahlen bis $x$
aufgeschrieben, die 2 umkreist und dann streicht man alle
Vielfachen von 2. Anschlie"send umkreist man die kleinste noch
nicht umkreiste oder gestrichene Zahl (3), streicht wieder alle
ihre Vielfachen, usw. Durchzuf"uhren braucht man das nur bis zu
der gr"o"sten Zahl, deren Quadrat kleiner oder gleich $x$ ist.

Abgesehen von 2 sind Primzahlen nie gerade. Abgesehen von 2 und 5 haben Primzahlen nie die
Endziffern 2, 5 oder 0. Also braucht man sowieso nur Zahlen mit den Endziffern 1, 3, 7, 9 zu
betrachten (es gibt unendlich viele Primzahlen mit jeder dieser letzten Ziffern; vergleiche \cite[Bd. 1,
S. 137]{Tietze1973}).

Inzwischen findet man im Internet auch viele fertige Programme,
oft mit komplettem Quellcode, so dass man auch selbst mit gro"sen
Zahlen experimentieren kann. Ebenfalls zug"anglich sind gro"se
Datenbanken, die entweder viele Primzahlen oder die Zerlegung in
Primfaktoren vieler zusammengesetzter Zahlen enthalten. Im
\index{Cunningham-Projekt} \textbf{Cunningham-Projekt} zum
Beispiel werden die Faktoren aller zusammengesetzten Zahlen
bestimmt, die sich in folgender Weise bilden:
$$ f(n) = b^n \pm 1  \quad {\rm f"ur~} b = 2, 3, 5, 6, 7, 10, 11, 12 $$
($b$ ist ungleich der Vielfachen von schon benutzten Basen wie $4, 8, 9$).

Details hierzu finden sich unter:

{\href{http://www.cerias.purdue.edu/homes/ssw/cun}{\tt
http://www.cerias.purdue.edu/homes/ssw/cun}}

% ---------------------------------------------------------------------------------------------------------------------------
\subsection{Anmerkungen}

\textbf{Bewiesene Aussagen / S"atze zu Primzahlen}

\begin{itemize}
  \item Zu jeder Zahl $n$ aus ${\bf N}$ gibt es $n$ aufeinanderfolgende nat"urliche Zahlen, die keine Primzahlen sind.
    Ein Beweis findet sich in \cite[S. 79]{Padberg1996}.
  \item Der ungarische Mathematiker Paul Erd"os (1913-1996) bewies:
    Zwischen jeder beliebigen Zahl ungleich $1$ und ihrem Doppelten gibt es mindestens eine Primzahl
    (er bewies das Theorem nicht als erster, aber auf einfachere Weise als andere vor ihm).
    Gerade diese Vermutung legt nahe, dass wir auch heute noch nicht in aller Tiefe den Zusammenhang zwischen der
    Addition und der Multiplikation der nat"urlichen Zahlen vertanden haben.
  \item Es existiert eine reelle Zahl a, so dass die Funktion $f: {\bf N} \rightarrow {\mathbb Z}$ mit $n \mapsto a^{3^n}$
    f"ur alle $n$ nur
    Primzahlenwerte annimmt (siehe \cite[S. 82]{Padberg1996}).
    Leider macht die Bestimmung von $a$ Probleme (siehe unten).
\end{itemize}

\textbf{Unbewiesene Aussagen / Vermutungen zu Primzahlen}

\begin{itemize}
  \item Der deutsche Mathematiker \index{Goldbach Christian} Christian Goldbach (1690-1764) vermutete:
    Jede gerade nat"ur"-liche Zahl gr"o"ser $2$ l"asst sich als die Summe zweier Primzahlen darstellen.
    Mit Computern ist die Goldbachsche Vermutung f"ur alle geraden Zahlen bis $4*10^{14}$ verifiziert\footnote{%
Da"s die Goldbachsche Vermutung wahr ist, d.h. f"ur alle geraden nat"urlichen Zahlen gr"o"ser als $2$ gilt, wird
    heute allgemein nicht mehr angezweifelt. Der Mathematiker J"org Richstein vom Institut f"ur Informatik der 
    Universit"at Gie"sen hat 1999 die geraden Zahlen bis 400 Billionen untersucht und kein Gegenbeispiel gefunden (Siehe 
    \href{http://www.informatik.uni-giessen.de/staff/richstein/de/Goldbach.html}
         {\tt http://www.informatik.uni-giessen.de/staff/richstein/de/Goldbach.html}).
    \index{Richstein 1999}
    Trotzdem ist das kein allgemeiner Beweis.\\ 
    Da"s die Goldbach�sche Vermutung trotz aller Anstrengungen bis heute nicht bewiesen wurde, f"ordert 
    allerdings einen Verdacht:
    Seit den bahnbrechenden Arbeiten des "osterreichischen Mathematikers Kurt G"odel\index{G\""odel Kurt}
    ist bekannt, dass nicht jeder wahre Satz in der Mathematik auch beweisbar ist (Siehe 
    \href{http://www.mathematik.ch/mathematiker/goedel.html}
         {\tt http://www.mathematik.ch/mathematiker/goedel.html}).
    M"oglicherweise hat Goldbach also Recht,
    und trotzdem wird nie ein Beweis gefunden werden. Das wiederum l"asst sich aber vermutlich auch nicht beweisen.
    },   % footnote
    aber allgemein noch nicht bewiesen\footnote{%
Der englische Verlag {\em Faber} und die amerikanische
    Verlagsgesellschaft {\em Bloomsbury} publizierten 2000 das 1992 erstmal
    ver"offentlichte Buch \glqq Onkel Petros und die Goldbachsche Vermutung\grqq~ von Apostolos Doxiadis (deutsch bei
    L"ubbe 2000 und bei BLT als Taschenbuch 2001). Es ist 
    die Geschichte eines Mathematikprofessors, der daran scheitert, ein mehr als 250 Jahre altes R"atsel zu l"osen.\\
    Um die Verkaufszahlen zu f"ordern, schreiben die beiden  Verlage einen Preis von 1 Million USD aus, wenn 
    jemand die Vermutung beweist -- ver"offentlicht in einer angesehenen mathematischen Fachzeitschrift bis 2004 (Siehe 
    \href{http://www.mscs.dal.ca/~dilcher/Goldbach/index.html}
         {\tt http://www.mscs.dal.ca/\~{}dilcher/Goldbach/index.html}).\\
    Erstaunlicherweise d"urfen nur englische und amerikanische Mathematiker daran teilnehmen.\\
    Der Beweis, der der Goldbach-Vermutung bisher am n"achsten kommt, wurde 1966 von Chen Jing-Run
    bewiesen -- in einer schwer nachvollziehbaren Art und Weise: Jede gerade Zahl gr"o"ser 2 ist die Summe einer 
    Primzahl und des Produkts zweier Primzahlen. Z.B. $20=5+3*5.$\\ 
    Die wichtigsten Forschungsergebnisse zur Goldbachschen Vermutung sind zusammengefasst in dem von Wang Yuan 
    herausgegebenen Band: \glqq Goldbach Conjecture\grqq, 1984, 
    World Scientific Series in Pure Maths, Vol. 4.
    Gerade diese Vermutung legt nahe, dass wir auch heute noch nicht in aller Tiefe den Zusammenhang zwischen
    der Addition und der Multiplikation der nat"urlichen Zahlen verstanden haben. \\
    }.   % footnote
\item Der deutsche \index{Riemann Bernhard} Mathematiker Bernhard Riemann (1826-1866) stellte eine bisher 
    unbewiesene, aber auch nicht widerlegte (?) Formel f"ur die Verteilung von Primzahlen auf, die die Absch"atzung 
    weiter verbessern w"urde.
\end{itemize}

\textbf{Offene Fragestellungen} \\

Primzahlzwillinge sind Primzahlen, die den Abstand 2 voneinander haben, zum Beispiel 5 und 7 oder
101 und 103. Primzahldrillinge gibt es dagegen nur eines: 3, 5, 7.  Bei allen anderen Dreierpacks
aufeinanderfolgender ungerader Zahlen ist immer eine durch 3 teilbar und somit keine Primzahl.
\begin{itemize}
  \item Offen ist die Anzahl der Primzahlzwillinge: unendlich viele oder eine begrenzte Anzahl?
  Das bis heute bekannte gr"o"ste Primzahlzwillingspaar ist
  $1.693.965 \cdot 2^{66.443} \pm 1.$
  \item Gibt es eine Formel f"ur die Anzahl der Primzahlzwillinge pro Intervall?
  \item Der Beweis oben zu der Funktion  $f: N \rightarrow Z$ mit $n \mapsto a^{3^n}$  garantiert nur die Existenz einer
    solchen Zahl $a$.  Wie kann diese Zahl $a$ bestimmt werden, und wird sie einen Wert haben, so dass
    die Funktion auch von praktischem Interesse ist?
  \item Gibt es unendlich viele Mersenne-Primzahlen?\index{Primzahlen!Mersenne}\index{Mersenne Marine!Mersenne-Primzahlen}
  \item Gibt es unendlich viele Fermatsche Primzahlen?
  \item Gibt es einen Polynomialzeit-Algorithmus zur Zerlegung einer Zahl in ihre Primfaktoren (vgl.
    \cite[S. 167]{Klee1997})?
    Diese Frage kann man auf die beiden folgenden Fragestellungen aufsplitten:
  \begin{itemize}
        \item Gibt es einen Polynomialzeit-Algorithmus, der entscheidet, ob eine Zahl prim ist?
        \item Gibt es einen Polynomialzeit-Algorithmus, mit dem sich f"ur eine zusammengesetzte Zahl $n$ ein
        nicht-trivialer (d.h. von $1$ und von $n$ verschiedener) Teiler von $n$ berechnen l"asst?
        \footnote{Vergleiche auch Kapitel \ref{RSABernstein} und Kapitel \ref{NoteFactorisation}.}
        \end{itemize}
\end{itemize}
\paragraph{Weitere interessante Themen rund um Primzahlen}
Nicht betrachtet wurden in diesem Kapitel folgende eher
zahlentheoretische Themen wie Teilbarkeitsregeln, Modulo-Rechnung,
modulare Inverse, modulare Potenzen und Wurzeln, chinesischer
Restesatz, Eulersche Phi-Funktion, perfekte Zahlen.

\newpage
\subsubsection{Anzahl von Primzahlen in verschiedenen Intervallen}
\hypertarget{primhfk}{}\label{s:primhfk}

\begin{tabular}{|l|l||l|l||l|l|}\hline
\multicolumn{2}{|l||}{Zehnerintervall} & \multicolumn{2}{l||}{Hunderterintervall} & \multicolumn{2}{l|}{Tausenderintervall} \\ \hline
Intervall  &     Anzahl &    Intervall  &  Anzahl &  Intervall  &    Anzahl\\ \hline \hline
1-10     &       4     &     1-100   &     25  &     1-1.000     &    168 \\
11-20    &       4     &     101-200 &     21  &     1.001-2.000  &    135 \\
21-30    &       2     &     201-300 &     16  &     2.001-3.000  &    127  \\
31-40    &       2     &     301-400 &     16  &     3.001-4.000  &    120 \\
41-50    &       3     &     401-500 &     17  &     4.001-5.000  &    119 \\
51-60    &       2     &     501-600 &     14  &     5.001-6.000  &    114 \\
61-70    &       2     &     601-700 &     16  &     6.001-7.000  &    117 \\
71-80    &       3     &     701-800 &     14  &     7.001-8.000  &    107 \\
81-90    &       2     &     801-900 &     15  &     8.001-9.000  &    110 \\
91-100   &       1     &     901-1.000 &     14 &     9.001-10.000 &    112 \\ \hline
\end{tabular} \\ \vskip +10 pt
Weitere Intervalle: \vskip +10 pt

\begin{tabular}{|l|r|r|}\hline
Intervall & Anzahl & Durchschnittl. Anzahl pro 1000 \\ \hline
1 - 10.000          &    1.229        &    122,900 \\
1 - 100.000         &    9.592        &    95,920 \\
1 - 1.000.000       &    78.498      &    78,498 \\
1 - 10.000.000       &   664.579     &    66,458 \\
1 - 100.000.000      &   5.761.455   &    57,615 \\
1 - 1.000.000.000    &   50.847.534  &    50,848 \\
1 - 10.000.000.000   &   455.052.512 &    45,505 \\ \hline
\end{tabular}
\vskip +6 pt

\newpage
\subsubsection{Indizierung von Primzahlen ($n$-te Primzahl)}
\hypertarget{ntePrimzahl}{}\label{s:ntePrimzahl}
\begin{tabular}{|l|l|l|l|}\hline
Index   &   Genauer Wert  &       Gerundeter Wert &    Bemerkung \\
\hline \hline
1       &   2             &     2              & \\
2       &   3             &     3  &  \\
3       &   5             &     5  & \\
4       &   7             &     7 & \\
5       &   11            &     11 & \\
6       &   13            &     13 & \\
7       &   17            &     17 & \\
8       &   19            &     19 & \\
9       &   23            &     23 & \\
10      &   29            &     29 & \\
100     &   541           &     541 & \\
1000    &   7917          &     7917 & \\
664.559  &  9.999.991     &     9,99999E+06 &   Alle Primzahlen bis zu 1E+07 waren am\\
        &                 &                 &   Beginn des 20. Jahrhunderts bekannt.\\
1E+06  &    15.485.863   &      1,54859E+07 & \\
6E+06  &    104.395.301    &    1,04395E+08  & Diese Primzahl wurde 1959 entdeckt.\\
1E+07  &    179.424.673     &    1,79425E+08 & \\
1E+09  &    22.801.763.489  &    2,28018E+10 & \\
1E+12  &    29.996.224.275.833 & 2,99962E+13 & \\ \hline
\end{tabular}
\vskip +2 pt Bemerkung: Mit L"ucke wurden fr"uh sehr gro"se
Primzahlen
entdeckt.  \\
\vskip +10pt Quell-URLs:

{\href{http://www.math.Princeton.EDU/~arbooker/nthprime.html}{http://www.math.Princeton.EDU/\~{}arbooker/nthprime.html}.}

\vskip +12 pt Ausgabe der $n$-ten Primzahl

Vergleiche {\href{http://www.utm.edu/research/primes/notes/by_year.html}{\tt http://www.utm.edu/research/primes/notes/by\_year.html}.}
\vskip +7 pt


\newpage
\subsubsection{Gr"o"senordnungen / Dimensionen in der Realit"at}
\label{s:grosord}
Bei der Beschreibung kryptographischer Protokolle und Algorithmen
treten Zahlen auf, die so gro"s bzw. so klein sind, dass sie einem
intuitiven Verst"andnis nicht zug"anglich sind. Es kann daher
n"utzlich sein, Vergleichszahlen aus der uns umgebenden realen
Welt bereitzustellen, so dass man ein Gef"uhl f"ur die Sicherheit
kryptographischer Algorithmen entwickeln kann. Die angegebenen
Werte stammen teilweise aus \cite{Schwenk1996} und
\cite[S.18]{Schneier1996}.  \hypertarget{grosord}{}
\begin{tabbing}
Wahrscheinlichkeit, dass Sie  auf ihrem n"achsten Flug entf"uhrt
werden:~~ \= abcdefhijk \= \kill
Wahrscheinlichkeit, dass Sie  auf ihrem n"achsten Flug entf"uhrt werden \> $ 5,5 \cdot 10^{-6} $\> \\
Wahrscheinlichkeit f"ur 6 Richtige im Lotto \> $ 7,1 \cdot 10^{-8} $\> \\
J"ahrliche Wahrscheinlichkeit, von einem Blitz getroffen zu werden \> $ 10^{-7} $\> \\
Risiko, von einem Meteoriten erschlagen zu werden \> $ 1,6 \cdot 10^{-12} $ \> \\
------------------------------------------------------------------------------------------------------------------\\
Zeit bis zur n"achsten Eiszeit (in Jahren) \> $14000 $ \> $ (2^{14})$ \\
Zeit bis die Sonne vergl"uht (in Jahren)   \> $10^{9} $ \> $(2^{30})$ \\
Alter der Erde (in Jahren)\> $ 10^9 $ \> $(2^{30}) $  \\
Alter des Universums (in Jahren) \> $ 10^{10} $ \> $(2^{34}) $ \\
Anzahl der Atome der Erde              \> $10^{51} $ \> $ (2^{170}) $ \\
Anzahl der Atome der Sonne             \> $10^{57}$ \> $ (2^{190})$ \\
Anzahl der Atome im Universum (ohne dunkle Materie)            \> $10^{77}$  \> $ (2^{265})$ \\
Volumen des Universums (in $cm^3$)     \> $10^{84}$ \> $(2^{280})$
\end{tabbing}

\subsubsection{Spezielle Werte des Zweier- und Zehnersystems}
\begin{tabbing}
DualSystem~~~ \= \kill
Dualsystem \> Zehnersystem \\*[4pt]
$2^{10}$ \> $1.024$ \\
$2^{40}$ \> $1,09951\cdot 10^{12}$ \\
$2^{56}$ \> $7,20576\cdot 10^{16}$ \\
$2^{64}$ \> $1,84467\cdot 10^{19}$ \\
$2^{80}$ \> $1,20893\cdot 10^{24}$ \\
$2^{90}$ \> $1,23794\cdot 10^{27}$ \\
$2^{112}$ \>    $5,19230\cdot 10^{33}$ \\
$2^{128}$ \>    $3,40282\cdot 10^{38}$ \\
$2^{150}$ \>    $1,42725\cdot 10^{45}$ \\
$2^{160}$ \>    $1,46150\cdot 10^{48}$ \\
$2^{250}$ \>    $1,80925\cdot 10^{75}$ \\
$2^{256}$ \>    $1,15792\cdot 10^{77}$ \\
$2^{320}$ \>    $2,13599\cdot 10^{96}$ \\
$2^{512}$ \>    $1,34078\cdot 10^{154}$ \\
$2^{768}$ \>    $1,55252\cdot 10^{231}$ \\
$2^{1024}$ \>   $1,79769\cdot 10^{308}$ \\
$2^{2048}$ \>   $3,23170\cdot 10^{616}$ \\
\end{tabbing}

\vskip -20 pt

Berechnung zum Beispiel per GMP:
{\href{http://www.gnu.ai.mit.edu}{\tt http://www.gnu.ai.mit.edu}}.

\newpage
\begin{thebibliography}{99999}
\addcontentsline{toc}{subsection}{Literaturverzeichnis}
\bibitem[Bartholome1996]{2Bartholome1996}  \index{Bartholome 1996}
    A. Bartholom\'e, J. Rung, H. Kern, \\
    Zahlentheorie f"ur Einsteiger, Vieweg 1995, 2. Auflage 1996.

\bibitem[Blum1999]{Blum1999} \index{Blum 1999}
    W. Blum, \\
    Die Grammatik der Logik, dtv, 1999.

\bibitem[Doxiadis2000]{Dioxadis2000}
    Apostolos Doxiadis, \\
    Onkel Petros und die Goldbachsche Vermutung, Bloomsbury 2000 (deutsch
    bei L"ubbe 2000 und bei BLT als Taschenbuch 2001).

\bibitem[Graham1989]{Graham1989} \index{Graham 1989}
    R.E. Graham, D.E. Knuth, O. Patashnik, \\
    Concrete Mathematics, Addison-Wesley, 1989.

\bibitem[Klee1997]{Klee1997} \index{Klee 1997}
    V. Klee, S. Wagon, \\
    Ungel"oste Probleme in der Zahlentheorie und der
    Geometrie der Ebene, Birkh"auser Verlag, 1997.

\bibitem[Knuth1981]{Knuth1981} \index{Knuth 1981}
    Donald E. Knuth, \\
        The Art of Computer Programming, vol 2: Seminumerical
    Algorithms, Addison-Wesley, 1969; second edition, 1981.

\bibitem[Lorenz1993]{Lorenz1993} \index{Lorenz 1993}
    F. Lorenz, \\
    Algebraische Zahlentheorie, BI Wissenschaftsverlag, 1993.

\bibitem[Padberg1996]{Padberg1996} \index{Padberg 1996}
    F. Padberg, \\
    Elementare Zahlentheorie, Spektrum Akademischer Verlag, 1996, 2. Auflage.

\bibitem[Pieper1983]{Pieper1983} \index{Pieper 1983}
    H. Pieper, \\
    Zahlen aus Primzahlen, Verlag Harri Deutsch, 1974, 3. Auflage 1983.

\bibitem[Richstein1999]{Richstein1999} \index{Richstein 1999}
    J. Richstein, \\
    Verifying the Goldbach Conjecture up to $4*10^{14},$ Mathematics of Computation. 

\bibitem[Schneier1996]{2Schneier1996} \index{Schneier 1996}
    Bruce Schneier, \\
    Applied Cryptography, Wiley and Sons, 2nd edition, 1996.

\bibitem[Schwenk1996]{Schwenk1996} \index{Schwenk 1996}
    J. Schwenk \\
    Conditional Access. In: taschenbuch der telekom praxis 1996, Hrgb. B. Seiler, Verlag Schiele und Sch"on, Berlin.

\bibitem[Tietze1973]{Tietze1973} \index{Tietze 1973}
    H. Tietze, \\
    Gel"oste und ungel"oste mathematische Probleme, Verlag C.H. Beck,
    1959, 6. Auflage 1973.

\end{thebibliography}

\newpage

\section*{Web-Links}\addcontentsline{toc}{subsection}{Web-Links}

\begin{enumerate}
   \item \href{http://www.utm.edu/}
              {\tt http://www.utm.edu/ }
   \item \href{http://prothsearch.net/index.html}
              {\tt http://prothsearch.net/index.html}
   \item \href{http://www.mersenne.org/prime.htm}
              {\tt http://www.mersenne.org/prime.htm }
   \item \href{http://reality.sgi.com/chongo/prime/prime_press.html}
              {\tt http://reality.sgi.com/chongo/prime/prime\_press.html }
   \item \href{http://www.eff.org/coop-awards/prime-release1.html}
              {\tt http://www.eff.org/coop-awards/prime-release1.html }
   \item \href{http://www.informatik.tu-darmstadt.de/TI/LiDIA/}
              {\tt http://www.informatik.tu-darmstadt.de/TI/LiDIA/ }
   \item \href{http://www.math.Princeton.EDU/\~{}arbooker/nthprime.html}
              {\tt http://www.math.Princeton.EDU/\~{}arbooker/nthprime.html }
   \item \href{http://www.utm.edu/research/primes/notes/by_year.html}
              {\tt http://www.utm.edu/research/primes/notes/by\_year.html }
   \item \href{http://www.cerias.purdue.edu/homes/ssw/cun}
              {\tt http://www.cerias.purdue.edu/homes/ssw/cun }
   \item \href{http://www.informatik.uni-giessen.de/staff/richstein/de/Goldbach.html}
              {\tt http://www.informatik.uni-giessen.de/staff/richstein/de/Goldbach.html}
   \item \href{http://www.mathematik.ch/mathematiker/goedel.html}
              {\tt http://www.mathematik.ch/mathematiker/goedel.html}
   \item \href{http://www.mscs.dal.ca/~dilcher/golbach/index.html}
              {\tt http://www.mscs.dal.ca/\~{}dilcher/goldbach/index.html}
\end{enumerate}

\subsection*{Dank} \addcontentsline{toc}{subsection}{Dank}
F"ur das konstruktive Korrekturlesen dieses Artikels: Hr.
Henrik Koy und Hr. Roger Oyono.

% Local Variables:
% TeX-master: "../script-de.tex"
% End:
