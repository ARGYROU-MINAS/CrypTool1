% .........................................................................................
%                V E R S C H L U E S S E L U N G S V E R F A H R E N
% ~~~~~~~~~~~~~~~~~~~~~~~~~~~~~~~~~~~~~~~~~~~~~~~~~~~~~~~~~~~~~~~~~~~~~~~~~~~~~~~~~~~~~~~~~
\newpage
\section{Verschl"usselungsverfahren}

\subsection{Verschl"usselung}

Sinn der Verschl"usselung  \index{Verschl""usselung} ist es, Daten so zu ver"andern, dass nur ein autorisierter Empf"an\-ger
in der Lage ist,
den Klartext zu rekonstruieren. Das hat den Vorteil, dass verschl"usselte Daten offen "ubertragen werden
k"onnen und trotzdem keine Gefahr besteht, dass ein Angreifer die Daten unberechtigterweise lesen kann. Der
autorisierte Empf"anger ist im Besitz einer geheimen Information, des sogenannten Schl"ussels, die
es ihm erlaubt, die Daten zu entschl"usseln, w"ahrend sie jedem anderen verborgen bleiben.\par \vskip + 3pt

Es gibt ein beweisbar sicheres Verschl"usselungsverfahren, das \index{One-�Time-�Pad} {\em One--Time--Pad}. Dieses weist allerdings einige
praktische Nachteile auf (der verwendete Schl"ussel mu"s zuf"allig gew"ahlt werden und genauso lang wie die zu
sch"utzende Nachricht sein), so dass es au"ser in geschlossenen Umgebungen, zum Beispiel beim hei"sen Draht
zwischen Moskau und Washington, kaum eine Rolle spielt.\par \vskip + 3pt

F"ur alle anderen Verfahren gibt es (theoretische) M"oglichkeiten, sie zu brechen. Bei guten Verfahren sind
diese jedoch so aufwendig, dass sie praktisch nicht durchf"uhrbar sind und diese Verfahren als (praktisch)
sicher angesehen werden k"onnen.\par \vskip + 3pt

Grunds"atzlich unterscheidet man zwischen symmetrischen und asymmetrischen Verfahren zur Verschl"usselung.

\subsubsection{Symmetrische Verschl"usselung}

Bei der {\em symmetrischen} Verschl"usselung \index{Verschl""usselung!symmetrisch} m"ussen Sender und Empf"anger "uber einen
gemeinsamen (geheimen)
Schl"ussel verf"ugen, den sie vor Beginn der eigentlichen Kommunikation ausgetauscht haben. Der Sender
benutzt diesen Schl"ussel, um die Nachricht zu verschl"usseln und der Empf"anger, um diese zu entschl"usseln.\par \vskip + 3pt

Vorteile von symmetrischen Algorithmen sind die hohe
Geschwindigkeit, mit denen Daten ver- und entschl"usselt werden.
Ein Nachteil ist das Schl"usselmanagement. Um miteinander
vertraulich kommunizieren zu k"onnen, m"ussen Sender und
Empf"anger vor Beginn der eigentlichen Kommunikation "uber einen
sicheren Kanal einen Schl"ussel ausgetauscht haben. Spontane
Kommunikation zwischen Personen, die sich vorher noch nie begegnet
sind, scheint so nahezu unm"oglich. Soll in einem Netz mit $ n $
Teilnehmern jeder mit jedem zu jeder Zeit spontan kommunizieren
k"onnen, so mu"s jeder Teilnehmer vorher mit jedem anderen der
$n-� 1$ Teilnehmer einen Schl"ussel ausgetauscht haben. Insgesamt
m"ussen also $n(n - 1)/2$ Schl"ussel ausgetauscht werden.\par \vskip + 3pt

Das bekannteste symmetrische Verschl"usselungsverfahren ist der \index{DES} DES--Algorithmus.
Der DES--Algorithmus ist eine Entwicklung von IBM in Zusammenarbeit mit der National Security Agency \index{NSA} (NSA). Er
wurde 1975 als Standard ver"offentlicht. Trotz seines relativ hohen Alters ist jedoch bis heute kein effektiver
Angriff auf ihn gefunden worden. Der effektivste Angriff besteht aus dem Durchprobieren aller m"oglichen
Schl"ussel, bis der richtige gefunden wird ({\em brute--force--attack})\index{Angriff!Brute-force}. Aufgrund der relativ kurzen
Schl"ussell"ange von effektiv 56 Bits (64 Bits, die allerdings 8 Parit"atsbits enthalten),
sind in der Vergangenheit schon mehrfach mit dem DES verschl"usselte Nachrichten gebrochen worden, so dass
er heute als nur noch bedingt sicher anzusehen ist. Symmetrische Alternativen zum DES sind zum Beispiel die Algorithmen
IDEA \index{IDEA} oder Triple--DES.\par \vskip + 3pt

Hohe Aktualit"at besitzen die symmetrischen AES-Verfahren. \index{AES} Das
dazu geh"orende Rijndael Verfahren wurde am 2. Oktober 2000 zum
Gewinner des AES-Ausschreibens erkl"art und ist damit Nachfolger
des DES-Verfahrens.

\subsubsection{Asymmetrische Verschl"usselung}

Bei der {\em asymmetrischen} Verschl"usselung \index{Verschl""usselung!asymmetrisch} hat jeder Teilnehmer ein
pers"onliches Schl"us\-selpaar, das aus einem
{\em geheimen} \index{Schl""ussel!geheim} und einem {\em "offentlichen} Schl"ussel \index{Schl""ussel!""offentlich} besteht. Der "offentliche Schl"ussel wird, der Name deutet es an,
"offentlich bekanntgemacht, zum Beispiel in einem Schl"usselverzeichnis im Internet.\par \vskip + 3pt

M"ochte Alice mit Bob kommunizieren, so sucht sie Bobs "offentlichen Schl"ussel aus dem Verzeichnis und benutzt
ihn, um ihre Nachricht an ihn zu verschl"usseln. Diesen verschl"usselten Text schickt sie dann an Bob, der mit
Hilfe seines geheimen Schl"ussels den Text wieder entschl"usseln kann. Da einzig Bob Kenntnis von seinem
geheimen Schl"ussel hat, ist auch nur er in der Lage, an ihn adressierte Nachrichten zu entschl"usseln.
Selbst Alice als Absenderin der Nachricht kann aus der von ihr versandten (verschl"usselten) Nachricht den
Klartext nicht wieder herstellen. Nat"urlich mu"s sichergestellt sein, dass man aus dem "offentlichen
Schl"ussel nicht auf den geheimen Schl"ussel schlie"sen kann.\par \vskip + 3pt

Veranschaulichen kann man sich ein solches Verfahren mit einer
Reihe von einbruchssicheren Briefk"asten. Wenn ich eine Nachricht
verfa"st habe, so suche ich den Briefkasten mit dem Namensschild
des Empf"angers und werfe den Brief dort ein. Danach kann ich die
Nachricht selbst nicht mehr lesen oder ver"andern, da nur der
legitime Empf"anger im Besitz des Schl"ussels f"ur den Briefkasten
ist.\par \vskip + 3pt

Vorteil von asymmetrischen Verfahren ist das einfache
\index{Schl""usselmanagement} Schl"usselmanagement. Betrachten wir
wieder ein Netz mit $n$ Teilnehmern. Um sicherzustellen, dass
jeder Teilnehmer jederzeit eine verschl"usselte Verbindung zu
jedem anderen Teilnehmer aufbauen kann, mu"s jeder Teilnehmer ein
Schl"usselpaar besitzen. Man braucht also $2n$ Schl"ussel oder $n$
Schl"usselpaare. Ferner ist im Vorfeld einer "Ubertragung kein
sicherer Kanal notwendig, da alle Informationen, die zur Aufnahme
einer vertraulichen Kommunikation notwendig sind, offen
"ubertragen werden k"onnen. Hier ist lediglich auf die
Unverf"alschtheit (Integrit"at und Authentizit"at)
\index{Authentizit""at} des "offentlichen Schl"ussels zu achten.
Nachteil: Im Vergleich zu symmetrischen Verfahren sind reine
asymmetrische Verfahren jedoch um ein Vielfaches langsamer.\par \vskip + 3pt

Das bekannteste asymmetrische Verfahren ist der \index{RSA} RSA--Algorithmus,
der nach seinen Entwicklern Ronald \index{Rivest Ronald} Rivest, Adi \index{Shamir Adi} Shamir und Leonard \index{Adleman Leonard} Adleman benannt wurde. Der
RSA--Algorithmus wurde 1978 ver"offentlicht. Das Konzept der asymmetrischen Verschl"usselung wurde erstmals
von Whitfield Diffie \index{Diffie Whitfield}  und Martin \index{Hellman Martin} Hellman in Jahre 1976 vorgestellt. Heute spielen auch die Verfahren nach
ElGamal \index{ElGamal} eine bedeutende Rolle, vor allem die
\index{Schnorr} Schnorr--Varianten im \index{DSA} DSA (Digital
\index{Signatur!digitale}\index{DSA-Signatur}\index{Signatur!DSA} Signature Algorithm).

\subsubsection{Hybridverfahren}
\index{Hybridverfahren}
Um die Vorteile von symmetrischen und asymmetrischen Techniken gemeinsam nutzen zu k"onnen, werden
(zur Verschl"usselung) in der Praxis meist Hybridverfahren verwendet.\par \vskip + 3pt

Hier werden die Daten mittels symmetrischer Verfahren
verschl"usselt: der Schl"ussel ist ein vom Absender zuf"allig
generierter Sitzungsschl"ussel (session key)\index{Session Key}, der nur f"ur diese
Nachricht verwendet wird. Anschlie"send wird dieser
Sitzungsschl"ussel mit Hilfe des asymmetrischen Verfahrens
verschl"usselt und zusammen mit der Nachricht an den Empf"anger
"ubertragen. Der Empf"anger kann den Sitzungsschl"ussel mit Hilfe
seines geheimen Schl"ussels bestimmen und mit diesem dann die
Nachricht entschl"usseln. Auf diese Weise nutzt man das bequeme
Schl"usselmanagement \index{Schl""usselmanagement} asymmetrischer
Verfahren und kann trotzdem gro"se Datenmengen schnell und effektiv mit
symmetrischen Verfahren verschl"usseln.

\subsubsection{Weitere Details}

Neben Informationen in der umfangreichen Fachliteratur und im Internet enth"alt auch die 
Online-Hilfe von CrypTool sehr viele details zu den einzelnen symmetrischen und asymmetrischen 
Verschl"usselungsverfahren.

\begin{thebibliography}{99999}
\addcontentsline{toc}{subsection}{Literatur}
\bibitem[Schmeh2001]{Schmeh2001}  \index{Schmeh 2001}
        Klaus Schmeh, \\
        Kryptografie und Public-Key Infrastrukturen im Internet, dpunkt.verlag, 2. akt. und erw. Auflage 2001. \\
        Sehr gut lesbares, hoch aktuelles und umfangreiches Buch "uber Kryptographie. Geht auch auf die praktischen
        Probleme, wie Standardisierung oder real existierende Software, ein.
        Bisher nur in deutsch erschienen.

\end{thebibliography}

