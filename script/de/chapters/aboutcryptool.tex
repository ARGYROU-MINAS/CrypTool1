% $Id$
% ............................................................................
%                 TEXT DER 2. SEITE (�berblick)
%
% ~~~~~~~~~~~~~~~~~~~~~~~~~~~~~~~~~~~~~~~~~~~~~~~~~~~~~~~~~~~~~~~~~~~~~~~~~~~~

% HACK to fix warning "destination with the same identifier .. has already been used, ...":
% REMOVED, because it caused missing index entries
%\makeatletter \renewcommand{\thepage}{~\csname @roman\endcsname \c@page~} \makeatother
\clearpage\phantomsection
% HACK to fix warning "destination with the same identifier .. has already been used, ...":
% \makeatletter \renewcommand{\thepage}{\csname @roman\endcsname \c@page} \makeatother
% REMOVED, because it caused missing index entries
% BE_2016-03-16: Nahm dies ohne ~ nach "page" raus. Scheint nichts zu �ndern?

\addcontentsline{toc}{chapter}{�berblick �ber den Inhalt des CrypTool-Buchs}
\chapter*{�berblick �ber den Inhalt des CrypTool-Buchs}

\parskip 4pt
%\vskip +12 pt
Der Erfolg des Internets hat zu einer verst�rkten 
Forschung der damit verbundenen Technologien gef�hrt, was auch im
Bereich Kryptographie viele neue Erkenntnisse schaffte.

In diesem {\em Buch zu den CrypTool-Programmen} \index{CrypTool} finden Sie
eher mathematisch orientierte Informationen zum Einsatz von
kryptographischen Verfahren. Zu einigen Verfahren gibt es Beispielcode,
geschrieben f�r das Computer-Algebra-System {\bf SageMath}\index{SageMath}
(siehe Anhang~\ref{s:appendix-using-sage}).
Die Hauptkapitel sind von verschiedenen {\bf Autoren} verfasst
(siehe Anhang~\ref{s:appendix-authors}) %\hyperlink{appendix-authors}{Autoren}
und in sich abgeschlossen. Am Ende der meisten Kapitel finden Sie
Literaturangaben und Web-Links.
Die Kapitel wurden reichlich mit {\em Fu�noten} versehen, in denen auch darauf verwiesen
wird, wie man die beschriebenen Funktionen in den verschiedenen CrypTool-Programmen 
aufruft.

Das \hyperlink{Kapitel_1}{erste Kapitel} beschreibt die Prinzipien der
symmetrischen und asymmetrischen {\bf Verschl�sselung} und Definitionen f�r
deren Widerstandsf�higkeit.

Im \hyperlink{Kapitel_PaperandPencil}{zweiten Kapitel} wird -- aus 
didaktischen Gr�nden -- eine ausf�hrliche �bersicht
�ber {\bf Papier- und Bleistiftverfahren} gegeben.

Ein gro�er Teil des Buchs ist dem faszinierenden Thema der
{\bf Primzahlen} (Kapitel \ref{Label_Kapitel_Primes})
% \hyperlink{Kapitel_2}{{\bf Primzahlen}} 
gewidmet.
Anhand vieler Beispiele wird in die {\bf modulare Arithmetik} und die 
{\bf elementare Zahlentheorie} (Kapitel \ref{Chapter_ElementaryNT})
% \hyperlink{Chapter_ElementaryNT} {{\bf elementare Zahlentheorie}} 
eingef�hrt. Die Eigenschaften des {\bf RSA-Verfahrens} bilden einen Schwerpunkt. 

Danach erhalten Sie Einblicke in die mathematischen Konzepte und
Ideen hinter der
{\bf modernen Kryptographie} (Kapitel \ref{Chapter_ModernCryptography}).
%\hyperlink{Chapter_ModernCryptography}{{\bf modernen Kryptographie}}.

%Ein \hyperlink{Chapter_Hashes-and-Digital-Signatures}{weiteres Kapitel}
Kapitel \ref{Chapter_Hashes-and-Digital-Signatures} gibt
einen �berblick zum Stand der Attacken gegen moderne {\bf Hashalgorithmen}
und widmet sich dann kurz den {\bf digitalen Signaturen}
 --- sie sind unverzichtbarer Bestandteil von E-Business-Anwendungen.

%Das \hyperlink{ellcurve}{letzte Kapitel} stellt {\bf Elliptische Kurven} vor:
Kapitel \ref{Chapter_EllipticCurves} stellt {\bf Elliptische Kurven} vor:
Sie sind eine Alternative zu RSA und f�r die Implementierung auf Chipkarten
besonders gut geeignet.

Kapitel \ref{Chapter_BitCiphers} f�hrt in die {\bf Boolesche Algebra} ein.
Boolesche Algebra ist Grundlage der meisten modernen, symmetrischen
Verschl�sselungsverfahren, da diese auf Bitstr�men und Bitbl�cken operieren.
Prinzipielle Konstruktionsmethoden dieser Verfahren werden beschrieben
und in SagMath implementiert.


Kapitel \ref{Chapter_HomomorphicCiphers} stellt {\bf Homomorphe Kryptofunktionen}
vor: Sie sind ein modernes Forschungsgebiet, das insbesondere im Zuge des
Cloud-Computing an Bedeutung gewann.

Das \hyperlink{Chapter_Crypto2020}{letzte Kapitel} {\bf Krypto 2020}
diskutiert Bedrohungen f�r bestehende kryptographische Verfahren und 
stellt alternative Forschungsans�tze f�r eine langfristige
kryptographische Sicherheit vor.

W�hrend die CrypTool-\textit{eLearning-Programme}\index{eLearning} eher den
praktischen Umgang motivieren und vermitteln, dient das \textit{Buch} dazu, 
dem an Kryptographie Interessierten ein tieferes Verst�ndnis f�r die 
implementierten mathematischen Algorithmen zu vermitteln -- und das 
didaktisch m�g"-lichst gut nachvollziehbar. 

Die {\bf Anh�nge}
\ref{s:appendix-menu-overview-CT1},
\ref{s:appendix-template-overview-CT2},
\ref{s:appendix-function-overview-JCT} und
\ref{s:appendix-function-overview-CTO}
erlauben einen schnellen �berblick �ber die Funktionen in den verschiedenen
CrypTool-Varianten\index{CrypTool 1}\index{CrypTool 2}\index{JCrypTool} via:
\begin{itemize}
  \item der Funktionsliste und 
        dem \hyperlink{appendix-menu-overview-CT1}
                      {Men�baum von CrypTool 1 (CT1)},
  \item der Funktionsliste und 
        den \hyperlink{appendix-template-overview-CT2}
                      {Vorlagen in CrypTool 2 (CT2)},
  \item der \hyperlink{appendix-function-overview-JCT}
                      {Funktionsliste von JCrypTool (JCT)}, und
  \item der \hyperlink{appendix-function-overview-CTO}
                      {Funktionsliste von CrypTool-Online (CTO)}.
\end{itemize}
%Anker sind \hypertarget{appendix-menutree}{} und \label{s:appendix-menutree}
%  - \hyperlink{}{} legt Link auf die Seite unter den Text in 2. Klammer
%  - \ref{} legt Link und f�gt Kapitelnummer ein.

% Bernhard Esslinger, Matthias B�ger, Bartol Filipovic, Henrik Koy, Roger Oyono
% und J�rg Cornelius Schneider
Die Autoren m�chten sich an dieser Stelle bedanken bei den Kollegen 
in der jeweiligen Firma und an den Universit�ten Bochum, Darmstadt, Frankfurt,
Gie�en, Karlsruhe und Siegen.

\enlargethispage{0.5cm}
Wie auch bei dem E-Learning-Programm CrypTool\index{CrypTool} w�chst 
die Qualit�t des Buchs mit den Anregungen und Verbesserungsvorschl�gen 
von Ihnen als Leser. Wir freuen uns �ber Ihre R�ck"-mel"-dung.



% Local Variables:
% TeX-master: "../script-de.tex"
% End:
