% $Id$
% ............................................................................
%                 TEXT DER 2. SEITE
% ~~~~~~~~~~~~~~~~~~~~~~~~~~~~~~~~~~~~~~~~~~~~~~~~~~~~~~~~~~~~~~~~~~~~~~~~~~~~

% --------------------------------------------------------------------------
\addcontentsline{toc}{chapter}{�berblick}  % Hier steht im Inhaltsverzeichnis
                    % die richtige Seitenzahl und beim Klicken springt es an
                    % die richtige Seite; aber alles was nach dem 
                    % Inhaltsverzeichnis hat ein Problem! be_2006 
\chapter*{�berblick �ber den Inhalt des CrypTool-Skripts}

\parskip 4pt
%\vskip +12 pt
{
In diesem {\em Skript zu dem Programm CrypTool} \index{CrypTool} finden Sie
eher mathematisch orientierte Informationen zum Einsatz von
kryptographischen Verfahren. Zu einigen Verfahren gibt es Beispielcode,
geschrieben f�r das CAS (Computer-Algebra-System) Sage.
Die Hauptkapitel sind von verschiedenen {\bf Autoren} verfasst
(siehe Anhang \ref{s:appendix-authors}) %\hyperlink{appendix-authors}{Autoren}
und in sich abgeschlossen. Am Ende der meisten Kapitel finden Sie
jeweils Literaturangaben und Web-Links.

Das \hyperlink{Kapitel_1}{erste Kapitel} beschreibt die Prinzipien der
symmetrischen und asymmetrischen {\bf Verschl�sselung} und erl�utert 
kurz die aktuellen Entschl�sselungsrekorde bei modernen symmetrischen
Verfahren.

Im \hyperlink{Kapitel_PaperandPencil}{zweiten Kapitel} wird -- aus 
didaktischen Gr�nden -- eine ausf�hrliche �bersicht
�ber {\bf Papier- und Bleistiftverfahren} gegeben.

Ein gro�er Teil des Skripts ist dem faszinierenden Thema der
{\bf Primzahlen} (Kapitel \ref{Label_Kapitel_2})
% \hyperlink{Kapitel_2}{{\bf Primzahlen}} 
gewidmet.
Anhand vieler Beispiele bis hin zum {\bf RSA-Verfahren} wird in die
{\bf modulare Arithmetik} und die 
{\bf elementare Zahlentheorie} (Kapitel \ref{Chapter_ElementaryNT})
% \hyperlink{Chapter_ElementaryNT} {{\bf elementare Zahlentheorie}} 
eingef�hrt.

Danach erhalten Sie Einblicke in die mathematischen Konzepte und
Ideen hinter der
{\bf modernen Kryptographie} (Kapitel \ref{Chapter_ModernCryptography}).
%\hyperlink{Chapter_ModernCryptography}{{\bf modernen Kryptographie}}.

%Ein \hyperlink{Chapter_Hashes-and-Digital-Signatures}{weiteres Kapitel}
Kapitel \ref{Chapter_Hashes-and-Digital-Signatures} gibt
einen �berblick zum Stand der Attacken gegen moderne {\bf Hashalgorithmen}
und widmet sich dann kurz den {\bf digitalen Signaturen}
 --- sie sind unverzichtbarer Bestandteil von E-Business-Anwendungen.

%Das \hyperlink{ellcurve}{letzte Kapitel} stellt {\bf Elliptische Kurven} vor:
Kapitel \ref{Chapter_EllipticCurves} stellt {\bf Elliptische Kurven} vor:
Sie sind eine Alternative zu RSA und f�r die Implementierung auf Chipkarten
besonders gut geeignet.

Das \hyperlink{Chapter_Crypto2020}{letzte Kapitel} {\bf Krypto 2020}
diskutiert Bedrohungen f�r bestehende kryptographische Verfahren und 
stellt alternative Forschungsans�tze f�r eine langfristige
kryptografische Sicherheit vor.

W�hrend das \textit{eLearning-Programm} CrypTool\index{CrypTool} eher den
praktischen Umgang motiviert und vermittelt, dient das \textit{Skript} dazu, 
dem an Kryptographie Interessierten ein tieferes Verst�ndnis f�r die 
implementierten mathematischen Algorithmen zu vermitteln -- und das 
didaktisch m�g"-lichst gut nachvollziehbar. 
Wer sich schon etwas mit der Materie auskennt, kann mit dem 
{\bf Men�baum} (siehe Anhang \ref{s:appendix-menutree})
einen schnellen �berblick �ber die Funktionen in CrypTool gewinnen.

% Bernhard Esslinger, Matthias B�ger, Bartol Filipovic, Henrik Koy, Roger Oyono
% und J�rg Cornelius Schneider
Die Autoren m�chten sich an dieser Stelle bedanken bei den Kollegen 
in der Firma und an den Universit�ten Frankfurt, Gie�en, Siegen,
Karlsruhe und Darmstadt.

\enlargethispage{0.5cm}
Wie auch bei dem E-Learning-Programm CrypTool\index{CrypTool} w�chst 
die Qualit�t des Skripts mit den Anregungen und Verbesserungsvorschl�gen 
von Ihnen als Leser. Wir freuen uns �ber Ihre R�ck"-mel"-dung.



% Local Variables:
% TeX-master: "../script-de.tex"
% End:
