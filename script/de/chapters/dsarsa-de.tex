Einige digitale Signaturverfahren basieren auf asymmetrischer
Verschl�sselung, das bekannteste Beispiel dieser Gattung ist RSA. F�r die
RSA-Signatur verwendet man die gleiche mathematische Operation wie zum
Entschl�sseln, nur wird sie auf den Hash-Wert des zu unterschreibenden
Dokuments angewendet.

Andere Systeme der digitalen Signatur wurden, wie DSA (Digital Signature
Algorithm), ausschliesslich zu diesem Zweck entwickelt, und stehen in
keiner direkten Verbindung zu einem entsprechenden
Verschl�sselungsverfahren.

Beide Signaturverfahren, RSA und DSA, werden in den folgenden beiden
Abschnitten n�her beleuchtet. Anschliessend gehen wir einen Schritt weiter
und zeigen, wie basierend auf der elektronischen Unterschrift das digitael
Pendent zum Personalausweis entwickelt wurde. Dies Verfahren nennt man
Public Key Zertifizierung.

\subsection{RSA Signatur}
\index{Signatur!digital}
\index{RSA Signatur}

\def\Mod#1{\ (\mbox{mod }#1)}

Wie im Kommentar am Ende von \hyperlink{RSAproof}{Abschnitt
  \ref{RSAproof}} bemerkt, ist es m�glich, die RSA Operationen mit dem
privaten und �ffentlichen Schl�ssel in umgekehrter Reihenfolge auszuf�hren,
d.~h.\ $M$ hoch $d$ hoch $e \Mod{N}$ ergibt wieder $M$. Wegen dieser
simplen Tatsache ist es m�glich, RSA als Signaturverferfahren zu
verwenden. 

Eine RSA Signatur $S$ zur die Nachricht $M$ wird durch folgende Operation
mit dem privaten Schl�ssel erzeugt:
$$ S \equiv M^d \Mod{N} $$
Zur Verifikation wird die korrespondierende Public-Key-Operation auf der
Signatur $S$ ausgef�hrt und das Ergebnis mit der Nachricht $M$ verglichen:
$$
S^e \equiv (M^d)^e \equiv (M^e)^d \equiv M \Mod{N}$$
Wenn das Ergebnis
$S^e$ mit der Nachricht $M$ �bereinstimmt, dann akzeptiert der Pr�fer die
Signatur, andernfalls ist die Nachrricht entweder ver�ndert worden, oder
sie wurde nie vom Inhaber von $d$ unterschrieben.

Wie weiter oben erkl�rt, werden Signaturen in der Praxis nie direkt auf der
Nachricht ausf�hrt, sondern auf einem kryptograhischen Hashwert davon. Um
verschiedene Attacken auf das Signaturverfahren (und seine Kombination mit
Verschl�sselung) auszuschliessen, ist es n�tig, den Hashwert vor der
Exponentiation auf spezielle Weise zu formatieren, wie in PKCS\#1 (Public
Key Cryptography Standard \#1 \cite{PKCS1}) beschrieben. Der Tatsache, dass
dieser Standard nach mehreren Jahren Einsatz revidiert werden musste, kann
als Beispiel daf�r dienen, wie schwer es ist, kryptographische Details
richtig hinzubekommen.

\subsection{DSA Signatur}
\index{Signatures!digital}
\index{DSA Signatur}

Im August 1991 hat das U.S. National Institute of Standards and Technology
(NIST) einen digitalen Signaturalgorithmus (DSA, Digital Signature
Algorithm) vorgestellt, der sp�ter zum U.S. Federal Information Processing
Standard (FIPS 186 \cite{FIPS186}) wurde.

Der Algorithmus ist eine Variante des ElGamal Verfahrens. Seine Sicherheit
beruhrt auf dem Diskreten Logarithmus Problem\index{Logarithm
  Problem!diskret}. Die Bestandteile des privaten und �ffentlichen DSA
Schl�ssels, sowie die Verfahren zur Signatur und Verifikation, sind im
Folgenden zusammengefasst.


\paragraph{�ffentlicher Schl�ssel}\strut\\
\begin{tabular}{l@{ }l}
$p$ & prim \\
$q$ & 160bit Primfaktor von $p - 1$ \\
$g$ & $ = h^{(p-1)/q}  \mbox{ mod } p$, wobei $h < p - 1$ und
$h^{(p-1)/q} > 1  \Mod{p}$ \\
$y$ & $\strut \equiv  g^x  \mbox{ mod } p$ 
\end{tabular}

\emph{Bemerkung:} Die Parameter $p,q$ und $g$ k�nnen von einer Gruppe von
Benutzern gemeinsam genutzt werden.

\paragraph{Privater Schl�ssel}\strut\\
\begin{tabular}{l@{ }l}
$x < q$ (160bit Zahl) 
\end{tabular}

\paragraph{Signatur}\strut\\
\begin{tabular}{l@{ }l}
$m$ & zu signierende Nachricht\\
$k$ & zuf�llig gew�hlte Primzahl, kleiner als $q$\\
$r$ & $= (g^k \mbox{ mod } p) \mbox{ mod } q$\\
$s$ & $= (k^{-1}(\mbox{SHA--1}(m) + xr)) \mbox{ mod } q$
\end{tabular}

\emph{Bemerkung:}
\begin{itemize}
\item $(s,r)$ ist die Signtur.
\item Die Sicherheit der Signatur h�ngt nicht nur von der Mathematik ab,
  sondern auch von der Verf�gbarkeit einer guten Zufallsquelle f�r $k$.
\item SHA--1 \index{SHA--1} ist eine in FIPS186 spezifizierte 160bit Hashfunktion.

\end{itemize}
\paragraph{Verifikation}\strut\\
\begin{tabular}{l@{ }l}
$w$ & $= s^{-1}  \mbox{ mod } q$\\
$u_1$ & $= (\mbox{SHA--1}(m)w) \mbox{ mod } q$\\
$u_2$ & $= (rw)  \mbox{ mod } q$\\
$v$ & $= (g^{u_1}y^{u_2}) \mbox{ mod } p)  \mbox{ mod } q$\\

\end{tabular}

\emph{Bemerkung:} Wenn $v = r$, dann ist die Signatur g�ltig.

Obwohl DSA unabh�ngig von einem Verschl�sselungsverfahren so spezifiziert
wurde, dass es aus L�nder exportiert werden kann, die den Export von
kryptographischer Hard- und Software einschr�nken, wie die USA zum
Zeitpunkt der Spezifikation, wurde festgestellt
\cite[p.~490]{5Schneier1996}, dass die Operationen des DSA dazu geeignet
sind, nach RSA bzw. ElGamal zu verschl�sseln.


