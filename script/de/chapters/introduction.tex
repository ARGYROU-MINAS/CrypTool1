% $Id$
% !Mode:: "TeX:DE"    % Setting document mode and submode for WinEdt
% ............................................................................
%      V O R W O R T  und  E I N F U E H R U N G  (Zusammenspiel CT-Buch und CT-Programme)
%
% ~~~~~~~~~~~~~~~~~~~~~~~~~~~~~~~~~~~~~~~~~~~~~~~~~~~~~~~~~~~~~~~~~~~~~~~~~~~~

\clearpage
%\setcounter{secnumdepth}{-1}  % Prevent this chapter title from having a number
\chapter*{Vorwort zur 12. Auflage des CrypTool-Buchs}
\setcounter{secnumdepth}{4}  % Set back default from CT-Book-de.tex (show numbers till level 4)

Das CrypTool-Buch versucht, einzelne Themen aus der Mathematik der Kryptologie
genau und trotzdem möglichst verständlich zu erläutern.

Dieses Buch wurde ab dem Jahr 2000 -- zusammen mit dem CrypTool-1-Paket
(CT1)\index{CT1} in Version 1.2.01 -- ausgeliefert.
Seitdem ist das Buch mit fast jeder neuen Version von CT1 und CT2 ebenfalls erweitert
und aktualisiert worden.

Themen aus Mathematik und Kryptographie wurden sinnvoll unterteilt und dafür
wurden jeweils eigenständig lesbare Kapitel geschrieben, damit
Entwickler/Autoren unabhängig voneinander mitarbeiten können. Natürlich gäbe
es viel mehr Themen aus der Kryptographie, die man vertiefen könnte -- deshalb
ist diese Auswahl auch nur eine von vielen möglichen.

In der anschließenden redaktionellen Arbeit wurden in LaTeX Querverweise ergänzt,
Fußnoten hinzugefügt, Index-Einträge vereinheitlicht und Korrekturen vorgenommen.

%In dieser Ausgabe des CT-Buchs wurden wieder einige Themen auf den
%aktuellen Stand gebracht:
Im Vergleich zu Ausgabe 11 des Buchs wurden in dieser Ausgabe
die TeX-Sourcen des Dokuments komplett überarbei"-tet (bspw. eine einzige
bibtex-Datei für alle Kapitel und beide Sprachen), und
etliche Themen ergänzt, korrigiert und auf den aktuellen Stand gebracht, z.B.:
\vspace{-5pt}

%%%%%%%%%%%%%BE_2016Jun: %% be_todo BEGIN
%% myNewSt = strrep(mySt,sprintf('\n'),'');
%% be_todo -- TUT noch nicht:
%% https://viaverdepaisagismo.com.br/index.cgi/ar/00/http/tex.stackexchange.com/questions/296643/how-to-use-nameref-with-xstring-package-to-check-string-length
%% http://tex.stackexchange.com/questions/296643/how-to-use-nameref-with-xstring-package-to-check-string-length/296648
%\StrLen{\getnamereftext{testLabel}}[\testStrNumOfChars1] yyy
%Number of characters in string1 = \testStrNumOfChars1
%\StrLen{\getnamereftext{\nameref{Chapter_Dlog-FactoringDead}}}[\testStrNumOfChars2]
%Number of characters in string2 = \testStrNumOfChars2

%\makeatletter
%\edef\@currentlabelname{This is a somewhat longxxx string}
%\label{testLabel}
%\makeatother
%\StrLen{\getnamereftext{testLabel}}[\testStrNumOfChars1] yyy
%Number of characters2 = \testStrNumOfChars1

%\makeatletter
%\edef\@currentlabelname{This is a somewhat longxxx string}
%\label{testLabel}
%\makeatother
%\StrLen{\getnamereftext{testLabel}}[\testStrNumOfChars1] yyy
%Nmbr3 = \testStrNumOfChars1

%BE_2016Jun: %% be_todo -- 3. Versuch: nameref erst einer Variablen zuweisen und diese dann um // strippen.
%                                      Aber auch beim Zuweisen zu einer Variablen wird es nicht korrekt expandiert.
% \def\mystring{XYZ}  % Das tut ok.
%\def\mystring{\nameref{Chapter_Dlog-FactoringDead}}
%\StrLen{\mystring}[\mystringlen]
%\mystring \\
%\mystringlen
%%%%%%%%%%%%%%%%%%%% %BE_2016Jun: %% be_todo END

\begin{itemize}
%  \item die Definitionen zur Stärke von Sicherheits-Funktionen
%        (Kapitel \ref{cm_Section_Security_Definitions}),
  \item die größten Primzahlen (Kap. \ref{search_for_very_big_primes}),
%        neue Faktorisierungsrekorde (Kap. \ref{RSA-768}), und
%  \item Fortschritte bei der Kryptoanalyse von Hashverfahren
%        (Kap. \ref{collision-attacks-against-sha-1}) und
%  \item Fortschritte bei den Ideen für neue Kryptoverfahren (RSA-Nachfolger)
%	(Kap. \ref{xxxxcm_Brute-force-versus-Symmetr})\index{xxxxxxxxxxxxxxxx} und
  \item die Auflistung, in welchen Filmen und Romanen Kryptographie eine
        wesentliche Rolle spielt (siehe Anhang \ref{s:appendix-movies}),
%       und Kurioses zu Primzahlen (siehe \ref{HT-Quaint-curious-Primes-usage}),
  \item die Funktionsübersichten
        \hyperlink{appendix-template-overview-CT2}{zu CrypTool~2 (CT2)},
        \hyperlink{appendix-function-overview-JCT}{zu JCrypTool (JCT)} und
        \hyperlink{appendix-function-overview-CTO}{zu CrypTool-Online (CTO)}
        (siehe Anhang),
  \item weitere SageMath-Skripte zu Kryptoverfahren, und die Einführung in das
        Computer-Algebra-System (CAS) SageMath (siehe
        Anhang \ref{s:appendix-using-sage}),
        % Nguyen und Massierer (vgl. auch Kap.~\ref{ec:Sage_Massierer})
  \item der Abschnitt über die Goldbach-Vermutung
        (siehe \ref{L-GoldbachConjecture}) und über Primzahl-Zwillinge
        (siehe \ref{L-TwinCousinPrimes}),
  \item der Abschnitt über gemeinsame Primzahlen in real verwendeten
        RSA-Modulen (siehe \ref{nt_Shared-Primes}),
%  \item der Abschnitt über RSA-Fixpunkte
%        (siehe \ref{l:NumberTheory_Sage_Number-of-RSA-FixedPoints}), und
%  \item Homomorphe Verschlüsselung
%        (siehe Kapitel \ref{Chapter_HomomorphicCiphers})
  \item die~\glqq \nameref{Chapter_BitCiphers}\grqq~ist völlig neu
	(siehe Kapitel \ref{Chapter_BitCiphers}),

  %% \item die~\glqq \StrSubstitute{ XXXXXXXXXXXX }{a}{AA}\grqq~ist völlig neu
  %% \item die~\glqq \StrSubstitute{ \nameref{Chapter_Dlog-FactoringDead} }{a}{AA}\grqq~ist völlig neu  %% tut nicht
  %% \item die~\glqq \StrSubstitute{FactFactoringFact}{a}{AA}\grqq~ist völlig neu  %% das tut
  %% \item die~\glqq \StrSubstitute{\nameref{Chapter_Dlog-FactoringDead}}{a}{AA}\grqq~ist völlig neu
  %% \item die~\glqq \nameref{Chapter_Dlog-FactoringDead}\grqq~ist völlig neu
  \item die Studie ~\glqq \nameref{Chapter_Dlog-FactoringDead}\grqq~ist völlig neu
	%% So kommt leider auch der Umbruch rein, der für den Titel nötig ist bei nameref.
	%% myNewSt = strrep(mySt,sprintf('\n'),'');
	%% \usepackage{xstring}
	%% \StrSubstitute{abcde}{d}{D}   %%  {//}{ }  Use of \@xs@StrSubstitute@@ doesn't match its definition.
	%% \StrSubstitute[<number>]{<string>}{<stringA>}{<stringB>}[<name>]
	%% Man muss erzwingen, dass nameref expandiert wird vor der Verwendung!
	%% 	  \StrLen{\getnamereftext{testLabel}}[\testStrNumOfChars]
	%% 	  Number of characters in string = \testStrNumOfChars
	%% http://www.golatex.de/wiki/%5Cmakeatletter
	(siehe Kapitel \ref{Chapter_Dlog-FactoringDead}). Das ist ein
	phantastischer und eingehender Überblick über die Grenzen der
	entsprechenden aktuellen kryptoanalytischen Methoden.
\end{itemize}

%Inzwischen bekommt das CT-Projekt Feedback und Testimonials aus nahezu allen
%Ländern der Erde.


\newpage
\textbf{Dank}

An dieser Stelle möchte ich explizit folgenden Personen danken,
die bisher in ganz besonderer Weise zum CrypTool-Projekt\index{CrypTool}
beigetragen haben.
Ohne ihre besonderen Fähigkeiten und ihr großes Engagement wäre CrypTool
nicht, was es heute ist:
\vspace{-6pt}
\begin{list}{\textbullet}{\addtolength{\itemsep}{-0.5\baselineskip}}
   \item Hr.\ Henrik Koy
   \item Hr.\ Jörg-Cornelius Schneider
   \item Hr.\ Florian Marchal
   \item Dr.\ Peer Wichmann
   \item Hr.\ Dominik Schadow
   \item Mitarbeiter in den Teams von
         Prof.\ Johannes Buchmann,
         Prof.\ Claudia Eckert,
         Prof.~Alexander May,
         Prof.~Torben Weis und insbesondere
         Prof.\ Arno Wacker.
\end{list}

Auch allen hier nicht namentlich Genannten ganz herzlichen Dank für das
(meist in der Freizeit) geleistete Engagement.

Danke auch an die Leser, die uns Feedback sandten. Und ein ganz besonderer
Dank für das konstruktive Gegenlesen dieser Version durch Helmut Witten und
Prof.\ Ralph-Hardo Schulz.

Ich hoffe, dass viele Leser mit diesem Buch mehr Interesse an und
Verständnis für dieses moderne und zugleich uralte Thema finden.



\par \vskip + 35pt
Bernhard Esslinger
\par \vskip + 10pt
Heilbronn/Siegen, August 2016 + August 2017 + Mai 2018



\par \vskip + 80pt
PS:\\
Wir würden uns freuen, wenn sich weitere Autoren finden, die vorhandene Kapitel verbessern oder fundierte
Kapitel z.B. zu einem der folgenden Themen ergänzen könnten:
\begin{compactitem}
   \item Riemannsche Zeta-Funktion,
   \item Hashverfahren und Passwort-Knacken,
   \item Gitter-basierte Kryptographie,
   \item Zufallszahlen,
   \item Format-erhaltende Verschlüsselung, Privacy-preserving Kryptographie,
   \item Diskussion der Wirkung verschiedener Blockmodi auf die Sicherheit,
   \item Design/Angriff auf Krypto-Protokolle (wie SSL).
\end{compactitem}


\par \vskip + 15pt
PPS:\\%xxxxxxxxxx Todo
Ausstehende Todos für Edition 12 dieses Buches (bis dahin nennen wir es weiterhin Draft):%xxxxxxxxxxxxx
%\begin{list}{\textbullet}{\addtolength{\itemsep}{-0.5\baselineskip}}
%\end{list}
% - Abstand nach/zwischen den Bulletzeilen per itemsep+baselineskip;
% - Abstand vor erstem Bulletpoint versucht per \vskip und negativem Wert, aber
%   \par \vskip -15pt wirkte weder VOR dem \begin{list} noch DANACH. --> Nutzte eben compactitem!
\begin{compactitem}
\item Updaten aller Informationen zu SageMath (Kap.~\ref{ec:Sage_Massierer} und Appendix) und Testen des Codes gegen das neueste SageMath (Version 8.x), sowohl von der Kommandozeile als auch mit dem SageMathCloud-Notebook. SageMath 8 ist auch für Windows verfügbar.
   \item Updaten der Funktionslisten zu den vier CT-Versionen (im Appendix).
\end{compactitem}


% [1.5\baselineskip]
% \enlargethispage*{2\baselineskip}
% \nopagebreak









% --------------------------------------------------------------------------
\clearpage
\setcounter{secnumdepth}{-1}  % Prevent this chapter title from having a number
\chapter{Einführung -- Zusammenspiel von Buch und Programmen}
\setcounter{secnumdepth}{4}  % Set back default from CT-Book-de.tex (show numbers till level 4)

\textbf{Das CrypTool-Buch}

Dieses Buch wird zusammen mit den Open-Source-Programmen des
CrypTool-Projektes\index{CrypTool} ausgeliefert. Es kann auch direkt auf der
Webseite des CT-Portals herunter geladen werden (\url{https://www.cryptool.org/de/ctp-dokumentation}).

Die Kapitel dieses Buchs sind weitgehend in sich abgeschlossen und können
auch unabhängig von den CrypTool-Programmen gelesen werden.

%Während für das Verständnis der meisten Kapitel Abiturwissen ausreicht,
%erfordern die Kapitel \ref{Chapter_ModernCryptography} (Moderne Kryptografie)
%und \ref{Chapter_EllipticCurves} (Elliptische Kurven) tiefere mathematische
%Kennt"-nisse.
Für das Verständnis der meisten Kapitel reicht Abiturwissen aus. Die Kapitel
\ref{Chapter_ModernCryptography} (\glqq Moderne Kryptografie\grqq), %% (\glqq \nameref{Chapter_ModernCryptography}\grqq)
\ref{Chapter_EllipticCurves} (\glqq \nameref{Chapter_EllipticCurves}\grqq),
\ref{Chapter_BitCiphers} (\glqq Bitblock- und Bitstrom-Verschlüsselung\grqq), %% (\glqq \nameref{Chapter_BitCiphers}\grqq)
\ref{Chapter_HomomorphicCiphers}~(\glqq \nameref{Chapter_HomomorphicCiphers}\grqq) und
\ref{Chapter_Dlog-FactoringDead} (\glqq Resultate für das Lösen diskreter Logarithmen und zur Faktorisierung\grqq) %% (\glqq \nameref{Chapter_Dlog-FactoringDead}\grqq)
erfordern tiefere mathematische Kennt"-nisse.

Die \hyperlink{appendix-authors}{Autoren}
% \hyperlink{appendix-authors}{(Autoren)}
% in \ref{s:appendix-authors}
% \hypertarget{appendix-authors}{}\label{s:appendix-authors}
haben sich bemüht, Kryptographie für eine möglichst breite Leserschaft
darzu"-stellen -- ohne mathematisch unkorrekt zu werden. Sie wollen die
Awareness für die IT-Sicherheit und den Einsatz standardisierter, moderner
Kryptographie fördern.
\par \vskip + 15pt


\textbf{Die Programme CrypTool~1\index{CT1},
  CrypTool~2\index{CT2} und JCrypTool\index{JCT}}
% \index{CrypTool~1}\index{CT1} \index{CrypTool~2}\index{CT2} \index{JCrypTool}\index{JCT}

CrypTool~1 (CT1) ist ein Lernprogramm, mit dem Sie unter einer einheitlichen Oberfläche
kryptographische Verfahren anwenden und analysieren können. Die umfangreicher Onlinehilfe
in CT1 enthält nicht nur Anleitungen zur Bedienung des Programms, sondern auch
Informationen zu den Verfahren selbst (aber weniger ausführlich und anders
strukturiert als im CT-Buch).

CrypTool~1 und die Nachfolgeversionen CrypTool~2 (CT2) und JCrypTool (JCT)
werden weltweit in Schule, Lehre, Aus- und Fortbildung eingesetzt.
\par \vskip + 15pt


\textbf{CrypTool-Online\index{CTO}}

Die Webseite CrypTool-Online (CTO) (\url{http://www.cryptool-online.org}), auf
der man im Browser oder vom Smartphone aus kryptographische Verfahren
ausprobieren und anwenden kann, gehört ebenfalls zum CT-Projekt. Der Umfang
von CTO ist bei weitem nicht so groß wie der der Standalone-Programme CT1,
CT2 und JCT. Jedoch wird CTO mehr und mehr als Erstkontakt genutzt, weshalb
wir Backbone und Frontend momentan mit moderner Webtechnologie neu designen,
um ein schnelles, konsistentes und responsives Look\&Feel anzubieten.
\par \vskip + 15pt


\textbf{MTC3\index{MTC3}}

Der internationale Kryptographie-Wettbewerb
MysteryTwister C3 (MTC3) (\url{http://www.mysterytwisterc3.org})
wird ebenfalls vom CT-Projekt getragen.
Hier findet man kryptographische Rätsel in vier verschiedenen Kategorien, eine
High-Score-Liste und ein moderiertes Forum. Stand 2016-06-16 sind über 7000
Teilnehmer dabei, und es gibt über 200 Aufgaben, von denen 162 von
zumindest einem Teilnehmer gelöst wurden.
%% Todo-repetitively: 2016-06-16: All: 75+84+48+12=219; Solved: 75+75+6+6=162. Diff = 57.
\par \vskip + 15pt



\textbf{Das Computer-Algebra-Programm SageMath\index{SageMath}}

SageMath ist Open-Source und ein umfangreiches Computer-Algebra-System (CAS)-Paket,
mit dem sich die in diesem Buch erläuterten mathematischen Verfahren leicht
Schritt-für-Schritt programmieren lassen. Eine Besonderheit dieses CAS ist,
dass als Skript-Sprache Python (z.Zt. Version 2.x) benutzt wird.
% be_xxxxxxxxxx_Update, if SageMath moves to 3.x
Dadurch stehen einem in Sage-Skripten nach einem import-Befehl auch alle Funktionen
der Sprache Pytho\index{Python}n zur Verfügung.\\
SageMath wird mehr und mehr zum Standard-CAS an Hochschulen.
\par \vskip + 15pt



\textbf{Die Schüler-Krypto-Kurse\index{Schüler-Krypto}}

Diese Initiative bietet Ein- und Zwei-Tages-Kurse in Kryptologie für Schüler
und Lehrer, um zu zeigen, wie attraktiv MINT-Fächer wie Mathematik,
Informatik und insbesondere Kryptologie sind.
Die Kursidee ist eine virtuelle Geheimagenten-Ausbildung.\\
Inzwischen finden diese Kurse seit mehreren Jahren in Deutschland in
unterschiedlichen Städten statt.\\
Alle Kursunterlagen sind frei erhältlich auf
\url{http://www.cryptool.org/schuelerkrypto/}.\\
Alle eingesetzte Software ist ebenfalls frei (meist wird CT1 und CT2
eingesetzt).\\
Wir würden uns freuen, wenn jemand die Kursunterlagen übersetzt und
einen entsprechenden Kurs in Englisch anbieten würde.
\par \vskip + 15pt



\textbf{Dank}

Herzlichen Dank an alle, die mit ihrem großem Einsatz zum
Erfolg und zur weiten Verbreitung dieses Projekts beigetragen haben.



\par \vskip + 45pt
Bernhard Esslinger
\par \vskip + 10pt
Heilbronn/Siegen, August 2017

% Local Variables:
% TeX-master: "../script-de.tex"
% End:
