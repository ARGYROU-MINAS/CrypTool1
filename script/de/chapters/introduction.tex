% ............................................................................
%                         E I N F � H R U N G
% ~~~~~~~~~~~~~~~~~~~~~~~~~~~~~~~~~~~~~~~~~~~~~~~~~~~~~~~~~~~~~~~~~~~~~~~~~~~~

\section*{Einf"uhrung}  \addcontentsline{toc}{section}{Einf"uhrung}

Dieses Skript wird zusammen mit CrypTool\index{CrypTool} ausgeliefert.

CrypTool ist ein Programm mit sehr umfangreicher Online-Hilfe, mit dessen
Hilfe Sie unter einer einheitlichen Oberfl"ache kryptographische Verfahren
anwenden und analysieren k"onnen.
\par \vskip + 3pt

CrypTool wurde im Zuge des End-User Awareness-Programmes entwickelt, um die
Sensibilit"at der Mitarbeiter f"ur IT-Sicherheit zu erh"ohen und um ein
tieferes Verst"andnis f"ur den Begriff Sicherheit zu erm"oglichen. 
\par \vskip +3pt 

Ein weiteres Anliegen war die Nachvollziehbarkeit der in der Deutschen Bank
eingesetzten kryptographischen Verfahren. So ist es mit CrypTool als 
verl"asslicher Referenzimplementierung der verschiedenen 
Verschl"usselungsverfahren (aufgrund der Nutzung der Industrie-bew"ahrten 
\index{Secude GmbH} Secude-Bibliothek) auch m"og\-lich, die in anderen 
Programmen eingesetzte Verschl"usselung zu testen.\par \vskip + 3pt

Inzwischen wird CrypTool\index{CrypTool} in Ausbildung und Lehre eingesetzt
und von mehreren Universit"aten mit weiterentwickelt. \par \vskip + 3pt

Da die Artikel dieses Skripts weitgehend in sich abgeschlossen sind, kann
dieses Skript auch unabh"angig von CrypTool\index{CrypTool} gelesen werden.
\par \vskip + 3pt

Die {\em Autoren} haben sich bem"uht, Kryptographie f"ur eine m"oglichst 
breite Leserschaft darzustellen -- ohne mathematisch unkorrekt zu werden. 
Dieser didaktische Anspruch ist am besten geeignet, die Awareness
f"ur die IT-Sicherheit und die Bereitschaft f"ur den Einsatz 
standardisierter, moderner Kryptographie zu f"ordern.

Au"serdem wurde versucht, die jeweils neuesten Erkenntnisse bez"uglich
Primzahlen, Faktorisierung und Kryptoanalyse von AES darzustellen, 
um top-aktuell zu sein.
\\

An dieser Stelle m"ochte ich explizit drei Personen nennen und danken, 
die bisher in ganz besonderer Weise zu CrypTool beigetragen haben.
Ohne ihre besonderen F"ahigkeiten und ihr gro�es Engagement w"are CrypTool
nicht, was es heute ist:
\begin{itemize}
   \item Hr. Henrik Koy
   \item Hr. Joerg-Cornelius Schneider und
   \item Dr. Peer Wichmann.
\end{itemize}
Auch allen hier nicht namentlich genannten ganz herzlichen Dank f"ur das 
(meist in der Freizeit) geleistete Engagement.
\\

Bernhard Esslinger

Frankfurt, March 2003


% Local Variables:
% TeX-master: "../script-de.tex"
% End:
