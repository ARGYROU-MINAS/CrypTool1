% $Id$
% ............................................................................
%                 TEXT OF THE 2nd PAGE (Overview)
% ~~~~~~~~~~~~~~~~~~~~~~~~~~~~~~~~~~~~~~~~~~~~~~~~~~~~~~~~~~~~~~~~~~~~~~~~~~~~

% --------------------------------------------------------------------------
% HACK to fix warning "destination with the same identifier .. has already been used, ...":
%\makeatletter \renewcommand{\thepage}{~\csname @roman\endcsname \c@page~} \makeatother
\clearpage\phantomsection
% HACK to fix warning "destination with the same identifier .. has already been used, ...":
% \makeatletter \renewcommand{\thepage}{\csname @roman\endcsname \c@page} \makeatother

\addcontentsline{toc}{chapter}{Overview about the Content of the CrypTool Book}
\chapter*{Overview about the Content of the CrypTool Book}  

\parskip 4pt
%\vskip +12 pt
The rapid spread of the Internet has led to intensified research in the
technologies involved, especially within the area of cryptography where a good
deal of new knowledge has arisen.

In this {\em book accompanying the CrypTool programs} \index{CrypTool}
you will find predominantly mathematically oriented information on using
cryptographic procedures. Also included are many sample code pieces written in the
computer algebra system {\bf SageMath}\index{SageMath} (see appendix~\ref{s:appendix-using-sage}).
The main chapters have been written by various {\bf authors}
(see appendix~\ref{s:appendix-authors}) %\hyperlink{appendix-authors}{authors}
and are therefore independent from one another. At the end of most chapters
you will find references and web links.
The sections have been enriched with many {\em footnotes}. Within the footnotes
you can see where the described functions can be called in the different CrypTool
versions.

The \hyperlink{Kapitel_1}{first chapter} explains the principles of symmetric
and asymmetric {\bf encryption} and definitions for their resistibility.

Because of didactic reasons the \hyperlink{Kapitel_PaperandPencil}
{second chapter} gives an exhaustive overview
about {\bf paper and pencil encryption methods}.

Big parts of this book are dedicated to the fascinating topic of 
{\bf prime numbers} (chap. \ref{Label_Kapitel_Primes}).
%\hyperlink{Kapitel_2}{{\bf prime numbers}}. 
Using numerous examples, {\bf modular arithmetic} and 
{\bf elementary number theory} (chap. \ref{Chapter_ElementaryNT})
are introduced. The features of the {\bf RSA procedure} are a key aspect.

By reading chapter \ref{Chapter_ModernCryptography}
you'll gain an insight into the mathematical ideas and concepts behind 
{\bf modern cryptography}.

Chapter \ref{Chapter_Hashes-and-Digital-Signatures} gives
an overview about the status of attacks against modern {\bf hash algorithms}
and is then shortly devoted to {\bf digital signatures}, 
which are an essential component of e-business applications.

Chapter \ref{Chapter_EllipticCurves} describes {\bf elliptic curves}:
They could be used as an alternative to RSA and in addition are extremely
well suited for implementation on smartcards.

Chapter \ref{Chapter_BitCiphers} introduces {\bf Boolean algebra}.
This is the foundation for most modern symmetric encryption algorithms
as these operate on bit streams and bit groups.

Chapter \ref{Chapter_HomomorphicCiphers} describes {\bf homomorphic crypto
functions}: They are a modern research topic which got especial attention
in the course of cloud computing.

The \hyperlink{Chapter_Crypto2020}{last chapter} {\bf Crypto2020}
discusses threats for currently used cryptographic methods and introduces
alternative research approaches to achieve long-term security
of cryptographic schemes.

Whereas the CrypTool \textit{e-learning programs}\index{e-learning} motivate
and teach you how to use cryptography in practice, the \textit{book} provides
those interested in the subject with a deeper understanding of the mathematical
algorithms used -- trying to do it in an instructive way.

Within the {\bf appendices}
\ref{s:appendix-menu-overview-CT1},
\ref{s:appendix-template-overview-CT2},
\ref{s:appendix-function-overview-JCT}, and
\ref{s:appendix-function-overview-CTO}
you can gain a fast overview about the functions delivered by the different
CrypTool variants\index{CrypTool 1}\index{CrypTool 2}\index{JCrypTool} via:
\begin{itemize}
  \item the function list and
        the \hyperlink{appendix-menu-overview-CT1}
                      {menu tree of CrypTool 1 (CT1)},
  \item the function list and
        the \hyperlink{appendix-template-overview-CT2}
                      {templates in CrypTool 2 (CT2)},
  \item the \hyperlink{appendix-function-overview-JCT}
                      {function list of JCrypTool (JCT)}, and
  \item the \hyperlink{appendix-function-overview-CTO}
                      {function list of CrypTool-Online (CTO)}.
\end{itemize}

% Bernhard Esslinger, Matthias B\"uger, Bartol Filipovic, Henrik Koy, 
% Roger Oyono and J\"org Cornelius Schneider
The authors would like to take this opportunity to thank their colleagues 
in the particular companies and at the universities of Bochum, Darmstadt,
Frankfurt, Gie\ss en, Karlsruhe and Siegen.

\enlargethispage{12pt}
As with the e-learning program CrypTool\index{CrypTool}, the quality of the 
book is enhanced by your suggestions and ideas for improvement. 
We look forward to your feedback.


% Local Variables:
% TeX-master: "../script-en.tex"
% End:
