% ............................................................................
%                         E I N F � H R U N G
% ~~~~~~~~~~~~~~~~~~~~~~~~~~~~~~~~~~~~~~~~~~~~~~~~~~~~~~~~~~~~~~~~~~~~~~~~~~~~

\section*{Introduction}  \addcontentsline{toc}{section}{Introduction}

This script is delivered together with CrypTool\index{CrypTool}.

CrypTool is a program with an extremely comprehensive online help enabling you
to use and analyse cryptographic procedures within a unified graphical user interface.\par \vskip + 3pt

CrypTool was developed during the end-user awareness program in order to
increase employee awareness of IT security and provide them with a deeper
understanding of the term security.\par \vskip + 3pt

A further aim was to enable users to understand the cryptographic procedures
implemented in the Deutsche Bank. In this way, using CrypTool as a reliable
reference implementation of the various encryption procedures (because of using the industry-proven Secude Library\index{Secude}),
you can test the encryption implemented in other programs. \par \vskip + 3pt

Since then CrypTool\index{CrypTool} has been used for education in companies and
universities and several universities help to develop further project.

Because the articles in this script are largely self-contained, this script can also
be read independently of CrypTool\index{CrypTool}.

The {\em authors} tried to describe cryptography for a broad audience --
without being mathematically incorrect. We believe, that this didactical
pretension is the best way to promote the awareness for IT security and the
readiness to use standardised modern cryptography.

Additionally we aimed to describe the newest results, as well as their
origins, concerning primes, factorization and cryptanalysis of AES, in order to be highly topical.
\\

At this point I'd like to thank explicitely 3 persons who espcially 
contributed to CrypTool. Without their talents and engagement 
CrypTool would not be what it is today:
\begin{itemize}
   \item Mr. Henrik Koy
   \item Mr. Joerg-Cornelius Schneider and
   \item Dr. Peer Wichmann.
\end{itemize}
Also I want to thank all the ones not named here for their engagement (mostly
performed in their spare time).
\\

Bernhard Esslinger

Frankfurt, March 2003


% Local Variables:
% TeX-master: "../script-en.tex"
% End:
