% $Id$
% ............................................................................
%      V O R W O R T  und  E I N F � H R U N G (Zusammenspiel Skript-CT) 
% ~~~~~~~~~~~~~~~~~~~~~~~~~~~~~~~~~~~~~~~~~~~~~~~~~~~~~~~~~~~~~~~~~~~~~~~~~~~~


% --------------------------------------------------------------------------
\clearpage\phantomsection
\addcontentsline{toc}{chapter}{Preface to the 11th Edition of the CrypTool Script}
\chapter*{Preface to the 11th Edition of the CrypTool Script}

Starting in the year 2000 this script became part of the CrypTool 1
(CT1)\index{CrypTool 1} package. The script's goal is to explain some mathematical
topics in more detail, but still in a way which is easy to understand.

Topics from mathematics and cryptography have been meaningfully split up and
for each topic an extra chapter has been written which can be read on its own.
This enables developers/authors to contribute independently of each other.
Naturally there are many more interesting topics in cryptography which could
be discussed in greater depth -- therefore this selection is only one of many
possible ways.

The later editorial work in TeX added footnotes and cross linkages between
different sections.

%This edition of the CT script again updated some topics:
This edition completely updated the TeX sources of the document, and of course
the content of the script was amended, corrected, and updated with some topics, e.g.:
\vspace{-5pt}
\begin{itemize}
  \item the definitions of the strength of security functions
        (chap. \ref{cm_Section_Security_Definitions}),
%  \item the largest prime numbers (chap. \ref{search_for_very_big_primes}),
%        new factorization records (chap. \ref{RSA-768}),
%  \item progress in cryptanalysis of AES 
%        (chap. \ref{NeueAES-Analyse}) and
%  \item progress in cryptanalysis of hash algorithms 
%        (chap. \ref{collision-attacks-against-sha-1}) and
%  \item progress in ideas for new crypto methods (RSA successor) 
%        (chap. \ref{xxxxxxxxxBrute-force-gegen-Symmetr})\index{xxxxxxxxxx} and
  \item the list of movies or novels, in which cryptography or number theory 
        played a major role (see appendix \ref{s:appendix-movies});
        and where primes are used as hangers  
        (see curiouses in \ref{HT-Quaint-curious-Primes-usage}),
  \item the overviews of all functions in
        \hyperlink{appendix-template-overview-CT2}{for CT2} and
        \hyperlink{appendix-function-overview-JCT}{for JCT},
  \item further Sage scripts for cryptography, and the appendix
        \ref{s:appendix-using-sage} about using the computer algebra system
        Sage. Sage becomes more and more the standard open-source CAS system,
        % Nguyen and Massierer (see also chap.~\ref{ec:Sage_Massierer}).
  \item the section about the Goldbach conjecture
        (see \ref{L-GoldbachConjecture}),
  \item the section about shared primes in RSA modules used in reality
        (see \ref{l:NumberTheory_Shared-Primesxxx}),xxxxxxxxxxxxxxxxxxxx
  \item the section about RSA fixed points
        (see \ref{l:NumberTheory_Sage_Number-of-RSA-FixedPoints}), and
  \item the chapter about homomorphic encryption (see \ref{Chapter_HomomorphicEnc}).
\end{itemize}

The script was delivered the first time with the CrypTool-1 package in
version 1.2.01. Since then it has been expanded and revised in almost
every new version of CrypTool 1.

In the meantime the CT project gets feedback and testimonials from almost all
countries of the planet.

I am deeply grateful to all the people helping with their impressive
commitment who have made this global project so successful.
Thanks also to the readers who sent us feedback.
% Especially I would like to acknowledge the English language proof-reading
% of this script version done by Richard Christensen and Lowell Montgomery.

I hope that many readers have fun with this script and that they get 
out of it more interest and greater understanding of this modern but 
also very ancient topic.
\\
\\
Bernhard Esslinger
\\
\\
Frankfurt and Siegen (Germany), October 2012



% --------------------------------------------------------------------------
\clearpage\phantomsection
\addcontentsline{toc}{chapter}{Introduction -- How do the Script and the Program Play together?}
\chapter*{Introduction -- How do the Script and the Program Play together?}

\textbf{This CrypTool script}

This document is delivered together with the open-source program
CrypTool 1\index{CrypTool 1}.

The articles in this script are largely self-contained and
can also be read independently of the CrypTool programs.

Chapters  \ref{Chapter_ModernCryptography} (Modern Cryptography) and 
\ref{Chapter_EllipticCurves} (Elliptic Curves) require a deeper knowledge
in mathematics, while the other chapters should be understandable with a 
school leaving certificate.

The \hyperlink{appendix-authors}{authors}
have attempted to describe cryptography for a broad 
audience -- without being mathematically incorrect. We believe that this
didactic pretension is the best way to promote the awareness for IT
security and the readiness to use standardized modern cryptography.
\par \vskip + 15pt


\noindent \textbf{The programs CrypTool 1\index{CrypTool}\index{CT1},
  CrypTool 2\index{CrypTool 2}\index{CT2} and JCrypTool\index{JCrypTool}\index{JCT}}

CrypTool 1 (CT1) is an educational program with a comprehensive
online help enabling you to use and analyse cryptographic procedures within a
unified graphical user interface. The online help in CrypTool 1 contains both
instructions how to use the program and explanations about the methods itself
(both not as detailled and in another structure than the CT script).

CrypTool 1\index{CrypTool 1} and the successor versions
CrypTool 2 (CT2)\index{CrypTool 2} and JCrypTool (JCT)\index{JCrypTool}
are used world-wide for training in companies and teaching at schools and
universities.
\par \vskip + 15pt


\noindent \textbf{CrypTool-Online and MTC3\index{CrypTool-Online}\index{CTO}\index{MTC3}}

Also part of the CT project is the website CrypTool-Online (CTO)
(\url{http://www.cryptool-online.org}), where you can check and apply cryptographic
methods within a browser or on a smartphone. The scope of CTO is far below from
the standalone programs CT1, CT2 and JCT.

Also based on the CT project is the international cryptography contest
MysteryTwister C3 (MTC3) (\url{http://www.mysterytwisterc3.org}).
Here you can find cryptographic riddles in four categories, a
high-score list and a moderated forum. In the meantime more than 3300 active
users participate, and more than 140 challenges are offered (currently 89 of
them are solved by at least one user).
\par \vskip + 15pt


\noindent \textbf{The Computer Algebra System Sage\index{Sage}}

Sage is an open source CAS package which can be used to easily program
the mathematical methods explained in this script.
\par \vskip + 15pt



\noindent \textbf{Acknowledgment}

At this point I'd like to thank explicitly the following people who
particularly contributed to the CrypTool projct\index{CrypTool}. They
applied their very special talents and showed really great engagement:
\vspace{-7pt}
%\begin{itemize}
\begin{list}{\textbullet}{\addtolength{\itemsep}{-0.5\baselineskip}}
   \item Mr.\ Henrik Koy
   \item Mr.\ J\"org-Cornelius Schneider
   \item Mr.\ Florian Marchal
   \item Dr.\ Peer Wichmann
   \item Hr.\ Dominik Schadow
   \item Staff of 
         Prof.\ Johannes Buchmann,
         Prof.\ Claudia Eckert,
         Prof.\ Alexander May,
         Prof.\ Torben Weis and
         Prof.\ Arno Wacker.
\end{list}
%\end{itemize}
Also I want to thank all the many people not mentioned here for their 
hard work (mostly carried out in their spare time).
\\
\\
Bernhard Esslinger
\\
\\
Frankfurt and Siegen (Germany), October 2012
\par \vskip + 15pt


\noindent PS:\\
We'd be glad if further authors would show up to add further chapters, e.g. about
hash functions, random numbers or the design and attack of crypto protocols (like SSL).

% Local Variables:
% TeX-master: "../script-en.tex"
% End:
