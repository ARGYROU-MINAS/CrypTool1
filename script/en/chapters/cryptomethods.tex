% ..............................................................................
%                V E R S C H L U E S S E L U N G S V E R F A H R E N
%
% Schreibregelung: Capitel header to capitalize words,
%                  All other headers in normal lower/upper case 
%                  like sentences without dot at the end.
%
% ~~~~~~~~~~~~~~~~~~~~~~~~~~~~~~~~~~~~~~~~~~~~~~~~~~~~~~~~~~~~~~~~~~~~~~~~~~~~~~

\newpage
\section{Encryption Procedures}
\hypertarget{Kapitel_1}{}
(Bernhard Esslinger, besslinger@web.de, May 1999, Updates Dec. 2001, Feb. 2003)


% --------------------------------------------------------------------------
\subsection{Encryption}

The purpose of encryption \index{Encryption} is to change data in such a way
that only an authorised recipient is able to reconstruct the plaintext. This has
the advantage that you can transmit encrypted data openly and nevertheless need
not fear a perpetrator reading the data without authorisation. Authorised
recipients possess a piece of secret information --- called the key --- which
allows them to decrypt the data while it remains hidden from everyone else.\par \vskip + 3pt

One encryption procedure has been proved to be secure --- the {\em One Time
  Pad}.
\index{One �Time �Pad} However, this procedure has several practical
disadvantages (the key used must be selected randomly and must be just as long
as the message to be protected), which means that it is hardly used except in
closed environments such as for the hot wire between Moscow and Washington.\par \vskip + 3pt

For all other procedures there is a (theoretical) possibility of breaking them.
If the procedures are good, however, the time taken to break them is so long
that it is practically impossible to do and these procedures can therefore be
considered (practically) secure.\par \vskip + 3pt

We basically distinguish between symmetric and asymmetric encryption procedures.

% --------------------------------------------------------------------------
\subsubsection[Symmetric encryption]
{Symmetric encryption\footnotemark}
  \footnotetext{%
    With CrypTool\index{CrypTool} v1.3 you can execute the following modern
    symmetric encryption algorithms 
    (using the menu {\bf Crypt \textbackslash{} Symmetric}): \\
    IDEA, RC2, RC4, DES (ECB), DES�(CBC), Triple-DES�(ECB), Triple-DES�(CBC),
    MARS (AES candidate), RC6 (AES candidate), Serpent (AES candidate), 
    Twofish (AES candidate), Rijndael (official AES algorithm).
  }

For {\em symmetric} encryption \index{Encryption!symmetric} the sender and
recipient must possess a common (secret) key which they have exchanged before
actually starting to communicate. The sender uses this key to encrypt the
message and the recipient uses it to decrypt it.\par \vskip + 3pt

The advantages of symmetric algorithms are the high speed with which data can be
encrypted and decrypted. One disadvantage is the need for key management. In
order to communicate with one another confidentially, sender and recipient must
have exchanged a key using a secure channel before actually starting to
communicate. Spontaneous communication between individuals who have never met
therefore seems virtually impossible. If everyone wants to communicate with
everyone else spontaneously at any time in a network of $ n $ subscribers, each
subscriber must have previously exchanged a key with each of the other $n-� 1$
subscribers. A total of $n(n - 1)/2$ keys must therefore be exchanged.\par \vskip + 3pt

The most well-known symmetric encryption procedure is the \index{DES} DES-algorithm. The DES-algorithm has been developed by IBM in collaboration with the
National Security Agency \index{NSA} (NSA), and was published as a standard in
1975. Despite the fact that the procedure is relatively old, no effective attack
on it has yet been detected. The most effective way of attacking consists of
testing all possible keys until the right one is found ({\em brute-force-attack}).
\index{Attack!brute-force} Due to the relatively short key length of
effectively 56 bits (64 bits, which however include 8 parity bits), numerous
messages encrypted using DES have in the past been broken. Therefore, the
procedure can now only be considered to be conditionally secure. Symmetric
alternatives to the DES procedure include the IDEA \index{IDEA} or Triple DES
algorithms.\par \vskip + 3pt

Up-to-the-minute procedures are the symmetric AES procedures. The associated
Rijndael procedure was declared winner the AES award on 2 October 2000 and thus
succeeds the DES procedure.

More details about the AES algorithms can be found within the 
Online help of CrypTool\index{CrypTool}
  \footnote{%
      CrypTool Online help\index{CrypTool}: the index see head-word {\bf AES}
      leads to the help pages: {\bf AES candidates}, 
      {\bf The AES Winner Rijndael} and 
      {\bf The Rijndael encryption algorithm}.
  }
.


% --------------------------------------------------------------------------
\subsubsection{New results about cryptanalysis of AES}

Below you will find some results, which have recently called into question the security of the AES algorithm -- from our point of view these doubts practically still remain unfounded
% = do not bring disrepute upon AES
. 
The following information is based on the original papers and the articles \cite{Wobst-iX2002} and \cite{Lucks-DuD2002}.

AES with a minimum key length of 128 bit is still in the long run sufficiently secure against brute-force attacks - as long as the quantum computers weren't powerful enough. When announced as new standard AES was immune against all known crypto attacks, mostly based on statistical considerations and earlier applied to DES: using pairs of clear and cipher texts expressions are constructed, which are not completely at random, so they allow conclusions to the used keys. These attacks required unrealistically large amounts of intercepted data.

Cryptanalysts already label methods as ``academic success'' or as ``cryptanalytic attack'' if they are theoretically faster than the complete testing of all keys (brute force analysis). In the case of AES with the maximal key length (256 bit) exhaustive key search on average needs $2^{255}$ encryption operations. A cryptanalytic attack needs to be better than this. At present between $2^{75}$ and $2^{90}$ encryption operations are estimated to be performable for organizations, for example a security agency.

In their 2001-paper Ferguson, Schroeppel and Whiting \cite{Ferguson2001} presented a new method of symmetric codes cryptanalysis: They described AES with a closed formula (in the form of a continued fraction) which was possible because of the "relatively" clear structure of AES. This formula consists of around 1000 trillion terms of a sum - so it does not help concrete practical cryptanalysis. Nevertheless -curiosity in the academic community awakened. It was already known, that the 128-bit AES could be described as an over-determined system of about 8000 quadratic equations (over an algebraic number field) with about 1600 variables (some of them are the bits of the wanted key) -- equation systems of that size are in practice not solvable. This special equation system is relatively sparse, so only very few of the quadratic terms (there are about 1,280,000 are possible quadratic terms in total) appear in the equation system.

The mathematicians Courois and Pieprzyk \cite{Courtois2002} published a paper in 2002, which got a great deal of attention amongst the crypto community: The pair had further developed the XL-method (eXtended Linearization), introduced at Eurocrypt 2000 by Shamir et al., to create the so called XSL-method (eXtended Sparse Linearization). The XL-method is a heuristic technique, which in some cases manages to solve big non-linear equation systems and which was till then used to analyze an asymmetric algorithm (HFE).  The innovation of Courois and Pieprzyk was, to apply the XL-method on symmetric codes: the XSL-method can be applied to very specific equation systems. A 256-bit AES could be attacked in roughly $2^{230}$ steps. This is still a purely academic attack, but also a direction pointer for a complete class of block ciphers. The major problem with this attack is that until now nobody has worked out, under what conditions it is successful: the authors specify in their paper necessary conditions, but it is not known, which conditions are sufficient.
There are two very new aspects of this attack: firstly this attack is not based on statistics but on algebra. So attacks seem to be possible, where only very small amounts of cipher text are available. Secondly the security of a product-algorithm does not exponentially increase with the number of rounds.

Currently there is a large amount of research in this area: for example Murphy and Robshaw presented a paper at Crypto 2002 \cite{Robshaw2002a}, which could dramatically improve cryptanalysis: the burden for a 128-bit key was estimated at about $2^{100}$ steps by describing AES as a special case of an algorithm called BES (Big Encryption System), which has an especially "round" structure. But even $2^{100}$ steps are beyond what is achievable in the foreseeable future. Using a 256 bit key the authors estimate that a XSL-attack will require $2^{200}$ operations.

More details can be found at: \\
\begin{itemize}
  \item[] \href{http://www.cryptosystem.net/aes}
               {\texttt{http://www.cryptosystem.net/aes}}
  \item[] \href{http://www.minrank.org/aes/}
               {\texttt{http://www.minrank.org/aes/}}
\end{itemize}

So for 256-AES the attack is much more effective than brute-force but still far more away from any computing power which could be accessible in the short-to-long term. 

The discussion is very controversial: Don Coppersmith (one of the inventors of DES) for example queries the practicability of the attack because XLS would provide no solution for AES \cite{Coppersmith2002}. This concludes that then the optimization of Murphy and Robshaw \cite{Robshaw2002b} would not work.


% --------------------------------------------------------------------------
\subsubsection[Asymmetric encryption]
{Asymmetric encryption\footnotemark}
  \footnotetext{%
    With CrypTool\index{CrypTool} v1.3 you can execute RSA encryption
    and decryption (using the menu {\bf Crypt \textbackslash{} Asymmetric}).
  }

In the case of {\em asymmetric} encryption \index{Encryption!asymmetric} each
subscriber has a personal pair of keys consisting of a {\em secret}
\index{Key!secret} key and a {\em public} key\index{Key!public}. The public
key, as its name implies, is made public, e.g. in a key directory on the
Internet.\par \vskip + 3pt

If Alice wants to communicate with Bob, then she finds Bob's public key 
in the directory and uses it to encrypt her message to him. She then sends
this cipher text to Bob, who is then able to decrypt it again using his 
secret key. As only Bob knows his secret key, only he can decrypt 
messages addressed to him.
Even Alice who sends the message cannot restore plaintext from the (encrypted)
message she has sent. Of course, you must first ensure that the public key
cannot be used to derive the private key.\par \vskip + 3pt

Such a procedure can be demonstrated using a series of thief-proof letter boxes.
If I have composed a message, I then look for the letter box of the recipient
and post the letter through it. After that, I can no longer read or change the
message myself, because only the legitimate recipient possesses the key for the
letter box.\par \vskip + 3pt

The advantage of asymmetric procedures is the easy \index{Key management} key management. Let's look again at a network with $n$
subscribers. In order to ensure that each subscriber can establish
an encrypted connection to each other subscriber, each subscriber
must possess a pair of keys. We therefore need $2n$ keys or $n$
pairs of keys. Furthermore, no secure channel is needed before
messages are transmitted, because all the information required in
order to communicate confidentially can be transmitted openly. In
this case, you simply have to pay attention to the accuracy
(integrity and authenticity) \index{Authenticity} of the public
key. Disadvantage: Pure asymmetric procedures take a lot longer to
perform than symmetric ones.\par \vskip + 3pt

The most well-known asymmetric procedure is the \index{RSA} RSA algorithm,
named after its developers Ronald \index{Rivest Ronald} Rivest, Adi
\index{Shamir Adi} Shamir and Leonard \index{Adleman Leonard} Adleman. The RSA algorithm
was published in 1978. The concept of asymmetric encryption was first
introduced by Whitfield Diffie \index{Diffie Whitfield}  and Martin
\index{Hellman Martin} Hellman in 1976. Today, the ElGamal \index{ElGamal}
procedures also play a decisive role, particularly the \index{Schnorr} Schnorr
variant in the \index{DSA} DSA (Digital \index{Signatures!digital}Signature
Algorithm).


% --------------------------------------------------------------------------
% \newpage
\subsubsection[Hybrid procedures]
{Hybrid procedures\footnotemark}
\footnotetext{%
Within CrypTool\index{CrypTool} v1.3 you can get a visualization of this
technique using the menu {\bf Crypt \textbackslash{} Hybrid Demonstration}: 
this dialogue shows the single steps and its dependencies with concrete
numbers.
}\index{Hybrid procedures}

In order to benefit from the advantages of symmetric and asymmetric techniques
together, hybrid procedures are usually used (for encryption) in practice.
\par \vskip + 3pt

In this case the data is encrypted using symmetric procedures: the key is a
session key\index{Session key} generated by the sender
randomly \footnote{%
An important part of cryptographically secure techniques is to generate 
random numbers. Within CrypTool\index{CrypTool} v1.3 you can check out
different random number generators using the menu {\bf Indiv. Procedures 
\textbackslash{} Generate Random Numbers}. 
Using the menu{\bf Analysis \textbackslash{} Random Tests} you can apply
different test methods for random data to binary documents. \\
Till now CrypTool concentrates on cryptographically strong 
pseudo number generators. Only the integrated Secude generator 
involves a "pure" random source. 
}\index{Random}
that is only used for this message.
This session key is then encrypted using the asymmetric procedure and
transmitted to the recipient together with the message. Recipients can determine
the session key using their secret keys and then use the session key to encrypt
the message. In this way, we can benefit from the easy key management
\index{Key management} of asymmetric procedures and encrypt large quantities of data
quickly and efficiently using symmetric procedures.


% --------------------------------------------------------------------------
\subsubsection{Further details}

Beside information you can find in many books and on a lot of websites the 
online help of CrypTool\index{CrypTool} also offers very many details about
the symmetric and asymmetric encryption methods.


% --------------------------------------------------------------------------
\newpage
\begin{thebibliography}{99999}
\addcontentsline{toc}{subsection}{Bibliography}

\bibitem[Schmeh2003]{Schmeh2003}  \index{Schmeh 2003}
        Klaus Schmeh, \\
        {\em Cryptography and Public Key Infrastructures on the Internet},\\ 
	John Wiley \& Sons Ltd., Chichester 2003. \\
        A considerable, up-to-date, good reading book, which also 
	considers practical problems like standardisation or
        real existing software.


\bibitem[Coppersmith2002]{Coppersmith2002}  \index{Coppersmith 2002}
        Don Coppersmith, \\
        {\em Re: Impact of Courtois and Pieprzyk results}, \\
	2002-09-19, ``AES Discussion Groups''~ at \\
        \href{http://aes.nist.gov/aes/}
        {\texttt{http://aes.nist.gov/aes/}}

\bibitem[Courtois2002]{Courtois2002}  \index{Courtois 2002}
        Nicolas Courtois, Josef Pieprzyk, \\
        {\em Cryptanalysis of Block Ciphers with Overdefined Systems of Equations}, \\
	received 10 Apr 2002, last revised 9 Nov 2002.\\
	A different version, so called compact version of the first XSL attack,
	was published at Asiacrypt Dec 2002. \\
        \href{http://eprint.iacr.org/2002/044}
        {\texttt{http://eprint.iacr.org/2002/044}}

\bibitem[Ferguson2001]{Ferguson2001}  \index{Ferguson 2001}
        Niels Ferguson, Richard Schroeppel, Doug Whiting, \\
        {\em A simple algebraic representation of Rijndael}, 
	Draft 2001/05/1, \\
        \href{http://www.xs4all.nl/~vorpal/pubs/rdalgeq.html}
        {\texttt{http://www.xs4all.nl/\~{}vorpal/pubs/rdalgeq.html}}

\bibitem[Lucks-DuD2002]{Lucks-DuD2002}  \index{Lucks 2002}
        Stefan Lucks, R"udiger Weis, \\
        {\em Neue Ergebnisse zur Sicherheit des Verschl�sselungsstandards AES}, 
	in DuD Dec. 2002.

\bibitem[Robshaw2002a]{Robshaw2002a}  \index{Robshaw 2002}
        S.P. Murphy, M.J.B. Robshaw, \\
        {\em Essential Algebraic Structure within the AES}, 
	June 5, 2002, Crypto 2002,  \\
        \href{http://www.isg.rhul.ac.uk/\~{}mrobshaw/rijndael/rijndael.html}
        {\texttt{http://www.isg.rhul.ac.uk/~mrobshaw/rijndael/rijndael.html}}

\bibitem[Robshaw2002b]{Robshaw2002b}  \index{Robshaw 2002}
        S.P. Murphy, M.J.B. Robshaw, \\
        {\em Comments on the Security of the AES and the XSL Technique}, 
	September 26, 2002, \\
        \href{http://www.isg.rhul.ac.uk/\~{}mrobshaw/rijndael/rijndael.html}
        {\texttt{http://www.isg.rhul.ac.uk/~mrobshaw/rijndael/rijndael.html}}

\bibitem[Wobst-iX2002]{Wobst-iX2002}  \index{Wobst 2002}
        Reinhard Wobst, \\
        {\em Angekratzt - Kryptoanalyse von AES schreitet voran}, 
	in iX Dec. 2002, \\
	plus the reader's remark by Johannes Merkle in iX Feb. 2003.

\end{thebibliography}


% --------------------------------------------------------------------------
\newpage
\section*{Web links}\addcontentsline{toc}{subsection}{Web links}

\begin{enumerate}
  \item[] \href{http://www.cryptosystem.net/aes}
               {\texttt{http://www.cryptosystem.net/aes}}
  \item[] \href{http://www.minrank.org/aes/}
               {\texttt{http://www.minrank.org/aes/}}
\end{enumerate}


% Local Variables:
% TeX-master: "../script-en.tex"
% End:
