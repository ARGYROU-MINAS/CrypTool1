% $Id$
%%%%%%%%%%%%%%%%%%%%%%%%%%%%%%%%%%%%%%%%%%%%%%%%%%%%%%%%%%%%%%%%%%%%%%%%%%
%
% C H A P T E R   T W O
%
%%%%%%%%%%%%%%%%%%%%%%%%%%%%%%%%%%%%%%%%%%%%%%%%%%%%%%%%%%%%%%%%%%%%%%%%%%

\newpage
\section{Prime Numbers}
\hypertarget{Kapitel_2}{}
(Bernhard Esslinger, besslinger@web.de, May 1999, Updates Nov. 2000, Dec. 2001, Feb. 2003)

\begin{center}
\fbox{\parbox{15cm}{
    \emph{Albert Einstein\footnotemark:}\\
    Progress requires exchange of knowledge.
}}
\end{center}
\addtocounter{footnote}{0}\footnotetext{%
  German physicist and Nobel Prize winner, March 14, 1879 $-$ April 14, 1955
}

% --------------------------------------------------------------------------
\subsection{What are prime numbers?}
\index{Prime number} \index{Number!prime}
Prime numbers are whole, positive numbers greater than or equal to $2$ that can
only be divided by 1 and themselves. All other natural numbers greater than or
equal to $2$ can be formed by multiplying prime numbers.

The {\em natural} \index{Number!natural} numbers $\mathbb{N}=\{1, 2, 3, 4,\cdots \}$ thus comprise 
\begin{itemize}
   \item the number $1$ (the unit value)
   \item the primes and
   \item the composite numbers.
\end{itemize}

Prime numbers are particularly important for 3 reasons:
\begin{itemize}
  \item In number theory, they are considered to be the basic components of
natural numbers, upon which numerous brilliant mathematical ideas are based.
  \item They are of extreme practical importance in modern
\index{Cryptography!modern} cryptography (public key \index{Cryptography!public key} cryptography). The most common public key procedure, invented at the end of
the 1970's, is \index{RSA} RSA encryption. Only using (large) prime numbers for
particular parameters can you guarantee that an algorithm is secure, both for
the RSA procedure and for even more modern procedures (digital
\index{Signatures!digital} signature, elliptic curves).
  \item The search for the largest known prime numbers does not have any
practical usage known to date, but requires the best computers, is an excellent
benchmark (possibility for determining the performance\index{Performance} of computers) and leads
to new calculation methods on many computers \\ (see also:
\href{http://www.mersenne.org/prime.htm}{\tt
http://www.mersenne.org/prime.htm}).
\end{itemize}
Many people have been fascinated by prime numbers over the past two millennia.
Ambition to make new discoveries about prime numbers has often resulted in
brilliant ideas and conclusions. The following section provides an easily
comprehensible introduction to the basics of prime numbers. We will also explain
what is known about the distribution (density, number of prime numbers in
particular intervals) of prime numbers and how prime number tests work.


% --------------------------------------------------------------------------
\subsection{Prime numbers in mathematics}\label{primesinmath}

Every whole number has a factor. The number 1 only has one factor,
itself, whereas the number $12$ has the six factors $1, 2, 3, 4,
6, 12$. Many numbers can only be divided by themselves and by $1$.
With respect to multiplication, these are the ``atoms'' in the
area of numbers. Such numbers are called prime numbers.

In mathematics, a slightly different (but equivalent) definition is used.

\begin{definition}\label{def-pz-prime}
A whole number $p \in {\bf N}$ is called prime \index{Number!prime} if
$p > 1$ and $p$ only possesses the trivial factors $\pm 1$ and $\pm p$.
\end{definition}


By definition, the number $1$ is not a prime number. In the following sections,
$p$ will always denote a prime number.

The sequence of prime numbers starts with $$ 2,~ 3,~ 5,~ 7, ~ 11, ~ 13, ~
17, ~ 19, ~ 23, ~ 29, ~ 31, ~ 37, ~ 41, ~ 43, ~ 47, ~ 53, ~ 59, ~ 61,
~ 67, ~ 71, ~ 73, ~ 79, ~ 83, ~ 89, ~ 97, \cdots . $$
The first 100 numbers include precisely 25 prime numbers. After this,
the percentage of primes constantly decreases. Prime numbers can be
factorised in a uniquely {\em trivial} way: 
$$5 = 1 \cdot 5,\quad  17 = 1 \cdot 17, \quad 1,013 = 1 \cdot 1,013,  \quad
1,296,409 = 1 \cdot 1,296,409.$$
All numbers that have $2$ or more factors not equal 1 are called 
\index{Number!composite} {\em composite} numbers. 
These include $$ 4 = 2 \cdot 2, \quad 6 = 2\cdot 3 $$ as well
as numbers that {\em look like primes}, but are in fact composite:
$$ 91 = 7 \cdot 13, \quad 161=7 \cdot 23, \quad 767 =13 \cdot 59. $$

\begin{theorem}\label{thm-pz-sqr}
Each whole number $m$ greater than $1$ possesses a lowest factor greater than
$1$. This is a prime number $p$. Unless $m$ is a prime number itself, then: $p$
is less than or equal to the square root of $m$.
\end{theorem}

All whole numbers greater than $1$ can be expressed as a product of prime
numbers --- in a unique way. This is the claim of the 1st fundamental theorem of
number theory (= fundamental theorem of arithmetic = fundamental building block
of all positive integers).\index{Number theory!fundamental theorem}

\begin{theorem}\label{thm-pz-prod}
Each element $n$ of the natural numbers greater than $1$ can be written as the
product $n = p_1 \cdot p_2 \dots p_m$ of prime numbers. If two such
factorisations $$n =  p_1 \cdot p_2 \cdot \cdots \cdot p_m = p'_1 \cdot p'_2 \cdots
p'_{m'}$$ are given, then they can be reordered such that $\;m = m'\;$ and for
all $i$:  $\;p_i = p'_i$. \\
($p_1, p_2, \dots, p_m$ are called the prime factors of n)

\end{theorem}

In other words: each natural number other than $1$ can be written as a product
of prime numbers in precisely one way, if we ignore the order of the factors.
The factors are therefore unique (the {\em expression as a product of factors}
is unique)! For example, $$ 60 = 2 \cdot 2 \cdot 3 \cdot 5 = 2^2\cdot 3^1 \cdot
5^1. $$
And this --- other than changing the order of the factors --- is the only way in
which the number $60$ can be factorised. If you allow numbers other than primes
as factors, there are several ways of factorising integers and the uniqueness \hypertarget{uniqueness}{} is
lost: $$ 60 = 1 \cdot 60 = 2 \cdot 30 = 4 \cdot 15 = 5 \cdot 12 =6 \cdot 10 = 2
\cdot 3 \cdot 10 = 2 \cdot 5 \cdot 6 = 3 \cdot 4 \cdot 5 = \cdots . $$

The following section is aimed more at those familiar with mathematical logic:
The 1st fundamental theorem only appears to be obvious \label{remFundTheoOfArithm}. We can construct
numerous other sets of numbers (i.e. other than positive whole numbers greater
than 1), for which numbers in the set cannot be expressed uniquely as a product
of the prime numbers of the set: In the set $M = \{1, 5, 10, 15, 20, \cdots\}$
there is no equivalent to the fundamental theorem under multiplication. The
first five prime numbers of this sequence are $5, 10, 15, 20, 30$ (note: $10$ is
prime, because $5$ is not a factor of $10$ in this set --- the result is not an
element of the given basic set $M$). Because the following applies in $M$: $$
100 = 5 \cdot 20 = 10 \cdot 10 $$ and $5, 10, 20$ are all prime numbers in this
set, the expression as a product of prime factors is not unique here.

% --------------------------------------------------------------------------
\subsection{How many prime numbers are there?}

For the natural numbers, the primes can be compared to elements in chemistry or
the elementary particles in physics (see \cite[p. 22]{Blum1999}).

Although there are only $92$ natural chemical elements, the number of prime
numbers is unlimited. Even the Greek, \index{Euclid} Euclid%
\footnote{Euclid,
a Greek mathematician of 4th and 3rd century B.C. He worked at the
Egyptian academy of Alexandria and wrote ``The Elements'', the most well 
known systematically textbook of the Greek mathematics.} 
knew this in the third century B.C.
\begin{theorem}[Euclid\footnote{The common usage of the term does not denote Euclid as the inventor of the theorem rather;
the true inventor is merely not as prominent. The theorem has already been distinguished
and proven in Euclid's Elements (Book IX, s. 20). The phraseology is remarkable due to 
the fact that the word infinite is not used. The text reads as followed
$$
O\acute{\iota}~\pi\varrho\tilde{\omega}\tau o \iota~\grave{\alpha}\varrho\iota\vartheta\mu o\grave{\iota}~
\pi\lambda\varepsilon\acute{\iota}o \upsilon\varsigma~\varepsilon\grave{\iota}\sigma\grave{\iota}~
\pi\alpha\nu\tau\grave{o}\varsigma~\tau o \tilde{\upsilon}~
\pi\varrho o \tau\varepsilon\vartheta\acute{\varepsilon}\nu\tau o \varsigma~
\pi\lambda\acute{\eta}\vartheta\ o \upsilon\varsigma~
\pi\varrho\acute{\omega}\tau\omega\nu~
\grave{\alpha}\varrho\iota\vartheta\mu\tilde{\omega}\nu,
$$
the English translation of which is: the prime numbers are more than 
any previously existing amount of prime numbers.
}]\label{thm-pz-euklid} % Ende der Fu�note
% Fussnote VOR "]" in [Euclid], damit kein Leerraum vor der Fussnotennummer.
% Vorher stand da: ...(Euclid). BLANK Fussnote und das Blank stoerte.
% Nun steht die Fussnote direkt hinter "Euclid" und vor der ")".
% Eigentlich h�tte ich sie gerne direkt hinter "(Euclid)", noch vor dem 
% automatisch gesetzten Punkt. 
The sequence of prime numbers does not discontinue.
Therefore, the quantity of prime numbers is infinite.
\end{theorem}
His proof that there is an infinite number of
primes is still considered to be a brilliant mathematical consideration and
conclusion today (proof by contradiction). He assumed that there is only a
finite number of primes and therefore a largest prime number. Based on this
assumption, he drew logical conclusions until he obtained an obvious
contradiction. This meant that something must be wrong. As there were no
mistakes in the chain of conclusions, it could only be the assumption that was
wrong. Therefore, there must be an infinite number of primes!

\hypertarget{euclid}{}
\paragraph{Euclid's proof by contradiction} goes as follows:

{\bf Assumption:} \quad There is a {\em finite} number of primes. \\*[4pt] {\bf
Conclusion:} \quad Then these can be listed $p_1 < p_2 < p_3 < \dots < p_n$,
where $n$ is the (finite) number of prime numbers. $p_n$ is therefore the
largest prime. Euclid now looks at the number $a = p_1 \cdot p_2 \cdots p_n +1$.
This number cannot be a prime number because it is not included in our list of
primes. It must therefore be divisible by a prime, i.e. there is a natural
number $i$ between $1$ and $n$, such that $p_i$ divides the number $a$. Of
course, $p_i$ also divides the product $a-1 = p_1 \cdot p_2 \cdots p_n$, because
$p_i$ is a factor of $a-1$. Since $ p_i $ divides the numbers $ a $ and $ a-1 $,
it also divides the difference of these numbers. Thus: $p_i$ divides  $a - (a-1)
= 1$. $p_i$ must therefore divide $1$, which is impossible. \\*[4pt] {\bf
Contradiction:} \quad Our assumption was false.

Thus there is an {\em infinite} number of primes
\hyperlink{primhfk}{(Cross-reference: overview under \ref{s:primhfk} of 
the number of prime numbers in various intervals)}.\par \vskip + 10pt

Here we should perhaps mention yet another fact which is initially somewhat surprising. 
Namely, in the prime numbers sequence $p_1, p_2, \cdots,$ gaps between prime numbers can have
an individually determined length $n$. It is undeniable that under the $n$
succession of natural numbers
$$(n+1)!+2,\cdots, (n+1)!+(n+1),
$$
none of them is a prime number since in order, the numbers $2,3,\cdots,(n+1)$  
are comprised respectively as real divisors. 
($n!$ means the product of the first $n$ natural numbers therefore 
$ n!= n*(n-1)*\cdots *2*1$).

% --------------------------------------------------------------------------
\subsection{The search for extremely large primes}

The largest prime numbers known today have several hundred
thousand digits, which is too big for us to imagine. The number of
elementary particles in the universe is ``only'' estimated to be a
$80$-digit number \hyperlink{grosord}{(See: overview under \ref{s:grosord}
of various orders of magnitude / dimensions)}.

% --------------------------------------------------------------------------
\hypertarget{MersenneNumbers01}{}
\subsubsection{Special number types -- Mersenne numbers} 
\index{Mersenne!number}

Almost all known huge prime numbers are special candidates, called
\index{Mersenne Marin} {\em Mersenne numbers} of the form $2^p -1,$ where $p$ is
a prime. Marin Mersenne (1588-1648) was a French priest and mathematician. Not
all Mersenne numbers are prime:

$$
\begin{array}{cl}
2^2 - 1 = 3 & \Rightarrow {\rm prime} \\
2^3 - 1 = 7 & \Rightarrow {\rm prime} \\
2^5 - 1 = 31    & \Rightarrow {\rm prime} \\
2^7 - 1 = 127    & \Rightarrow {\rm prime} \\
2^{11} - 1 = 2.047 = 23 \cdot 89    & \Rightarrow  {\rm NOT~prime} !
\end{array}
$$

\index{Number!Mersenne}\index{Mersenne!number}
\index{Mersenne!theorem} 

Even Mersenne knew that not all Mersenne numbers
are prime (see exponent $p = 11$). 
A prime Mersenne number \index{Mersenne!prime number} is called
Mersenne prime number.  \\
However, he is to be thanked for the interesting conclusion that a number of the form $2^n-1$ is not a prime number
if $n$ is a composite number:

\begin{theorem}[Mersenne]\label{thm-pz-mersenne} 
  If $2^n - 1$ is a prime number, then $n$ is also a prime number.
\end{theorem}

\begin{Proof}{}
The theorem of Mersenne can be proved by contradiction. We therefore assume that
there exists a composite natural number $ n $ (with real factorisation)  
$ n=n_1 \cdot n_2 $, with the property that $ 2^n -1 $ is a prime number.

From \begin{eqnarray*} (x^r-1)((x^r)^{s-1} + (x^r)^{s-2} + \cdots + x^r +1) & =
&  ((x^r)^s + (x^r)^{s-1} + (x^r)^{s-2} + \cdots + x^r) \\ &  & -((x^r)^{s-1} +
(x^r)^{s-2} + \cdots + x^r +1)  \\ & = & (x^r)^s -1 = x^{rs } -1,
\end{eqnarray*} we conclude \[ 2^{n_1 n_2} - 1 = (2^{n_1} -1)((2^{n_1})^{n_2 -1}
+ (2^{n_1})^{n_2 -2} + \cdots + 2^{n_1} + 1). \]
Because $ 2^n - 1 $ is a prime number, one of the above two factors on the
right-hand side must be equal to 1. This is the case if and only if $ n_1 =1 $
or $ n_2 =1$. But this contradicts our assumption. Therefore the assumption is
false. This means that there exists no composite number $ n, $ such that $ 2^n -
1 $ is a prime.
\end{Proof} 

\vskip + 5pt
\hypertarget{Mer-nums-not-always-prim}{}
Unfortunately this theorem only applies in one direction (the inverse 
statement does not apply, no equivalence): that means that there exist 
prime exponent for which the Mersenne number is {\bf not} prime (see the 
above example $2^{11}-1, $ where $11$ is prime, but $2^{11}-1$ not).

Mersenne claimed that $2^{67}-1$ is a prime number. There is also a mathematical
history behind this claim: it first took over 200 years before \index{Lucas
  Edouard} Edouard Lucas (1842-1891) proved that this number is composite.
However, he argued indirectly and did not name any of the factors. Then Frank
Nelson showed in 1903 which factors make up this composite number: 
$$ 2^{67} -1
=147, 573, 952, 589, 676, 412, 927 = 193, 707, 721 \cdot 761, 838, 257, 287. $$
He admitted to having worked 20 years on the factorisation 
\index{Factorisation} (expression as a product of prime factors)\footnote{%
Using CrypTool\index{CrypTool} v1.3 you can factorize numbers in the 
following way: menu {\bf Indiv. Procedures \textbackslash{} RSA Demonstration 
\textbackslash{} Factorisation of a Number}.
}
of this 21-digit decimal number!

Due to the fact that the exponents of the Mersenne numbers
\index{Mersenne!number} do not use all
natural numbers, but only the primes, the {\em experimental space} is limited
considerably. The currently known Mersenne prime numbers 
\index{Mersenne!prime number} have the exponents 
$$
\begin{array}{c}
2; ~ 3; ~ 5; ~ 7; ~ 13; ~ 17; ~ 19; ~ 31; ~ 61; ~ 89; ~ 107; ~ 127;
~ 521; ~ 607; ~ 1,279; ~ 2,203; ~ 2,281; ~ 3,217; ~ 4,253;\\
 4,423; ~ 9,689; ~ 9,941, ~ 11,213; ~ 19,937; ~ 21,701; ~ 23,207; ~ 44,497; ~
86,243; ~ 110,503; ~ 132,049; \\
 216,091; ~ 756,839; ~ 859,433; ~ 1,257,787; ~ 1,398,269; ~ 2,976,221; ~ 3,021,377; ~
6,972,593; \\13,466,917.
\end{array}
$$
Thus $39$ Mersenne prime numbers are currently known
\index{Prime number!Mersenne}\index{Mersenne!prime number}. 
For the first 38 Mersenne prime numbers we know that this list is complete.
The exponents until the 39th Mersenne prime number have not yet been checked
completely (see chapter 2.5 prime number tests\index{Prime number!test}).

The $19$th number with the exponent $4,253$ was the first with at least $1,000$ digits in  decimal system
(the mathematician Samual \index{Yates Samual} Yates coined the expression {\em
titanic} \index{Prime number!titanic} prime for this; it was discovered by
Hurwitz in 1961); the $27$th number with the exponent $44,497$ was the first
with at least $10,000$ digits in the decimal system (Yates coined the expression
\index{Prime number!gigantic}  {\em gigantic} prime for this. These names are
now long outdated).

These numbers can be found at the following URLs:
\vspace{-10pt}
\begin{itemize}
  \item[] \href{http://reality.sgi.com/chongo/prime/prime_press.html}
               {\texttt{http://reality.sgi.com/chongo/prime/prime\_press.html}}\\
          \href{http://www.utm.edu/}
               {\texttt{http://www.utm.edu/}}
\end{itemize}


\vskip +15 pt
\paragraph{M-37 -- January 1998}\index{Mersenne!prime number!M-37}\mbox{}
\vskip +10 pt

The 37th Mersenne prime, $$ 2^{3,021,377} - 1 $$
was found in January 1998 and has 909,526
digits in the decimal system, which corresponds to 33 pages in the newspaper!


\vskip +15 pt
\paragraph{M-38 -- June 1999}\index{Mersenne!prime number!M-38}\mbox{}
\vskip +10 pt

The 38th Mersenne prime, called M-38, $$ 2^{6,972,593} - 1 $$
was discovered in June 1999 and has $2,098,960$ digits in the decimal system
(that corresponds to around 77 pages in the newspaper).

Right now (Feb. 2003) all exponents smaller than $ 6,972,593 $ have been
checked: so we can be certain, that this is really the 38th Mersenne prime
number.
This means there is no other Mersenne prime number between the 37th 
Mersenne prime number and this one.


\vskip +15 pt
\paragraph{M13466917 -- December 2001} \index{Mersenne!prime number!M-39}\mbox{}
\vskip +10 pt

This number was discovered as 39th Mersenne prime (and already called M-39,
despite it has not been proven yet, whether no further Mersenne prime
numbers between M-38 und M13466917 do indeed exist), $$2^{13,466,917}-1,$$
at December 6, 2001 -- more exactly, the verification of this number,
found at November 14, 2001 by the Canadian student Michael Cameron, was
successfully completed. This number has about 4 million decimal digits
(exactly 4,053,946 digits).
Trying only to print this number 
$$(924947738006701322247758 \cdots 1130073855470256259071)$$
would require around 200 pages in the Financial Times.


\vskip +15 pt
\paragraph{GIMPS}\index{GIMPS}\mbox{}
\vskip +10 pt

Discovering the 39th Mersenne prime the GIMPS project (Great Internet Mersenne Prime Search)\index{GIMPS} founded in 1996 already discovered for the 5th time the greatest Mersenne number to be proven being prime
({\href{http://www.mersenne.org} {\tt http://www.mersenne.org}}).

Now more than 130,000 volunteers, amateurs and experts, are working for the GIMPS project. They connect their computers into the so called ``primenet'', organized by the company entropia, to find such numbers using distributed computer programs.

Currently the 4 largest known primes are Mersenne prime numbers. Only the 5th biggest known prime is another number type: it belongs to the so called  \hyperlink{generalizedFermatprimes}{generalized Fermat primes}.

% --------------------------------------------------------------------------
\vskip +15 pt
\subsubsection{EFF}\index{EFF}
This search is also spurred on by a competition started by the non-profit
organisation EFF (Electronic Frontier Foundation) using the means of an unknown
donator. The participants are rewarded with a total of 500,000 USD if they find
the longest prime number. In promoting this project, the unknown donator is not
looking for the quickest computer, but rather wants to draw people's attention
to the opportunities offered by {\em cooperative networking} \\
{\href{http://www.eff.org/coopawards/prime-release1.html}{\tt http://www.eff.org/coopawards/prime-release1.html}}

The discoverer of M-38 received 50,000 USD from the EFF for discovering 
the first prime with more than 1 million decimal digits. 
The next prize of 100,000 USD offered by EFF is for a proven prime with more
than 10 million decimal digits.
%{\href{http://www.octocad.demon.co.uk/mersenne/prime.htm }{\tt http://www.octocad.demon.co.uk/mersenne/prime.htm}}.

Edouard Lucas\index{Lucas Edouard} (1842-1891) held the record for the
longest prime number for over 70 years by proving that $2^{127}-1$ is prime.
No new record is likely to last that long.


% --------------------------------------------------------------------------
\subsection{Prime number tests}
\index{Prime number!test}
In order to implement secure encryption procedures we need extremely large prime
numbers (in the region of $2^{2,048}$, i.e. numbers with $600$ digits in the
decimal system!).

Up to now we have looked for the prime factors in order to decide whether
a number is prime. However, even if the smallest prime factor is enormous, the
search takes too long. Factorising numbers using systematic computational
division or using the \hyperlink{SieveEratosthenes01}{sieve of Eratosthenes} 
\index{Eratosthenes!sieve} is only feasible using current computers for
numbers with up to around $20$ digits in the decimal system.
The biggest number factorized into its 2 almost equal prime factors 
has 158 digits (see \hyperlink{C158-chap3}{chapter 3.11.4}).


However, if we know something about the {\em construction} of the number in
question, there are extremely highly developed procedures that are much quicker.
These procedures can determine the primality attribute of a number, but they
cannot determine the prime factors of a number, if it is compound.

\hypertarget{FermatNumbers01}{}
In the 17th century, Fermat wrote to Mersenne \index{Mersenne Marin} that he presumed that all numbers
of the form $$ F(n) = 2^{2^n} + 1 $$ are prime for all whole numbers $n$ greater
than or equal to $0$ (\hyperlink{FermatNumbers02}{see below}) 
\index{Number!Fermat}\index{Fermat!number}

As early as in the 19th century, it was discovered that the $29$-digit number $$
F(7) = 2^{2^7} + 1 $$ is not prime. However, it was not until 1970 that
Morrison/Billhart managed to factorise it.
\begin{eqnarray*}
F(7) & = & 340,282,366,920,938,463,463,374,607,431,768,211,457 \\
& = & 59, 649, 589, 127, 497, 217 \cdot  5,704,689,200,685,129,054,721
\end{eqnarray*}

\vspace{12pt}
Despite Fermat was wrong with this supposition, he is the originator of
an important theorem in this area: Many rapid prime number tests are 
based on the (little) Fermat theorem put forward by Fermat in 1640
(\hyperlink{KleinerSatzFermat-chap3}{see chapter 3.8.3}).

\hypertarget{KleinerSatzFermat-chap2}{}
\index{Fermat!little theorem}
\begin{theorem}[``little'' Fermat]\label{thm-pz-fermat1}
Let $p$ be a prime number and $a$ be any whole number, then for all $a$ $$a^p
\equiv a \; {\rm mod} \; p.$$
This could also be formulated as follows: \\ Let $p$ be a prime number and $a$
be any whole number that is not a multiple of $p$ (also $a \not\equiv 0 \; {\rm
mod} \; p$), then $a^{p-1} \equiv 1 \; {\rm mod} \; p$.
\end{theorem}

If you are not used to calculating with remainders (modulo), please simply
accept the theorem. What is important here is that this sentence implies that if
this equation is not met for any whole number $a$, then $p$ is not a prime! The
tests (e.g. for the first formulation) can easily be performed using the {\em
test basis} $a = 2$.

This gives us a criterion for non-prime numbers, i.e. a negative test, but no
proof that a number $a$ is prime. Unfortunately Fermat's theorem does not apply
--- otherwise we would have a simple proof of the prime number property (or to
put it in other words, we would have a simple prime number criterion).

Comment: Numbers n that have the property $$ 2^n \equiv 2 \;{\rm mod}\; n $$ but
are not prime are called \index{Prime number!pseudo prime}\index{Number!pseudo prime} {\em pseudo prime numbers}. The first pseudo prime number (i.e. not a
prime) is $$ 341 = 11 \cdot 31. $$
There are numbers that pass the Fermat test with all bases and yet are not
prime: these numbers are called \index{Number!Carmichael} {\em
Carmichael numbers}. The first of these is $$ 561 = 3 \cdot 11 \cdot 17. $$

A stronger test is provided by\index{Miller}\index{Rabin} Miller/Rabin: it is
only passed by so-called {\em strong pseudo prime numbers}. Again, there are
strong pseudo prime numbers that are not primes, but this is much less often the
case than for (simple) pseudo prime numbers. The smallest strong pseudo prime
number base $2$ is $$ 15,841 = 7 \cdot 31 \cdot 73. $$
If you test all 4 bases, $2, 3, 5$ and $7$, you will find only one strong
pseudo prime number up to $25 \cdot 10^9$, i.e. a number that passes the 
test and yet is not a prime number.

More extensive mathematics behind the Rabin test delivers the probability that
the number examined is prime (such probabilities are currently around $10^{-
60}$).

Detailed descriptions of tests for finding out whether a number is prime
can be found on Web sites such as:
\vspace{-10pt}
\begin{itemize}
  \item[] \href{http://www.utm.edu/research/primes/mersenne.shtml}
               {\texttt{http://www.utm.edu/research/primes/mersenne.shtml}} \\
          \href{http://www.utm.edu/research/primes/prove/index.html}
               {\texttt{http://www.utm.edu/research/primes/prove/index.html}} 
\end{itemize}


% --------------------------------------------------------------------------
\vskip +20 pt
\subsection{Further special number types and the search for a formula for primes}
\index{Prime number!formula}
There are currently no useful, open (i.e. not recursive) formulae known that
only deliver prime numbers (recursive means that in order to calculate the
function the same function is used with a smaller variable). Mathematicians
would be happy if they could find a formula that leaves gaps (i.e. does not
deliver all prime numbers) but does not deliver any composite (non-prime)
numbers.

Ideally, we would like, for the number $n$, to immediately be able to 
obtain the $n$-th prime number, i.e. for $f(8) = 19\,$ or for  $f(52) = 239$.

Ideas for this can be found at
\vspace{-10pt}
\begin{itemize}
  \item[] {\href{http://www.utm.edu/research/primes/notes/faq/p_n.html}
          {\tt http://www.utm.edu/research/primes/notes/faq/p\_n.html}}.
\end{itemize}


Cross-reference:  \hyperlink{ntePrimzahl}{the table under \ref{s:ntePrimzahl}}
contains the precise values for the $n$th prime numbers for selected $ n.$
\\

The following enumeration contains the most common ideas for 
``prime number formulae'':
\begin{enumerate}
\item {\em Mersenne numbers}  $f(n) = 2^n - 1$  \quad for $ n $ prime: \\
    As shown \hyperlink{MersenneNumbers01}{above}, this formula seems
    to deliver relatively large prime numbers but - as for $n=11$ 
    [$f(n)=2,047$] - it is repeatedly the case that the result even with
    prime exponents is \hyperlink{Mer-nums-not-always-prim}{not} prime.\\
    Today, all the Mersenne primes having less than around 2,000,000 digits are 
    known (M-38\index{M-38}):
\vspace{-10pt}
\begin{itemize}
  \item[] {\href{http://perso.wanadoo.fr/yves.gallot/primes/index.html}
           {\tt http://perso.wanadoo.fr/yves.gallot/primes/index.html}}
\end{itemize}
\item $F(k,n) = k \cdot 2^n \pm 1$  for $ n $ prime and $ k $ small primes:\\
   For this generalisation of the Mersenne 
   numbers\index{Mersenne!number!generalized}
   there are (for small $k$)
   also extremely quick prime number tests (see \cite{Knuth1981}). This can
   be performed in practice using software such as the Proths software from
   Yves Gallot\index{Gallot Yves}
\vspace{-10pt}
\begin{itemize}
  \item[] {\href{http://www.prothsearch.net/index.html}
    {\tt http://www.prothsearch.net/index.html}}.
\end{itemize}


\vskip +15 pt
\hypertarget{FermatNumbers02}{}
\item {\em Fermat numbers}\footnote{The Fermat prime numbers play a role in circle division.
As proven by Gauss\index{Gauss}, a regular $p$-edge
can only be constructed with the use of a pair of compasses and a ruler, when $p$ is a Fermat prime number.}
\index{Number!Fermat}\index{Fermat!number}  $F(n) = 2^{2^n} + 1$:\\
As mentioned \hyperlink{FermatNumbers01}{above}, Fermat wrote to Mersenne regarding his assumption, that all numbers of this type are primes.
Surprisingly he would have been able obtain a positive result using the negative
prime number test for $n=5$ based on his small theorem.
$$
\begin{array}{lll}
F(0) = 2^{2^0} + 1  = 2^1 + 1 & = 3 &   \mapsto {\rm ~prime}  \\
F(1) = 2^{2^1} + 1  = 2^2 + 1 & = 5 &   \mapsto {\rm ~prime}  \\
F(2) = 2^{2^2} + 1  = 2^4 + 1 & = 17 &  \mapsto {\rm ~prime}  \\
F(3) = 2^{2^3} + 1  = 2^8 + 1 & = 257 & \mapsto {\rm ~prime}  \\
F(4) = 2^{2^4} + 1  = 2^{16} + 1 &  = \mbox{65,537} &  \mapsto {\rm ~prime}  \\
F(5) = 2^{2^5} + 1  = 2^{32} + 1 &  = \mbox{4,294,967,297} = 641 \cdot \mbox{6,700,417} &  \mapsto {\rm ~NOT~prime} ! \\
F(6) = 2^{2^6} + 1  = 2^{64} + 1 &  = \mbox{18,446,744,073,709,551,617} \\
                                 &  = \mbox{274,177} \cdot \mbox{67,280,421,310,721} & \mapsto {\rm ~NOT~prime} !
\end{array} 
$$

    Within the project ``Distributed Search for Fermat Number Dividers''
    offered by Leonid Durman there is also progress in finding new monster
    primes: 
\vspace{-10pt}
\begin{itemize}
  \item[] {\href{http://www.fermatsearch.org/}
       {\tt http://www.fermatsearch.org/}}\\
       This website links to other webpages in Russian, Italian and German.
\end{itemize}
    The discovered factors can be compound integers or primes.
     
    On February 22, 2003 John Cosgrave discovered 
    \begin{itemize} 
     \item the largest composite Fermat number to date and
     \item the largest prime non-Mersenne number so far with 645,817 digits.
    \end{itemize}
    
    The Fermat number
    $$ F[2,145,351] = 2^{(2^{2,145,351})} + 1 $$ 
    is divisible by the prime
    $$ p = 3*2^{2,145,353} + 1 $$ \\
    This prime p is the largest known prime generalized Mersenne 
    number\index{Mersenne!number!generalized}. 
    It was discovered only a few days after Michael Angel has come across the 
    largest generalized Fermat number (\hyperlink{generalizedFermatprimes}
    {see below}).  This moved GIMPS' \index{GIMPS} first prime discovery,
    M-35 = M1398269, into 7th place.

    This work was done using NewPGen from Paul Jobling's, PRP from 
    George Woltman's, Proth from Yves Gallot's programs\index{Gallot Yves} and also the
    Proth-Gallot group at St. Patrick's College, Dublin.

    More details are in
    \vspace{-10pt}
    \begin{itemize}
      \item[] \href{http://www.fermatsearch.org/history/cosgrave_record.htm/}
          {\texttt{http://www.fermatsearch.org/history/cosgrave\_record.htm/}}
    \end{itemize}

%    ({\href{http://perso.wanadoo.fr/yves.gallot/primes/index.html}
%           {\tt http://perso.wanadoo.fr/yves.gallot/primes/index.html}}).
% M.E. ist die Zahl auf seiner Webseite zu gro�, da M-39 "nur" 4 Mio. Stellen hat !
%    Heute kennt man alle Fermatschen Primzahlen mit bis zu 2.000.000.000 ??? 
%    Dezimalstellen. \\


\vskip +15 pt
\hypertarget{generalizedFermatprimes}{}
\item {\em Generalized Fermat numbers}\footnote{%
The base of this power is no longer restricted to 2 !}  $F(b,m) = b^{2^m} + 1$:
\index{Fermat!number!generalized}  \\
    Generalized Fermat numbers are more numerous than Mersenne numbers at
    equal size and many of them are waiting to be discovered to fill the
    big gaps between the Mersenne primes already found or still undiscovered.
    Progress in number theory made it possible that numbers, where the
    representation is not limited to the base 2, can be tested at almost
    the same speed than a Mersenne number.
    
    Yves Gallot\index{Gallot Yves} wrote the program Proth.exe to investigate generalized
    Fermat numbers.
    
    Using this program at February 16, 2003 Michael Angel discovered the
    largest of them till then with 628,808 digits, which so became the
    5th largest known prime number:
    $$ b^{2^{17}} + 1  =  62,722^{131,072} + 1. $$ 
    This moved GIMPS' \index{GIMPS} first prime discovery, M-35 = M1398269,
    into 6th place.

    More details are in
    \vspace{-10pt}
    \begin{itemize}
      \item[] \href{http://www.prothsearch.net/index.html}
                   {\texttt{http://www.prothsearch.net/index.html}} \\
              \href{http://perso.wanadoo.fr/yves.gallot/primes/index.html}
               {\texttt{http://perso.wanadoo.fr/yves.gallot/primes/index.html}} 
    \end{itemize}



  \vskip +15 pt
  \item Carmichael numbers: see above.

  \item Pseudo prime numbers: see above.
  
  \item Strong pseudo prime numbers: see above.
  
  \item Idea based on \hyperlink{euclid}{Euclid's proof} (infinite many prime
numbers) $p_1 \cdot p_2 \cdots p_n +1$:

$$
\begin{array}{lll}
2{\cdot}3 +1 &      = 7 &          \mapsto {\rm ~prime} \\
2{\cdot}3{\cdot}5 +1 &      = 31    &      \mapsto {\rm ~prime} \\
2{\cdot}3{\cdot}5{\cdot}7 +1 &      = 211   &      \mapsto {\rm ~prime} \\
2{\cdot}3{\cdots}11 +1 &        = 2,311  &      \mapsto {\rm ~prime} \\
2\cdot3 \cdots 13 +1 &  = 59 \cdot 509 &    \mapsto {\rm ~NOT~prime} ! \\
2\cdot3 \cdots 17 +1 &  = 19 \cdot 97 \cdot 277 &   \mapsto {\rm ~NOT~prime} !
\\
\end{array}
$$


  \item As above but except $+1$: $p_1 \cdot p_2 \cdots p_n -1$

$$
\begin{array}{lll}
2\cdot 3 -1     &   = 5 &   \mapsto {\rm ~prime} \\
2\cdot 3 \cdot  5  -1   &   = 29 &  \mapsto {\rm ~prime} \\
2\cdot 3 \cdots 7  -1   &   = 11 \cdot 19 & \mapsto {\rm ~NOT~prime} ! \\
2\cdot 3 \cdots 11 -1   &   = 2,309 &    \mapsto {\rm ~prime} \\
2\cdot 3 \cdots 13 -1   &   = 30,029 &    \mapsto {\rm ~prime} \\
2\cdot 3 \cdots 17 -1    &  = 61 \cdot 8,369 &   \mapsto {\rm ~NOT~prime!}
\end{array} 
$$

  \item \index{Euclidean number} {\em Euclidean numbers} $e_n = e_0 
    \cdot e_1 \cdots e_{n-1} + 1$  with $n$ greater than or equal to 
    $1$ and $e_0 := 1$. \\
    $e_{n-1}$ is not the $(n-1)$th prime number, but the number 
    previously found here.
    Unfortunately this formula is not open but recursive.
    The sequence starts with 

$$
\begin{array}{lll}
e_1 = 1 + 1 &   = 2 &   \mapsto {\rm ~prime} \\
e_2 = e_1 + 1   &   = 3 &   \mapsto {\rm ~prime} \\
e_3 = e_1 \cdot e_2 + 1 &   = 7 &   \mapsto {\rm ~prime} \\
e_4 = e_1 \cdot e_2 \cdot e_3 + 1 & = 43 &  \mapsto {\rm ~prime} \\
e_5 = e_1 \cdot e_2 \cdots e_4 + 1 &    = 13 \cdot 139 &    \mapsto {\rm
~NOT~prime} ! \\
e_6 = e_1 \cdot e_2 \cdots e_5 + 1 &    = 3,263,443 &   \mapsto {\rm ~prime} \\
e_7 = e_1 \cdot e_2 \cdots e_6 + 1 &    = 547 \cdot 607 \cdot 1,033 \cdot 31,051
& \mapsto {\rm ~NOT~prime} ! \\
e_8 = e_1 \cdot e_2 \cdots e_7 + 1 &    = 29,881\cdot 67,003 \cdot 9,119,521
\cdot 6,212,157,481 & \mapsto {\rm ~NOT~prime} !
\end{array} 
$$

$e_9, \cdots, e_{17}$ are also composite, which means that this formula is not
particularly useful. Comment: However, what is special about these numbers
is that any pair of them does not have a common factor other than $1$. They are
therefore\index{Prime number!relative prime}\index{Relatively prime} {\em relatively prime}.


  \item $f(n) = n^2 + n + 41$:\\
  This sequence starts off very {\em promisingly},   but is far from being a
proof.

$$
 \begin{array}{ll}
f(0) = 41 & \mapsto {\rm ~prime} \\
f(1) = 43 & \mapsto {\rm ~prime} \\
f(2) = 47 & \mapsto {\rm ~prime} \\
f(3) = 53 & \mapsto {\rm ~prime} \\
f(4) = 61 & \mapsto {\rm ~prime} \\
f(5) = 71 & \mapsto {\rm ~prime} \\
f(6) = 83 & \mapsto {\rm ~prime} \\
f(7) = 97 & \mapsto {\rm ~prime} \\
\vdots \\
f(39) = 1,601 & \mapsto {\rm ~prime} \\
f(40) = 11 \cdot 151 &  \mapsto {\rm ~NOT~prime}! \\
f(41) = 41 \cdot 43 &   \mapsto {\rm ~NOT~prime}! \\
\end{array} 
$$

The first $40$ values are prime numbers (which have the obvious regularity that
their difference starts with $2$ and increases by $2$ each time), but the $41$th
and $42$th values are not prime numbers. 
It is easy to see that $f(41)$ cannot be a prime number:
    $f(41) = 41^2 + 41 + 41 = 41 (41 + 1 + 1) = 41 \cdot 43$.

  \item $f(n) = n^2 - 79 \cdot n + 1,601$: \\
    This function delivers prime numbers for all values from $n=0$ to $n=79$.
Unfortunately $f(80) = 1,681 = 11 \cdot 151$ is not a prime number. To this
date, no function has been found that delivers more prime numbers in a row. On
the other hand, each prime occurs twice (first in the decreasing then in the
increasing sequence), which means that the algorithm delivers a total of 40
difference prime values (the same ones as the function from point 11).
$$
\begin{array}{|ll||ll|}
\hline
f(0) = 1,601    & \mapsto {\rm ~prime} &  f(28) = 173    & \mapsto {\rm ~prime}
\\
f(1) = 1,523    & \mapsto {\rm ~prime} &  f(29) = 151    & \mapsto {\rm ~prime}
\\
f(2) = 1,447    & \mapsto {\rm ~prime} &  f(30) = 131 & \mapsto {\rm ~prime} \\
f(3) = 1,373    & \mapsto {\rm ~prime} &  f(31) = 113 & \mapsto {\rm ~prime} \\
f(4) = 1,301    & \mapsto {\rm ~prime} &  f(32) = 97 & \mapsto {\rm ~prime} \\
f(5) = 1,231    & \mapsto {\rm ~prime} &  f(33) = 83 & \mapsto {\rm ~prime} \\
f(6) = 1,163    & \mapsto {\rm ~prime} &  f(34) = 71 & \mapsto {\rm ~prime} \\
f(7) = 1,097    & \mapsto {\rm ~prime} &  f(35) = 61 & \mapsto {\rm ~prime} \\
f(8) = 1,033    & \mapsto {\rm ~prime} &  f(36) = 53 & \mapsto {\rm ~prime} \\
f(9) = 971  & \mapsto {\rm ~prime} &  f(37) = 47 & \mapsto {\rm ~prime} \\
f(10) = 911 & \mapsto {\rm ~prime} &  f(38) = 43 & \mapsto {\rm ~prime} \\
f(11) = 853 & \mapsto {\rm ~prime} &  f(39) = 41 & \mapsto {\rm ~prime} \\
f(12) = 797 & \mapsto {\rm ~prime} &  f(40) = 41 & \mapsto {\rm ~prime} \\               
f(13) = 743 & \mapsto {\rm ~prime} &  f(41) = 43 & \mapsto {\rm ~prime} \\               
f(14) = 691 & \mapsto {\rm ~prime} &  f(42) = 47 & \mapsto {\rm ~prime} \\               
f(15) = 641 & \mapsto {\rm ~prime} &  f(43) = 53 & \mapsto {\rm ~prime} \\               
f(16) = 593 & \mapsto {\rm ~prime} &  \cdots  &  \\                                      
f(17) = 547 & \mapsto {\rm ~prime} &  f(77) = 1,447  & \mapsto {\rm ~prime} \\           
f(18) = 503 & \mapsto {\rm ~prime} &  f(78) = 1,523  & \mapsto {\rm ~prime} \\           
f(19) = 461 & \mapsto {\rm ~prime} &  f(79) = 1,601  & \mapsto {\rm ~prime} \\           
f(20) = 421 & \mapsto {\rm ~prime} &  f(80) = 11 \cdot 151 & \mapsto {\rm ~NOT~prime!} \\
f(21) = 383 & \mapsto {\rm ~prime} &  f(81) = 41 \cdot 43 & \mapsto {\rm ~NOT~prime!} \\ 
f(22) = 347 & \mapsto {\rm ~prime} &  f(82) = 1,847  & \mapsto {\rm ~prime} \\           
f(21) = 383 & \mapsto {\rm ~prime} &  f(83) = 1,933  & \mapsto {\rm ~prime} \\           
f(22) = 347 & \mapsto {\rm ~prime} &  f(84) = 43 \cdot 47 &  \mapsto {\rm ~NOT~prime!} \\
f(23) = 313 & \mapsto {\rm ~prime} & & \\
f(24) = 281 & \mapsto {\rm ~prime} & & \\
f(25) = 251 & \mapsto {\rm ~prime} & & \\
f(26) = 223 & \mapsto {\rm ~prime} & & \\
f(27) = 197 & \mapsto {\rm ~prime} & & \\
\hline
\end{array} 
$$

\newpage
\item Polynomial functions $f(x) = a_n x^n + a_{n-1}x^{n-1} + \cdots + a_1 x^1 +
a_0$  ($a_i$ in ${\mathbb Z}$, $n \geq 1$):

    There exists no such polynomial that for all $x$ in ${\mathbb Z}$ only
delivers prime values.
    For a proof of this, please refer to \cite[p. 83 f.]{Padberg1996}, where you
will also find further details about prime number formulae.

    This means there is no hope in looking for further formulae 
    similar to that of type 11. or 12.

  \item  \index{Catalan number} Catalan, after whom the so-called
{\em Catalan numbers} $A(n) = (1 / (n+1) ) * (2n)! / (n!)^2$ are named,
conjectured that $C_4$ is a prime:
$$
 \begin{array}{l}
C_0 = 2, \\
C_1 = 2^{C_0} - 1,  \\
C_2 = 2^{C_1} - 1,  \\
C_3 = 2^{C_2} - 1, \\
C_4 = 2^{C_3} - 1, \cdots \\
\end{array}
$$
    (see {\href{http://www.utm.edu/research/primes/mersenne.shtml}{\tt
http://www.utm.edu/research/primes/mersenne.shtml}} under Conjectures and
Unsolved Problems).

    This sequence is also defined recursively and increases extremely quickly.
Does it only consist of primes?
$$
\begin{array}{lll}
C(0) = 2 & & \mapsto {\rm \:prime}\\
C(1) = 2^2 - 1 &    = 3 & \mapsto {\rm \:prime}\\
C(2) = 2^3 - 1 &    = 7 & \mapsto {\rm \:prime} \\
C(3) = 2^7 - 1 &    = 127& \mapsto {\rm \:prime} \\
C(4) = 2^{127} - 1 &      = 170, 141, 183, 460, 469, 231, 731, 687, 303, 715,
884, 105, 727 & \mapsto {\rm \:prime} \\
\end{array} 
$$
It is not (yet) known whether $C_5$ and higher elements are prime, but this is
not very likely.
In any case, it has not been proved that this formula delivers only primes.
\end{enumerate}

\vskip +10 pt
% --------------------------------------------------------------------------
\subsection{Density and distribution of the primes}

As Euclid discovered, there is an infinite number of primes. However, some
infinite sets are {\em denser} \index{Prime number!density} than others. Within
the set of natural numbers, there is an infinite number of even, uneven and
square numbers.

The following proves that there are more even numbers than square ones:
\begin{itemize}
  \item the size of the $n$th element: \\
    The $n$th element of the even numbers is $2n$; the $n$th element of the
square numbers is $n^2$. Because for all $n>2$: $2n < n^2$, the $n$th even
number occurs much earlier than the $n$th square number.
    Thus the even numbers are distributed more densely and we can say that there
are more even numbers than square ones.
  \item the number of values that are less than or equal to a certain {\em
maximum value} $x$ in ${\mathbb R}$ is: \\
    There are $[x/2]$ such even numbers and $[\sqrt{x}]$ square numbers. Because
for large $x$ the value $x/2$ is much greater than the square root of $2$, we
can again say that there are more even numbers.
\end{itemize}
\vskip +3 pt
\begin{theorem}\label{thm-pz-density}
For large n: The value of the $n$-th prime $P(n)$ is asymptotic to $n \cdot
ln(n)$, i.e. the limit of the relation  $P(n)/(n\cdot \ln n)$ is equal to $1$ if
$n$ tends to infinity.
\end{theorem}


The definition is similar for the number of prime numbers $PI(x)$
that do not exceed the maximum value $x$:

\begin{theorem}\label{thm-pz-pi-x}
$PI(x)$  is asymptotic to  $x / ln(x)$.
\end{theorem}


This is the \index{Prime number!theorem} \textbf{prime number theorem}. It was
put forward by Legendre \index{Legendre} (1752-1833) and Gauss\index{Gauss} (1777-1855) but not proved until
over 100 years later.

\hyperlink{primhfk}{(Cross-reference: overview under \ref{s:primhfk} of the number of
prime numbers in various intervals).}

For large $n$, $P(n)$  lies between  $2n$  and  $n^2$. This means that there are
fewer prime numbers than even natural numbers but more prime numbers than
square numbers.

These formulae, which only apply when n tends to infinity, can be replaced by
more precise formulae.
For $x \geq 67$:
$$ ln(x) - 1,5 < x / PI(x) < ln(x) - 0,5 $$
Given that we know $PI(x)  =  x / \ln x$ only for very large $x$ ($x$ tending
towards infinity), we can create the following overview:
$$
\begin{array}{ccccc}
x     &  ln(x)  &  x / ln(x) & PI(x)(counted) &       PI(x) / (x/ln(x)) \\
10^3  &  6.908  &   144      &  168        &       1.160 \\
10^6  &  13.816 &   72,386    &  78,498          &       1.085 \\
10^9  & 20.723  &   48,254,942 &  50,847,534       &       1.054
\end{array}
$$

For a binary number (number in the binary system) of the length $250$ bits
($2^{250}$ is approximately = $1.809 251 * 10^{75}$): $$ PI(250) = 2^{250} /
(250 \cdot \ln 2) \; {\rm is~approximately} \; =  2^{250} /173.28677 = 1.045 810
\cdot 10^{73}. $$
We can therefore expect that the set of numbers with a bit length of less than
250 contains approximately $10^{73}$ primes (a reassuring result?!).

We can also express this as follows: Let us consider a {\em random} natural
number $n$. Then the probability that this number is prime is around $1 / \ln(n)$.
For example, let us take numbers in the region of $10^{16}$. Then we must
consider $ 16 \cdot \ln 10 = 36,8 $ numbers (on average) until we find a prime.
A precise investigation shows: There are $10$ prime numbers between $10^{16}-
370$ and $10^{16}-1$.

Under the heading {\em How Many Primes Are There} at
\vspace{-10pt}
\begin{itemize}
  \item[] \href{http://www.utm.edu/research/primes/howmany.shtml}
               {\tt http://www.utm.edu/research/primes/howmany.shtml}
\end{itemize}
\vspace{-10pt}
you will find numerous other details.

Using the following Web site:
\vspace{-10pt}
\begin{itemize}
  \item[] \href{http://www.math.Princeton.EDU/~arbooker/nthprime.html}
               {\tt http://www.math.Princeton.EDU/\~{}arbooker/nthprime.html}
\end{itemize}
\vspace{-10pt}
you can easily determine $PI(x)$.


The \textbf{distribution} of primes displays several irregularities for which no
�system� has yet been found: On the one hand, many occur closely together, like
$2$ and $3,$ $ 11$ and $13, $ $ 809$ and $811$, on the other hand large gaps
containing no primes also occur. For example, no primes lie between $113$ and
$127, $ $ 293$ and $307, $ $317$ and $331, $ $ 523$ and $541, $ $ 773$ and $787,
$ $ 839$ and $853$ as well as between $887$ and $907$. \\
For details, please see: 
\vspace{-10pt}
\begin{itemize}
  \item[] \href{http://www.utm.edu/research/primes/notes/gaps.html}
               {\tt http://www.utm.edu/research/primes/notes/gaps.html}
\end{itemize}

This is precisely part of what motivates mathematicians to discover its secrets.

%----------------------------------------
\index{Eratosthenes!sieve}
\paragraph{Sieve of Eratosthenes}\mbox{}
\hypertarget{SieveEratosthenes01}{}

An easy way of calculating all $PI(x)$ primes less than or equal to $x$ is to
use the sieve of Eratosthenes. In the 3rd century B.C., he found an
extremely easy, automatic way of finding this out. To begin with, you write down
all numbers from 2 to $x$, circle 2, then cross out all multiples of 2. Next,
you circle the lowest number that hasn't been circled or crossed out (3) and
again cross out all multiples of this number, etc. You only need to continue
until you reach the largest number whose square is less than or equal to $x$.

Apart from 2, prime numbers are never even. Apart from 2 and 5, prime numbers
never end in 2, 5 or 0. So you only need to consider numbers ending in 1, 3, 7,
9 anyway (there are infinite primes ending in these numbers; see \cite[vol. 1,
p. 137]{Tietze1973}).

You can now find a large number of finished programs on the Internet - often
complete with source code - allowing you to experiment with large numbers
yourself. You also have access to large databases that contain either a large
number of primes or the factorisation of numerous composite numbers. The
\index{Cunningham project} \textbf{Cunningham project} for example, determines
the factors of all composite numbers that are formed as follows: $$ f(n) = b^n
\pm 1  \quad {\rm for~} b = 2, 3, 5, 6, 7, 10, 11, 12 $$ ($b$ is not equal to
multiples of bases already used, such as $4, 8, 9$).

Details of this can be found at:
\vspace{-10pt}
\begin{itemize}
  \item[] \href{http://www.cerias.purdue.edu/homes/ssw/cun}
               {\tt http://www.cerias.purdue.edu/homes/ssw/cun}
\end{itemize}

% --------------------------------------------------------------------------
\subsection{Notes}

\textbf{Proven statements / theorems about primes}

\begin{itemize}
  \item For each number $n$ in ${\bf N}$ there are $n$ consecutive natural
      numbers that are not primes.
      A proof of this can be found in \cite[p. 79]{Padberg1996}.
  \item The Hungarian mathematician Paul Erd\"os (1913-1996) proved:
      Between each random number not equal to $1$ and its double, there is
      at least one prime. He was not the first to prove this theorem, but
      proved it in a much simpler manner than those before him.
  \item There is a real number a such that the function $f: {\bf N}
      \rightarrow {\mathbb Z}$ where $n \mapsto a^{3^n}$ 
      only delivers primes for all $n$ (see \cite[p. 82]{Padberg1996}).
      Unfortunately, problems arise when we try to determine $a$ (see below).
\end{itemize}

\textbf{Proven statements / conjectures about primes}

\begin{itemize}
  \item The German mathematician \index{Goldbach Christian} Christian Goldbach
      (1690-1764) conjectured: Every even natural number greater than 2 can
      be represented as the sum of two prime numbers.
      Computers have verified\footnote{%
It is generally accepted today, that the Goldbach Conjecture is true, 
i.~e.\ valid for all even natural numbers bigger than $2$.
In 1999, mathematician J\"org Richstein from the computer sciences 
institute at the University of Giessen, studied even numbers up to 
400 billion and found no contradictory example (see
\href{http://www.informatik.uni-giessen.de/staff/richstein/de/Goldbach.html}
{\tt http://www.informatik.uni-giessen.de/staff/richstein/de/Goldbach.html}).
\index{Richstein 1999}
Nevertheless, this does not provide us with general proof.\\
The fact is that despite all efforts, Goldbach's conjecture has to date
not been proven. This leads one to believe that since the pioneer work of
the Austrian mathematician Kurt G\"odel \index{G\""odel Kurt} is 
well-known, not every true mathematical theorem is provable (see
\href{http://www.mathematik.ch/mathematiker/goedel.html}
{http://www.mathematik.ch/mathematiker/goedel.html}).
Perhaps Goldbach's conjecture was correct, but in any case the proof
will never be found. Conversely, that will presumably also remain unproven.}
      the Goldbach conjecture for all even numbers up to
      $4*10^{14}$ but no general proof has yet been found\footnote{%
The English publisher {\em Faber} and the American publisher {\em Bloomsbury}
issued in 2000 the 1992 published book ``Uncle Petros and Goldbach's 
Conjecture'' by Apostolos Doxiadis.
It's the story of an old maths professor who fails to prove a more than
250 year old puzzle. To boost the sales figures the English and American
publishers have offered a prize of 1 million USD, if someone can prove
the conjecture -- which should be published by 2004 in a well-known
mathematical journal.\\
Surprisingly only British and American citizens are allowed to participate.\\
The theorem which has come closest so far to Goldbach's conjecture
was proved by Chen Jing-Run in 1966 in a way which is somewhat hard
to understand:
Each even integer greater than $2$ is the sum of one prime and of 
the product of two primes. E.g.: $20=5+3*5.$\\
Most of the research about the Goldbach conjecture is collected in 
the book: ``Goldbach Conjecture'', ed. Wang Yuan, 1984, World scientific
Series in Pure Maths, Vol. 4.\\
Especially this conjecture makes it clear, that even today we do not
have a complete understanding of the deeper connections between addition and
multiplication of natural numbers.}.

  \item The German \index{Riemann Bernhard} mathematician Bernhard Riemann
      (1826-1866) put forward a formula for the distribution of primes
      that would further improve the estimate. However, this has neither
      been proved nor disproved so far.
\end{itemize}


\textbf{Open questions} \\

Twin primes are prime numbers whose difference is 2. Examples include 5 and 7 or
101 and 103. Triplet primes, however, only occur once: 3, 5, 7.  For all other
sets of three consecutive uneven numbers, one of them is always divisible by 3
and thus not a prime.
\begin{itemize}
  \item The number of twin primes is an open question: infinite or limited
number?
  The largest twin primes known today are $1,693,965 \cdot 2^{66,443} \pm 1.$
  \item Does a formula exist for calculating the number of twin primes per
interval?
  \item The above proof of the function  $f: \: N \rightarrow Z$ with $n \mapsto
a^{3^n}$  only guarantees the existence of such a number $a$.  How can we
determine this number $a$ and will it have a value, making the function also of
some practical interest?
\item Is there an infinite number of Mersenne prime numbers?\index{Prime number!Mersenne}\index{Mersenne!prime number}
\item Is there an infinite number of Fermat prime numbers?

  \item Does a polynomial time algorithm exist for calculating the prime factors of a number (see
    \cite[p. 167]{Klee1997})? 
    This question can be divided into the two following questions:
\begin{itemize} 
 \item Does a polynomial time algorithm exist that decides whether a number is
prime?
\item Does a polynomial time algorithm exist that calculates for a
  composite number $n$ a non-trivial (i.e. other than $1$ and $n$) factor
  of $n$?\footnote{Please see chapters \ref{RSABernstein} and
    \ref{NoteFactorisation}.}
\end{itemize}
\end{itemize}


\paragraph{Further interesting topics regarding prime numbers}
\mbox{}

This chapter doesn't consider other number theory topics such as
divisibility rules, modulus calculation, modular inverses, modular powers and
roots, Chinese remainder theorem, Euler PHI function, perfect numbers.
Some of these are considered in the next chapter.


% --------------------------------------------------------------------------
\newpage
\subsubsection{Number of prime numbers in various intervals}\label{s:primhfk}
\hypertarget{primhfk}{}
\vskip +10 pt

\begin{tabular}{|l|l||l|l||l|l|}\hline
\multicolumn{2}{|l||}{Ten-sized intervals} & \multicolumn{2}{l||}{Hundred-sized
intervals} & \multicolumn{2}{l|}{Thousand-sized intervals} \\ \hline
Interval  &     Number &    Interval  & Number &  Interval  & Number\\ \hline
\hline
1-10     &       4     &     1-100   &     25  &     1-1000     &    168 \\
11-20    &       4     &     101-200 &     21  &     1001-2000  &    135 \\
21-30    &       2     &     201-300 &     16  &     2001-3000  &    127  \\
31-40    &       2     &     301-400 &     16  &     3001-4000  &    120 \\
41-50    &       3     &     401-500 &     17  &     4001-5000  &    119 \\
51-60    &       2     &     501-600 &     14  &     5001-6000  &    114 \\
61-70    &       2     &     601-700 &     16  &     6001-7000  &    117 \\
71-80    &       3     &     701-800 &     14  &     7001-8000  &    107 \\
81-90    &       2     &     801-900 &     15  &     8001-9000  &    110 \\
91-100   &       1     &     901-1000 &     14 &      9001-10000 &    112 \\
\hline
\end{tabular} \\ \vskip +10 pt
Further intervals: \vskip +10 pt

\begin{tabular}{|l|l|l|}\hline
Interval & Number & Average number per 1000 \\ \hline
1 - 10,000          &    1,229        &    122.900 \\
1 - 100,000         &    9,592        &    95.920 \\
1 - 1,000,000       &    78,498      &    78.498 \\
1 - 10,000,000       &   664,579     &    66.458 \\
1 - 100,000,000      &   5,761,455   &    57.615 \\
1 - 1,000,000,000    &   50,847,534  &    50.848 \\
1 - 10,000,000,000   &   455,052,512 &    45.505 \\ \hline
\end{tabular}
\vskip +6 pt

% --------------------------------------------------------------------------
\newpage
\subsubsection{Indexing prime numbers ($n$-th prime number)} 
\hypertarget{ntePrimzahl}{}\label{s:ntePrimzahl}
\vskip +10 pt

\begin{tabular}{|l|l|l|l|}\hline
Index   &   Precise value  & Rounded value &
Comment \\
\hline \hline
1       &   2             &     2              & \\
2       &   3             &     3  &  \\
3       &   5             &     5  & \\
4       &   7             &     7 & \\
5       &   11            &     11 & \\
6       &   13            &     13 & \\
7       &   17            &     17 & \\
8       &   19            &     19 & \\
9       &   23            &     23 & \\
10      &   29            &     29 & \\
100     &   541           &     541 & \\
1,000    &   7,917          &     7,917 & \\
664,559  &  9,999,991     &     9.99999E+06 &   All prime numbers up to 1E+07 were known\\
        &                 &                 &  at the beginning of the 20th century. \\
1E+06  &    15,485,863   &      1.54859E+07 & \\
6E+06  &    104,395,301    &    1.04395E+08  & This prime was discovered in 1959.\\
1E+07  &    179,424,673     &    1.79425E+08 & \\
1E+09  &    22,801,763,489  &    2.28018E+10 & \\
1E+12  &    29,996,224,275,833 & 2.99962E+13 & \\ \hline
\end{tabular}
\vskip +2pt Comment: With gaps, extremely large prime numbers were discovered
at an early stage.  \\


\vskip +12pt Web links:

{\href{http://www.math.Princeton.EDU/~arbooker/nthprime.html}{http://www.math.Princeton.EDU/\~{}arbooker/nthprime.html}.} 

\vskip +16pt Output of the $n$-th prime number

See {\href{http://www.utm.edu/research/primes/notes/by_year.html}
    {\tt http://www.utm.edu/research/primes/notes/by\_year.html}.}



% --------------------------------------------------------------------------
\newpage
\subsubsection{Orders of magnitude / dimensions in reality}\label{s:grosord}
In the description of cryptographic protocols and algorithms, numbers occur that
are so large or so small that they are inaccessible to our intuitive
understanding. It may therefore be useful to provide comparative numbers from
the real world around us so that we can develop a feeling for the
security of cryptographic algorithms. Some of the numbers listed below originate
from \cite{Schwenk1996} and \cite[p.18]{Schneier1996p}.  \hypertarget{grosord}{}
\begin{tabbing}
Probability that you will be hijacked on your next flight:~~ \= abcdefhijk \=
\kill
Probability that you will be hijacked on your next flight \> $ 5.5 \cdot 10^{-6}
$\> \\
Probability of 6 correct numbers in the lottery \> $ 7.1 \cdot 10^{-8} $\> \\
Annual probability of being hit by lightning \> $ 10^{-7} $\> \\
Risk of being hit by a meteorite \> $ 1.6 \cdot 10^{-12} $ \> \\
--------------------------------------------------------------------------------
----------------------------------\\
Time until the next ice age (in years) \> $14000 $ \> $ (2^{14})$ \\
Time until the sun dies (in years)   \> $10^{9} $ \> $(2^{30})$ \\
Age of the Earth (in years)\> $ 10^9 $ \> $(2^{30}) $  \\
Age of the universe (in years) \> $ 10^{10} $ \> $(2^{34}) $ \\
Number of the Earth's atoms            \> $10^{51} $ \> $ (2^{170}) $ \\
Number of the sun's atoms              \> $10^{57}$ \> $ (2^{190})$ \\
Number of atoms in the universe (without dark material)        \> $10^{77}$  \>
$ (2^{265})$ \\
Volume of the universe (in $cm^3$)     \> $10^{84}$ \> $(2^{280})$
\end{tabbing}

% --------------------------------------------------------------------------
\subsubsection{Special values in the binary and decimal systems}
\begin{tabbing}
DualSystem~~~ \= \kill
Dual system \> Decimal system \\*[4pt]
$2^{10}$ \> $1024$ \\
$2^{40}$ \> $1.09951\cdot 10^{12}$ \\
$2^{56}$ \> $7.20576\cdot 10^{16}$ \\
$2^{64}$ \> $1.84467\cdot 10^{19}$ \\
$2^{80}$ \> $1.20893\cdot 10^{24}$ \\
$2^{90}$ \> $1.23794\cdot 10^{27}$ \\
$2^{112}$ \>    $5.19230\cdot 10^{33}$ \\
$2^{128}$ \>    $3.40282\cdot 10^{38}$ \\
$2^{150}$ \>    $1.42725\cdot 10^{45}$ \\
$2^{160}$ \>    $1.46150\cdot 10^{48}$ \\
$2^{250}$ \>    $1.80925\cdot 10^{75}$ \\
$2^{256}$ \>    $1.15792\cdot 10^{77}$ \\
$2^{320}$ \>    $2.13599\cdot 10^{96}$ \\
$2^{512}$ \>    $1.34078\cdot 10^{154}$ \\
$2^{768}$ \>    $1.55252\cdot 10^{231}$ \\
$2^{1024}$ \>   $1.79769\cdot 10^{308}$ \\
$2^{2048}$ \>   $3.23170\cdot 10^{616}$ \\
\end{tabbing}

\vskip -20 pt

Calculation using GMP, for example:
{\href{http://www.gnu.ai.mit.edu}{\tt http://www.gnu.ai.mit.edu}}.


% --------------------------------------------------------------------------
\newpage
\begin{thebibliography}{99999}
\addcontentsline{toc}{subsection}{Bibliography}


\bibitem[Aaronson2003]{Aaronson2003} \index{Aaronson 2003}
    Scott Aaronson, \\
    {\em The Prime Facts: From Euclid to AKS}, \\
    \href{http://www.cs.berkeley.edu/~aaronson/prime.ps}
         {\texttt{http://www.cs.berkeley.edu/\~{}aaronson/prime.ps}}. \\
    Only after I had completed this article, did I come across the 
    extremely well-written paper by Scott Aaronson, which also offers
    a didactically well done intoduction to this topic. It is 
    humorous and easy to read but at the same time precise and erudite.


%already defined in elementaryNumberTheory.inc -> 2 davor
\bibitem[Bartholome1996]{2Bartholome1996}  \index{Bartholome 1996}
    A. Bartholom�, J. Rung, H. Kern, \\     
    {\em Zahlentheorie f\"ur Einsteiger}, Vieweg 1995, 2nd edition 1996.

\bibitem[Blum1999]{Blum1999} \index{Blum 1999}   
    W. Blum, \\     
    {\em Die Grammatik der Logik}, dtv, 1999.

\bibitem[Bundschuh1998]{Bundschuh1998} \index{Bundschuh 1998}
    Peter Bundschuh, \\
    {\em Einf\"uhrung in die Zahlentheorie}, Springer 1988, 4th edition 1998.

\bibitem[Doxiadis2000]{Dioxadis2000}
    Apostolos Doxiadis, \\
    {\em Uncle Petros and the Goldbach's Conjecture}, \\
    Faber/Bloomsbury, 2000.

\bibitem[Graham1989]{Graham1989} \index{Graham 1989}     
   R.E. Graham, D.E. Knuth, O. Patashnik, \\
   {\em Concrete Mathematics}, Addison-Wesley, 1989.

\bibitem[Klee1997]{Klee1997} \index{Klee 1997}     
   V. Klee, S. Wagon, \\
   {\em Ungel\"oste Probleme in der Zahlentheorie und der Geometrie der 
   Ebene}, \\ Birkh\"auser Verlag, 1997.

\bibitem[Knuth1981]{Knuth1981} \index{Knuth 1981}     
   Donald E. Knuth, \\ 
   {\em The Art of Computer Programming, vol 2: Seminumerical Algorithms}, \\
   Addison-Wesley, 1969, 2nd edition 1981.

\bibitem[Lorenz1993]{Lorenz1993} \index{Lorenz 1993}     
   F. Lorenz, \\
   {\em Algebraische Zahlentheorie}, BI Wissenschaftsverlag, 1993.

\bibitem[Padberg1996]{Padberg1996} \index{Padberg 1996}     
   F. Padberg, \\
   {\em Elementare Zahlentheorie}, 
   Spektrum Akademischer Verlag 1988, 2nd edition 1996.

\bibitem[Pieper1983]{Pieper1983} \index{Pieper 1983}     
   H. Pieper, \\
   {\em Zahlen aus Primzahlen}, 
   Verlag Harri Deutsch 1974, 3rd edition 1983.

\bibitem[Richstein1999]{Richstein1999} \index{Richstein 1999}
    J. Richstein, \\
    {\em Verifying the Goldbach Conjecture up to $4*10^{14},$}
    Mathematics of Computation. 

\bibitem[Schneier1996]{Schneier1996p} \index{Schneier 1996}     
    Bruce Schneier, \\
    {\em Applied Cryptography, Protocols, Algorithms, and Source Code in C},\\
    Wiley and Sons, 2nd edition 1996.

\bibitem[Schroeder1999]{Schroeder1999} \index{Schroeder 1999}
    M.R. Schroeder, \\
    {\em Number Theory in Science and Communication}, \\ 
    Springer 1984, 3rd edition 1997, Corrected Printing 1999.

\bibitem[Schwenk1996]{Schwenk1996} \index{Schwenk 1996}     
    J. Schwenk \\
    {\em Conditional Access}, 
    in taschenbuch der telekom praxis 1996, \\
    Hrgb. B. Seiler, Verlag Schiele und Sch\"on, Berlin.

\bibitem[Tietze1973]{Tietze1973} \index{Tietze 1973}     
    H. Tietze, \\
    {\em Gel\"oste und ungel\"oste mathematische Probleme}, \\
    Verlag C. H. Beck 1959, 6th edition 1973.

\end{thebibliography}


% --------------------------------------------------------------------------
\newpage
\section*{Web links} \addcontentsline{toc}{subsection}{Web links}

\begin{enumerate}
\item GIMPS (Great Internet Mersenne-Prime Search) 
      \index{Mersenne!prime number}  \index{GIMPS} \\
      www.mersenne.org is the home page of the GIMPS project, \\
      \href{http://www.mersenne.org/prime.htm}{\tt http://www.mersenne.org/prime.htm }

\item The Proth Search Page with the Windows program by Yves Gallot \\
      \href{http://prothsearch.net/index.html}{\tt http://prothsearch.net/index.html}

\item Generalized Fermat Prime Search \\
      \href{http://perso.wanadoo.fr/yves.gallot/primes/index.html}
        {\tt http://perso.wanadoo.fr/yves.gallot/primes/index.html}

\item Distributed Search for Fermat Number Dividers \\
      \href{http://www.fermatsearch.org/}{\tt http://www.fermatsearch.org/}

\item At the University of Tennessee you will find extensive research
      results about prime numbers. \\
      \href{http://www.utm.edu/}{\tt http://www.utm.edu/ }

\item The best overview about prime numbers is offered from my point of view 
      by ~``The Prime Pages'' from Chris Caldwell.
      \index{Chris Caldwell} \\
      \href{http://www.utm.edu/research/primes}
       {\tt http://www.utm.edu/research/primes }

\item Descriptions e.g. about prime number tests \\
      \href{http://www.utm.edu/research/primes/mersenne.shtml}
           {\texttt{http://www.utm.edu/research/primes/mersenne.shtml}} \\
      \href{http://www.utm.edu/research/primes/prove/index.html}
           {\texttt{http://www.utm.edu/research/primes/prove/index.html}} 

\item Showing the $n$-th prime number \\
      \href{http://www.utm.edu/research/primes/notes/by_year.html}
           {\tt http://www.utm.edu/research/primes/notes/by\_year.html }

\item The supercomputer manufacturer SGI Cray Research not only 
      employed brilliant mathematicians but also used the prime 
      number tests as benchmarks for their machines. \\
      \href{http://reality.sgi.com/chongo/prime/prime_press.html}
      {\tt http://reality.sgi.com/chongo/prime/prime\_press.html }

\item \href{http://www.eff.org/coop-awards/prime-release1.html}{\tt
http://www.eff.org/coop-awards/prime-release1.html }

%   \item \href{http://www.informatik.tu-darmstadt.de/TI/LiDIA/}{\tt http://www.informatik.tu-darmstadt.de/TI/LiDIA/ }

\item \href{http://www.math.Princeton.EDU/~arbooker/nthprime.html}{\tt
http://www.math.Princeton.EDU/\~{}arbooker/nthprime.html }

\item \href{http://www.cerias.purdue.edu/homes/ssw/cun}{\tt
http://www.cerias.purdue.edu/homes/ssw/cun }

\item \href{http://www.informatik.uni-giessen.de/staff/richstein/de/Goldbach.html}{\tt http://www.informatik.uni-giessen.de/staff/richstein/de/Goldbach.html}

\item \href{http://www.mathematik.ch/mathematiker/goedel.html}{\tt http://www.mathematik.ch/mathematiker/goedel.html}

\item \href{http://www.mscs.dal.ca/~dilcher/golbach/index.html}{\tt http://www.mscs.dal.ca/\~{}dilcher/goldbach/index.html}

\end{enumerate}


\vskip +10 pt
% --------------------------------------------------------------------------
\subsection*{Acknowledgments} \addcontentsline{toc}{subsection}{Acknowledgments}

I would like to take this opportunity to thank Mr.\ Henrik Koy and Mr.\ Roger
Oyono for their very constructive proof-reading of this article.

% Local Variables:
% TeX-master: "../script-en.tex"
% End:
