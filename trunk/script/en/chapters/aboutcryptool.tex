% $Id$
% ............................................................................
%                 TEXT DER 1. SEITE
% ~~~~~~~~~~~~~~~~~~~~~~~~~~~~~~~~~~~~~~~~~~~~~~~~~~~~~~~~~~~~~~~~~~~~~~~~~~~~

\parskip 4pt
%\vskip +12 pt
{
\noindent In this CrypTool\index{CrypTool} script you will find predominantly
mathematically oriented information on using cryptographic procedures.
The main chapters have been written by various
\hyperlink{appendix-authors}{authors}
(see appendix \ref{s:appendix-authors})
and are therefore independent from one another. At the end of most chapters
you will find literature and web links.

You will obtain information about the principles of symmetrical and
asymmetrical {\bf encryption}. A large section of this script is dedicated
to the fascinating topic of {\bf prime numbers}. Using numerous examples,
the {\bf elementary number theory} and {\bf modular arithmetic} are
introduced and applied in an exemplary manner for the {\bf RSA procedure}.
By reading the following chapter you'll gain an insight into the mathematical
ideas behind {\bf modern cryptography}. 

A further chapter is devoted to {\bf digital signatures}, 
which are an essential component of e-business applications. 
The last chapter describes {\bf elliptic curves}: they could be used
as an alternative to RSA and in addition are extremely well suited for 
implementation on smartcards.

Whereas the \textit{program} CrypTool\index{CrypTool} teaches you how to use
cryptography in practice, the \textit{script} provides those interested in the
subject with a deeper understanding of the mathematical algorithms
used -- trying to do it in an instructive way.

The authors Bernhard Esslinger, Matthias B\"uger, Bartol Filipovic, 
Henrik Koy, Roger Oyono and J\"org Cornelius Schneider
would like to take this opportunity to thank their colleagues 
in the company and at the universities of Frankfurt, Gie\ss en, 
Siegen, Karlsruhe and Darmstadt. They are particularly indepted to Dr.~Peer
Wichmann from the Karlsruhe computer science research centre (Forschungszentrum
Informatik, FZI) for his down-to-earth support.

\enlargethispage{12pt}
As at CrypTool\index{CrypTool}, the quality of the script is enhanced
by your suggestions and ideas for improvement. We look forward 
to your feedback.

\vskip +7 pt \noindent
You will find the current version of CrypTool\index{CrypTool} under \newline
  \href{http://www.cryptool.org}{\texttt{http://www.cryptool.org}},~
  \href{http://www.cryptool.com}{\texttt{http://www.cryptool.com}}~ or~
  \href{http://www.CrypTool.de}{\texttt{http://www.cryptool.de}}.

\vskip +7 pt \noindent
The contact people for this free open-source tool are listed 
in the ``readme'' file delivered within the CrypTool\index{CrypTool} package.}


% Local Variables:
% TeX-master: "../script-en.tex"
% End:
