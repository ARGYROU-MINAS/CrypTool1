% $Id$
% ..............................................................................
%                V E R S C H L U E S S E L U N G S V E R F A H R E N
%
% Schreibregelung / Syntax rules English:
%         - Capitel header with all nouns/verbs-words capitalized,
%         - All other headers in the normal lower/upper case manner
%           like sentences, but without dot at the end.
%
% ~~~~~~~~~~~~~~~~~~~~~~~~~~~~~~~~~~~~~~~~~~~~~~~~~~~~~~~~~~~~~~~~~~~~~~~~~~~~~~

\hypertarget{Kapitel_1}{}
\chapter{Encryption Procedures}
\label{Label_Kapitel_1}
(Bernhard Esslinger, Joerg-Cornelius Schneider, May 1999; Updates Dec. 2001, Feb. 2003,
June 2005, July 2007, August 2009)

\begin{center}
\fbox{\parbox{15cm}{%
{\em Saying from India}:\\
Explain it to me, I will forget it.\\
Show it to me, maybe I will remember it.\\
Let me do it, and I will be good at it.}}
\end{center}

This chapter introduces the topic in a more descriptive way without using too much mathematics.


The purpose of encryption \index{Encryption} is to change data in such a way
that only an authorized recipient is able to reconstruct the plaintext. This
allows us to transmit data without worrying about it getting into unauthorized
hands. Authorized recipients possess a piece of secret information -- called the
key -- which allows them to decrypt the data while it remains hidden from
everyone else.\par \vskip + 3pt

One encryption procedure has been mathematically proved to be secure, the {\em
One Time Pad}\index{One Time Pad}. However, this procedure has several
practical disadvantages (the key used must be randomly selected and must be at
least as long as the message being protected), which means that it is hardly
used except in closed environments such as for the hot wire between Moscow and
Washington.\par \vskip + 3pt

For all other procedures there is a (theoretical) possibility of breaking them.
If the procedures are good, however, the time taken to break them is so long
that it is practically impossible to do so, and these procedures can therefore be
considered (practically) secure.\par \vskip + 3pt

The book of Bruce Schneier \cite{cm:Schneier1996cm} offers a very good overview of
the different algorithms. We basically distinguish between symmetric and
asymmetric encryption procedures.

% --------------------------------------------------------------------------
\section[Symmetric encryption]
{Symmetric encryption\footnotemark}
  \footnotetext{%
    With CrypTool\index{CrypTool} you can execute the following modern
    symmetric encryption algorithms 
    (using the menu path {\bf Crypt \textbackslash{} Symmetric (modern)}): \\
    IDEA, RC2, RC4, DES (ECB), DES (CBC), Triple-DES (ECB), Triple-DES (CBC),
    MARS (AES candidate), RC6 (AES candidate), Serpent (AES candidate), 
    Twofish (AES candidate), Rijndael (official AES algorithm).
  }
\nopagebreak
For {\em symmetric} encryption \index{Encryption!symmetric} sender and
recipient must be in possession of a common (secret) key which they have
exchanged before actually starting to communicate. The sender uses this
key to encrypt the message and the recipient uses it to decrypt it.
\par \vskip + 3pt

All classical methods are of this type. Examples can be found within the 
program CrypTool, in chapter  \ref{Kapitel_PaperandPencil} 
(Paper and Pencil Encryption Methods) of this script or in \cite{cm:Nichols1996}. 
Now we want to consider more modern mechanisms.

The advantages of symmetric algorithms are the high speed with which data can be
encrypted and decrypted. One disadvantage is the need for key management. In
order to communicate with one another confidentially, sender and recipient must
have exchanged a key using a secure channel before actually starting to
communicate. Spontaneous communication between individuals who have never met
therefore seems virtually impossible. If everyone wants to communicate with
everyone else spontaneously at any time in a network of $ n $ subscribers, each
subscriber must have previously exchanged a key with each of the other $n - 1$
subscribers. A total of $n(n - 1)/2$ keys must therefore be exchanged.\par \vskip + 3pt

The most well-known symmetric encryption procedure is the \index{DES} DES-algorithm. The DES-algorithm has been developed by IBM in collaboration with the
National Security Agency \index{NSA} (NSA), and was published as a standard in
1975. Despite the fact that the procedure is relatively old, no effective attack
on it has yet been detected. The most effective way of attacking consists of
testing (almost) all possible keys until the right one is found ({\em brute-force-attack}).
\index{Attack!brute-force} Due to the relatively short key length of
effectively 56 bits (64 bits, which however include 8 parity bits), numerous
messages encrypted using DES have in the past been broken. Therefore, the
procedure can now only be considered to be conditionally secure. Symmetric
alternatives to the DES procedure include the IDEA \index{IDEA} or Triple DES
algorithms.\par \vskip + 3pt

Up-to-the-minute procedure is the symmetric AES standard. The associated
Rijndael algorithm was declared winner of the AES award on October 2nd, 2000
and thus succeeds the DES procedure.

More details about the AES algorithms and the AES candidates of the last round
can be found within the online help of CrypTool\index{CrypTool}%
\footnote{%
      CrypTool online help\index{CrypTool}: the index head-word {\bf AES}
      leads to the 3 help pages: {\bf AES candidates}, 
      {\bf The AES winner Rijndael} and 
      {\bf The Rijndael encryption algorithm}.
  }.


% --------------------------------------------------------------------------
\subsection{New results about cryptanalysis of AES}
\index{Cryptanalysis} \label{NeueAES-Analyse}

Below you will find some results, which have recently called into question the security of the AES algorithm -- from our point of view these doubts practically still remain unfounded
% = do not bring disrepute upon AES
. 
The following information is based on the original papers and the articles \cite{cm:Wobst-iX2002} and \cite{cm:Lucks-DuD2002}.

AES with a minimum key length of 128 bit is still in the long run sufficiently secure against brute-force attacks\index{Attack!brute-force} -- as long as the quantum computers aren't powerful enough. When announced as new standard AES was immune against all known crypto attacks, mostly based on statistical considerations and earlier applied to DES: using pairs of clear and cipher texts expressions are constructed, which are not completely at random, so they allow conclusions to the used keys. These attacks required unrealistically large amounts of intercepted data.

Cryptanalysts already label methods as ``academic success'' or as ``cryptanalytic attack'' if they are theoretically faster than the complete testing of all keys (brute force analysis). In the case of AES with the maximal key length (256 bit) exhaustive key search on average needs $2^{255}$ encryption operations. A cryptanalytic attack needs to be better than this. At present between $2^{75}$ and $2^{90}$ encryption operations are estimated to be performable only just for organizations, for example a security agency.

In their 2001-paper Ferguson, Schroeppel and Whiting \cite{cm:Ferguson2001}
presented a new method of symmetric codes cryptanalysis: They described AES with
a closed formula (in the form of a continued fraction) which was possible
because of the "relatively" clear structure of AES. This formula consists of
around 1000 trillion terms of a sum - so it does not help concrete practical
cryptanalysis. Nevertheless curiosity in the academic community was awakened.
It was already known, that the 128-bit AES could be described as an
over-determined system of about 8000 quadratic equations (over an algebraic
number field) with about 1600 variables (some of them are the bits of the wanted
key) -- equation systems of that size are in practice not solvable. This special
equation system is relatively sparse, so only very few of the quadratic terms
(there are about 1,280,000 are possible quadratic terms in total) appear in the
equation system.

The mathematicians Courtois and Pieprzyk \cite{cm:Courtois2002} published a paper
in 2002, which got a great deal of attention amongst the crypto community: The
pair had further developed the XL-method (eXtended Linearization), introduced at
Eurocrypt 2000 by Shamir et al., to create the so called XSL-method (eXtended
Sparse Linearization). The XL-method is a heuristic technique, which in some
cases manages to solve big non-linear equation systems and which was till then
used to analyze an asymmetric algorithm (HFE).  The innovation of Courtois and
Pieprzyk was, to apply the XL-method on symmetric codes: the XSL-method can be
applied to very specific equation systems. A 256-bit AES could be attacked in
roughly $2^{230}$ steps. This is still a purely academic attack, but also a
direction pointer for a complete class of block ciphers. The major problem with
this attack is that until now nobody has worked out, under what conditions it is
successful: the authors specify in their paper necessary conditions, but it is
not known, which conditions are sufficient.  There are two very new aspects of
this attack: firstly this attack is not based on statistics but on algebra. So
attacks seem to be possible, where only very small amounts of cipher text are
available. Secondly the security of a product algorithm\index{Cascade cipher}%
\footnote{%
A cipher text can be used as input for another encryption algorithm. 
A cascade cipher is build up as a composition of different encryption 
transformations. The overall cipher is called product algorithm or cascade cipher (sometimes depending whether the used keys are statistically dependent or not).\\
Cascading does not always improve the security.\\
This process is also used {\em within} modern algorithms:
They usually combine simple and, considered at its own, cryptologically
relatively insecure single steps in several rounds into an efficient 
overall procedure.  Most block ciphers (e.g. DES, IDEA) are cascade ciphers.\\
Also serial usage of the same cipher with different keys (like with Triple-DES)
is called cascade cipher. 
}
does not exponentially increase with the number of rounds.

Currently there is a large amount of research in this area: for example Murphy and Robshaw presented a paper at Crypto 2002 \cite{cm:Robshaw2002a}, which could dramatically improve cryptanalysis: the burden for a 128-bit key was estimated at about $2^{100}$ steps by describing AES as a special case of an algorithm called BES (Big Encryption System), which has an especially "round" structure. But even $2^{100}$ steps are beyond what is achievable in the foreseeable future. Using a 256 bit key the authors estimate that a XSL-attack will require $2^{200}$ operations.

More details can be found at:
\vspace{-10pt}
\begin{itemize}
  \item[] \href{http://www.cryptosystem.net/aes}
               {\texttt{http://www.cryptosystem.net/aes}}\\
          \href{http://www.minrank.org/aes/}
               {\texttt{http://www.minrank.org/aes/}}
\end{itemize}

So for 256-AES the attack is much more effective than brute-force\index{Attack!brute-force} but still far more away from any computing power which could be accessible in the short-to-long term. 

The discussion is very controversial at the moment: Don Coppersmith (one of the
inventors of DES) for example queries the practicability of the attack because
XLS would provide no solution for AES \cite{cm:Coppersmith2002}. This implies that
then the optimization of Murphy and Robshaw \cite{cm:Robshaw2002b} would not work.

In 2009 Biryukov und Khovratovich \cite{cm:Biryukov2009} published another
theoretical attack on AES. This attack uses different methods from the ones
described above. They applied methods from hash function cryptanalysis (local
collisions and boomerang switching) to construct a related-key attack on
AES-256. I.~e.\ the attacker not only needs to be able to encrypt arbitrary data
(chosen plain text), in addition he needs to be able to manipulate the unknown key
(related-key). 

Based on those assumptions, the effort to find a AES-256 key is reduced to
$2^{119}$ time and $2^{77}$ memory. In the case of AES-192 the attack is even less
practical, for AES-128 the authors do not provide an attack.

% --------------------------------------------------------------------------
\subsection{Current status of brute-force attacks on symmetric algorithms (RC5)}
\index{Attack!brute-force}
\index{RC5}
\label{Brute-force-gegen-Symmetr}

The current status of brute-force attacks on symmetric encryption algorithms can be explained with the block cipher RC5.

Brute-force (exhaustive search, trial-and-error) means to completely examine all keys of the key space: so no special analysis methods have to be used. Instead, the cipher text is decrypted with all possible keys and for each resulting text it is checked, whether this is a meaningful clear text. A key length of 64 bit means at most $2^{64}$ = 18,446,744,073,709,551,616 or about 18 trillion (GB) / 18 quintillion (US)  keys to check\index{CrypTool}%
\footnote{%
    With CrypTool\index{CrypTool} you can also try brute-force attacks
    of modern symmetric algorithms (using the menu path
    {\bf Analysis \textbackslash{} Symmetric Encryption (modern)}): here
    the weakest knowledge of an attacker is assumed, he performs a 
    ciphertext-only attack.
    To achieve a result in an appropriate time with a single PC you should 
    mark not more than 20 bit of the key as unknown.
}.

Companies like RSA Security provide so-called cipher challenges in order to quantify the security offered by well-known symmetric ciphers as DES, Triple-DES or RC5\footnote{\href{http://www.rsasecurity.com/rsalabs/challenges/secretkey/index.html}{\tt http://www.rsasecurity.com/rsalabs/challenges/secretkey/index.html}\\
In May 2007 RSA announced that they will not confirm the correctness of the not yet solved RC5-72 challenge.}. They offer prizes for those who manage to decipher cipher texts, encrypted with different algorithms and different key lengths, and to unveil the symmetric key (under controlled conditions). So theoretical estimates can be confirmed.

It is well-known, that the ``old'' standard algorithm DES with a fixed key length of 56 bit is no more secure: this was demonstrated already in January 1999 by the Electronic Frontier Foundation (EFF). With their specialized computer Deep Crack they cracked a DES encrypted message within less than a day\footnote{\href{http://www.rsasecurity.com/rsalabs/challenges/des3/index.html}{\tt http://www.rsasecurity.com/rsalabs/challenges/des3/index.html}}.

The current record for strong symmetric algorithms unveiled a key 64 bit long. The algorithm used was RC5, a block cipher with variable key size. 

The RC5-64 challenge has been solved by the distributed.net team after 5 years\footnote{\href{http://distributed.net/pressroom/news-20020926.html}{\tt http://distributed.net/pressroom/news-20020926.html}}.  In total 331,252 individuals co-operated over the internet to find the key\footnote{%
An overview of current distributed computing projects can be found here:\\
\href{http://distributedcomputing.info/}{\tt http://distributedcomputing.info/}
}. More than 15 trillion (GB) / 15 quintillion (US)  keys were checked, until they found the right key.

This makes clear, that symmetric algorithms (even if they have no cryptographic weakness) using keys of size 64 bit are no more appropriate to keep sensible data private.

Similar cipher challenges are there for asymmetric algorithms (please see chapter \ref{NoteFactorization}).




% --------------------------------------------------------------------------
\section[Asymmetric encryption]
{Asymmetric encryption\footnotemark}
  \footnotetext{%
    With CrypTool\index{CrypTool} you can execute RSA encryption
    and decryption (using the menu path {\bf Crypt \textbackslash{} Asymmetric}).
    In both cases you must select a RSA key pair. Only in the case of decryption
    the secret RSA key is necessary: so here you are asked to enter the PIN. 
  }
\nopagebreak
In the case of {\em asymmetric} encryption \index{Encryption!asymmetric} each
subscriber has a personal pair of keys consisting of a {\em secret}
\index{Key!secret} key and a {\em public} key\index{Key!public}. The public
key, as its name implies, is made public, e.g. in a key directory on the
Internet.\par \vskip + 3pt

If Alice\index{Alice}%
\footnote{%
      In order to describe cryptographic protocols participants
      are often named Alice, Bob\index{Bob}, \dots (see \cite[p. 23]{cm:Schneier1996cm}). 
      Alice and Bob perform all 
      2-person-protocols. Alice will initiate all protocols and 
      Bob answers. The attackers are named Eve (eavesdropper) and
      Mallory (malicious active attacker).
  } wants to communicate with Bob, then she finds Bob's public key 
in the directory and uses it to encrypt her message to him. She then sends
this cipher text to Bob, who is then able to decrypt it again using his 
secret key. As only Bob knows his secret key, only he can decrypt 
messages addressed to him.
Even Alice who sends the message cannot restore plaintext from the (encrypted)
message she has sent. Of course, you must first ensure that the public key
cannot be used to derive the private key.\par \vskip + 3pt

Such a procedure can be demonstrated using a series of thief-proof letter boxes.
If I have composed a message, I then look for the letter box of the recipient
and post the letter through it. After that, I can no longer read or change the
message myself, because only the legitimate recipient has the key for the
letter box.\par \vskip + 3pt

The advantage of asymmetric procedures is the easy \index{Key management} key management. Let's look again at a network with $n$
subscribers. In order to ensure that each subscriber can establish
an encrypted connection to each other subscriber, each subscriber
must possess a pair of keys. We therefore need $2n$ keys or $n$
pairs of keys. Furthermore, no secure channel is needed before
messages are transmitted, because all the information required in
order to communicate confidentially can be sent openly. In
this case, you simply have to pay attention to the accuracy
(integrity and authenticity) \index{Authenticity} of the public
key. Disadvantage: Pure asymmetric procedures take a lot longer to
perform than symmetric ones.\par \vskip + 3pt

The most well-known asymmetric procedure is the \index{RSA} 
RSA algorithm\index{CrypTool}%
\footnote{%
The RSA algorithm is extensively described in chapter \ref{rsabeweis} and later
within this script. 
The RSA cryptosystem can be executed in many variations with 
CrypTool\index{CrypTool} (using the menu path
{\bf Individual Procedures \textbackslash{} RSA Cryptosystem \textbackslash{}
RSA Demonstration}). 
The topical research results concerning RSA are described 
in chapter \ref{SecurityRSA}.
}%
,
named after its developers Ronald \index{Rivest, Ronald} Rivest, Adi
\index{Shamir, Adi} Shamir and Leonard \index{Adleman, Leonard} Adleman. The RSA algorithm
was published in 1978. The concept of asymmetric encryption was first
introduced by Whitfield Diffie \index{Diffie, Whitfield}  and Martin
\index{Hellman, Martin} Hellman in 1976. Today, the ElGamal \index{ElGamal, Tahir}
procedures also play a decisive role, particularly the \index{Schnorr, C.P.} Schnorr
variant in the \index{DSA} DSA (Digital \index{Signature!digital}Signature
Algorithm).



% --------------------------------------------------------------------------
% \newpage
\section[Hybrid procedures]
{Hybrid procedures\footnotemark}
\footnotetext{%
Within CrypTool\index{CrypTool} you can find this technique using the menu
path {\bf Crypt \textbackslash{} Hybrid}: 
There you can follow the single steps and its dependencies with concrete
numbers. The variant with RSA as the asymmetric algorithm is graphically
visualized; the variant with ECC uses the standard dialogs. In both cases
AES is used as the symmetric algorithm.
}\index{Hybrid procedure}

In order to benefit from the advantages of symmetric and asymmetric
techniques together, hybrid procedures \index{Encryption!hybrid} are
usually used (for encryption) in practice. \par \vskip + 3pt

In this case the bulk data is encrypted using symmetric procedures: The key
used for this is a secret session key\index{Session key} generated by the
sender randomly\footnote{%
An important part of cryptographically secure techniques is to generate 
random numbers. Within CrypTool\index{CrypTool} you can check out
different random number generators using the menu path
{\bf Indiv. Procedures \textbackslash{} Generate Random Numbers}. 
Using the menu path {\bf Analysis \textbackslash{} Analyze Randomness}
you can apply different test methods for random data to binary documents. \\
Up to now CrypTool has concentrated on cryptographically strong 
pseudo random number generators. Only the integrated Secude\index{SECUDE IT Security}
generator involves a "pure" random source. 
}\index{Random}
that is only used for this message.

This session key is then encrypted using the asymmetric procedure, and
transmitted to the recipient together with the message.

Recipients can determine the session key using their private keys and
then use the session key to encrypt the message.

In this way, we can benefit from the easy key management\index{Key management}
of asymmetric procedures (using public/private keys) and we benefit from the
efficiency of symmetric procedures to encrypt large quantities of data
(using secret keys).







% --------------------------------------------------------------------------
\newpage
% \vskip +40 pt
\begin{center}
\fbox{\parbox{15cm}{
    \emph{IETF\footnotemark:} \\
    There is an old saying inside the US National Security Agency (NSA):\\
    "Attacks always get better; they never get worse."
}}
\end{center}
\addtocounter{footnote}{0}\footnotetext{%
  \url{http://tools.ietf.org/html/rfc4270}\index{IETF}\index{NSA}
  }
\section{Ciphers and cryptanalysis for educational purposes} \index{Cryptanalysis}

Compared to public-key ciphers based on mathematics like RSA, the structure of AES,
and most other modern symmetric ciphers, is very complex and cannot be explained
as easily as RSA.

So some simplified variants of modern symmetric ciphers like DES, IDEA or AES
were developed for educational purposes in order to allow beginners to perform
encryption and decryption by hand and gain a better understanding of how the
algorithms work in detail.
These also help to understand and apply the according cryptanalysis methods.

One of these variants is e.g. S-AES (Simplified-AES) by Prof. Ed Schaefer
and his students \cite{cm:Musa-etal2003}%
\footnote{
    See the articles
    ``Devising a Better Way to Teach and Learn the Advanced Encryption Standard''\\
    at the university's news at\\
    \url{http://math.scu.edu/~eschaefe/getfile.pdf} \\
    \url{http://www.scu.edu/cas/research/cryptography.cfm}
}.
Another one is Mini-AES \cite{cm:Phan2002}
(see chapter~\ref{CM_Sage_Mini-AES} ``\nameref{CM_Sage_Mini-AES}''):
\begin{itemize}

\item Edward F. Schaefer: {\em A Simplified Data Encryption Standard Algorithm} 
      \cite{cm:Schaefer1996}.

\item Raphael Chung-Wei Phan: {\em Mini Advanced Encryption Standard (Mini-AES):
                                   A Testbed for Cryptanalysis Students} 
      \cite{cm:Phan2002}.

\item Raphael Chung-Wei Phan: {\em Impossible differential cryptanalysis of Mini-AES} 
      \cite{cm:Phan2003}.

\item Mohammad A. Musa, Edward F. Schaefer, Stephen Wedig:
      {\em A simplified AES algorithm and its linear and differential cryptanalyses} 
      \cite{cm:Musa-etal2003}.

\item Nick Hoffman: {\em A SIMPLIFIED IDEA ALGORITHM} 
      \cite{cm:Hoffman2006}.

\item S. Davod. Mansoori, H. Khaleghei Bizaki: 
      {\em On the vulnerability of Simplified AES Algorithm Against Linear Cryptanalysis} 
      \cite{cm:Mansoori-etal2007}.

\end{itemize}




% --------------------------------------------------------------------------
\section{Further details}

Beside the information you can find in the following chapters, in many other
books and on a good number of websites, the online help of 
CrypTool\index{CrypTool} also offers very many details about the 
symmetric and asymmetric encryption methods.




	

% ---------------------------------------------------------------------------
% ---------------------------------------------------------------------------
\newpage
\hypertarget{CM_Appendix_SageCode}{}
\section{Appendix: Examples using Sage}
\label{CM_Sage_samples}
\index{Sage!Code examples}
\index{Sage}

\noindent Below is Sage source code related to contents of the
chapter~\ref{Label_Kapitel_1} (``\nameref{Label_Kapitel_1}''). 


% ---------------------------------------------------------------------------
\subsection{Mini-AES}
\label{CM_Sage_Mini-AES}

The Sage module \texttt{crypto/block\_cipher/miniaes.py} supports Mini-AES to allow
students to explore the working of a modern block cipher.

Mini-AES, originally described at \cite{cm:Phan2002}, is a simplified variant of the Advanced
Encryption Standard (AES) to be used for cryptography education.

How to use Mini-AES is exhaustively described at the Sage reference page\\
\url{http://www.sagemath.org/doc/reference/sage/crypto/block_cipher/miniaes.html}.

The following Sage code from the release note of Sage 4.1 calls the implementation
of the Mini Advanced Encryption Standard.\\
See \url{http://mvngu.wordpress.com/2009/07/12/sage-4-1-released/}.

\begin{sagecode}
\begin{Verbatim}%
[fontsize=\footnotesize,fontshape=tt]
# We can encrypt a plaintext using Mini-AES as follows:
sage: from sage.crypto.block_cipher.miniaes import MiniAES
sage: maes = MiniAES()
sage: K = FiniteField(16, "x")
sage: MS = MatrixSpace(K, 2, 2)
sage: P = MS([K("x^3 + x"), K("x^2 + 1"), K("x^2 + x"), K("x^3 + x^2")]); P

[  x^3 + x   x^2 + 1]
[  x^2 + x x^3 + x^2]
sage: key = MS([K("x^3 + x^2"), K("x^3 + x"), K("x^3 + x^2 + x"), K("x^2 + x + 1")]); key

[    x^3 + x^2       x^3 + x]
[x^3 + x^2 + x   x^2 + x + 1]
sage: C = maes.encrypt(P, key); C

[            x       x^2 + x]
[x^3 + x^2 + x       x^3 + x]

# Here is the decryption process:
sage: plaintxt = maes.decrypt(C, key)
sage: plaintxt == P
True

# We can also work directly with binary strings:
sage: from sage.crypto.block_cipher.miniaes import MiniAES
sage: maes = MiniAES()
sage: bin = BinaryStrings()
sage: key = bin.encoding("KE"); key
0100101101000101
sage: P = bin.encoding("Encrypt this secret message!")
sage: C = maes(P, key, algorithm="encrypt")
sage: plaintxt = maes(C, key, algorithm="decrypt")
sage: plaintxt == P
True

# Or work with integers n such that 0 <= n <= 15:
sage: from sage.crypto.block_cipher.miniaes import MiniAES
sage: maes = MiniAES()
sage: P = [n for n in xrange(16)]; P
[0, 1, 2, 3, 4, 5, 6, 7, 8, 9, 10, 11, 12, 13, 14, 15]
sage: key = [2, 3, 11, 0]; key
[2, 3, 11, 0]
sage: P = maes.integer_to_binary(P)
sage: key = maes.integer_to_binary(key)
sage: C = maes(P, key, algorithm="encrypt")
sage: plaintxt = maes(C, key, algorithm="decrypt")
sage: plaintxt == P
True
\end{Verbatim}
\caption{Encryption and decryption with Mini-AES}
\end{sagecode}








% --------------------------------------------------------------------------
\newpage
\begin{thebibliography}{99999}
\addcontentsline{toc}{section}{Bibliography}

\bibitem[Biryukov2009]{cm:Biryukov2009} \index{Biryukov 2009}
	Alex Biryukov, Dmitry Khovratovich, \\
	{\em Related-key Cryptanalysis of the Full AES-192 and AES-256}, \\
	2009 \\
	\url{http://eprint.iacr.org/2009/317}

\bibitem[Coppersmith2002]{cm:Coppersmith2002}  \index{Coppersmith 2002}
        Don Coppersmith, \\
        {\em Re: Impact of Courtois and Pieprzyk results}, \\
        2002-09-19, ``AES Discussion Groups''~ at \\
        \href{http://aes.nist.gov/aes/}
        {\texttt{http://aes.nist.gov/aes/}}

\bibitem[Courtois2002]{cm:Courtois2002}  \index{Courtois 2002}
        Nicolas Courtois, Josef Pieprzyk, \\
        {\em Cryptanalysis of Block Ciphers with Overdefined Systems
             of Equations}, \\
        received 10 Apr 2002, last revised 9 Nov 2002.\\
        A different version, so called compact version of the first XSL attack,
        was published at Asiacrypt Dec 2002. \\
        \href{http://eprint.iacr.org/2002/044}
        {\texttt{http://eprint.iacr.org/2002/044}}

\bibitem[Ferguson2001]{cm:Ferguson2001}  \index{Ferguson 2001}
       Niels Ferguson, Richard Schroeppel, Doug Whiting, \\
       {\em A simple algebraic representation of Rijndael}, 
       Draft 2001/05/1, \\
       \href{http://www.xs4all.nl/~vorpal/pubs/rdalgeq.html}
       {\texttt{http://www.xs4all.nl/\~{}vorpal/pubs/rdalgeq.html}}

\bibitem[Hoffman2006]{cm:Hoffman2006} \index{Hoffman 2006} 
       Nick Hoffman, \\
	2006 \\
       {\em A SIMPLIFIED IDEA ALGORITHM},\\
       \url{http://www.nku.edu/~christensen/simplified%20IDEA%20algorithm.pdf}

\bibitem[Lucks-DuD2002]{cm:Lucks-DuD2002}  \index{Lucks 2002}
       Stefan Lucks, R\"udiger Weis, \\
       {\em Neue Ergebnisse zur Sicherheit des Verschl\"usselungsstandards AES},\\
       In: DuD, Dec 2002.

\bibitem[Mansoori-etal2007]{cm:Mansoori-etal2007} \index{Mansoori, Bizaki 2007} 
       S. Davod. Mansoori, H. Khaleghei Bizaki, \\
       {\em On the vulnerability of Simplified AES Algorithm Against Linear Cryptanalysis}, \\
       In: IJCSNS International Journal of Computer Science and Network Security, VOL.7 No.7,
           July 2007, pp~257-263 \\
       See also:\\
       \url{http://paper.ijcsns.org/07_book/200707/20070735.pdf}

\bibitem[Musa-etal2003]{cm:Musa-etal2003} \index{Musa, Schaefer, Wedig 2003} 
       Mohammad A. Musa, Edward F. Schaefer, Stephen Wedig, \\
       {\em A simplified AES algorithm and its linear and differential cryptanalyses}, \\
       In: Cryptologia 17 (2), April 2003, pp~148-177.\\
       See also:\\
       \url{http://findarticles.com/p/articles/mi_qa3926/is_200304/ai_n9181658}\\
       \url{http://www.rose-hulman.edu/~holden/Preprints/s-aes.pdf}\\
       \url{http://math.scu.edu/~eschaefe/}   Ed Schaefer's Homepage

\bibitem[Nichols1996]{cm:Nichols1996} \index{Nichols 1996} 
       Randall K. Nichols, \\
       {\em Classical Cryptography Course, Volume 1 and 2}, \\
       Aegean Park Press 1996;
       or in 12 lessons online at \\
       \href{http://www.fortunecity.com/skyscraper/coding/379/lesson1.htm}
       {\texttt{http://www.fortunecity.com/skyscraper/coding/379/lesson1.htm}}

\bibitem[Phan2002]{cm:Phan2002} \index{Phan 2002} 
       Raphael Chung-Wei Phan, \\
       {\em Mini Advanced Encryption Standard (Mini-AES): A Testbed for
            Cryptanalysis Students}, \\
       In: Cryptologia 26 (4), 2002, pp~283-306.

\bibitem[Phan2003]{cm:Phan2003} \index{Phan 2003} 
       Raphael Chung-Wei Phan, \\
       {\em Impossible differential cryptanalysis of Mini-AES}, \\
       In: Cryptologia, Oct 2003\\
       \url{http://findarticles.com/p/articles/mi_qa3926/is_200310/ai_n9311721/}

\bibitem[Robshaw2002a]{cm:Robshaw2002a}  \index{Robshaw 2002}
       S.P. Murphy, M.J.B. Robshaw, \\
       {\em Essential Algebraic Structure within the AES}, \\
       June 5, 2002, Crypto 2002,  \\
       \href{http://www.isg.rhul.ac.uk/\~{}mrobshaw/rijndael/rijndael.html}
       {\texttt{http://www.isg.rhul.ac.uk/~mrobshaw/rijndael/rijndael.html}}

\bibitem[Robshaw2002b]{cm:Robshaw2002b}  \index{Robshaw 2002}
       S.P. Murphy, M.J.B. Robshaw, \\
       {\em Comments on the Security of the AES and the XSL Technique}, \\
       September 26, 2002, \\
       \href{http://www.isg.rhul.ac.uk/\~{}mrobshaw/rijndael/rijndael.html}
       {\texttt{http://www.isg.rhul.ac.uk/~mrobshaw/rijndael/rijndael.html}}

\bibitem[Schaefer1996]{cm:Schaefer1996} \index{Schaefer 1996} 
       Edward F. Schaefer, \\
       {\em A Simplified Data Encryption Standard Algorithm}, \\
       In: Cryptologia 20 (1), 1996, pp~77-84

\bibitem[Schmeh2003]{cm:Schmeh2003}  \index{Schmeh 2003}
       Klaus Schmeh, \\
       {\em Cryptography and Public Key Infrastructures on the Internet},\\ 
       John Wiley \& Sons Ltd., Chichester 2003. \\
       An up-to-date, easy to read book, which also considers practical
       problems such as standardisation or real existing software.

\bibitem[Schneier1996]{cm:Schneier1996cm} \index{Schneier 1996} 
       Bruce Schneier, \\
       {\em Applied Cryptography, Protocols, Algorithms, and Source Code in C}, \\
       Wiley 1994, 2nd edition 1996.

\bibitem[Stallings2006]{cm:Stallings2006} \index{Stallings 2006} 
       William Stallings, \\
       {\em Cryptography and Network Security}, \\
       Prentice Hall 2006,\\
       \url{http://williamstallings.com/}

\bibitem[Stamp2007]{cm:Stamp2007} \index{Stamp 2007} 
       Mark Stamp, Richard M. Low, \\
       {\em Applied Cryptanalysis: Breaking Ciphers in the Real World}, \\
       Wiley-IEEE Press 2007, \\
       \href{http://cs.sjsu.edu/faculty/stamp/crypto/}
       {\texttt{http://cs.sjsu.edu/faculty/stamp/crypto/}}

\bibitem[Swenson2008]{cm:Swenson2008} \index{Swenson 2008} 
       Christopher Swenson, \\
       {\em Modern Cryptanalysis: Techniques for Advanced Code Breaking}, \\
       Wiley 2008.

\bibitem[Wobst-iX2002]{cm:Wobst-iX2002}  \index{Wobst 2002}
       Reinhard Wobst, \\
       {\em Angekratzt - Kryptoanalyse von AES schreitet voran}, \\
       In: iX, Dec 2002; \\
       plus the reader's remark by Johannes Merkle in iX Feb 2003.

\end{thebibliography}



% --------------------------------------------------------------------------
\newpage
\section*{Web links}
\addcontentsline{toc}{section}{Web links}

\begin{enumerate}

\item AES or Rijndael cryptosystem \\
        \href{http://www.cryptosystem.net/aes}
         {\tt http://www.cryptosystem.net/aes} \\
	\href{http://www.minrank.org/aes/}
         {\tt http://www.minrank.org/aes/}

  \item AES discussion groups at NIST \\
	\href{http://aes.nist.gov/aes}{\tt http://aes.nist.gov/aes}

  \item distributed.net: ``RC5-64 has been solved'' \\
        \href{http://distributed.net/pressroom/news-20020926.html}
         {\tt http://distributed.net/pressroom/news-20020926.html}

  \item RSA: ``The RSA Secret Key Challenge'' \\
      \href{http://www.rsasecurity.com/rsalabs/challenges/secretkey/index.html}
       {\tt http://www.rsasecurity.com/rsalabs/challenges/secretkey/index.html}

  \item RSA: ``DES Challenge'' \\
        \href{http://www.rsasecurity.com/rsalabs/challenges/des3/index.html}
         {\tt http://www.rsasecurity.com/rsalabs/challenges/des3/index.html}

  \item Further links can be found at the CrypTool homepage \\
        \href{http://www.cryptool.org}
         {\tt http://www.cryptool.org}
	       
\end{enumerate}



% Local Variables:
% TeX-master: "../script-en.tex"
% End:
