% ............................................................................
%      V O R W O R T  und  E I N F � H R U N G (Zusammenspiel Skript-CT) 
% ~~~~~~~~~~~~~~~~~~~~~~~~~~~~~~~~~~~~~~~~~~~~~~~~~~~~~~~~~~~~~~~~~~~~~~~~~~~~


% --------------------------------------------------------------------------
\section*{Preface to the 6th Edition of the CrypTool Script}  \addcontentsline{toc}{section}{Preface to the 6th Edition of the CrypTool Script}

Starting in the year 2000 this script became a part of the 
CrypTool\index{CrypTool} package. It should accomplish the program 
CrypTool by explaining some mathematical topics in more details, 
but still in a way which is easy to.

In order to also enable independent developers/authors work together 
the topics have been split up and for each topic an extra chapter was 
written readable by its own. The later editorial work in TeX added 
cross linkages between different sections and footnotes, where you



can find the according 
functions within the CrypTool\index{CrypTool} program. Naturally 
there are much more interesting topics in mathematics and cryptography 
which could be more expanded -- therefore this is only one of many
ways to do it.

Rapid spread of the Internet has also lead to an intensified research 
in the technologies involved, especially within the area of cryptography 
where a good deal of new knowledge came up.

This edition of the script updates the summaries of the following 
topical research areas:
\vspace{-7pt}
\begin{itemize}
  \item the search for the largest prime numbers (generalized Mersenne 
        and Fermat primes), 
  \item progress in number theory ("Primes in P"), 
  \item the factorisation of big numbers (TWIRL) and 
  \item progress in cryptanalysis of the AES standard. 
% \item usage of elliptic curves [not only over GP(p)].
\end{itemize}

The first time the script was delivered with CrypTool\index{CrypTool} 
was in version 1.2.01. Since then it has been extended and updated with each 
new version of CrypTool (1.2.02, 1.3.00, 1.3.02, 1.3.03 and now 1.3.04).

I'd be more than happy if this keeps on going also in the further 
open source versions of CrypTool\index{CrypTool}.

I am deeply grateful to all the people helping with their impressive
committment to make this project successful and wide-spreading.

I hope that many readers have fun with this script and that they get 
out of it more interest and greater understanding for this modern but 
also very ancient topic.
\\

Bernhard Esslinger

Frankfurt, March 2003



% --------------------------------------------------------------------------
\newpage
\section*{Introduction -- How do the Script and CrypTool Play together?}  \addcontentsline{toc}{section}{Introduction -- How do the Script and CrypTool Play together?}


\textbf{This script}

This script is delivered together with CrypTool\index{CrypTool}. \par \vskip + 3pt

Because the articles in this script are largely self-contained, this script
can also be read independently of CrypTool\index{CrypTool}.

The {\em authors} have attempted to describe cryptography for a broad 
audience -- without being mathematically incorrect. We believe that this
didactical pretension is the best way to promote the awareness for IT
security and the readiness to use standardised modern cryptography.
\par \vskip + 15pt


\textbf{The program CrypTool\index{CrypTool}}

CrypTool\index{CrypTool} is a program with an extremely comprehensive online
help enabling you to use and analyse cryptographic procedures within a
unified graphical user interface.\par \vskip + 3pt

CrypTool\index{CrypTool} was developed during the end-user awareness program
in order to increase employee awareness of IT security and provide them with
a deeper understanding of the term security.\par \vskip + 3pt

A further aim has been to enable users to understand the cryptographic
procedures implemented in the Deutsche Bank. In this way, using CrypTool
as a reliable reference implementation of the various encryption procedures
(because of using the industry-proven Secude Library\index{Secude}),
you can test the encryption implemented in other programs. \par \vskip + 3pt

Since then CrypTool\index{CrypTool} has often been used for 
education in companies and teaching at school and universities and
moreover several universities are helping to further develop the project.
\par \vskip + 15pt


\textbf{Acknowledgment}

At this point I'd like to thank explicitely 3 people who especially 
contributed to CrypTool\index{CrypTool}. Without their talents and engagement 
CrypTool would not be what it is today:
\vspace{-7pt}
\begin{itemize}
   \item Mr. Henrik Koy
   \item Mr. J\"org-Cornelius Schneider and
   \item Dr. Peer Wichmann.
\end{itemize}
Also I want to thank all the many people not mentioned here for their 
hard work (mostly carried out in their spare time).
\\

Bernhard Esslinger


% Local Variables:
% TeX-master: "../script-en.tex"
% End:
