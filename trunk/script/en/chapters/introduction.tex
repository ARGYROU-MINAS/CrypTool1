% $Id$
% ............................................................................
%      P R E F A C E  and  I N T R O D U C T I O N  (Playing together of book and programs)
%
% ~~~~~~~~~~~~~~~~~~~~~~~~~~~~~~~~~~~~~~~~~~~~~~~~~~~~~~~~~~~~~~~~~~~~~~~~~~~~

\clearpage\phantomsection
\addcontentsline{toc}{chapter}{Preface to the 12th Edition of the CrypTool Book}
\chapter*{Preface to the 12th Edition of the CrypTool Book}

The book's goal is to explain some mathematical topics from cryptology in
exact detail, nevertheless in a way which is easy to understand.

This book was delivered since the year 2000 -- together with the CrypTool~1
(CT1)\index{CT1} package in version 1.2.01. 
Since then it has been expanded and revised in almost every new version of CT1 and CT2.

Topics from mathematics and cryptography have been meaningfully split up and
for each topic an extra chapter has been written which can be read on its own.
This enables developers/authors to contribute independently of each other.
Naturally there are many more interesting topics in cryptography which could
be discussed in greater depth -- therefore this selection is only one of many
possible ways.

The later editorial work in LaTeX added footnotes and cross linkages between
different sections, harmonized the index entries, and made some corrections.

%This edition of the CT book again updated some topics:
Compared to edition 11, this edition completely updated the TeX sources
of the document (e.g. one single bibtex file for all chapters and both languages),
and of course, the content of the book was amended, corrected, and
updated with many topics, for instance:
\vspace{-5pt}
\begin{itemize}
%  \item the definitions of the strength of security functions
%        (chapter \ref{cm_Section_Security_Definitions}),
  \item the largest prime numbers (chap. \ref{search_for_very_big_primes}),
%        new factorization records (chap. \ref{RSA-768}),
%  \item progress in cryptanalysis of AES 
%        (chap. \ref{NeueAES-Analyse}) and
%  \item progress in cryptanalysis of hash algorithms 
%        (chap. \ref{collision-attacks-against-sha-1}) and
%  \item progress in ideas for new crypto methods (RSA successor) 
%        (chap. \ref{xxxxxxxxxcm_Brute-force-versus-Symmetr})\index{xxxxxxxxxx} and
  \item the list of movies or novels, in which cryptography or number theory 
        played a major role (see appendix \ref{s:appendix-movies}),
%       and where primes are used as hangers  
%        (see curiouses in \ref{HT-Quaint-curious-Primes-usage}),
  \item the overviews of all functions in
        \hyperlink{appendix-template-overview-CT2}{CrypTool~2 (CT2)},
        \hyperlink{appendix-function-overview-JCT}{JCrypTool (JCT)}, and
        \hyperlink{appendix-function-overview-CTO}{CrypTool-Online (CTO)}
        (see appendix),
  \item further SageMath scripts for cryptography, and the appendix
        \ref{s:appendix-using-sage} about using the computer algebra system
        SageMath,
        % Nguyen and Massierer (see also chap.~\ref{ec:Sage_Massierer}).
  \item the section about the Goldbach conjecture
        (see \ref{L-GoldbachConjecture}) and about twin primes
        (see \ref{L-TwinCousinPrimes}),
  \item the section about shared primes in RSA modules used in reality
        (see \ref{nt_Shared-Primes}),
%  \item the section about RSA fixed points
%        (see \ref{l:NumberTheory_Sage_Number-of-RSA-FixedPoints}), and
%  \item homomorphic encryption
%        (see chapter \ref{Chapter_HomomorphicCiphers}).
  \item the ``\nameref{Chapter_BitCiphers}'' is completely new
        (see chapter \ref{Chapter_BitCiphers}),
  \item the ``\nameref{Chapter_Dlog-FactoringDead}'' is completely new too
        (see chapter \ref{Chapter_Dlog-FactoringDead}). It's a phantastic
	in-depth summary about the limits of the according current
	cryptanalytical methods.
\end{itemize}

%In the meantime the CT project gets feedback and testimonials from almost all
%countries of the planet.


\newpage
\noindent \textbf{Acknowledgment}

At this point I'd like to thank explicitly the following people who
particularly contributed to the CrypTool project\index{CrypTool}. They
applied their very special talents and showed really great engagement:
\vspace{-7pt}
%\begin{itemize}
\begin{list}{\textbullet}{\addtolength{\itemsep}{-0.5\baselineskip}}
   \item Mr.\ Henrik Koy
   \item Mr.\ J\"org-Cornelius Schneider
   \item Mr.\ Florian Marchal
   \item Dr.\ Peer Wichmann
   \item Mr.\ Dominik Schadow
   \item Staff of 
         Prof.\ Johannes Buchmann,
         Prof.\ Claudia Eckert,
         Prof.\ Alexander May,
         Prof.~Torben Weis, and especially
         Prof.~Arno Wacker.
\end{list}

Also I want to thank all the many people not mentioned here for their 
hard work (mostly carried out in their spare time).

Thanks also to the readers who sent us feedback. And especial thanks for the
free proof reading of this edition done by Helmut Witten and Prof.\ Ralph-Hardo Schulz.
% Especially I would like to acknowledge the English language proof-reading
% of this book version done by Richard Christensen and Lowell Montgomery. Tom.

I hope that many readers have fun with this book and that they get 
out of it more interest and greater understanding of this modern but 
also very ancient topic.



\par \vskip + 35pt
\noindent Bernhard Esslinger
\par \vskip + 10pt
\noindent Heilbronn/Siegen (Germany), August 2016



\par \vskip + 90pt
\noindent PS:\\
We'd be glad if new authors would show up to improve existing chapters or to add further chapters,
e.g. about
\begin{compactitem}
   \item the Riemann Zeta function,
   \item hash functions and password attacks,
   \item lattice-based cryptography,
   \item random numbers, or
   \item the design and attack of crypto protocols (like SSL).
\end{compactitem}


\par \vskip + 15pt
\noindent PPS:\\%xxxxxxxxxx Todo
Todos to be dealt with to make edition 12 of this CTB a release (till then we still call it a draft):%xxxxxxxxxxxxx
\begin{compactitem}
   \item Update all information about SageMath (chap.~\ref{ec:Sage_Massierer}
      and appendix) and run the code against the newest SageMath (version 7.x) -- 
      both command line and SageMathCloud notebook.
   \item Update the function lists of the four CT versions (in the appendix).
\end{compactitem}









% --------------------------------------------------------------------------
\clearpage\phantomsection
\addcontentsline{toc}{chapter}{Introduction -- How do the Book and the Programs Play together?}
\chapter*{Introduction -- How do the Book and the Programs Play together?}

\textbf{This CrypTool book}

This document is delivered together with the open-source programs of the
CrypTool project\index{CrypTool}. You can also download it directly from the
website of the CT portal: \url{https://www.cryptool.org/en/ctp-documentation}.

The chapters in this book are largely self-contained and
can also be read independently of the CrypTool programs.

Chapters \ref{Chapter_ModernCryptography} (``Modern Cryptography''),
\ref{Chapter_EllipticCurves} (``\nameref{Chapter_EllipticCurves}''),
\ref{Chapter_BitCiphers} (``Bitblock and Bitstream Ciphers''), %% (``\nameref{Chapter_BitCiphers}'')
\ref{Chapter_HomomorphicCiphers} (``\nameref{Chapter_HomomorphicCiphers}''), and
\ref{Chapter_Dlog-FactoringDead} (``Results for Solving Discrete
Logarithms and for Factoring'') % Shortened as original title is too long.
require a deeper knowledge in mathematics, while the other chapters should
be understandable with a school leaving certificate.

The \hyperlink{appendix-authors}{authors}
have attempted to describe cryptography for a broad 
audience -- without being mathematically incorrect. We believe that this
didactic pretension is the best way to promote the awareness for IT
security and the readiness to use standardized modern cryptography.
\par \vskip + 15pt


\noindent \textbf{The programs CrypTool~1, CrypTool~2, and JCrypTool}
\index{CT1} \index{CT2} \index{JCT}
% \noindent \textbf{The programs CrypTool~1\index{CrypTool~1}\index{CT1},
%   CrypTool~2\index{CrypTool~2}\index{CT2}, and JCrypTool\index{JCrypTool}\index{JCT}}
%%% Solved-Problem__Only in English__Separated index at p144 (4.11.2) and p217 (5.3) and
%%%     at page xviii, but \index{CrypTool~1} was called like everywhere else!
%%%     In *.idx stand statt \indexentry{CrypTool~1|hyperpage}{iii}
%%%     ein                  \indexentry{CrypTool\nobreakspace  {}1|hyperpage}{xviii}
%%%  --> Fixed here by moving the \index out of \textbf.
%%%  --> Fixed in chap 4 (below \label{chptSecurityParam}) by moving the \index out of
%%%      \paragraph (done via moving footnote out of paragraph using footnotemark).
%%%  --> Fixed in 5.3 by moving the \index out of \subsection-Klammern using footnotemark).
%%%      BEFORE_MOVING_OUT_INDEX: \subsection[The RSA procedure]{The RSA procedure\footnote{%
%%%  See: http://wwws.htwk-leipzig.de/~myagovki/latex/fussnoten_und_randnotizen/
%%%       https://en.wikibooks.org/wiki/LaTeX/Counters

CrypTool~1 (CT1) is an educational program enabling you to use and analyze
cryptographic procedures within a unified graphical user interface. The
comprehensive online help in CrypTool~1 contains both
instructions how to use the program and explanations about the methods itself
(both not as detailed and in another structure as in the CT book).

CrypTool~1 and the successor versions CrypTool~2 (CT2) and JCrypTool (JCT)
are used world-wide for training in companies and teaching at schools and
universities.
\par \vskip + 15pt


\noindent \textbf{CrypTool-Online\index{CTO}}

Another part of the CT project is the website CrypTool-Online (CTO)
(\url{http://www.cryptool-online.org}), where you can apply
crypto methods within a browser or on a smartphone. The scope
of CTO is far below from the standalone programs CT1, CT2 and JCT.
However, this is what people more and more use as the first contact,
so we currently redesign the whole backbone system with most modern
web technology to have a fast and consistent look\&feel.
\par \vskip + 15pt


\noindent \textbf{MTC3\index{MTC3}}

MTC3 is the abbreviation for MysteryTwister C3, an international cryptography
contest (\url{http://www.mysterytwisterc3.org}), which is also based on the
CT project.
Here you can find cryptographic riddles in four categories, a high-score list
and a moderated forum. As of 2016-06-16 more than 7000 users participate,
and more than 200 challenges are offered (162 of them are solved by
at least one participant).
\par \vskip + 25pt  % vskip: more than +15 to start SageMat the next page



\noindent \textbf{The Computer Algebra System SageMath\index{SageMath}}

SageMath is a comprehensive open-source CAS package which can be used to easily
program the mathematical methods explained in this book. A speciality of
this CAS is, that the scripting language is Python (currently version 2.x).
So in a Sage script, you have after an import statement all functions from the
Python language at your disposal.\\
SageMath becomes more and more the standard CAS system at universities.
\par \vskip + 15pt



\noindent \textbf{The Pupil's Crypto Courses\index{pupil's crypto}}

Within this initiative, one and two 2 day courses in cryptology are offered
for pupils and teachers in order to show how attractive MINT subjects like
mathematics, computer science and especially cryptology are.
The course agenda is a virtual secret agent story.\\
In the meantime, these courses took place for several years in Germany in
different towns.\\
All course material is freely available at
\url{http://www.cryptool.org/schuelerkrypto/}.\\
All software used is free software (using mostly CT1 and CT2).\\
As all course material is currently available only in German -- we'd be happy
if someone could translate the course material and build an according course
in English.
\par \vskip + 15pt



\noindent \textbf{Acknowledgment}

I am deeply grateful to all the people helping with their impressive
commitment who have made this global project so successful.



\par \vskip + 45pt
\noindent Bernhard Esslinger
\par \vskip + 10pt
\noindent Heilbronn/Siegen (Germany), August 2016

% Local Variables:
% TeX-master: "../script-en.tex"
% End:
