% $Id$
% ............................................................................
%      V O R W O R T  und  E I N F � H R U N G (Zusammenspiel Skript-CT) 
% ~~~~~~~~~~~~~~~~~~~~~~~~~~~~~~~~~~~~~~~~~~~~~~~~~~~~~~~~~~~~~~~~~~~~~~~~~~~~


% --------------------------------------------------------------------------
\clearpage\phantomsection
\addcontentsline{toc}{chapter}{Preface to the 10th Edition of the CrypTool Script}
\chapter*{Preface to the 10th Edition of the CrypTool Script}

Starting in the year 2000 this script became part of the 
CrypTool 1 (CT1)\index{CrypTool 1.x} package. It is designed to accompany the
program CrypTool by explaining some mathematical topics in more detail, 
but still in a way which is easy to understand.

In order to also enable developers/authors to work together independently 
the topics have been split up and for each topic an extra chapter has been 
written which can be read on its own. The later editorial work in TeX added 
cross linkages between different sections and footnotes describing where you
can find the according functions within the CrypTool 1\index{CrypTool 1.x} 
program \hyperlink{appendix-menutree}{(see menu tree} in appendix \ref{s:appendix-menutree}).
% \hypertarget{appendix-menutree}{}\label{s:appendix-menutree}
Naturally there are many more interesting topics in mathematics and
cryptography which could be discussed in greater depth -- therefore this
is only one of many ways to do it.

The rapid spread of the Internet has lead to intensified research in the
technologies involved, especially within the area of cryptography where a good
deal of new knowledge has arisen.

%This edition of the script adds some topics, but mainly updates areas (e.g. the
%summaries of topical research areas):
This edition completely updated the TeX sources of the document, and of course
the content of the script was corrected, amended and updated with some topics, e.g.:
\vspace{-5pt}
\begin{itemize}
  \item the largest prime numbers (chap. \ref{search_for_very_big_primes}),
        new factorization records (chap. \ref{RSA-768}),
  \item progress in cryptanalysis of AES 
        (chap. \ref{NeueAES-Analyse}) and
%  \item progress in cryptanalysis of hash algorithms 
%        (chap. \ref{collision-attacks-against-sha-1}) and
%  \item progress in ideas for new crypto methods (RSA successor) 
%        (chap. \ref{xxxxxxxxxBrute-force-gegen-Symmetr})\index{xxxxxxxxxxxxxxxx} and
  \item the list of movies or novels, in which cryptography or number theory 
        played major role (see appendix \ref{s:appendix-movies});
        and where primes are used as hangers  
        (see curiouses in \ref{HT-Quaint-curious-Primes-usage}).
  \item Newly added are e.g. the section about Benford's law and primes
        (chap. \ref{primes:Benford-Law}), and 
        the appendix \ref{s:appendix-using-sage} about using the
        computer algebra system Sage. Sage becomes
        more and more the standard open-source CAS system. Accordingly all
        samples written before in Pari-GP and Mathematica have been substituted
        with Sage code. Thanks to Nguyen and Massierer, a lot of new code
        samples could be added (see also chap.~\ref{ec:Sage_Massierer}).
\end{itemize}

The first time the document was delivered with CrypTool\index{CrypTool} 
was in version 1.2.01. Since then it has been expanded and revised in almost
every new version of CrypTool.

I am deeply grateful to all the people helping with their impressive
commitment who have made this global project so successful.
Thanks also to the readers who sent us feedback.
% Especially I would like to acknowledge the English language proof-reading
% of this script version done by Richard Christensen and Lowell Montgomery.

I hope that many readers have fun with this script and that they get 
out of it more interest and greater understanding of this modern but 
also very ancient topic.
\\
\\
% [1.5\baselineskip]
% \enlargethispage*{2\baselineskip}
% \nopagebreak
Bernhard Esslinger
\\
\\
% [\baselineskip]
Frankfurt (Germany), January 2010



% --------------------------------------------------------------------------
\clearpage\phantomsection
\addcontentsline{toc}{chapter}{Introduction -- How do the Script and the Program Play together?}
\chapter*{Introduction -- How do the Script and the Program Play together?}

\textbf{This script}

This document is delivered together with the open-source program CrypTool 1\index{CrypTool}.

The articles in this script are largely self-contained and
can also be read independently of CrypTool\index{CrypTool}.

Chapters  \ref{Chapter_ModernCryptography} (Modern Cryptography) and 
\ref{Chapter_EllipticCurves} (Elliptic Curves) require a deeper knowledge
in mathematics, while the other chapters should be understandable with a 
school leaving certificate.

The \hyperlink{appendix-authors}{authors}
have attempted to describe cryptography for a broad 
audience -- without being mathematically incorrect. We believe that this
didactic pretension is the best way to promote the awareness for IT
security and the readiness to use standardized modern cryptography.
\par \vskip + 15pt


\noindent \textbf{The program CrypTool 1\index{CrypTool}}

CrypTool 1 (CT1)\index{CrypTool} is an educational program with a comprehensive
online help enabling you to use and analyse cryptographic procedures within a
unified graphical user interface.

CrypTool\index{CrypTool} is used world-wide for training in companies and
teaching at schools and universities worldwide, and
several universities are helping to further develop the project.
\par \vskip + 15pt


\noindent \textbf{Acknowledgment}

At this point I'd like to thank explicitly the following people who
particularly contributed to CT1\index{CrypTool}. They applied their
very special talents and showed really great engagement:
\vspace{-7pt}
%\begin{itemize}
\begin{list}{\textbullet}{\addtolength{\itemsep}{-0.5\baselineskip}}
   \item Mr.\ Henrik Koy
   \item Mr.\ J\"org-Cornelius Schneider
   \item Mr.\ Florian Marchal
   \item Dr.\ Peer Wichmann
   \item Staff of Prof.\ Claudia Eckert, Prof.\ Johannes Buchmann,
         Prof.\ Torben Weis and Prof.\ Arno Wacker..
\end{list}
%\end{itemize}
Also I want to thank all the many people not mentioned here for their 
hard work (mostly carried out in their spare time).
\\
\\
Bernhard Esslinger
\\
\\
Frankfurt (Germany), July 2012

% Local Variables:
% TeX-master: "../script-en.tex"
% End:
