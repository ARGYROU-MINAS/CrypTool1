% ............................................................................
%                 TEXT DER 1. SEITE
% ~~~~~~~~~~~~~~~~~~~~~~~~~~~~~~~~~~~~~~~~~~~~~~~~~~~~~~~~~~~~~~~~~~~~~~~~~~~~

\parskip 4pt
\vskip + 30 pt
{
In diesem Skript zu dem Programm CrypTool\index{CrypTool} finden Sie
eher mathematisch orientierte Informationen zum Einsatz von
kryptographischen Verfahren. Die Hauptkapitel sind von
{\em verschiedenen Autoren\/} verfasst und in sich abgeschlossen. Am
Ende der meisten Kapitel finden Sie jeweils Literaturangaben und
URL's.

Sie erhalten Informationen "uber die Prinzipien der symmetrischen
und asymmetrischen Verschl"usselung. Ein gro"ser Teil des
Skripts ist dem faszinierenden Thema der Primzahlen gewidmet.
Anhand vieler Beispiele wird in die modulare Arithmetik und die elementare
Zahlentheorie eingef"uhrt, die dann beispielhaft beim RSA-Verfahren
angewandt werden. Danach erhalten Sie Einblick in die mathematischen Ideen
hinter der modernen Kryptographie.

Ein weiteres Kapitel widmet sich kurz den digitalen Signaturen ---
sie sind unverzichtbarer Bestandteil von e-Business Applikationen.
Dazu passt das letzte Kapitel: Elliptische Kurven. Die Mathematik der
Elliptischen Kurven ist Grundlage f"ur sehr schnelle kryptographische
Algorithmen zur digitalen Signatur; diese Algorithmen sind f"ur die
Implementierung auf Chipkarten sehr gut geeignet.

W"ahrend das Programm CrypTool\index{CrypTool} eher den praktischen Umgang
vermittelt, dient das Skript dazu, dem an Kryptographie Interessierten 
ein tieferes Verst"andnis f"ur die implementierten mathematischen 
Algorithmen zu vermitteln -- und das didaktisch
m"oglichst gut nachvollziehbar.

Die {\em Autoren\/} Bernhard Esslinger, Mathias B"uger, Bartol Filipovic, 
Henrik Koy, Roger Oyono und J"org Cornelius Schneider 
m"ochten sich an dieser Stelle bedanken bei den Kollegen 
in der Firma und an den Universit"aten Frankfurt, Gie"sen, Siegen und Karlsruhe, insbesondere
bei Dr. Peer Wichmann vom Forschungszentrum Informatik (FZI) Karlsruhe f"ur
die unkomplizierte Unterst"utzung.

\enlargethispage{0.5cm}
Wie auch bei CrypTool w"achst die Qualit"at des Skripts mit Ihren Anregungen
und Verbesserungen. Wir freuen uns "uber Ihre R"uckmeldung.


\vskip +7 pt \noindent
Die aktuelle Version von CrypTool\index{CrypTool} finden Sie unter \newline
  \href{http://www.CrypTool.de}{\texttt{http://www.CrypTool.de}},~
  \href{http://www.cryptool.com}{\texttt{http://www.CrypTool.com}}~ oder ~
  \href{http://www.cryptool.org}{\texttt{http://www.CrypTool.org}}.
\vskip + 7 pt \noindent
Im Readme zu CrypTool sind die Ansprechpartner f"ur dieses kostenlose Tool genannt.
}



% Local Variables:
% TeX-master: "../script-de.tex"
% End:
