% $Id$
% ............................................................................
%                 TEXT DER 1. SEITE
% ~~~~~~~~~~~~~~~~~~~~~~~~~~~~~~~~~~~~~~~~~~~~~~~~~~~~~~~~~~~~~~~~~~~~~~~~~~~~

\parskip 4pt
%\vskip +12 pt
{
In diesem {\em Skript zu dem Programm CrypTool} \index{CrypTool} finden Sie
eher mathematisch orientierte Informationen zum Einsatz von
kryptographischen Verfahren. Die Hauptkapitel sind von
verschiedenen 
\hyperlink{appendix-authors}{Autoren}
(siehe Anhang \ref{s:appendix-authors})
verfasst und in sich abgeschlossen. Am Ende der meisten Kapitel finden Sie
jeweils Literaturangaben und Web-Links.

Sie erhalten Informationen "uber die Prinzipien der symmetrischen
und asymmetrischen {\bf Verschl"usselung}. Ein gro"ser Teil des
Skripts ist dem faszinierenden Thema der {\bf Primzahlen} gewidmet.
Anhand vieler Beispiele bis hin zum {\bf RSA-Verfahren} wird in die
{\bf modulare Arithmetik} und die {\bf elementare Zahlentheorie} eingef"uhrt.
Danach erhalten Sie Einblicke in die mathematischen Ideen
hinter der {\bf modernen Kryptographie}.

Ein weiteres Kapitel widmet sich kurz den {\bf digitalen Signaturen} ---
sie sind unverzichtbarer Bestandteil von E-Business-Anwendungen.
Das letzte Kapitel stellt {\bf Elliptische Kurven} vor: sie sind eine
Alternative zu RSA und f"ur die Implementierung auf Chipkarten
besonders gut geeignet.

W"ahrend das Programm CrypTool\index{CrypTool} eher den praktischen Umgang
vermittelt, dient das Skript dazu, dem an Kryptographie Interessierten 
ein tieferes Verst"andnis f"ur die implementierten mathematischen 
Algorithmen zu vermitteln -- und das didaktisch
m"oglichst gut nachvollziehbar.

Die Autoren Bernhard Esslinger, Matthias B"uger, 
Bartol Filipovic, Henrik Koy, Roger Oyono und J"org Cornelius
Schneider m"ochten sich an dieser Stelle bedanken bei den Kollegen 
in der Firma und an den Universit"aten Frankfurt, Gie"sen, Siegen,
Karlsruhe und Darmstadt, insbesondere bei Dr. Peer Wichmann vom 
Forschungszentrum Informatik (FZI) Karlsruhe f"ur
die unkomplizierte Unterst"utzung.

\enlargethispage{0.5cm}
Wie auch bei CrypTool\index{CrypTool} w"achst die Qualit"at des 
Skripts mit den Anregungen und Verbesserungsvorschl"agen von Ihnen
als Leser. Wir freuen uns "uber Ihre R"uckmeldung.


\vskip +7 pt \noindent
Die aktuelle Version von CrypTool\index{CrypTool} finden Sie jeweils
unter \newline
  \href{http://www.CrypTool.de}{\texttt{http://www.CrypTool.de}},~
  \href{http://www.cryptool.com}{\texttt{http://www.CrypTool.com}}~ oder ~
  \href{http://www.cryptool.org}{\texttt{http://www.CrypTool.org}}.
\vskip + 7 pt \noindent
Die Ansprechpartner f"ur dieses kostenlose Open Source-Tool\index{Open
Source} sind in der zum CrypTool-Paket dazugeh"orenden Readme-Datei genannt.
}



% Local Variables:
% TeX-master: "../script-de.tex"
% End:
