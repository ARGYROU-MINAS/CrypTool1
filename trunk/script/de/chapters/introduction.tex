% ............................................................................
%      V O R W O R T  und  E I N F � H R U N G (Zusammenspiel Skript-CT) 
% ~~~~~~~~~~~~~~~~~~~~~~~~~~~~~~~~~~~~~~~~~~~~~~~~~~~~~~~~~~~~~~~~~~~~~~~~~~~~


% --------------------------------------------------------------------------
\section*{Vorwort zur 6. Ausgabe des CrypTool-Skripts}  \addcontentsline{toc}{section}{Vorwort zur 6. Ausgabe des CrypTool-Skripts}

Seit dem Jahr 2000 wurde mit CrypTool\index{CrypTool} auch ein Skript 
ausgeliefert, das die Mathematik einzelner Themen genauer, aber doch 
m"oglichst verst"andlich erl"autern sollte.

Um auch hier die M"oglichkeit zu schaffen, dass getrennte Entwickler/Autoren
mitarbeiten k"onnen, wurden die Themen sinnvoll unterteilt und daf"ur jeweils
eigenst"andig lesbare Kapitel geschrieben. In der anschlie"senden 
redaktionellen Arbeit wurden in TeX Querverweise eingef"ugt. Dabei wurde
dann jeweils auch eingef"ugt, wo man die entsprechenden Funktionen in CrypTool\index{CrypTool}
aufrufen kann. Nat"urlich g"abe es viel mehr Themen in Mathematik und 
Kryptographie, die man vertiefen k"onnte -- deshalb ist diese Auswahl auch 
nur eine von vielen m"oglichen.

Die rasante Verbreitung des Internets hat auch zu einer verst"arkten 
Forschung der damit verbundenen Technologien gef"uhrt. Gerade im Bereich 
der Kryptographie gibt es viele neue Erkenntnisse.

Auf den aktuellen Stand gebracht wurden in dieser Ausgabe des Skripts die
Zusammenfassungen zu den Forschungsergebnissen "uber:                   
\begin{itemize}
\item die Suche nach den gr"o"sten Primzahlen (verallgemeinerte Mersenne- und Fermatzahlen),
\item Fortschritte in der Zahlentheorie ("Primes in P"),
\item die Faktorisierung gro"ser Zahlen (TWIRL) und
\item die Fortschritte bei der Kryptoanalyse des AES-Standard.
% \item den Einsatz Elliptischer Kurven [nicht nur GP(p)].
\end{itemize}

Seit das Skript dem CrypTool-Paket in Version 1.2.01 das 1. Mal beigelegt 
wurde, ist es mit jeder neuen Version von CrypTool (1.2.02, 1.3.00, 
1.3.02, 1.3.03 und nun 1.3.04) ebenfalls erweitert und aktualisiert worden.

Ich w"urde mich sehr freuen, wenn sich dies auch innerhalb der ab nun 
folgenden Open Source-Versionen von CrypTool\index{CrypTool} so weiter
fortsetzt.

Herzlichen Dank nochmal an alle, die mit Ihrem gro"sem Einsatz zum 
Erfolg und der weiten Verbreitung dieses Projektes beigetragen haben. 

Ich hoffe, dass viele Leser mit diesem Skript mehr Interesse an und 
Verst"andnis f"ur dieses moderne und zugleich uralte Thema finden.

Bernhard Esslinger

Frankfurt, M"arz 2003




% --------------------------------------------------------------------------
\newpage
\section*{Einf"uhrung -- Zusammenspiel Skript und CrypTool}  \addcontentsline{toc}{section}{Einf"uhrung -- Zusammenspiel Skript und CrypTool}


\textbf{Das Skript}

Dieses Skript wird zusammen mit CrypTool\index{CrypTool} ausgeliefert. \par \vskip + 3pt

Da die Artikel dieses Skripts weitgehend in sich abgeschlossen sind, kann
dieses Skript auch unabh"angig von CrypTool\index{CrypTool} gelesen werden.
\par \vskip + 3pt

Die {\em Autoren} haben sich bem"uht, Kryptographie f"ur eine m"oglichst 
breite Leserschaft darzustellen -- ohne mathematisch unkorrekt zu werden. 
Dieser didaktische Anspruch ist am besten geeignet, die Awareness
f"ur die IT-Sicherheit und die Bereitschaft f"ur den Einsatz 
standardisierter, moderner Kryptographie zu f"ordern.
\par \vskip + 15pt


\textbf{Das Programm CrypTool\index{CrypTool}}

CrypTool\index{CrypTool} ist ein Programm mit sehr umfangreicher Online-Hilfe,
mit dessen Hilfe Sie unter einer einheitlichen Oberfl"ache
kryptographische Verfahren anwenden und analysieren k"onnen.
\par \vskip + 3pt

CrypTool\index{CrypTool} wurde im Zuge des End-User Awareness-Programmes
entwickelt, um die Sensibilit"at der Mitarbeiter f"ur IT-Sicherheit
zu erh"ohen und um ein tieferes Verst"andnis f"ur den Begriff Sicherheit
zu erm"oglichen. 
\par \vskip +3pt 

Ein weiteres Anliegen war die Nachvollziehbarkeit der in der Deutschen Bank
eingesetzten kryptographischen Verfahren. So ist es mit CrypTool als 
verl"asslicher Referenzimplementierung der verschiedenen 
Verschl"usselungsverfahren (aufgrund der Nutzung der Industrie-bew"ahrten 
\index{Secude GmbH} Secude-Bibliothek) auch m"og\-lich, die in anderen 
Programmen eingesetzte Verschl"usselung zu testen.\par \vskip + 3pt

Inzwischen wird CrypTool\index{CrypTool} vielerorts
in Ausbildung und Lehre eingesetzt und von mehreren Hochschulen
mit weiterentwickelt. \par \vskip + 13pt


An dieser Stelle m"ochte ich explizit drei Personen danken, 
die bisher in ganz besonderer Weise zu CrypTool\index{CrypTool} 
beigetragen haben.
Ohne ihre besonderen F"ahigkeiten und ihr gro"ses Engagement w"are CrypTool
nicht, was es heute ist:
\begin{itemize}
   \item Hr. Henrik Koy
   \item Hr. J"org-Cornelius Schneider und
   \item Dr. Peer Wichmann.
\end{itemize}
Auch allen hier nicht namentlich Genannten ganz herzlichen Dank f"ur das 
(meist in der Freizeit) geleistete Engagement.
\\

Bernhard Esslinger


% Local Variables:
% TeX-master: "../script-de.tex"
% End:
