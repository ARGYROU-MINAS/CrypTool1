
% ++++++++++++++++++++++++++++++++++++++++++++++++++++++++++++++++++++++++++
\hypertarget{appendix-authors}{}
\subsection{Autoren des CrypTool-Skripts\index{CrypTool}}
\label{s:appendix-authors}

Dieser Anhang f"uhrt die Autoren\index{Autoren} dieses Dokuments auf.\\
Die Autoren sind namentlich am Anfang jedes Kapitels aufgef"uhrt,
zu dem sie beigetragen haben.

\begin{description}

\item[Bernhard Esslinger,] \mbox{}\\  % be_2005: Das \\ allein wird ignoriert (brauche beide)!  
Initiator des CrypTool-Projekts, Hauptautor dieses Skripts, Leiter IT-Security
in der Deutschen Bank und Dozent an der Universit"at Siegen. E-Mail:
besslinger@web.de.

\item[Matthias B"uger,] \mbox{}\\
Mitautor des Kapitels ``Elliptische Kurven'', Research
Analyst bei der Deutschen Bank.

\item[Bartol Filipovic,] \mbox{\\}
Urspr"unglicher Autor der Elliptische-Kurven-Implementierung
in CrypTool und des entsprechenden Kapitels in diesem Skript.

\item[Henrik Koy, ] \mbox{}\\
Hauptentwickler und Koordinator der CrypTool-Entwicklung
seit Version 1.3, Reviewer des Skripts und \TeX{}-Guru, 
Projektleiter IT und Kryptologe bei der Deutschen Bank.

\item[Roger Oyono, ] \mbox{}\\
Implementierer des Faktorisierungs-Dialogs in CrypTool und urspr"unglicher
Autor des Kapitels "`Die mathematischen Ideen hinter der modernen
Kryptographie"'.

\item[J"org Cornelius Schneider,] \mbox{}\\
Design und Support von CrypTool, Kryptographie-Enthusiast und 
Senior-Projektleiter IT bei der Deutschen Bank.

\item[Christine St"otzel,] \mbox{}\\
Diplom Wirtschaftsinformatikerin an der Universit"at Siegen.


\end{description}






% ++++++++++++++++++++++++++++++++++++++++++++++++++++++++++++++++++++++++++
\newpage
\hypertarget{appendix-movies}{}
\subsection{Filme und belletristische Literatur mit Bezug zur Kryptographie,\\
B"ucher mit Sammlungen einfacher Verfahren f"ur Kinder}
\label{s:appendix-movies}  
% {\bf Filme und Literatur mit Bezug zur Kryptographie} (siehe Anhang \ref{s:appendix-movies})
\index{Filme}
\index{Literatur}


Kryptographische Verfahren -- sowohl klassische wie moderne -- fanden auch
Eingang in die Literatur und in Filme. In manchen Medien werden diese nur
erw"ahnt und sind reine Beigabe, in anderen sind sie tragend und werden
genau erl"autert, und manchmal ist die Rahmenhandlung nur dazu da, dieses
Wissen motivierend zu transportieren.

Anbei der Beginn eines "Uberblicks:

%be_2005: Hatte zuerst \begin{thebibliography}{99999} und \bibitem[... ,
%         aber dann wurde immer der feste Titel "Literatur" bzw. "References"
%         geschrieben und wir fanden keine M�glichkeit, ihn weg zu bekommen.
%         L�sung: Stattdessen \begin{description} \item[...


\begin{description}

\item[\textrm{[Poe1843]}] \index{Poe 1843}
    Edgar Allan Poe\index{Poe, Edgar Allan}, \\
    {\em Der Goldk"afer}, 1843. \\
    Diese Kurzgeschichte erschien in Deutsch z.B. in der illustrierten und
    mit Kommentaren in den Marginalspalten versehenen Ausgabe "`Der Goldk"afer
    und andere Erz"ahlungen"', Gerstenbergs visuelle Weltliteratur,
    Gerstenberg Verlag, Hildesheim, 2002.\\
    In dieser Kurzgeschichte beschreibt Poe als Ich-Erz"ahler seine
    Bekanntschaft mit dem sonderbaren Legrand. Mit Hilfe eines an der
    K"uste Neuenglands gefundenen Goldk"afers, einem alten Pergament und
    den Dechiffrierk"unsten von Legrand finden Sie den sagenhaften Schatz
    von Kapit"an Kidd.\\
    Die Geheimschrift besteht aus 203 kryptischen Zeichen und erweist sich
    als allgemeine monoalphabetische Substitutions-Chiffre (vgl. 
    Kapitel~\ref{monoalphabeticSubstitutionCiphers}). Ihre schrittweise 
    Dechiffrierung durch semantische und syntaktische Analyse
    (H"aufigkeiten der einzelnen Buchstaben in englischen Texten)
    wird ausf"uhrlich erl"autert.\\
    Der Entschl"usseler Legrand sagt darin (S. 39) den ber"uhmten Satz:
    "`Und es ist wohl sehr zu bezweifeln, ob menschlicher Scharfsinn
    ein R"atsel ersinnen kann, das menschlicher Scharfsinn bei
    entsprechender Hingabe nicht wieder zu l"osen vermag."'\\
    % D: Poe: 1809-1849, "Vater des Kriminalromans", er schloss Wetten ab,
    % dass er alle verschl�sselten Botschaften, die ihm Freunde und Leser
    % vorlegten, im Handumdrehen entschl�sseln k�nne. Gutes Gesp�r durch
    % viel �bung.
    % E: Poe, 1809-1849 was named "Father of the crime novel". He claimed,
    % that he will be able to decrypt any cipher sent to him by friends or
    % readers.
 
    % Originalausgabe dieser illustrierten Ausgabe: "Le scarab�e d'or et 
    %                           autre nouvelles", Gallimard, Paris, 1998.
    % In "Der Goldk�fer" wird detailliert beschrieben, wie der verarmte
    % Hugenotte William Legrand in South Carolina die monoalphabetische
    % Geheimschrift des Piratenkapit�ns Kidd knackt, die zu einem
    % sagenhaften Schatz f�hrt.



\item[\textrm{[Verne1885]}] \index{Verne 1885}
    Jules Verne\index{Verne, Jules}, \\
    {\em Mathias Sandorf}, 1885. \\
    Dies ist einer der bekanntesten Romane des franz"osischen Schriftstellers
    Jules Verne (1828-1905), der auch als "`Vater der Science Fiction"'
    bezeichnet wurde.\\
    Erz"ahlt wird die spannende Geschichte des Freiheitsk"ampfers Graf
    Sandorf, der an die Polizei verraten wird, aber schlie"slich fliehen
    kann.\\
    M"oglich wurde der Verrat nur, weil seine Feinde eine Geheimbotschaft an
    ihn abfangen und entschl"usseln konnten. Dazu ben"otigten sie eine
    besondere Schablone, die sie ihm stahlen. Diese Schablone bestand aus
    einem quadratischen St"uck Karton mit 6x6 K"astchen, wovon 1/4, also neun,
    ausgeschnitten waren (vgl. die
    \hyperlink{turning-grille}{Flei"sner-Schablone}
    in Kapitel~\ref{introsamplesTranspositionCiphers}).\\


% Gefunden in: CRYPTO-GRAM, January 15, 2007, by Bruce Schneier
\item[\textrm{[Kipling1901]}] \index{Kipling 1901}
    Rudyard Kipling\index{Kipling, Rudyard}, \\
    {\em Kim}, 1901. \\
    Dieser Roman wird in der Besprechung von Rob Slade%
    \footnote{See
          \href{http://catless.ncl.ac.uk/Risks/24.49.html\#subj12} %% \ vor # n�tig !
       {\texttt{http://catless.ncl.ac.uk/Risks/24.49.html\#subj12}}.
    }
    folgenderma"sen beschrieben:
    "`Kipling packte viele Informationen und Konzepte in seine Geschichten.
    In "`Kim"' geht es um das gro"se "`Spiel"' Spionage und Bespitzelung.
    Schon auf den ersten 20 Seite finden sich Authentisierung "uber Besitz,
    Denial of Service, Sich-f"ur-jemand-anderen-Ausgeben (Impersonation),
    Heimlichkeit, Maskerade, Rollen-basierte Autorisierung (mit 
    Ad-hoc-Authentisierung durch Wissen), Abh"oren, und Vertrauen basierend
    auf Datenintegrit"at.
    Sp"ater kommen noch Contingency Planning gegen Diebstahl und
    Kryptographie mit Schl"usselwechsel hinzu."'\\
    Das Copyright des Buches ist abgelaufen%
    \footnote{Sie k"onnen es lesen unter:\\
          \href{http://whitewolf.newcastle.edu.au/words/authors/K/KiplingRudyard/prose/Kim/index.html}
       {\texttt{http://whitewolf.newcastle.edu.au/words/authors/K/KiplingRudyard/prose/Kim/index.html}},\\
          \href{http://kipling.thefreelibrary.com/Kim}
       {\texttt{http://kipling.thefreelibrary.com/Kim}} or\\
          \href{http://www.readprint.com/work-935/Rudyard-Kipling}
       {\texttt{http://www.readprint.com/work-935/Rudyard-Kipling}}.
    }.\\


\item[\textrm{[Doyle1905]}] \index{Doyle 1905}
    Arthur Conan Doyle\index{Doyle, Sir Arthur Conan}, \\
    {\em Die tanzenden M"annchen}, 1905. \\
    In der Sherlock-Holmes-Erz"ahlung {\em Die tanzenden M"annchen} 
    (erschienen erstmals 1903 im "`Strand Magazine"', und dann 1905 im 
    Sammelband "`Die R"uckkehr des Sherlock Holmes"' erstmals in Buchform)
    wird Sherlock Holmes mit einer Geheimschrift konfrontiert, die zun"achst
    wie eine harmlose Kinderzeichnung aussieht. \\
    Sie erweist sich als monoalphabetische Substitutions-Chiffre (vgl. 
    Kapitel~\ref{monoalphabeticSubstitutionCiphers}) des Verbrechers Abe
    Slaney.
    Sherlock Holmes knackt die Geheimschrift mittels H"aufigkeitsanalyse.\\


\item[\textrm{[Sayer1932]}] \index{Sayer 1932}
    Dorothy L. Sayer, \\
    {\em Zur fraglichen Stunde und Der Fund in den Teufelsklippen 
    (Orginaltitel: Have his carcase)}, Harper, 1932 \\
    (1. dt. "Ubersetzung {\em Mein Hobby: Mord} bei A. Scherz, 1964; \\
    dann {\em Der Fund in den Teufelsklippen} bei Rainer Wunderlich-Verlag, 
    1974;\\
    Neu"ubersetzung 1980 im Rowohlt-Verlag). \\
    In diesem Roman findet die Schriftstellerin Harriet Vane eine Leiche 
    am Strand und die Polizei h"alt den Tod f"ur einen Selbstmord. 
    Doch Harriet Vane und der elegante Amateurdetektiv Lord Peter Wimsey
    kl"aren in diesem zweiten von Sayers's ber"uhmten Harriet Vane's
    Geschichten den widerlichen Mord auf. \\
    Dazu ist ein Chiffrat zu l"osen. Erstaunlicherweise beschreibt der
    Roman nicht nur detailliert die Playfair-Chiffre, sondern auch deren
    Kryptoanalyse (vgl. \hyperlink{playfair}{Playfair}
    in Kapitel~\ref{polygraphicSubstitutionCiphers}).\\


\item[\textrm{[Arthur196x]}] \index{Arthur 196x}
    Robert Arthur, \\
    {\em Die 3 ???: Der geheime Schl"ussel nach Alfred Hitchcock (Band 119)},
    Kosmos-Verlag (ab 1960) \\
    Darin m"ussen die drei Detektive Justus, Peter und Bob verdeckte und
    verschl"usselte Botschaften entschl"usseln, um herauszufinden, was
    es mit dem Spielzeug der Firma Kopperschmidt auf sich hat.\\
    % Vgl.: http://de.wikipedia.org/wiki/Die_drei_%3F%3F%3F


\item[\textrm{[Seed1990]}] \index{Seed 1990}
    Regie Paul Seed (Paul Lessac), \\
    {\em Das Kartenhaus (Orginaltitel: House of Cards)}, 1990 (dt. 1992). \\
    In diesem Film versucht Ruth, hinter das Geheimnis zu kommen, das ihre
    Tochter verstummen lie"s. Hierin unterhalten sich Autisten mit Hilfe von
    5- und 6-stelligen Primzahlen (siehe 
    Kapitel~\ref{Label_Kapitel_2}).
    Nach "uber eine Stunde kommen im Film die folgenden beiden (nicht
    entschl"usselten) Primzahlfolgen vor:
%  \vskip -30pt  %be_2005 Bewirkt anscheinend nichts -- Abstand etwas zu gro�.
    \begin{center}
    $21.383, \;\;176.081, \;\;18.199, \;\;113.933, \;\;150.377, \;\;304.523, \;\;113.933$\\
    $193.877, \;\;737.683, \;\;117.881, \;\;193.877$
    \end{center}
    \vskip +20 pt   % da "\\" hier nicht geht!


\item[\textrm{[Robinson1992]}] \index{Robinson 1992}
    Regie Phil Alden Robinson, \\
    {\em Sneakers - Die Lautlosen (Orginaltitel: Sneakers)},
    Universal Pictures Film, 1992. \\
    In diesem Film versuchen die "`Sneakers"', Computerfreaks um ihren Boss
    Martin Bishop, den "`B"osen"' das Dechiffrierungsprogramm SETEC abzujagen.
    SETEC wurde von einem genialen Mathematiker vor seinem gewaltsamen Tod
    erfunden und kann alle Geheimcodes dieser Welt entschl"usseln.\\
    Das Verfahren wird nicht beschrieben.\\


\item[\textrm{[Becker1998]}] \index{Becker 1998}
    Regie Harold Becker, \\
    {\em Das Mercury Puzzle (Orginaltitel: Mercury Rising)}, 
    Universal Pictures Film, 1998. \\
    Die NSA hat einen neuen Code entwickelt, der angeblich weder von Menschen
    noch von Computern geknackt werden kann. Um die Zuverl"assigkeit zu testen,
    verstecken die Programmierer eine damit verschl"usselte Botschaft in
    einem R"atselheft.\\
    Simon, eine neunj"ahriger autistischer Junge, knackt den Code.
    Statt den Code zu fixen, schickt ihm ein Sicherheitsbeamter einen Killer.
    Der FBI-Agent Art Jeffries (Bruce Willis) besch"utzt den Jungen und
    stellt den Killern eine Falle.\\    
    Das Chiffrier-Verfahren wird nicht beschrieben.\\


\item[\textrm{[Brown1998]}] \index{Brown 1998}
    Dan Brown, \\
    {\em Diabolus (Orginaltitel: Digital Fortress)}, L"ubbe, 2005. \\
    Dan Browns erster Roman "`The Digital Fortress"' erschien 1998 als E-Book,
    blieb jedoch damals weitgehend erfolglos.\\
    Die National Security Agency (NSA) hat f"ur mehrere Milliarden US-Dollar
    einen gewaltigen Computer gebaut, mit dem sie in der Lage ist, auch nach
    modernsten Verfahren verschl"usselte Meldungen (nat"urlich nur die von
    Terroristen und Verbrechern) innerhalb weniger Minuten zu entziffern.\\
    Ein abtr"unniger Angestellter erfindet einen unbrechbaren Code und
    sein Computerprogramm Diabolus zwingt damit den Supercomputer zu
    selbstzerst"orerischen Rechenoperationen. Der Plot, in dem auch die
    sch"one Computerexpertin Susan Fletcher eine Rolle spielt, ist ziemlich
    vorhersehbar.\\
    Die Idee, dass die NSA oder andere Geheimdienste jeden Code knacken
    k"onnen, wurde schon von mehreren Autoren behandelt: Hier hat der
    Supercomputer 3 Millionen Prozessoren -- trotzdem ist es aus heutiger
    Sicht damit auch nicht ann"aherungsweise m"oglich, diese modernen Codes
    zu knacken.\\


\item[\textrm{[Elsner1999]}] \index{Elsner 1999}
    Dr.~C.~Elsner, \\
    {\em Der Dialog der Schwestern}, c't, Heise-Verlag, 1999. \\
    In dieser Geschichte, die dem CrypTool-Paket\index{CrypTool} als PDF-Datei
    beigelegt ist, unterhalten sich die Heldinnen vertraulich mit einer
    Variante des RSA-Verfahrens (vgl. Kapitel~\ref{rsabeweis} ff.).
    Sie befinden sich in einem Irrenhaus unter st"andiger Bewachung.\\
    

\item[\textrm{[Stephenson1999]}] \index{Stephenson 1999}
    Neal Stephenson, \\
    {\em Cryptonomicon}, Harper, 1999. \\
    Der sehr dicke Roman besch"aftigt sich mit Kryptographie sowohl im 
    zweiten Weltkrieg als auch in der Gegenwart.
    Die zwei Helden aus den 40er-Jahren sind der gl"anzende Mathematiker und
    Kryptoanalytiker Lawrence Waterhouse, und der "ubereifrige,
    morphiums"uchtige Bobby Shaftoe von den US-Marines. 
    Sie geh"oren zum Sonderkommando 2702, einer Alliiertengruppe, die
    versucht, die gegnerischen Kommunikationscodes zu knacken und dabei
    ihre eigene Existenz geheim zu halten. \\
    Diese Geheimnistuerei spiegelt sich in der Gegenwartshandlung wider,
    in der sich die Enkel der Weltkriegshelden -- der Programmierfreak 
    Randy Waterhouse und die sch"one Amy Shaftoe -- zusammentun. \\
    Cryptonomicon ist f"ur nicht-technische Leser teilweise schwierig zu
    lesen. Mehrere Seiten erkl"aren detailliert Konzepte der Kryptographie.
    Stephenson legt eine ausf"uhrliche Beschreibung der Solitaire-Chiffre
    (siehe Kapitel~\ref{Further-PaP-methods}) bei, ein 
    Papier- und Bleistiftverfahren\index{Papier- und Bleistiftverfahren},
    das von Bruce Schneier entwickelt wurde und im
    Roman "`Pontifex"' genannt wird. Ein anderer, moderner Algorithmus
    namens "`Arethusa"' wird dagegen nicht im Detail beschrieben.
    nicht offengelegt.\\


\item[\textrm{[Elsner2001]}] \index{Elsner 2001}
    Dr.~C.~Elsner, \\
    {\em Das Chinesische Labyrinth}, c't, Heise-Verlag, 2001. \\
    In dieser Geschichte, die dem CrypTool-Paket\index{CrypTool} als PDF-Datei
    beigelegt ist, muss Marco Polo in einem Wettbewerb Probleme aus der
    Zahlentheorie l"osen, um Berater des gro"sen Khan zu werden.\\


\item[\textrm{[Colfer2001]}] \index{Colfer 2001}
    Eoin Colfer, \\
    {\em Artemis Fowl}, List-Verlag, 2001. \\  
    In diesem Jugendbuch gelangt der junge Artemis, ein Genie und Meisterdieb,
    an eine Kopie des streng geheimen "`Buches der Elfen"'. Nachdem er es mit 
    Computerhilfe entschl"usselt hat, erf"ahrt er Dinge, die kein Mensch
    erfahren d"urfte. \\
    Der Code wird in dem Buch nicht genauer beschrieben.\\


\item[\textrm{[Howard2001]}] \index{Howard 2001}
    Regie Ron Howard, \\
    {\em A Beautiful Mind}, 2001. \\
    Verfilmung der von Sylvia Nasar verfassten Biographie des
    Spieltheoretikers John Nash. 
    Nachdem der brillante, aber unsoziale Mathematiker geheime kryptografische
    Arbeiten annimmt, verwandelt sich sein Leben in einen Alptraum. Sein 
    unwiderstehlicher Drang, Probleme zu l"osen, gef"ahrden ihn und sein 
    Privatleben.
    Nash ist in seiner Vorstellungswelt ein staatstragender Codeknacker. \\
    Konkrete Angaben zur seinen Analyseverfahren werden nicht beschrieben.\\


\item[\textrm{[Apted2001]}] \index{Apted 2001}
    Regie Michael Apted, \\
    {\em Enigma -- Das Geheimnis}, 2001. \\
    Verfilmung des von Robert Harris verfassten "`historischen Romans"' 
    {\em Enigma} (Hutchinson, London, 1995) "uber die ber"uhmteste 
    Verschl"usselungsmaschine in der Geschichte, die in
    Bletchley Park nach polnischen Vorarbeiten gebrochen wurde. 
    Die Geschichte spielt 1943, als der eigentliche Erfinder Alan Turing
    schon in Amerika war. So kann der Mathematiker Tom Jericho als Hauptperson
    in einem spannenden Spionagethriller brillieren.\\
    Konkrete Angaben zu dem Analyseverfahren werden nicht gemacht.\\


\item[\textrm{[Isau2003]}] \index{Isau 2003}
    Ralf Isau, \\
    {\em Das Museum der gestohlenen Erinnerungen}, Thienemann-Verlag, 2003. \\
    In diesem spannenden Roman kann der letzte Teil des Spruches nur durch
    die vereinten Kr"afte der Computergemeinschaft gel"ost werden.\\


\item[\textrm{[Brown2003]}] \index{Brown 2003}
    Dan Brown, \\
    {\em Sakrileg (Orginaltitel: The Da Vinci Code)}, L"ubbe, 2004. \\
    Der Direktor des Louvre wird in seinem Museum vor einem Gem"alde Leonardos
    ermordet aufgefunden, und der Symbolforscher Robert Langdon ger"at in eine
    Verschw"orung.\\
    Innerhalb der Handlung werden verschiedene klassische Codes (Substitution
    wie z.B. Caesar oder Vigenere, sowie Transposition und Zahlencodes) 
    angesprochen. Au"serdem klingen interessante Nebenbemerkungen "uber 
    Schneier oder die Sonnenblume an.
    Der zweite Teil des Buches ist sehr von theologischen Betrachtungen 
    gepr"agt. \\
    Das Buch ist einer der erfolgreichsten Romane der Welt.\\

% Rezensionen aus der Amazon.de-Redaktion:
% Bestsellerautor Dan Brown bietet mit Sakrileg erneut spannende und intelligente Unterhaltung vom Feinsten. Der Direktor des Louvre wird in seinem Museum vor einem Gem�lde Leonardos ermordet aufgefunden, und der Symbolforscher Robert Langdon ger�t ins Fadenkreuz der Polizei, war er doch mit dem Opfer just zur Tatzeit verabredet. Eine Verschw�rung ist immer noch das Sch�nste. Stimmt, wenn sie schriftstellerisch so �berzeugend und raffiniert inszeniert ist, wie es dem Amerikaner Dan Brown in diesem Thriller gelingt. Genaue Recherchen an den Schaupl�tzen und penible historische Studien in Zusammenarbeit mit seiner Frau Blythe, einer Kunsthistorikerin, machen das umfangreiche Werk nicht nur f�r Historiker und Religionswissenschaftler, sondern gerade auch f�r ein gro�es Publikum zu einem echten Vergn�gen. Der Symbolologe Robert Langdon sitzt in der Klemme. Er gilt als Hauptverd�chtiger im Fall Jacques Sauni�re, des ermordeten Direktors des Louvre, und ger�t als solcher in die F�nge von Capitaine Bezu Fache, der als �beraus gerissener Ermittler gilt. Sauni�re hatte im Todeskampf einen Hinweis auf Langdon gegeben. Mithilfe von Sophie Neveu, der Enkelin des Ermordeten, gelingt Langdon die Flucht. Beide sind der �berzeugung, dass Sauni�re vielmehr Informationen �ber eine Verschw�rung des Opus Dei und der katholischen Kirche liefern wollte. Im Verlauf einer atemlosen Flucht von Frankreich nach England haben Langdon und Neveu knifflige Codes zu knacken, um Sauni�res Geheimnis zu l�ften, der sich als Gro�meister der Geheimorganisation Prieur� de Sion entpuppt. Auf ihren Fersen befindet sich nicht nur die Polizei. Die Handlung einer Nacht und eines Tages auf 600 fesselnden Seiten, die �berdies Lust machen auf mehr Informationen zu Templern, Prieur� de Sion, Opus Dei sowie auf mehr historische Fakten -- was will man mehr. Und wer das Ganze nicht allzu ernst nimmt, wird die Lekt�re sehr genie�en -- am besten innerhalb einer Nacht und eines Tages.
% --Ulrich Deurer




\item[\textrm{[McBain2004]}] \index{McBain 2004}
    Scott McBain, \\
    {\em Der Mastercode (Orginaltitel: Final Solution)}, Knaur, 2005. \\
    In einer nahen Zukunft haben Politiker, Milit"ars und Geheimdienstchefs
    aus allen Staaten in korrupter Weise die Macht "ubernommen. Mit einem
    gigantischen Computernetzwerk names "`Mother"' und vollst"andiger
    "Uberwachung wollen sie die Machtverteilung und Kommerzialisierung f"ur
    immer festschreiben.
    Menschen werden ausschlie"slich nach ihrem Kreditrating bewertet und
    global agierende Unternehmen entziehen sich jeder demokratischen
    Kontrolle.
    Innerhalb des Thrillers wird die offensichtliche Ungerechtigkeit,
    aber auch die realistische M"oglichkeit dieser Entwicklung immer wieder
    neu betont.\\
    In den Supercomputer "`Mother"' wurde m.H. eines Kryptologen ein Code zur
    Deaktivierung eingebaut: In einem Wettrennen mit der Zeit versuchen
    Lars Pedersen, Oswald Plevy, die amerikanische Pr"asidentin, der britische
    Regierungschef und eine unbekannte Finnin namens Pia, die den Tod ihres
    Bruders r"achen will, den Code zur Deaktivierung zu starten. Auf der
    Gegenseite agiert eine Gruppe m"orderischer Verschw"orer unter F"uhrung
    des britischen Au"senministers und des CIA-Chefs.\\
    Die englische Originalfassung "`The Final Solution"' wurde als Manuskript
    an Harper Collins, London verkauft, ist dort aber nicht erschienen.\\



\item[\textrm{[Burger2006]}] \index{Burger 2006}
    Wolfgang Burger, \\
    {\em Heidelberger L"ugen}, Piper, 2006. \\
    In diesem Kriminalroman mit vielen oft
    unabh"angigen Handlungsstr"angen und lokalen Geschichten geht es vor
    allem um den Kriminalrat Gerlach aus Heidelberg. Auf S. 207 f. wird aber
    auch der kryptologische Bezug von einem der Handlungsstr"ange kurz
    erl"autert: der Soldat H"orrle hatte Schaltpl"ane eines neuen digitalen 
    NATO-Entschl"usselungsger"ates kopiert und der Ermordete hatte versucht,
    seine Erkenntnisse an die Chinesen zu verkaufen.\\
    % siehe: www.wolfgang-burger.com



\item[\textrm{[Vidal2006]}] \index{Vidal 2006}
    Agustin Sanchez Vidal, \\
    {\em Kryptum}, Dtv, 2006. \\
    Der erste Roman des spanischen Professors der Kunstgeschichte "ahnelt
    Dan Browns "`Sakrileg"' aus dem Jahre 2003, aber angeblich hat Vidal schon
    1996 begonnen, daran zu schreiben. Vidals Roman ist zwischen historischem
    Abenteuerroman und Mystery-Thriller angesiedelt und war in Spanien ein
    Riesenerfolg.\\
    Im Jahre 1582 wartet Raimundo Randa, der sein Leben lang einem Geheimnis
    auf der Spur war, im Alkazar auf seinen Inquisitionsprozess.
    Dieses Geheimnis rankt sich um ein mit kryptischen Zeichen beschriftetes
    Pergament, von dem eine mysteri"ose Macht ausgeht.
    Rund 400 Jahre sp"ater kann sich die amerikanische Wissenschaftlerin Sara
    Toledano dieser Macht nicht entziehen, bis sie in Antigua verschwindet.
    Ihr Kollege, der Kryptologe David Calderon, und ihre Tochter Rachel machen
    sich auf die Suche nach ihr und versuchen gleichzeitig, den Code zu knacken.
    Aber auch Geheimorganisationen wie die NSA sind hinter dem
    Geheimnis des "`letzten Schl"ussels"' her. Sie sind bereit, daf"ur 
    "uber Leichen zu gehen.\\
    % Korrekte Schreibweise ? :  August�n S�nchez Vidal, David Calder�n



\end{description}

% Filme: Matrix, Tron, ...
% B�cher: xxx, ...


\vskip +20 pt
%be_2006: Hier ist ein Link zu lange.
%   \href{http://www.staff.uni-mainz.de/pommeren/Kryptologie99/Klassisch/1\_Monoalph/Literat.html}
%   {\texttt{http://www.staff.uni-mainz.de/pommeren/Kryptologie99/Klassisch/1\_Monoalph/Literat.html}}
% Damit er nicht �ber den Seitenrand hinausragt, wurden 2 \href mit dem
% gleichen Link, aber verschiedenem Text direkt aneinandergef�gt --
% mit "\\" dazwischen.
% Eigentlich wollten wir \discretionary{}{}{} dazwischen einf�gen, aber
% das bewirkte nichts.
Weitere Beispiele f"ur Kryptologie in der belletristischen Literatur finden
sich z.B. auf der folgenden Webseite: 
\begin{center}
    \href{http://www.staff.uni-mainz.de/pommeren/Kryptologie99/Klassisch/1\_Monoalph/Literat.html}
   {\texttt{http://www.staff.uni-mainz.de/pommeren/Kryptologie99/Klassisch/1\_Monoalph/}}\\
    \href{http://www.staff.uni-mainz.de/pommeren/Kryptologie99/Klassisch/1\_Monoalph/Literat.html}{\texttt{Literat.html}}
\end{center}
F"ur die "altere Literatur (z.B. von Jules Verne, Karl May, Arthur Conan
Doyle, Edgar Allen Poe) sind darauf sogar die Links zu den eigentlichen
Textstellen enthalten.




\vskip +60 pt
Die folgende Auflistung enth"alt Kinderb"ucher mit Sammlungen von einfachen
kryptografischen Verfahren, didaktisch und spannend aufbereitet:

\begin{description}

\item[\textrm{[Mosesxxxx]}] \index{Moses xxxx}
    [Ohne Autorenangabe], \\
    {\em Streng geheim -- Das Buch f"ur Detektive und Agenten},
    Edition moses, [ohne Angabe der Jahreszahl]. \\
    Ein d"unnes Buch f"ur kleinere Kinder mit Inspektor Fox und Dr. Chicken.\\


\item[\textrm{[Para1988]}] \index{Para 1988}
    Para, \\
    {\em Geheimschriften},
    Ravensburger Taschenbuch Verlag, 1988 (erste Auflage 1977). \\
    Auf 125 eng beschriebenen Seiten werden in dem klein-formatigen
    Buch viele verschiedene Verfahren erl"autert, die Kinder selbst 
    anwenden k"onnen, um ihre Nachrichten zu verschl"usseln oder unlesbar
    zu machen. Ein kleines Fachw"orter-Verzeichnis und eine kurze
    Geschichte der Geheimschriften runden das B"uchlein ab.

    Gleich auf S. 6 steht in einem old-fashioned Style f"ur den Anf"anger
    "`Das Wichtigste zuerst"' "uber Papier- und Bleistiftverfahren
    (vergleiche Kapitel~\ref{Kapitel_PaperandPencil}):
    "`Wollte man Goldene Regeln f"ur Geheimschriften aufstellen, so
    k"onnten sie lauten:
    \begin{itemize}
      \item[-] Deine geheimen Botschaften m"ussen sich an jedem Ort zu
               jeder Zeit mit einfachsten Mitteln bei geringstem Aufwand
               sofort anfertigen lassen.
      \item[-] Deine Geheimschrift muss f"ur Deine Partner leicht zu merken
               und zu lesen sein. Fremde hingegen sollen sie nicht entziffern
               k"onnen.\\
               Merke: Schnelligkeit geht vor Raffinesse, Sicherheit vor
               Sorglosigkeit.
      \item[-] Deine Botschaft muss immer knapp und pr"azise sein wie
               ein Telegramm.
               K"urze geht vor Grammatik und Rechtschreibung. Alles
               "Uberfl"ussige weglassen (Anrede, Satzzeichen).
               M"oglichst immer nur Gro"s- oder immer nur
               Kleinbuchstaben verwenden."'
    \end{itemize}
    \vskip +20 pt   % da "\\" hier nicht geht!


\item[\textrm{[M"uller-Michaelis2000]}] \index{M"uller-Michaelis 2000}
    Matthias M"uller-Michaelis, \\
    {\em Streng geheim. Das gro"se Buch der Detektive}, Moses, 2003. \\


\item[\textrm{[Kippenhahn2002]}] \index{Kippenhahn 2002}
    Rudolf Kippenhahn, \\
    {\em Streng geheim! -- Wie man Botschaften verschl"usselt und 
    Zahlencodes knackt}, rororo, 2002. \\
    In dieser Geschichte bringt ein Gro"svater, ein Geheimschriftexperte,
    seinen 4 Enkeln und deren Freunden bei, wie man Botschaften
    verschl"usselt, die niemand lesen soll. Da es jemand gibt, der die
    Geheimnisse knackt, muss der Gro"svater den Kindern immer komplizierte
    Verfahren beibringen. \\
    In dieser puren Rahmenhandlung werden die wichtigsten klassischen
    Verfahren und ihre Analyse kindgerecht und spannend erl"autert.\\


\item[\textrm{[Harder2003]}] \index{Harder 2003}
    Corinna Harder und Jens Schumacher, \\
    {\em Das Handbuch f"ur Detektive. Alles "uber Geheimsprachen, Codes,
     Spurenlesen und die gro"sen Detektive dieser Welt}, S"udwest, 2000. \\


\item[\textrm{[Flessner2004]}] \index{Flessner 2004}
    Bernd Flessner, \\
    {\em Die 3 ???: Handbuch Geheimbotschaften},
    Kosmos, 2004. \\
    Auf 127 Seiten wird sehr verst"andlich, spannend, und strukturiert nach
    Verfahrenstyp geschildert, welche Geheimsprachen (Navajo-Indianer oder
    Dialekte) und welche Geheimschriften (echte Verschl"usselung, aber auch
    technische und linguistische Steganographie) es gab und wie man
    einfache Verfahren entschl"usseln kann.\\
    Bei jedem Verfahren steht, wenn es in der Geschichte oder in Geschichten
    Verwendung fand [wie in Edgar Allan Poe's "`Der Goldk"afer"', wie von
    Jules Verne's Held Mathias Sandorf oder von Astrid Lindgren's
    Meisterdetektiv Kalle Blomquist, der die ROR-Sprache verwendet
    ("ahnliche Einf"uge-Chiffren sind auch die L"offel- oder die B-Sprache)].
    \\
    Dieses Buch ist ein didaktisch hervorragender Einstieg f"ur j"ungere
    Jugendliche.\\

%   www.detektiv-club.de
%   Singh's Codebook in der Kinder-Fassung

\end{description}%\\ Hier dahinter kann man kein Newline haben
         %Macht man eines, kommt: "! LaTeX Error: There's no line here to end."

\mbox{}\\

Wenn Sie weitere Literatur und Filme wissen, wo Kryptographie eine
wesentliche Rolle spielt oder wenn Sie weitere B"ucher kennen, die
Kryptographie didaktisch kindergerecht aufbereiten, dann w"urden wir
uns sehr freuen, wenn Sie uns den genauen Buchtitel und eine kurze
Erl"auterung zum Buchinhalt zusenden w"urden. Herzlichen Dank.


\clearpage
\listoffigures
\addcontentsline{toc}{section}{\listfigurename}


\clearpage

\listoftables
\addcontentsline{toc}{section}{\listtablename}




%    \href{http://www....}
%    {\texttt{http://www....}}


