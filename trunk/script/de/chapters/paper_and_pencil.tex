% $Id$
% ............................................................................
%             V E R S C H L U E S S E L U N G S V E R F A H R E N
% ~~~~~~~~~~~~~~~~~~~~~~~~~~~~~~~~~~~~~~~~~~~~~~~~~~~~~~~~~~~~~~~~~~~~~~~~~~~~
% "`xx"'
\newpage
%------------------------------------------------------------------------------
% First Editor: Christine St�tzel, April 2004
% Update and corrections: B. Esslinger, June 2005
% corrected by C. Esslinger, June 2005 
%------------------------------------------------------------------------------
\hypertarget{Kapitel_PaperandPencil}{}

\begin{center}
\fbox{\parbox{15cm}{%
{\em Edgar Allan Poe\index{Poe, Edgar Allan}: 
A Few Words on Secret Writing, 1841}\\
Nur Wenige kann man glauben machen, dass es keine leichte Sache ist,
eine Geheimschrift zu erdenken, die sich der Untersuchung widersetzt.
Dennoch kann man rundheraus annehmen, dass menschlicher Scharfsinn
keine Chiffre erdenken kann, die menschlicher Scharfsinn nicht l"osen kann.}}
\end{center}

\section{Papier- und Bleistift-Verschl"usselungsverfahren}
\label{Kapitel_PaperandPencil}
(Christine St"otzel, April 2004; Updates B.+C. Esslinger, Juni 2005)
\index{Papier- und Bleistiftverfahren}

Das folgende Kapitel bietet einen recht vollst"andigen "Uberblick "uber Papier- 
und Bleistiftverfahren\footnote{Jeweils mit Verweisen zu vertiefenden Informationen.}.
Unter diesem Begriff lassen sich alle Verfahren zusammenfassen, die Menschen von
Hand anwenden k"onnen, um Nachrichten zu ver- und entschl"usseln.
Besonders popul"ar waren diese Verfahren f"ur Geheimdienste (und 
sind es immer noch), da ein Schreibblock und ein Stift -- im 
Gegensatz zu elektronischen Hilfsmitteln -- vollkommen unverd"achtig sind.

Die ersten Papier- und Bleistiftverfahren entstanden bereits vor rund 
3000 Jahren, aber auch w"ahrend des vergangenen Jahrhunderts kamen 
noch zahlreiche neue Methoden hinzu. Bei allen Papier- und 
Bleistiftverfahren handelt es sich um symmetrische
Verfahren\index{Verschl""usselung!symmetrisch}. Selbst in den 
"altesten Verschl"usselungsmethoden steckten schon die
grunds"atzlichen Konstruktionsprinzipien wie Transposition, 
Substitution, Blockbildung und deren Kombination. Daher
lohnt es sich vor allem aus didaktischen Gesichtspunkten, 
diese "`alten"' Verfahren genauer zu betrachten.

Erfolgreiche bzw. verbreiteter eingesetzte Verfahren mussten die gleichen 
Merkmale erf"ullen wie moderne Verfahren:
\begin{itemize}
\item Vollst"andige Beschreibung, klare Regeln, ja fast Standardisierung 
      (inkl. der Sonderf"alle, dem Padding, etc.).
\item Gute Balance zwischen Sicherheit und Benutzbarkeit 
      (denn zu kompliziert zu bedienende Verfahren waren fehlertr"achtig
      oder unangemessen langsam).
\end{itemize}


\vskip +30 pt
%------------------------------------------------------------------------------
\subsection{Transpositionsverfahren}

Bei der Verschl"usselung durch Transposition\index{Transposition} 
bleiben die urspr"unglichen Zeichen der Nachricht erhalten,
nur ihre Anordnung wird ge"andert (Transposition = 
Vertauschung)\footnote{Ein anderer Begriff f"ur eine Transposition
ist Permutation\index{Permutation}.}.

%------------------------------------------------------------------------------
\subsubsection{Einf"uhrungs-Beispiele unterschiedlicher
               Transpositionsverfahren}
\label{introsamplesTranspositionCiphers}  %be_2006 so ist es gut.

\begin{itemize}

\item {\bf Gartenzaun}\footnote{In CrypTool
   kann man dieses Verfahren "uber den Men"ueintrag {\bf Ver-/Entschl"usseln
   \textbackslash{} Symmetrisch (klassisch) \textbackslash{} Permutation} 
   abbilden: f"ur einen Gartenzaun mit 2 Zeilen gibt man als Schl"ussel
   "`B,A"' ein und l"asst ansonsten die Standardeinstellungen (nur 1
   Permutation, in der man zeilenweise ein- und spaltenweise ausliest). 
   Mit dem Schl"ussel "`A,B"' w"urde man man das Zick-zack-Muster unten so 
   beginnen, dass der 1. Buchstabe in der ersten statt in der 2. Zeile steht.}
   \cite{Singh2001}\index{Gartenzaun-Verschl""usselung}:
                        % Kein Blank VOR "\index", sonst ist der
                        % Doppelpunkt nicht direkt nach dem Fussnotenindex.
   Die Buchstaben des Klartextes werden abwechselnd in zwei (oder mehr) 
   Zeilen geschrieben, so dass ein Zickzack-Muster entsteht. 
   Dann werden die Zeichen zeilenweise nacheinander gelesen.\\
   Dieses Verfahren ist eher eine Kinderverschl"usselung.
	
   Klartext\footnote{Wenn das Alphabet nur die 26 Buchstaben verwendet, 
   schreiben wir den Klartext in Kleinbuchstaben und den Geheimtext in 
   Gro"sbuchstaben.}: ein beispiel zur transposition

   \begin{table}[h]
   \begin{center}
   \begin{tabular}{r@{\:}r@{\:}r@{\:}r@{\:}r@{\:}r@{\:}r@{\:}r@{\:}r@{\:}
                r@{\:}r@{\:}r@{\:}r@{\:}r@{\:}r@{\:}r@{\:}r@{\:}r@{\:}
                r@{\:}r@{\:}r@{\:}r@{\:}r@{\:}r@{\:}r@{\:}r@{\:}r@{\:}}
	  & i &   & b &   & i &   & p &   & e &   & z &   & r &   & 
            r &   & n &   & p &   & s &   & t &   & o &   \\
	e &   & n &   & e &   & s &   & i &   & l &   & u &   & t &
          & a &   & s &   & o &   & i &   & i &   & n 
   \end{tabular}
   \caption{Gartenzaun-Verschl"usselung}
   \end{center}
   \end{table}

   Geheimtext\footnote{Die Buchstaben des Klartextes sind hier -- wie 
   historisch "ublich -- in 5-er Bl"ocken gruppiert. Man k"onnte o.E.d.A.
   auch eine andere (konstante) Blockl"ange oder gar keine Trennung durch 
   Leerzeichen w"ahlen.}%
   : IBIPE ZRRNP STOEN ESILU TASOI IN\\


\item {\bf Skytale von Sparta}\footnote{%
   Das Ergebnis dieses Verschl"usselungsverfahrens entspricht dem einer 
   einfachen Spaltentransposition.  
   In CrypTool kann man dieses Verfahren "uber den Men"ueintrag {\bf 
   Ver-/Entschl"usseln \textbackslash{} Symmetrisch (klassisch) 
   \textbackslash{} Permutation} abbilden: F"ur die Skytale braucht man in
   der Dialogbox nur die erste Permutation. Darin gibt man bei z.B. 4 
   Kanten als Schl"ussel  "`1,2,3,4"'  ein. Dies w"are so, als w"urde man
   den Text in 4-er Bl"ocken waagrecht in eine Tabelle schreiben und 
   senkrecht auslesen. 
   Weil der Schl"ussel aufsteigend geordnet ist, bezeichnet man die Skytale
   auch als identische Permutation. Weil das Schreiben und Auslesen nur 
   einmal durchgef"uhrt wird, als einfache (und nicht als doppelte)
   Permutation.}
   \cite{Singh2001}\index{Skytale}%
   : 
   Dieses Verfahren wurde wahrscheinlich das erste Mal um 600 v.Chr.
   benutzt und es wurde von dem griechischen Schriftsteller und Philosophen
   Plutarch (50-120 v.Chr.) zuerst beschrieben.\\
   Um einen Holzstab wird ein Streifen Papier o."a. gewickelt. Dann wird
   darauf zeilenweise der Klartext geschrieben. 
   Wickelt man den Streifen ab, h"alt man den Geheimtext in den H"anden.

\item {\bf Schablone} \cite{Goebel2003}: Sender und Empf"anger benutzen
   die gleiche Schablone. In deren L"ocher werden zeilenweise die
   Klartextzeichen geschrieben, die dann spaltenweise ausgelesen werden. 
   Bleibt Klartext "ubrig, wird der Vorgang wiederholt, unter Umst"anden 
   mit einer anderen Ausrichtung der Schablone\footnote{%
   Dieses Verfahren kann man nicht durch eine einfache Spaltentransposition
   darstellen.}.

   \hypertarget{turning-grille}{}  %be_2006 so ist es gut.
\item {\bf Flei"sner-Schablone} \cite{Savard1999}% verhindert, dass Blank vor ":"
   \index{Flei""sner-Schablone}: 
   Die Flei"sner-Schablone
   wurde im Ersten Weltkrieg von deutschen Soldaten benutzt\footnote{%
   Erfunden wurde die Flei"sner-Schablone bereits 1881 von Eduard Flei"sner
   von Wostrowitz.\\
   Eine gute Visualisierung findet sich unter www.turning-grille.com.}%
   .
   Ein quadratisches Gitter dient als Schablone, wobei ein Viertel der 
   Felder L"ocher hat. Der erste Teil des Klartextes wird zeichenweise 
   durch die L"ocher auf ein Blatt Papier geschrieben, dann wird die
   Schablone um 90 Grad gedreht, und der zweite Teil des Textes wird
   auf das Papier geschrieben, usw. Die Kunst besteht in der richtigen 
   Wahl der L"ocher: Kein Feld auf dem Papier darf frei bleiben, es 
   darf aber auch keines doppelt beschriftet werden. Der Geheimtext 
   wird zeilenweise ausgelesen.

   In die Beispiel-Flei"sner-Schablone in der folgenden Tabelle k"onnen
   4 Mal je 16 Zeichen des Klartextes auf ein Blatt geschrieben werden:
   \begin{table}[h]
   \begin{center}
   \begin{tabular}{|cccc|cccc|}
   \hline 	
	O & - & - & - & - & O & - & - \\
	- & - & - & O & O & - & - & O \\
	- & - & - & O & - & - & O & - \\
	- & - & O & - & - & - & - & - \\
   \hline 	
	- & - & - & - & O & - & - & - \\
	O & - & O & - & - & - & O & - \\
	- & O & - & - & - & - & - & O \\
	- & - & - & O & O & - & - & - \\
   \hline
   \end{tabular}  
   \caption{8x8-Flei"sner-Schablone}
   \end{center} 
   \end{table}

\end{itemize}


%------------------------------------------------------------------------------
\subsubsection[Spalten- und Zeilentranspositionsverfahren]
    {Spalten- und Zeilentranspositionsverfahren\footnotemark}
    \footnotetext{%
Die meisten der folgenden Verfahren k"onnen in CrypTool\index{CrypTool}
mit dem Men"upunkt {\bf Ver-/Entschl"usseln \textbackslash{} 
Symmetrisch (klassisch) \textbackslash{} Permutation} abgebildet werden.}

\begin{itemize}

\item {\bf Einfache Spaltentransposition} \cite{Savard1999}: Zun"achst wird
   ein Schl"usselwort bestimmt, das "uber die Spalten eines Gitters geschrieben
   wird. Dann schreibt man den zu verschl"usselnden Text zeilenweise in dieses
   Gitter.
   Die Spalten werden entsprechend des Auftretens der Buchstaben des 
   Schl"usselwortes im Alphabet durchnummeriert. In dieser Reihenfolge werden 
   nun auch die Spalten ausgelesen und so der Geheimtext gebildet\footnote{%
   Darstellung mit CrypTool: Eingabe eines Schl"ussels f"ur die 1. Permutation,
   zeilenweise einlesen, spaltenweise permutieren und auslesen.}.
	
   Klartext: ein beispiel zur transposition

   \begin{table}[h]
   \begin{center}
   \begin{tabular}{|c|c|c|}
   \hline
	K & E & Y \\
   \hline
	e & i & n \\
	b & e & i \\
	s & p & i \\
	e & l & z \\
	u & r & t \\
	r & a & n \\
	s & p & o \\
	s & i & t \\
	i & o & n \\
   \hline
   \end{tabular}
   \caption{Einfache Spaltentransposition}
   \end{center}
   \end{table}

   Transpositionsschl"ussel: K=2; E=1; Y=3. \\
   Geheimtext: IEPLR APIOE BSEUR SSINI IZTNO TN \\

\item {\bf AMSCO} \cite{ACA2002}\index{AMSCO}: Die Klartextzeichen werden 
   abwechselnd in Einer- und Zweiergruppen in ein Gitter geschrieben.
   Dann erfolgt eine Vertauschung der Spalten, anschlie"send das Auslesen.

\item {\bf Doppelte Spaltentransposition / "`Doppelw"urfel"'} 
   \cite{Savard1999}\index{Doppelw""urfel}: 
   Die doppelte 
   Spaltentransposition wurde h"aufig im Zweiten Weltkrieg und zu Zeiten des 
   Kalten Krieges angewendet. Dabei werden zwei Spaltentranspositionen 
   nacheinander durchgef"uhrt, f"ur die zweite Transposition wird ein neuer
   Schl"ussel benutzt\footnote{%
   Darstellung mit CrypTool: Eingabe eines Schl"ussels f"ur die 1. Permutation,
   zeilenweise einlesen, spaltenweise permutieren und auslesen.
   Eingabe eines (neuen) Schl"ussels f"ur die 2. Permutation, Ergebnis der
   1. Permutation zeilenweise einlesen, spaltenweise permutieren und
   auslesen.}.

\item {\bf Spaltentransposition, General Luigi Sacco} \cite{Savard1999}: Die 
   Spalten eines Gitters werden den Buchstaben des Schl"usselwortes
   entsprechend nummeriert. Der Klartext wird dann zeilenweise eingetragen,
   in der ersten Zeile bis zur Spalte mit der Nummer 1, in der zweiten Zeile
   bis zur Spalte mit der Nummer 2 usw. Das Auslesen erfolgt wiederum
   spaltenweise.

   Klartext: ein beispiel zur transposition

   \begin{table}[h]
   \begin{center}
   \begin{tabular}{|c|c|c|c|c|c|c|}
   \hline
	G & E & N & E & R & A & L\\
	4 & 2 & 6 & 3 & 7 & 1 & 5\\
   \hline
	e & i & n & b & e & i &  \\
	s & p &   &   &   &   &  \\
	i & e & l & z &   &   &  \\
	u &   &   &   &   &   &  \\
	r & t & r & a & n & s & p\\
	o & s & i &   &   &   &  \\
	t & i & o & n &   &   &  \\
   \hline
   \end{tabular}
   \caption{Spaltentransposition nach General Luigi Sacco}
   \end{center}
   \end{table}

   Geheimtext: ESIUR OTIPE TSINL RIOBZ ANENI SP\\


\item {\bf Spaltentransposition, Franz"osische Armee im Ersten Weltkrieg}
   \cite{Savard1999}: 
   Nach der Durch\-f"uhrung einer Spaltentransposition werden diagonale Reihen
   ausgelesen.


\item {\bf Zeilentransposition} \cite{Savard1999}: Der Klartext wird in gleich 
   lange Bl"ocke zerlegt, was auch mit Hilfe eines Gitters erfolgen kann. Dann
   wird die Reihenfolge der Buchstaben bzw. der Spalten vertauscht. Da das 
   Auslesen zeilenweise erfolgt, wird nur jeweils innerhalb der Bl"ocke 
   permutiert\footnote{%
   Darstellung mit CrypTool: Eingabe eines Schl"ussels f"ur die 1. Permutation, 
   zeilenweise einlesen, spaltenweise permutieren und zeilenweise auslesen.}.

\end{itemize}
% Ende \subsubsection{Spalten- und Zeilentranspositionsverfahren}



%------------------------------------------------------------------------------
\subsubsection{Weitere Transpositionsverfahren}

\begin{itemize}

\item {\bf Geometrische Figuren} \cite{Goebel2003}: Einem bestimmten Muster
   folgend wird der Klartext in ein Gitter geschrieben (Schnecke, R"osselsprung
   o."a.), einem zweiten Muster folgend wird der Geheimtext ausgelesen.

\item {\bf Union Route Cipher} \cite{Goebel2003}: Die Union Route Cipher hat 
   ihren Ursprung im Amerikanischen B"urgerkrieg. Nicht die Buchstaben einer
   Nachricht werden umsortiert, sondern die W"orter. F"ur besonders pr"agnante 
   Namen und Bezeichnungen gibt es Codew"orter, die zusammen mit den Routen
   in einem Codebuch festgehalten werden. Eine Route bestimmt die Gr"o"se des
   Gitters, in das der Klartext eingetragen wird und das Muster, nach dem der
   Geheimtext ausgelesen wird. Zus"atzlich gibt es eine Anzahl von
   F"ullw"ortern.

\item {\bf Nihilist-Transposition} \cite{ACA2002}%
   \index{Nihilist-Transposition}: 
   Der Klartext wird in eine
   quadratische Matrix eingetragen, an deren Seiten jeweils der gleiche 
   Schl"ussel steht. Anhand dieses Schl"ussels werden sowohl die Zeilen als 
   auch die Spalten alphabetisch geordnet und der Inhalt der Matrix
   dementsprechend umsortiert. Der Geheimtext wird zeilenweise ausgelesen.

   Klartext: ein beispiel zur transposition

   \begin{table}[h]
   \begin{center}
   \begin{tabular}{|c|ccccc||cc|ccccc|}
   \hline
	  & K & A & T & Z & E &   &   & A & E & K & T & Z\\
   \hline
	K & e & i & n & b & e &   & A & s & e & i & p & i\\
	A & i & s & p & i & e &   & E & s & i & o & i & t\\
	T & l & z & u & r & t &   & K & i & e & e & n & b\\
	Z & r & a & n & s & p &   & T & z & t & l & u & r\\
	E & o & s & i & t & i &   & Z & a & p & r & n & s\\
   \hline
   \end{tabular}
   \caption[Nihilist-Transposition]{Nihilist-Transposition\footnotemark}
   \end{center}
   \end{table}

   Geheimtext: SEIPI SIOIT IEENB ZTLUR APRNS\\
   \footnotetext{%
   Der linke Block ist das Resultat nach dem Einlesen. Der rechte Block ist 
   das Resultat nach dem Vertauschen von Zeilen und Spalten.}


\item {\bf Cadenus} \cite{ACA2002}\index{Cadenus}:
   Hierbei handelt es sich um eine 
   Spalten"-transposition, die zwei Schl"us"-sel"-worte benutzt.\\
   Das erste Schl"usselwort wird benutzt, um die Spalten zu vertauschen.\\
   Das zweite Schl"usselwort wird benutzt, um das Startzeichen jeder Spalte
   festzulegen: dieses zweite Schl"usselwort ist eine beliebige Permutation
   des benutzten Alphabets. Diese schreibt man links vor die erste Spalte.
   Jede Spalte wird dann vertikal so verschoben (wrap-around), dass sie mit dem
   Buchstaben beginnt, der in derjenigen Zeile, wo der Schl"usselbuchstabe des
   ersten Schl"usselwortes in dem zweiten Schl"usselwort zu finden ist.\\
   Der Geheimtext wird zeilenweise ausgelesen.

   Siehe Tabelle \ref{Cadenus-table-reference}.

   Klartext: ein laengeres beispiel zur transposition mit cadenus

   \begin{table}[h]
   \begin{center}
   \begin{tabular}{|c|ccc|ccc|ccc|}
   \hline
	  & K & {\bf E} & Y & {\bf E} & K & Y & {\bf E} & K & Y\\
   \hline
	A & e & i & n & i & e & n & {\bf p} & r & n\\
	D & l & a & e & a & l & e & i & b & o\\
	X & n & g & e & g & n & e & o & s & t\\
	K & r & e & s & e & {\bf r} & s & i & e & n\\
	C & b & e & i & e & b & i & a & u & t\\
	W & s & p & i & p & s & i & n & r & d\\
	N & e & l & z & l & e & z & i & s & u\\
	S & u & r & t & r & u & t & a & s & n\\
	Y & r & a & n & a & r & {\bf n} & g & i & e\\
	{\bf E} & s & {\bf p} & o & {\bf p} & s & o & e & m & e\\
	D & s & i & t & i & s & t & e & c & s\\
	T & i & o & n & o & i & n & p & e & i\\	
	U & m & i & t & i & m & t & l & s & i\\
	B & c & a & d & a & c & d & r & e & z\\
	R & e & n & u & n & e & u & a & l & t\\
	G & s & - & - & - & s & - & - & n & -\\
   \hline
   \end{tabular}
%be_2005: Fu�note zur Tabellen�berschrift wird nur angezeigt, wenn man das
%         \footnotetext-Statement nicht direkt hinter \caption{...} schreibt,
%         sondern au�erhalb!
%\caption{Cadenus\footnotemark}
   \caption[Cadenus]{Cadenus\footnotemark}
   \label{Cadenus-table-reference}
   \end{center}
   \end{table}%  be_2005: Das % ist n�tig: sonst ist "Geheimtext" etwas einger�ckt.
           %           Noch besser: Man l�sst das % weg und f�gt daf�r eine
           %           Leerzeile nach \end{table} ein, denn dann wird nicht
           %           mehr der "Geheimtext..." VOR die Tabelle gedruckt !!

   Geheimtext: PRNIB OOSTI ENAUT NRDIS UASNG IEEME ECSPE ILSIR EZALT N\\
   \footnotetext{%
   In dem zweiten Dreierblock sind diejenigen Zeichen fett, die nach 
   der Anwendung des zweiten Schl"usselwortes oben im dritten Dreierblock
   stehen.}

\end{itemize}



%------------------------------------------------------------------------------
\subsection{Substitutionsverfahren}
\index{Substitution}


%------------------------------------------------------------------------------
\subsubsection{Monoalphabetische Substitutionsverfahren}
Monoalphabetische Substitutionsverfahren\index{Substitution!monoalphabetisch}
ordnen jedem Klartextzeichen ein Geheimtextzeichen fest zu, d.h. diese
Zuordnung ist w"ahrend des ganzen Verschl"usselungsprozesses dieselbe.
\label{monoalphabeticSubstitutionCiphers}

\begin{itemize}

\item {\bf Allgemeine monoalphabetische Substitution / Zuf"allige
   Buchstabenzuordnung\footnotemark}
   \footnotetext{%
   Dieses Verfahren kann in CrypTool\index{CrypTool} mit dem Men"upunkt 
   {\bf Ver-/Entschl"usseln \textbackslash{} Symmetrisch (klassisch) \textbackslash{}
   Substitution / Atbash} abgebildet werden.}
   \cite{Singh2001}: Die Substitution erfolgt aufgrund einer festgelegten
   Zuordnung der Einzelbuchstaben.

\item {\bf Atbash\footnotemark}
   \footnotetext{%
   Dieses Verfahren kann in CrypTool\index{CrypTool} mit dem Men"upunkt 
   {\bf Ver-/Entschl"usseln \textbackslash{} Symmetrisch (klassisch) \textbackslash{}
   Substitution / Atbash} abgebildet werden.} 
   \cite{Singh2001}\index{Atbash}: Der erste Buchstabe des Alphabets wird 
   durch den letzten Buchstaben des Alphabets ersetzt, 	der zweite durch den
   vorletzten, usw.

\item {\bf Verschiebechiffre, z.B. Caesar}\footnote{In CrypTool
   kann man dieses Verfahren an drei verschiedenen Stellen im Men"u finden:\\
   - {\bf Ver-/Entschl"usseln \textbackslash{} Symmetrisch (klassisch)
     \textbackslash{} Caesar / ROT13} \\
   - {\bf Analyse \textbackslash{} Symmetrische Verschl"usselung (klassisch)
     \textbackslash{} Ciphertext only \textbackslash{} Caesar} \\
   - {\bf Einzelverfahren \textbackslash{} Visualisierung von Algorithmen 
     mit ANIMAL \textbackslash{} Caesar}. }
   \cite{Singh2001}\index{Caesar}%
   : Klartext- und Geheimtextalphabet werden um eine bestimmte Anzahl von
   Zeichen gegeneinander verschoben.

   Klartext: 	bei der caesarchiffre wird um drei stellen verschoben

   Geheimtext:	EHL GHU FDHVDUFKLIIUH ZLUG XP GUHL VWHOOHQ YHUVFKREHQ\\

\item {\bf Substitution mit Symbolen, z.B. Freimaurerchiffre}  
   \cite{Singh2001}: Ein Buchstabe wird durch ein Symbol ersetzt.

\item {\bf Varianten}: F"uller, absichtliche Fehler \cite{Singh2001}.

\item {\bf Nihilist-Substitution}\footnote{Eine Animation zu diesem
   Nihilist-Verfahren findet sich in CrypTool unter dem Men"upunkt 
     {\bf Einzelverfahren \textbackslash{} Visualisierung von Algorithmen 
     mit ANIMAL \textbackslash{} Nihilist}. }
   \cite{ACA2002}\index{Nihilist-Substitution}: Das Alphabet wird in 
   eine 5x5-Matrix eingetragen und jeder Klartextbuchstabe durch das 
   entsprechende Ziffernpaar ersetzt. Die so entstandenen zweistelligen 
   Zahlen werden in eine Tabelle eingetragen. Dazu wird zun"achst ein
   Schl"usselwort gew"ahlt und "uber die Tabelle geschrieben,
   wobei die Zeichen des Schl"usselwortes ebenfalls durch Zahlenpaare
   substituiert werden. Die Geheimtextzeichen sind die Summen aus den
   Zahlen des Klartextes und den Zahlen des Schl"usselwortes. Bei Zahlen
   zwischen 100 und 110 wird die f"uhrende "`1"' ignoriert, so dass jeder
   Buchstabe durch eine zweistellige Zahl repr"asentiert wird.

   Siehe Tabelle \ref{Nihilist-substitution-table-reference}.
	
   Klartext: ein beispiel zur substitution

% \begin{minipage}  Das um die Table herum f�hrt nur zu Tex-Fehlern!
% Hier nun um 2 Tabellen nur 1 TABLE: Table hat 2 Wirkungen:
%   a) sagt TeX, dass er es frei vertikal positionieren kann.
%   b) dass man eine Nummerierung erh�lt und einen Index erstellen kann.
   \begin{table}[h]

   \begin{center}
   Matrix~~   % Text links vor eine Table und 2 Blanks dazu
   \begin{tabular}{|c|ccccc|}
   \hline 	
	  & 1 & 2 & 3 & 4 & 5\\
   \hline
	1 & S & C & H & L & U\\
	2 & E & A & B & D & F\\
	3 & G & I & K & M & N\\
	4 & O & P & Q & R & T\\
	5 & V & W & X & Y & Z\\
   \hline
   \end{tabular}  
   \end{center} 

   \begin{center}
   Tabelle~~
   \begin{tabular}{|ccc|}
   \hline 	
	K & E & Y\\
	(33) & (21) & (54)\\
   \hline
	e & i & n\\
	(54) & (53) & (89)\\
	b & e & i\\
	(56) & (42) & (86)\\
	s & p & i\\
	(44) & (63) & (86)\\
	e & l & z\\
	(54) & (35) & (109)\\
	u & r & s\\
	(48) & (65) & (65)\\
	u & b & s\\
	(48) & (44) & (65)\\
	t & i & t\\
	(78) & (53) & (99)\\
	u & t & i\\
	(48) & (66) & (86)\\
	o & n &   \\	
	(74) & (56)&   \\
   \hline
   \end{tabular}  
   \caption{Nihilist-Substitution}
   \label{Nihilist-substitution-table-reference} %Ans Ende legen; analog bei figure!
   \end{center} 

   \end{table}%
   % \end{minipage}  Vertr�gt sich nicht mit TABLE.
	
   Geheimtext: 
   54 53 89 56 42~~~86 44 63 86 54~~~35 09 48 65 65
   48 44 65 78 53~~~99 48 66 86 74~~~56\\


\newpage % Damit die Tables immer direkt nach dem zugeh�rigen Item. 
\item {\bf Codierung} \cite{Singh2001}: Im Laufe der Geschichte wurden immer
   wieder Codeb"ucher verwendet. In diesen B"uchern wird jedem m"oglichen 
   {\bf Wort} eines Klartextes ein Codewort, ein Symbol oder eine Zahl
   zugeordnet.
   Voraussetzung f"ur eine erfolgreiche geheime Kommunikation ist, dass 
   Sender und Empf"anger exakt das gleiche Codebuch besitzen und die 
   Zuordnung der Codew"orter zu den Klartextw"ortern nicht offengelegt wird.

\item {\bf Nomenklatur} \cite{Singh2001}\index{Nomenklatur}: Eine Nomenklatur 
   ist ein Verschl"usselungssystem, das auf einem Geheimtextalphabet basiert,
   mit dem ein Gro"steil der Nachricht chiffriert wird. F"ur besonders
   h"aufig auftretende oder geheim zu haltende W"orter existieren eine 
   begrenzte Anzahl von Codew"ortern.

\item {\bf Landkarten-Chiffre} \cite{ThinkQuest1999}\index{Landkarten-Chiffre}: 
   Diese Methode stellt eine Kombination aus 
   Substitution und Steganographie\footnote{Statt eine Nachricht zu 
   verschl"usseln, versucht man bei der reinen
   Steganographie\index{Steganographie}, die Existenz
   der Nachricht zu verbergen.} dar.
   Klartextzeichen werden durch Symbole ersetzt, diese werden nach bestimmten
   Regeln in Landkarten angeordnet.


\item {\bf Straddling Checkerboard} 
   \cite{Goebel2003}\index{Straddling Checkerboard}: 
   Eine 3x10-Matrix wird
   mit den Buchstaben des Alphabets und zwei beliebigen Sonderzeichen oder
   Zahlen gef"ullt, indem zun"achst die voneinander verschiedenen Zeichen
   eines Schl"usselwortes und anschlie"send die restlichen Buchstaben des
   Alphabetes eingef"ugt werden. Die Spalten der Matrix werden mit den Ziffern
   0 bis 9, die zweite und dritte Zeile der Matrix mit den Ziffern 1 und 2
   nummeriert. Jedes Zeichen des Geheimtextes wird durch die entsprechende
   Ziffer bzw. das entsprechende Ziffernpaar ersetzt. Da die 1 und die 2
   die ersten Ziffern der m"oglichen Ziffernkombinationen sind, werden
   sie nicht als einzelne Ziffern verwendet. 

   Siehe Tabelle \ref{Straddling-Checkerboard-table-reference}.

   Klartext: substitution bedeutet ersetzung  % hiernach Leerzeile, nichts stattdessen
                                              % ein \\, sonst wird, wenn die Tabelle
                                              % sp�ter kommt, der Abstand doppelt. 
                                              % Warum auch immer?!

   \begin{table}[h]
   \begin{center}
   \begin{tabular}{|c|cccccccccc|}
   \hline
	  & 0 & 1 & 2 & 3 & 4 & 5 & 6 & 7 & 8 & 9\\
   \hline
	  & S & - & - & C & H & L & U & E & A & B\\
	1 & D & F & G & I & J & K & M & N & O & P\\
	2 & Q & R & T & V & W & X & Y & Z & . & /\\
   \hline
   \end{tabular}
   \caption{Straddling Checkerboard mit Passwort "`Schluessel"'}
   \label{Straddling-Checkerboard-table-reference}
   \end{center}
   \end{table}

   Geheimtext: 06902 21322 23221 31817 97107 62272 27210 72227 61712\\

   Auff"allig ist die H"aufigkeit der Ziffern 1 und 2, 
   dies wird jedoch durch die folgende Variante behoben.\\


\item {\bf Straddling Checkerboard, Variante} \cite{Goebel2003}%
   \index{Straddling Checkerboard}: 
   Diese Form des Straddling Checkerboards wurde von sowjetischen Spionen
   im Zweiten Weltkrieg entwickelt. Angeblich haben auch Ernesto (Ch\'e)
   Guevara\index{Ch\'e Guevara} und Fidel Castro diese Chiffre zur geheimen
   Kommunikation benutzt.  
   Das Alphabet wird in ein Gitter eingetragen (Spaltenanzahl = L"ange des 
   Schl"usselwortes), und es werden zwei beliebige Ziffern als "`reserviert"' 
   festgelegt, die sp"ater die zweite und dritte Zeile einer
   3x10-Matrix bezeichnen (in unserem Bsp. 3 und 7). Nun wird das Gitter mit 
   dem erzeugten Alphabet spaltenweise durchlaufen und Buchstaben zeilenweise
   in die Matrix "ubertragen:
   Die acht h"aufigsten Buchstaben (ENIRSATD f"ur die deutsche Sprache) 
   bekommen zur schnelleren Chiffrierung die Ziffern 0 bis 9 zugewiesen, 
   dabei werden die reservierten Ziffern nicht vergeben. Die "ubrigen
   Buchstaben werden der Reihe nach in die Matrix eingetragen.
   Gegebenenfalls wird als zweite Stufe der Verschl"usselung zum Geheimtext
   noch eine beliebige Ziffernfolge addiert.

   Siehe Tabelle \ref{Straddling-Checkerboard-variant-table-reference}.

   Klartext: substitution bedeutet ersetzung

   \begin{table}[h]

   \begin{center}
   Gitter~~
   \begin{tabular}{|c|c|c|c|c|c|}
   \hline 		
	{\bf S} & C & H & L & U & {\bf E}\\
   \hline
	{\bf A} & B & {\bf D} & F & G & {\bf I}\\
   \hline
	J & K & M & {\bf N} & O & P\\
   \hline
	Q & {\bf R} & {\bf T} & V & W & X\\
   \hline
	Y & Z & . & / &   &    \\
   \hline
   \end{tabular}  
   \end{center} 

   \begin{center}
   Matrix~~
   \begin{tabular}{|c|cccccccccc|}
   \hline 	
	  & 0 & 1 & 2 & 3 & 4 & 5 & 6 & 7 & 8 & 9\\
   \hline
	  & {\bf S} & {\bf A} & {\bf R} & - & {\bf D} & {\bf T} & {\bf N} & - & {\bf E} & {\bf I}\\
	3 & J & Q & Y & C & B & K & Z & H & M & .\\
	7 & L & F & V & / & U & G & O & W & P & X\\
   \hline
   \end{tabular}  
   \caption{Variante des Straddling Checkerboards}
   \label{Straddling-Checkerboard-variant-table-reference}
   \end{center}
 
   \end{table}

   Geheimtext: 07434 05957 45976 63484 87458 58208 53674 675\\


   \begin{itemize}
      \item {\bf Ch\'e Guevara};
      Einen Spezialfall dieser Variante benutzte Ch\'e Guevara (mit einem 
      zus"atzlicher Substitutionsschritt und einem leicht modifizierten Checkerboard):
	 \begin{itemize}
	    \item Die sieben h"aufigsten Buchstaben im Spanischen werden auf
               die erste Zeile verteilt.
            \item Es werden vier statt drei Zeilen benutzt.
            \item Damit konnte man $10*4 - 4 = 36 $ verschiedene Zeichen 
               verschl"usseln.\\
         \end{itemize}
   \end{itemize}
  


\item {\bf Tri-Digital} \cite{ACA2002}: 
   Aus einem Schl"ussel"-wort der L"ange 10 wird ein numerischer Schl"us"-sel 
   gebildet, indem die Buchstaben entsprechend ihres Auftretens im Alphabet
   durchnummeriert werden. Dieser Schl"ussel wird "uber ein Gitter mit zehn
   Spalten geschrieben. In dieses Gitter wird unter Verwendung eines 
   Schl"usselwortes zeilenweise das Alphabet eingetragen, wobei die letzte
   Spalte frei bleibt. Die Klartextzeichen werden durch die Zahl "uber der
   entsprechenden Spalte substituiert, die Zahl "uber der freien Spalte dient
   als Trennzeichen zwischen den einzelnen W"ortern.\\


\item {\bf Baconian Cipher} \cite{ACA2002}\index{Baconian Cipher}: 
   Jedem Buchstaben des Alphabets und 6 Zahlen oder Sonderzeichen wird ein
   f"unfstelliger Bin"arcode zugeordnet (zum Beispiel 00000 = A, 00001 = B, 
   usw.). Die Zeichen der Nachricht werden entsprechend ersetzt. Nun benutzt
   man eine zweite, unverd"achtige Nachricht, um darin den Geheimtext zu 
   verbergen. Dies kann zum Beispiel durch Klein- und Gro"sschreibung oder
   kursiv gesetzte Buchstaben geschehen: man schreibt z.B. alle Buchstaben
   in der unverd"achtigen Nachricht gro"s, die unter einer \glqq 1\grqq~
   stehen.

   Siehe Tabelle \ref{Baconian-table-reference}.

   \begin{table}[h]
   \begin{center}
   \begin{tabular}{|c|ccccc|}
   \hline
        Nachricht                &  H   &   I   &   L   &   F   &   E     \\
   \hline
	Geheimtext               & {\tt 00111} & {\tt 01000} & {\tt 01011} & {\tt 00101} & {\tt 00100}   \\
	Unverd"achtige Nachricht & {\tt esist} & {\tt warmu} & {\tt nddie} & {\tt sonne} & {\tt scheint} \\
   \hline
	Baconian Cipher          & {\tt esIST} & {\tt wArmu} & {\tt nDdIE} & {\tt soNnE} & {\tt scHeint} \\
   \hline
   \end{tabular}  
   \caption{Baconian Cipher}
   \label{Baconian-table-reference}
   \end{center} 
   \end{table}

   \end{itemize}


%------------------------------------------------------------------------------
\subsubsection{Homophone Substitutionsverfahren}

Homophone Verfahren\index{Substitution!homophon} stellen eine Sonderform der
monoalphabetischen Substitution dar. Jedem Klartextzeichen werden mehrere
Geheimtextzeichen zugeordnet.

\begin{itemize}

\item {\bf Homophone monoalphabetische Substitution}\footnote{In CrypTool
   kann man dieses Verfahren "uber den Men"ueintrag {\bf Ver-/Entschl"usseln
   \textbackslash{} Symmetrisch (klassisch) \textbackslash{} Homophone}
   aufrufen.}
   \cite{Singh2001}: 
   Um die typische H"aufigkeitsverteilung der Buchstaben einer nat"urlichen
   Sprache zu verschleiern, werden einem Klartextbuchstaben mehrere 
   Geheimtextzeichen fest zugeordnet. Die Anzahl der zugeordneten Zeichen
   richtet sich gew"ohnlich nach der H"aufigkeit des zu verschl"usselnden
   Buchstabens.

\item {\bf Beale-Chiffre} \cite{Singh2001}\index{Beale-Chiffre}: 
   Die Beale-Chiffre ist eine Buchchiffre, bei der die W"orter eines 
   Schl"usseltextes durchnummeriert werden. Diese Zahlen ersetzen die
   Buchstaben des Klartextes durch die Anfangsbuchstaben der W"orter.

\item {\bf Grandpr\'e Cipher} \cite{Savard1999}: 
   Eine 10x10-Matrix (auch andere Gr"o"sen sind m"oglich) wird mit zehn W"orter
   mit je zehn Buchstaben gef"ullt, so dass die Anfangsbuchstaben ein elftes
   Wort ergeben. Da die Spalten und Zeilen mit den Ziffern 0 bis 9 
   durchnummeriert werden, l"asst sich jeder Buchstabe durch ein Ziffernpaar
   darstellen. Es ist offensichtlich, dass bei einhundert Feldern die meisten
   Buchstaben durch mehrere Ziffernpaare ersetzt werden k"onnen. Wichtig ist,
   dass die zehn W"orter m"oglichst alle Buchstaben des Alphabets enthalten.

\item {\bf Buchchiffre}\index{Buchchiffre}: 
   Die W"orter eines Klartextes werden durch Zahlentripel der Form 
   "`Seite-Zeile-Position"' ersetzt. Diese Methode setzt eine genaue 
   Absprache des verwendeten Buches voraus, so muss es sich insbesondere um 
   die gleiche Ausgabe handeln (Layout, Fehlerkorrekturen, etc.).

\end{itemize}



%------------------------------------------------------------------------------
\subsubsection{Polygraphische Substitutionsverfahren}
\label{polygraphicSubstitutionCiphers}

Bei der polygraphische Substitution\index{Substitution!polygraphisch} werden 
keine einzelne Buchstaben ersetzt, sondern Buchstabengruppen. Dabei kann es
sich um Digramme, Trigramme, Silben, etc. handeln.

\begin{itemize}

\item {\bf Gro"se Chiffre} \cite{Singh2001}: 
   Die Gro"se Chiffre wurde von Ludwig XIV. verwendet und erst kurz vor 
   Beginn des 20. Jahrhunderts entschl"usselt. Die Kryptogramme enthielten 587
   verschieden Zahlen, jede Zahl repr"asentierte eine Silbe. Die Erfinder
   dieser Chiffre (Rossignol, Vater und Sohn) hatten zus"atzliche Fallen
   eingebaut, um die Sicherheit der Methode zu erh"ohen. Eine Zahl konnte
   beispielsweise die vorangehende l"oschen oder ihr eine andere Bedeutung
   zuweisen.

   \hypertarget{playfair}{}
\item {\bf Playfair}\footnote{In CrypTool
   kann man dieses Verfahren "uber den Men"ueintrag {\bf Ver-/Entschl"usseln
   \textbackslash{} Symmetrisch (klassisch) \textbackslash{} Playfair}
   aufrufen.}
   \cite{Singh2001}\index{Playfair}: 
   Eine 5x5-Matrix wird mit dem Alphabet gef"ullt, z.B. erst mit den 
   verschiedenen Zeichen eines Schl"usselwortes, dann mit den restlichen 
   Buchstaben des Alphabets. Der Klartext wird in Digramme unterteilt, die 
   nach den folgenden Regeln verschl"usselt werden:
   \begin{enumerate}
      \item Befinden sich die Buchstaben in der selben Spalte, werden sie durch 
         die Buchstaben ersetzt, die direkt darunter stehen.
      \item Befinden sich die Buchstaben in der selben Zeile, nimmt man jeweils
         den Buchstaben rechts vom Klartextzeichen.
      \item Befinden sich die Buchstaben in unterschiedlichen Spalten und 
         Zeilen, nimmt man jeweils den Buchstaben, der zwar in der selben
         Zeile, aber in der Spalte des anderen Buchstabens steht.
      \item F"ur Doppelbuchstaben (falls sie in einem Digramm vorkommen)
         gelten Sonderregelungen, wie zum Beispiel
         die Trennung durch einen F"uller.
   \end{enumerate}

   Siehe Tabelle \ref{Playfair-table-reference}.

   Klartext: buchstaben werden paarweise verschluesxselt

   \begin{table}[h]
   \begin{center}
   \begin{tabular}{|c|c|c|c|c|}
   \hline
	S & C & H & L & U\\
   \hline
	E & A & B & D & F\\
   \hline
	G & I & K & M & N\\
   \hline
	O & P & Q & R & T\\
   \hline
	V & W & X & Y & Z\\
   \hline
   \end{tabular}
   \caption{5x5-Playfair-Matrix}
   \label{Playfair-table-reference}
   \end{center}
   \end{table}

   Geheimtext: FHHLU OBDFG VAYMF GWIDP VAGCG SDOCH LUSFH VEGUR\\


\item {\bf Playfair f"ur Trigramme} \cite{Savard1999}: 
Zun"achst f"ullt man eine 5x5-Matrix mit dem Alphabet und teilt den Klartext
in Trigramme auf. F"ur die Verschl"usselung gelten folgende Regeln:
   \begin{enumerate}
      \item Drei gleiche Zeichen werden durch drei gleiche Zeichen ersetzt, 
         es wird der Buchstabe verwendet, der rechts unter dem urspr"unglichen
         Buchstaben steht (Beispiel anhand von Tabelle 11: 
         BBB $ \Rightarrow $ MMM).
      \item Bei zwei gleichen Buchstaben in einem Trigramm gelten die Regeln
         der Original-Playfair-Verschl"usselung.
      \item Bei drei unterschiedlichen Buchstaben kommen relativ komplizierte
         Regeln zur Anwendung, mehr dazu unter \cite{Savard1999}.
   \end{enumerate}


\item {\bf Ersetzung von Digrammen durch Symbole} \cite{Savard1999}: 
   Giovanni Battista della Porta, 15. Jahrhundert. Er benutzte eine
   20x20-Matrix, in die er f"ur jede m"ogliche Buchstabenkombination (das 
   verwendete Alphabet bestand aus nur zwanzig Zeichen) ein Symbol eintrug.


\item {\bf Four Square Cipher} \cite{Savard1999}: 
   Diese Methode "ahnelt Playfair, denn es handelt sich um ein 
   Koordinatensystem, dessen vier Quadranten jeweils mit dem Alphabet gef"ullt
   werden, wobei die Anordnung des Alphabets von Quadrant zu Quadrant 
   unterschiedlich sein kann. Um eine Botschaft zu verschl"usseln, geht man wie
   folgt vor: 
   Man sucht den ersten Klartextbuchstaben im ersten Quadranten und den zweiten
   Klartextbuchstaben im dritten Quadranten. Denkt man sich ein Rechteck mit 
   den beiden Klartextbuchstaben als gegen"uberliegende Eckpunkte, erh"alt man
   im zweiten und vierten Quadranten die zugeh"origen Geheimtextzeichen.

   Siehe Tabelle \ref{Four-Square-Cipher-table-reference}.

   Klartext: buchstaben werden paarweise verschluesselt

   \begin{table}[h]
   \begin{center}
   \begin{tabular}{|ccccc|ccccc|}
   \hline
	d & w & x & y & m & E & P & T & O & L\\
	r & q & e & k & i & C & V & I & Q & Z\\
	u & v & h & p & s & R & M & A & G & U\\
	a & l & {\bf b} & z & n & F & W & {\bf Y} & H & S\\
	g & c & o & f & t & B & N & D & X & K\\
   \hline
	Q & T & B & L & E & v & q & i & p & g\\
	Z & H & {\bf N} & D & X & s & t & {\bf u} & o & h\\
	P & M & I & Y & C & n & r & d & x & y\\
	V & S & K & W & O & b & l & w & m & f\\
	U & A & F & R & G & c & z & k & a & e\\
   \hline
   \end{tabular}
   \caption{Four Square Cipher}
   \label{Four-Square-Cipher-table-reference}
   \end{center}
   \end{table}

   Geheimtext:  YNKHM XFVCI LAIPC IGRWP LACXC BVIRG MKUUR XVKT\\


\item {\bf Two Square Cipher} \cite{Savard1999}: 
   Die Vorgehensweise gleicht der der Four Square Cipher, allerdings enth"alt
   die Matrix nur zwei Quadranten. Befinden sich die beiden zu ersetzenden
   Buchstaben in der gleichen Reihe, werden sie nur vertauscht. Andernfalls
   werden die beiden Klartextzeichen als gegen"uberliegende Eckpunkte eines
   Rechtecks betrachtet und durch die anderen Eckpunkte ersetzt. Die Anordnung
   der beiden Quadranten ist horizontal und vertikal m"oglich.


\item {\bf Tri Square Cipher} \cite{ACA2002}: 
   Drei Quadranten werden jeweils mit dem Alphabet gef"ullt. Der erste 
   Klartextbuchstabe wird im ersten Quadranten gesucht und kann mit jedem 
   Zeichen der selben Spalte verschl"usselt werden. Der zweite 
   Klartextbuchstabe wird im zweiten Quadranten (diagonal gegen�berliegend) 
   gesucht und kann mit jedem Buchstaben derselben Zeile verschl"usselt werden.
   Zwischen diese Geheimtextzeichen wird der Buchstabe des Schnittpunktes 
   gesetzt.

\item {\bf Dockyard Cipher/Werftschl"ussel} \cite{Savard1999}: 
   Angewendet von der Deutschen Marine im Zweiten Weltkrieg.

\end{itemize}
~\\


%------------------------------------------------------------------------------
\subsubsection{Polyalphabetische Substitutionsverfahren}

Bei der polyalphabetischen Substitution\index{Substitution!polyalphabetisch}
ist die Zuordnung Klartext-/Geheimtextzeichen nicht fest, sondern variabel
(meist abh"angig vom Schl"ussel).

\begin{itemize}

\item {\bf Vigen\`ere}\footnote{%
   In CrypTool kann man dieses Verfahren "uber den Men"ueintrag {\bf 
   Ver-/Entschl"usseln \textbackslash{} Symmetrisch (klassisch) 
   \textbackslash{} Vigen\`ere} aufrufen.}
   \cite{Singh2001}\index{Vigen\`ere}: 
   Entsprechend den Zeichen eines Schl"usselwortes wird jedes Klartextzeichen
   mit einem anderen Geheimtextalphabet verschl"usselt (als Hilfsmittel dient
   das sog. Vigen\`ere-Tableau). Ist der Klartext l"anger als der Schl"ussel,
   wird dieser wiederholt.

   Siehe Tabelle \ref{Vigenere-table-reference}.

   \begin{table}[h]

   \begin{center}
   \begin{tabular}{|c|c|c|c|c|}
   \hline
   Klartext:   & {\tt das} & {\tt alphabet} & {\tt wechselt} & {\tt staendig}\\
   \hline
   Schl"ussel: & {\tt KEY} & {\tt KEYKEYKE} & {\tt YKEYKEYK} & {\tt EYKEYKEY}\\
   \hline
   Geheimtext: & {\tt NEQ} & {\tt KPNREZOX} & {\tt UOGFCIJD} & {\tt WRKILNME}\\
   \hline
   \end{tabular}  
   \end{center} 

   {
   \textmd \small
   \begin{center}
   \begin{tabular}{|@{\:}r@{\:}@{\:}|r@{\:}r@{\:}r@{\:}r@{\:}r@{\:}r@{\:}r@{\:}r@{\:}r@{\:}r@{\:}r@{\:}r@{\:}r@{\:}r@{\:}r@{\:}r@{\:}r@{\:}r@{\:}r@{\:}r@{\:}r@{\:}r@{\:}r@{\:}r@{\:}r@{\:}r@{\:}|}
   \hline
	- & A & B & C & {\bf D} & E & F & G & H & I & J & K & L & M & N & O & P & Q & R & S & T & U & V & W & X & Y & Z\\
   \hline
	A & A & B & C & D & E & F & G & H & I & J & K & L & M & N & O & P & Q & R & S & T & U & V & W & X & Y & Z\\
	B & B & C & D & E & F & G & H & I & J & K & L & M & N & O & P & Q & R & S & T & U & V & W & X & Y & Z & A\\
	C & C & D & E & F & G & H & I & J & K & L & M & N & O & P & Q & R & S & T & U & V & W & X & Y & Z & A & B\\
	D & D & E & F & G & H & I & J & K & L & M & N & O & P & Q & R & S & T & U & V & W & X & Y & Z & A & B & C\\
	E & E & F & G & H & I & J & K & L & M & N & O & P & Q & R & S & T & U & V & W & X & Y & Z & A & B & C & D\\
	F & F & G & H & I & J & K & L & M & N & O & P & Q & R & S & T & U & V & W & X & Y & Z & A & B & C & D & E\\
	G & G & H & I & J & K & L & M & N & O & P & Q & R & S & T & U & V & W & X & Y & Z & A & B & C & D & E & F\\
	H & H & I & J & K & L & M & N & O & P & Q & R & S & T & U & V & W & X & Y & Z & A & B & C & D & E & F & G\\
	I & I & J & K & L & M & N & O & P & Q & R & S & T & U & V & W & X & Y & Z & A & B & C & D & E & F & G & H\\
	J & J & K & L & M & N & O & P & Q & R & S & T & U & V & W & X & Y & Z & A & B & C & D & E & F & G & H & I\\
	{\bf K} & K & L & M & {\bf N} & O & P & Q & R & S & T & U & V & W & X & Y & Z & A & B & C & D & E & F & G & H & I & J\\
	... & ... & ... &   &   &   &   &   &   &   &   &   &   &   &   &   &   &   &   &   &   &   &   &   &   &   &  \\
   \hline
   \end{tabular}  
   \caption{Vigen\`ere-Tableau}
   \label{Vigenere-table-reference}
   \end{center} 
   }

   \end{table}
%~ \\  % N�tig f�r einheitlichen Abstand (sonst folgt nach der Tabelle ja oft eine 
      % Zeile mit dem Geheimtext. Aber so nie sicher, ob der Abstand nach der
      % Tabelle auf der gleichen Seite (ok), oder nach dem Text (zuviel), falls
      % die Tabelle auf die kommende Seite kommt.


\newpage % Damit die Tables immer direkt nach dem zugeh�rigen Item. 
\begin{itemize}
   \item {\bf Unterbrochener Schl"ussel}: 
     Der Schl"ussel wird nicht fortlaufend wiederholt, sondern beginnt mit 
     jedem neuen Klartextwort von vorne.

 
   \item {\bf Autokey-Variante} \cite{Savard1999}: 
      Nachdem der vereinbarte Schl"ussel abgearbeitet wurde, geht man dazu
      "uber, die Zeichen der Nachricht als Schl"ussel zu benutzen.

      Siehe Tabelle \ref{Autokey-table-reference}.

   \begin{table}[h]
   \begin{center}
   \begin{tabular}{|c|c|c|c|c|}
   \hline
   Klartext:   & {\tt das} & {\tt alphabet} & {\tt wechselt} & {\tt staendig}\\
   \hline
   Schl"ussel: & {\tt KEY} & {\tt DASALPHA} & {\tt BETWECHS} & {\tt ELTSTAEN}\\
   \hline
   Geheimtext: & {\tt NEQ} & {\tt DLHHLQLA} & {\tt XIVDWGSL} & {\tt WETWGDMT}\\
   \hline
   \end{tabular}  
   \caption{Autokey-Variante}
   \label{Autokey-table-reference}	
   \end{center} 
   \end{table}  % Hiernach u. in die n�chste Zeile darf man kein // setzen 
                % -> TeX-Fehler !
                %~ \\


   \item {\bf Progressive-Key-Variante} \cite{Savard1999}: 
      Der Schl"ussel "andert sich im Laufe der Chiffrierung, indem er das 
      Alphabet durchl"auft. So wird aus KEY LFZ.

   \item {\bf Gronsfeld} \cite{Savard1999}: 
      Vigen\`ere-Variante, die einen Zahlenschl"ussel verwendet.

   \item {\bf Beaufort} \cite{Savard1999}\index{Beaufort}: 
      Vigen\`ere-Variante, keine Verschl"usselung durch Addition, sondern durch
      Subtraktion. Auch mit r"uckw"arts geschriebenem Alphabet.

   \item {\bf Porta} \cite{ACA2002}: 
      Vigen\`ere-Variante, die nur 13 Alphabete verwendet. Das bedeutet, dass 
      jeweils zwei Schl"usselbuchstaben dasselbe Geheimtextalphabet zugeordnet
      wird, und die erste und zweite H"alfte des Alphabets reziprok sind.

   \item {\bf Slidefair} \cite{ACA2002}: 
      Kann als Vigen\`ere-, Gronsfeld- oder Beaufort-Variante verwendet werden. 
      Dieses Verfahren verschl"usselt Digramme. Den ersten Buchstaben sucht man
      im Klartextalphabet "uber dem Tableau, den zweiten in der Zeile, die dem
      Schl"usselbuchstaben entspricht. Diese beiden Punkte bilden 
      gegen"uberliegende Punkte eines gedachten Rechtecks, die verbleibenden
      Ecken bilden die Geheimtextzeichen.

\end{itemize}


\item {\bf Superposition}\index{Superposition}
   \begin{itemize}
      \item {\bf Buchchiffre}: 
         Addition eines Schl"usseltextes (z.B. aus einem Buch) zum Klartext.
      \item {\bf "Uberlagerung mit einer Zahlenfolge}: 
         Eine M"oglichkeit sind mathematische Folgen wie die Fibonacci-Zahlen.
   \end{itemize}


\item {\bf Phillips} \cite{ACA2002}: 
   Das Alphabet wird in eine 5x5-Matrix eingetragen. Dann werden 7 weitere 
   Matrizen erzeugt, indem zun"achst immer die erste, dann die zweite Zeile
   um eine Position nach unten verschoben wird. Der Klartext wird in Bl"ocke
   der L"ange 5 unterteilt, die jeweils mit Hilfe einer Matrix verschl"usselt
   werden. Dazu wird jeweils der Buchstabe rechts unterhalb des 
   Klartextzeichens verwendet.


\item {\bf Ragbaby} \cite{ACA2002}: 
   Zuerst wird ein Alphabet mit 24 Zeichen konstruiert. Die Zeichen des 
   Klartextes werden durchnummeriert, wobei die Nummerierung der Zeichen des
   ersten Wortes mit 1 beginnt, die des zweiten Wortes mit 2 usw. Die Zahl 25
   entspricht wieder der Zahl 1. Ein Buchstabe der Nachricht wird chiffriert,
   indem man im Alphabet entsprechend viele Buchstaben nach rechts geht.

   Alphabet: SCHLUEABDFGIKMNOPQRTVWXZ\\
   \begin{table}[h]
   \begin{center}
   \begin{tabular}{|c||r@{\:}r@{\:}r@{\:}|r@{\:}r@{\:}r@{\:}r@{\:}r@{\:}r@{\:}r@{\:}r@{\:}|r@{\:}r@{\:}r@{\:}r@{\:}r@{\:}r@{\:}r@{\:}r@{\:}|r@{\:}r@{\:}r@{\:}r@{\:}r@{\:}r@{\:}r@{\:}r@{\:}|}
   \hline
   Klartext: & d & a & s & a & l & p & h & a & b & e & t & w & e & c & h & s & e & l & t & s & t & a & e & n & d & i & g\\
   Nummerierung: & 1 & 2 & 3 & 2 & 3 & 4 & 5 & 6 & 7 & 8 & 9 & 3 & 4 & 5 & 6 & 7 & 8 & 9 & 10 & 4 & 5 & 6 & 7 & 8 & 9 & 10 & 11\\
   Geheimtext: & F & D & L & D & A & V & B & K & N & M & U & S & F & A & D & B & M & K & E & U & S & K & K & X & Q & W & W\\
   \hline
   \end{tabular}  
   \caption{Ragbaby}
   \end{center} 
   \end{table}

\end{itemize}
 


%------------------------------------------------------------------------------
\subsection{Kombination aus Substitution und Transposition}

In der Geschichte der Kryptographie sind h"aufig Kombinationen der oben 
angef"uhrten Verfahrensklassen anzutreffen. Diese haben sich - im Durchschnitt - 
als sicherer erwiesen als Verfahren, die nur auf einem der Prinzipien Transposition oder Substitution beruhen.

\begin{itemize}

\item {\bf ADFG(V)X}\footnote{%
   In CrypTool kann man dieses Verfahren "uber den Men"ueintrag {\bf 
   Ver-/Entschl"usseln \textbackslash{} Symmetrisch (klassisch) 
   \textbackslash{} ADFGVX} aufrufen.}
   \cite{Singh2001}\index{ADFGVX}%
   : 
   Die ADFG(V)X-Verschl"usselung wurde in Deutschland im ersten Weltkrieg 
   entwickelt. Eine 5x5- oder 6x6-Matrix wird mit dem Alphabet gef"ullt, die 
   Spalten und Zeilen werden mit den Buchstaben ADFG(V)X versehen. Jedes 
   Klartextzeichen wird durch das entsprechende Buchstabenpaar ersetzt. 
   Abschlie"send wird auf dem so entstandenen Text eine (Zeilen-)Transposition
   durchgef"uhrt.

\item {\bf Zerlegung von Buchstaben, auch Fractionation genannt} 
   \cite{Savard1999}: Sammelbegriff f"ur die Verfahren, die erst ein
   Klartextzeichen durch mehrere Geheimtextzeichen verschl"usseln und auf 
   diese Verschl"usselung dann eine Transposition anwenden, so dass die 
   urspr"unglich zusammengeh"orenden Geheimtextzeichen voneinander getrennt
   werden. 

   \begin{itemize}
      \item {\bf Bifid/Polybius square/Checkerboard} \cite{Goebel2003}: 
         Bei der Grundform dieser Verschl"usselungsmethode wird eine 5x5-Matrix
         mit den Buchstaben des Alphabets gef"ullt (siehe Playfair). Die
         Spalten und Zeilen dieser Matrix m"ussen durchnummeriert sein, damit
         jedes Zeichen des Klartextes durch ein Ziffernpaar (Zeile/Spalte)
         ersetzt werden kann. Die Zahlen k"onnen jede beliebige Permutation
         von (1,2,3,4,5,) sein. Dies ist ein "`Schl"ussel"' oder
         Konfigurations-Parameter dieses Verfahrens. Eine m"ogliche Variante
         besteht darin, Schl"usselw"orter statt der Zahlen 1 bis 5
         zu verwenden.
	 Meist wird der Klartext vorher in Bl"ocke gleicher L"ange
         zerlegt. Die Blockl"ange (hier 5) ist ein weiterer
         Konfigurations-Parameter diese Verfahrens. Um den Geheimtext zu
         erhalten, werden zun"achst alle Zeilennummern, dann alle
         Spaltennummern eines Blocks ausgelesen. 
         Anschlie"send werden die Ziffern paarweise in Buchstaben umgewandelt.
        

         Siehe Tabelle \ref{Bifid-table-reference}.

         \begin{table}[h]

         \begin{center}
         \begin{tabular}{|c|ccccc|}
         \hline
            & 2 & 4 & {\bf 5} & 1 & 3\\
         \hline
          1 & S & C & H & L & U\\
          4 & E & A & B & D & F\\
	  {\bf 2} & G & I & {\bf K} & M & N\\
          3 & O & P & Q & R & T\\
          5 & V & W & X & Y & Z\\
         \hline
         \end{tabular}  
         \end{center} 

         \begin{center}
         \begin{tabular}{|c|cccccc|}
         \hline
         Klartext: & {\tt {\bf k}ombi} & {\tt natio} & {\tt nenme} & {\tt hrere} & {\tt rverf} & {\tt ahren}\\
         \hline
         Zeilen:	& {\tt {\bf 2}3242}	& {\tt 24323} & {\tt 24224} & {\tt 13434} & {\tt 35434} & {\tt 41342}\\
         Spalten: & {\tt {\bf 5}2154} & {\tt 34342} & {\tt 32312} & {\tt 51212} & {\tt 12213} & {\tt 45123}\\
         \hline
         \end{tabular}  
         \caption{Bifid}
         \label{Bifid-table-reference}
         \end{center} 

         \end{table}

         23242 52154 24323 34342 24224 32312 13434 51212 35434 12213 41342 45123\\

         Geheimtext: NIKMW IOTFE IGFNS UFBSS QFDGU DPIYN\\		


      \item {\bf Trifid} \cite{Savard1999}: 
         27 Zeichen (Alphabet + 1 Sonderzeichen) k"onnen durch Tripel aus den 
         Ziffern 1 bis 3 repr"asentiert werden. Die zu verschl"usselnde 
         Botschaft wird in Bl"ocke der L"ange 3 zerlegt und unter jeden 
         Buchstaben wird das ihm entsprechende Zahlentripel geschrieben. Die 
         Zahlen unter den Bl"ocken werden wiederum als Tripel zeilenweise 
         ausgelesen und durch entsprechende Zeichen ersetzt.
   \end{itemize}


\item {\bf Bazeries} \cite{ACA2002}: 
   Eine 5x5-Matrix wird spaltenweise mit dem Alphabet gef"ullt, eine zweite 
   Matrix wird zeilenweise mit dem Schl"ussel (einer ausgeschriebene Zahl 
   unter 1.000.000) und den "ubrigen Buchstaben des Alphabets gef"ullt. Der
   Text wird in beliebige Bl"ocke unterteilt, die Reihenfolge dieser Zeichen
   wird jeweils umgekehrt und zu jedem Zeichen entsprechend seiner Position
   in der ersten Matrix sein Gegenst"uck in der Schl"usselmatrix gesucht.

   Siehe Tabelle \ref{Bazeries-table-reference}.

   Klartext: kombinationen mehrerer verfahren\\
   Schl"usselwort: 900.004 (neunhunderttausendundvier)

   \begin{table}[h]

   \begin{center}
   \begin{tabular}{|ccccccccccc|}
   \hline
	a & f & l & q & v & & N & E & U & H & D\\
	b & g & {\bf m} & r & w & & R & T & {\bf A} & S & V\\
	c & h & n & s & x & & I & B & C & F & G\\
	d & i & o & t & y & & K & L & M & O & P\\
	e & k & p & u & z & & Q & W & X & Y & Z\\
   \hline
   \end{tabular}  
   \end{center} 

   \begin{center}
   \begin{tabular}{|ccccccccc|}
   \hline
	{\tt kom} & {\tt bi} & {\tt nation} & {\tt enm} & {\tt ehr} & {\tt ere} & {\tt rverf} & {\tt ahr} & {\tt en}\\
	{\tt {\bf m}ok} & {\tt ib} & {\tt noitan} & {\tt mne} & {\tt rhe} & {\tt ere} & {\tt frevr} & {\tt rha} & {\tt ne}\\
	{\tt {\bf A}MW} & {\tt LR} & {\tt CMLONC} & {\tt ACQ} & {\tt SBQ} & {\tt QSQ} & {\tt ESQDS} & {\tt SBN} & {\tt CQ}\\
   \hline
   \end{tabular}  
   \caption{Bazeries}
   \label{Bazeries-table-reference}
   \end{center} 

   \end{table}	


\item {\bf Digrafid} \cite{ACA2002}: 
   Zur Substitution der Digramme wird die nachfolgende Matrix benutzt (der 
   Einfachheit halber wird das Alphabet hier in seiner urspr"unglichen 
   Reihenfolge verwendet). Der erste Buchstabe eines Digramms wird im
   waagerechten Alphabet gesucht, notiert wird die Nummer der Spalte. Der 
   zweite Buchstabe wird im senkrechten Alphabet gesucht, notiert wird die 
   Nummer der Zeile. Zwischen diese beiden Ziffern wird die Ziffer des
   Schnittpunktes gesetzt. Die Tripel werden vertikal unter die Digramme, 
   die in Dreierbl"ocken angeordnet sind, geschrieben. Dann werden die 
   horizontal entstandenen dreistelligen Zahlen ausgelesen und in Buchstaben
   umgewandelt.

   {\bf Bemerkung:} Da das Verfahren immer ganze Dreiergruppen ben"otigt,
   ist eine vollst"andige Verfahrensbeschreibung zwischen Sender und Empf"anger
   notwendig, die auch den Umgang mit Texten erl"autert, die im letzten Block
   nur 1-5 Klartextbuchstaben enthalten. Vom Weglassen bis Padding mit
   zuf"allig oder fix vorher festgelegten Buchstaben ist alles m"oglich.

   Siehe Tabelle \ref{Digrafid-table-reference}.

   \begin{table}[h]

   \begin{center}
   \begin{tabular}{|ccccccccc|ccc|c|}
   \hline	
	1 & {\bf 2} & 3 & 4 & 5 & 6 & 7 & 8 & 9 &   &   &   &  \\
   \hline

	A & B & C & D & E & F & G & H & I & 1 & 2 & 3 &  \\
	J & {\bf K} & L & M & N & O & P & Q & R & 4 & {\bf 5} & 6 &  \\
	S & T & U & V & W & X & Y & Z & . & 7 & 8 & 9 &  \\
   \hline
	  &   &   &   &   &   &   &   &   & A & J & S & 1\\
	  &   &   &   &   &   &   &   &   & B & K & T & 2\\
	  &   &   &   &   &   &   &   &   & C & L & U & 3\\
	  &   &   &   &   &   &   &   &   & D & M & V & 4\\
	  &   &   &   &   &   &   &   &   & E & N & W & 5\\
	  &   &   &   &   &   &   &   &   & F & {\bf O} & X & {\bf 6}\\
	  &   &   &   &   &   &   &   &   & G & P & Y & 7\\
	  &   &   &   &   &   &   &   &   & H & Q & Z & 8\\
	  &   &   &   &   &   &   &   &   & I & R & . & 9\\
   \hline
   \end{tabular}  
   \end{center} 

   \begin{center}
   \begin{tabular}{|ccccccccccccccccccc|}
   \hline		
	ko & mb & in &   & at & io & ne &   & nm & eh & re &   & re & rv & er &   & fa & hr & en\\
   \hline
	2  & 4  & 9  &   & 1  & 9  & 5  &   & 5  & 5  & 9  &   & 9  & 9  & 5  &   & 6  & 8  & 5\\ 
	5  & 4  & 2  &   & 3  & 2  & 4  &   & 5  & 1  & 4  &   & 4  & 6  & 2  &   & 1  & 2  & 2\\ 
	6  & 2  & 5  &   & 2  & 6  & 5  &   & 4  & 8  & 5  &   & 5  & 4  & 9  &   & 1  & 9  & 5\\
   \hline
	KI & NB & FN &   & SW & CM & KW	&   & NR & ED & VN &   & .W & MT & NI &   & XN & AK & SW\\
   \hline
   \end{tabular}
   \caption{Digrafid}
   \label{Digrafid-table-reference}
   \end{center}

   \end{table}

\clearpage   % Damit Tabelle 19 (Nicodemus) noch vor Kapitel 2.4. kommt.
\item {\bf Nicodemus} \cite{ACA2002}: 
   Zun"achst wird eine einfache Spaltentransposition durchgef"uhrt. Noch vor 
   dem Auslesen erfolgt eine Vigen\`ere-Verschl"usselung (die Buchstaben einer
   Spalte werden mit dem entsprechenden Zeichen des Schl"usselwortes
   chiffriert).
   Das Auslesen erfolgt in vertikalen Bl"ocken.

   Siehe Tabelle \ref{Nicodemus-table-reference}.

   Klartext: kombinationen mehrerer verfahren

   \begin{table}[h]
   \begin{center}
   \begin{tabular}{|ccccccccccc|}
   \hline
	K & E & Y & & E & K & Y & & E & K & Y\\
   \hline
	k & o & m & & o & k & m & & S & U & K\\
	b & i & n & & i & b & n & & M & L & L\\
	a & t & i & & t & a & i & & X & K & G\\
	o & n & e & & n & o & e & & R & Y & C\\
	n & m & e & & m & n & e & & Q & X & C\\
	h & r & e & & r & h & e & & V & R & C\\
	r & e & r & & e & r & r & & I & B & P\\
	v & e & r & & e & v & r & & I & F & P\\
	f & a & h & & a & f & h & & E & P & F\\
	r & e & n & & e & r & n & & I & B & L\\
   \hline
   \end{tabular}
   \caption{Nicodemus}
   \label{Nicodemus-table-reference}
   \end{center}
   \end{table}

   Geheimtext: SMXRQ ULKYX KLGCC VIIEI RBFPB CPPFL\\

\end{itemize}


% \clearpage  % Damit Tabelle 19 (Nicodemus) noch vor Kapitel 2.4. kommt.
             % Leider dann Tab. 19 allein auf der Folgeseite und 2.4 beginnt
             % auf �bern�chster Seite.
%------------------------------------------------------------------------------
\subsection{Andere Verfahren}
\label{Further-PaP-methods}

\begin{itemize}

\item {\bf Nadelstich-Verschl"usselung} \cite{Singh2001}: 
   Dieses simple Verfahren wurde aus den unterschiedlichsten Gr"unden "uber
   viele Jahrhunderte hinweg praktiziert. So markierten zum Beispiel im 
   Viktorianischen Zeitalter kleine L"ocher unter Buchstaben in 
   Zeitungsartikeln die Zeichen des Klartextes, da das Versenden einer Zeitung
   sehr viel billiger war als das Porto f"ur einen Brief.

\item {\bf Lochschablone}: 
   Die Lochschablone ist auch unter der Bezeichnung
   Kardinal-Richelieu-Schl"ussel bekannt.
   Eine Schablone wird "uber einen vorher vereinbarten Text gelegt
   und die Buchstaben, die sichtbar bleiben, bilden den Geheimtext.

\item {\bf Kartenspiele} \cite{Savard1999}: 
   Der Schl"ussel wird mit Hilfe eines Kartenspiels und vorher festgelegter
   Regeln erzeugt. Alle im Folgenden genannten Verfahren sind als 
   Papier- und Bleistiftverfahren ausgelegt, also ohne elektronische 
   Hilfsmittel durchf"uhrbar. Ein Kartenspiel ist f"ur Au"senstehende 
   unverd"achtig, das Mischen der Karten bietet ein gewisses Ma"s an 
   Zuf"alligkeit, die Werte der Karten lassen sich leicht in Buchstaben
   umwandeln und Transpositionen lassen sich ohne weitere Hilfsmittel 
   (sei es schriftlich oder elektronisch) durchf"uhren.
   \begin{itemize}
      \item {\bf Solitaire (Bruce Schneier)\footnotemark}
         \footnotetext{%
         %Dieses Verfahren wird in einem zuk"unftigen Release von CrypTool
         %enthalten sein.}
         In CrypTool kann man dieses Verfahren "uber den Men"ueintrag {\bf 
         Ver-/Entschl"usseln \textbackslash{} Symmetrisch (klassisch) 
         \textbackslash{} Solitaire} aufrufen.}
         \cite{Schneier1999}\index{Solitaire}: 
         Sender und Empf"anger der Botschaft m"ussen jeweils ein Kartenspiel
         besitzen, bei dem alle Karten in der gleichen Ausgangsanordnung
         im Stapel liegen. Mit den Karten wird ein Schl"usselstrom erzeugt,
         der ebenso viele Zeichen besitzen muss wie der zu verschl"usselnde
         Klartext. 

         Als Basis zur Erzeugung des Schl"ussels wird ein gemischtes 
         Bridge-Kartenspiel mit 54 Karten (As, 2 - 10, Bube, Dame, K"onig
         in vier Farben + 2 Joker) benutzt. Der Kartenstapel wird dazu offen
         in die Hand genommen:
         \begin{enumerate}
            \item Der erste Joker wandert um eine Position nach hinten.
            \item Der zweite Joker wandert um zwei Positionen nach hinten.
	    \item Die Karten "uber dem ersten Joker werden mit den Karten unter 
               dem zweiten Joker vertauscht.
            \item Der Wert der untersten Karte (1 bis 53; Kreuz, Karo, Herz, 
               Pik; Joker = 53) wird notiert. Genau so viele Karten werden von
               oben abgez"ahlt und mit den "ubrigen Karten vertauscht, wobei
               die unterste Karte des Stapels liegen bleibt.
            \item Der Wert der obersten Karte wird notiert. Entsprechend viele 
               Karten werden von oben abgez"ahlt. 
            \item Der Wert der darauffolgenden Karte ist das erste 
               Schl"usselzeichen, Kreuz und Herz = 1 bis 13, Karo und Pik = 14 
               bis 26. Handelt es sich um einen Joker, wird wieder bei Schritt
              1 begonnen. 
	 \end{enumerate}
         Diese 6 Schritte werden f"ur jedes neue Schl"usselzeichen
         abgearbeitet. Dieser Vorgang dauert -- von Hand -- relativ lange
         (4 h f"ur 300 Zeichen, je nach "Ubung) und erfordert hohe
         Konzentration.
      
         Die Verschl"usselung erfolgt dann durch Addition mod 26. Sie geht im
         Vergleich zur Schl"usselstromerzeugung relativ rasch.\\

      \item {\bf Mirdek (Paul Crowley)} \cite{Crowley2000}: 
         Hierbei handelt es sich um ein relativ kompliziertes Verfahren, der
         Autor liefert aber anhand eines Beispiels eine sehr anschauliche 
         Erkl"arung.

      \item {\bf Playing Card Cipher (John Savard)} \cite{Savard1999}: 
         Dieses Verfahren verwendet ein bereits gemischtes Kartenspiel ohne 
         Joker, wobei das Mischen gesondert geregelt ist. Ein Schl"ussel wird 
         erzeugt ,indem:
         \begin{enumerate}
            \item Der Stapel liegt verdeckt vor dem Anwender, der solange 
               Karten aufdeckt und in einer Reihe ablegt, bis die Summe der 
               Werte gr"o"ser oder gleich 8 ist.
            \item Ist die letzte Karte ein Bube, eine Dame oder ein K"onig, 
               wird der Wert dieser Karte notiert, andernfalls notiert man die
               Summe der Werte der ausgelegten Karten (eine Zahl zwischen 8 und
               17). In einer zweiten Reihe wird nun die notierte Anzahl von 
               Karten ausgelegt.
            \item Die dann noch verbleibenden Karten werden reihenweise
               abgelegt und zwar in der ersten Reihe bis zur Position der
               niedrigsten Karte aus 2., in der zweiten Reihe bis zur Position
               der zweitniedrigsten Karte aus 2. usw. Sind 2 Karten gleich,
               ist rot niedriger zu bewerten als schwarz.
            \item Die in 3. ausgeteilten Karten werden spaltenweise
               eingesammelt und auf einem Stapel offen abgelegt
               (Spaltentransposition). 
               Dabei wird mit der Spalte unter der niedrigsten Karte
	       begonnen.(aufgedeckt)
            \item Die in 1. und 2. ausgeteilten Karten werden eingesammelt 
               (die letzte Karte wird zuerst entfernt).
            \item Der Stapel wird umgedreht, so dass die Karten verdeckt
               liegen. Im Anschluss werden die Schritte 1 bis 6 noch zweimal
               ausgef"uhrt.
         \end{enumerate}
         Um ein Schl"usselzeichen zu erzeugen, wird der Wert der ersten Karte,
         die nicht Bube, Dame oder K"onig ist, notiert und die entsprechende
         Anzahl Karten abgez"ahlt (auch der Wert dieser Karte muss zwischen 1
         und 10	liegen). Die gleichen Schritte werden ausgehend von der letzten
         Karte angewendet. Die Werte der beiden ausgew"ahlten Karten werden 
         addiert und die letzte Stelle dieser Summe ist das Schl"usselzeichen.
   \end{itemize}

\item {\bf VIC cipher} \cite{Savard1999}: 
   Dies ist ein extrem aufw"andiges, aber verh"altnism"a"sig sicheres 
   Papier- und Bleistiftverfahren. Es wurde von sowjetischen Spionen
   eingesetzt.
   Unter anderem musste der Anwender aus einem Datum, einem Satzanfang und 
   einer beliebigen f"unfstelligen Zahl nach bestimmten Regeln zehn
   Pseudozufallszahlen erzeugen. Bei der Verschl"usselung findet unter 
   anderem auch ein Straddling Checkerboard Verwendung. Eine genaue 
   Beschreibung der Vorgehensweise findet sich unter \cite{Savard1999}.

\end{itemize}



%------------------------------------------------------------------------------
\newpage
\begin{thebibliography}{99999}
\addcontentsline{toc}{subsection}{Literaturverzeichnis}

\bibitem[ACA2002]{ACA2002} \index{ACA 2002}
   American Cryptogram Association, \\
   {\em Length and Standards for all ACA Ciphers}, \\
   2002.\\
   \href{http://www.cryptogram.org/cdb/aca.info/aca.and.you/chap08.html#}
   {\texttt{http://www.cryptogram.org/cdb/aca.info/aca.and.you/chap08.html\#}}

\bibitem[Bauer1995]{Bauer1995} \index{Bauer 1995}
   Friedrich L. Bauer, \\
   {\em Entzifferte Geheimnisse}, Springer, 1995.

\bibitem[Bauer2000]{Bauer2000} \index{Bauer 2000}
   Friedrich L. Bauer, \\
   {\em Decrypted Secrets}, Springer 1997, 2nd edition 2000.

\bibitem[Crowley2000]{Crowley2000} \index{Crowley 2000}
   Paul Crowley, \\
   {\em Mirdek: A card cipher inspired by "`Solitaire"'}, \\
   2000.\\
   \href{http://www.ciphergoth.org/crypto/mirdek/}
   {\texttt{http://www.ciphergoth.org/crypto/mirdek/}}

\bibitem[DA1999]{DA1999} \index{DA 1999}
   Data encryption page des ThinkQuest Team 27158 f"ur ThinkQuest 1999 \\
   (Kein Update seit 1999, keine Suchm"oglichkeit), \\
   1999.\\
   \href{http://library.thinkquest.org/27158/}
   {\texttt{http://library.thinkquest.org/27158/}}

\bibitem[Goebel2003]{Goebel2003} \index{Goebel 2003}
   Greg Goebel, \\
   {\em Codes, Ciphers and Codebreaking}, \\
   2003.\\
   \href{http://www.vectorsite.net/ttcode.htm}
   {\texttt{http://www.vectorsite.net/ttcode.htm}}
	
\bibitem[Nichols1996]{Nichols1996} \index{Nichols 1996} 
    Randall K. Nichols, \\
    {\em Classical Cryptography Course, Volume 1 and 2}, \\
    Aegean Park Press 1996; \\
    oder in 12 Lektionen online unter \\
    \href{http://www.fortunecity.com/skyscraper/coding/379/lesson1.htm}
    {\texttt{http://www.fortunecity.com/skyscraper/coding/379/lesson1.htm}}

\bibitem[Savard1999]{Savard1999} \index{Savard 1999}
	John J. G. Savard, \\
	{\em A Cryptographic Compendium}, \\
	1999.\\
	\href{http://www.hypermaths.org/quadibloc/crypto/jscrypt.htm}
	{\texttt{http://www.hypermaths.org/quadibloc/crypto/jscrypt.htm}}
% manchmal schlecht zu erreichen, auch wenn man von seiner Homepage
% http://www.hypermaths.org/quadibloc/main.htm    aus startet.
% http://www.hypermaths.org/quadibloc/crypto/jscrypt.htm
	
\bibitem[Schmeh2004]{Schmeh2004}  \index{Schmeh 2004}
        Klaus Schmeh, \\
        {\em Die Welt der geheimen Zeichen. Die faszinierende Geschichte der Verschl"usselung},\\ 
        W3L Verlag Bochum, 1. Auflage 2004.

\bibitem[Schneier1999]{Schneier1999}
	Bruce Schneier, \\
	{\em The Solitaire Encryption Algorithm}, \\
	version 1.2, 1999.\\
	\href{http://www.schneier.com/solitaire.html}
	{\texttt{http://www.schneier.com/solitaire.html}}

\bibitem[Singh2001]{Singh2001} \index{Singh 2001}
	Simon Singh, \\
	{\em Geheime Botschaften. Die Kunst der Verschl"usselung von der 
        Antike bis in die Zeiten des Internet}, \\
	dtv, 2001.

\bibitem[ThinkQuest1999]{ThinkQuest1999} \index{ThinkQuest 1999}
	ThinkQuest Team 27158, \\
	{\em Data Encryption}, \\
	1999.\\
	\href{http://library.thinkquest.org/27158/}
	{\texttt{http://library.thinkquest.org/27158/} }

	
%xxxxxxxxxxx
%\bibitem[x]{x}  \index{x}
%        x, \\
%        {\em x}, 
%       x x.
%xxxxxxxxxxx

% ddddddddddddddddddddddddddddddddddddddddx



\end{thebibliography}

% Local Variables:
% TeX-master: "../script-de.tex"
% End:

