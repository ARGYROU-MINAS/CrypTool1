%\def\QM {{,\kern -0.9 pt ,}}
\setcounter{satz}{0}
\setcounter{definition}{0}

\newpage
\hypertarget{Kapitel_3}{}
\section{Einf"uhrung in die elementare Zahlentheorie mit Beispielen}
(Bernhard Esslinger, besslinger@web.de, Juli 2001, Updates: Nov. 2001, Juni/ 2002)
\begin{center}
\fbox{\parbox{15cm}{{\em \index{Berne} Eric Berne\footnotemark\ :}\newline
Die mathematische Spielanalyse postuliert Spieler, die rational reagieren.
Die transaktionale Spielanalyse dagegen befa"st sich mit Spielen, die
unrational, ja sogar {\bf irrational und damit wirklichkeitsn"aher} sind.}}
\end{center}
\addtocounter{footnote}{0}\footnotetext{Eric Berne, \glqq Spiele der Erwachsenen'', rororo, (c) 1964, S. 235.}\vskip +4pt

Diese \glqq Einf"uhrung\grqq~ bietet einen Einstieg f"ur mathematisch Interessierte.
Erforderlich sind nicht mehr Vorkenntnisse als die, die im Grundkurs Mathematik am Gymnasium vermittelt werden.\par
Wir haben uns bewu"st an \glqq Einsteigern\grqq~ und \glqq Interessenten\grqq~ orientiert, und nicht an den
Gepflogenheiten mathematischer Lehrb"ucher, die auch dann \glqq Einf"uhrung\grqq~ genannt werden,
wenn sie schon auf der 5. Seite nicht mehr auf Anhieb zu verstehen sind und sie eigentlich den Zweck haben, dass
man danach auch spezielle Monographien zu dem Thema lesen k"onnen soll.



% +++++++++++++++++++++++++++++++++++++++++++++++++++++++++++++++++++++++++++++++++++++++++++
\subsection{Mathematik und Kryptographie}
Ein gro"ser Teil der modernen, asymmetrischen Kryptographie beruht auf mathematischen 
Erkenntnissen -- auf den Eigenschaften (\glqq Gesetzen'') ganzer 
Zahlen, die in der elementaren \index{Zahlentheorie!elementare} Zahlentheorie untersucht werden. \glqq Elementar\grqq~ bedeutet hier, 
dass die zahlentheoretischen Fragestellungen im wesentlichen
in der Menge der nat"urlichen und der ganzen Zahlen durchgef"uhrt werden.

Weitere mathematische Disziplinen, die heute in der Kryptographie Verwendung finden, sind 
(vgl. \cite[S. 2]{Bauer1995}, \cite[Seite 3]{Bauer2000}) :
\begin{itemize}
    \item Gruppentheorie
    \item Kombinatorik
    \item Komplexit"atstheorie
    \item Ergodentheorie
    \item Informationstheorie.
\end{itemize}

Die Zahlentheorie oder Arithmetik (hier wird mehr der Aspekt des Rechnens
mit Zahlen betont) wurde von Carl Friedrich Gauss \index{Gauss} als besondere
mathematische Disziplin begr"undet. Zu ihren elementaren Gegenst"anden
geh"oren: gr"o"ster gemeinsamer Teiler\footnote{
Auf ggT\index{ggT}, englisch gcd (greatest common divisor), geht dieser Artikel im 
\hyperlink{Appendix_A}{Anhang A} ein.
} (ggT), Kongruenzen (Restklassen), Faktorisierung, Satz von Euler-Fermat und
primitive Wurzeln. Kernbegriff sind jedoch die Primzahlen und ihre
multiplikative Verkn"upfung.

Lange Zeit galt gerade die Zahlentheorie als Forschung pur, als
Paradebeispiel f"ur die Forschung im Elfenbeinturm. Sie erforschte die
\glqq geheimnisvollen Gesetze im Reich der Zahlen'' und gab Anla"s zu
philosophischen Er"orterungen, ob sie beschreibt, was "uberall in der Natur
schon da ist, oder ob sie ihre Elemente (Zahlen, Operatoren, Eigenschaften)
nicht k"unstlich konstruiert.

Inzwischen wei"s man, dass sich zahlentheoretische Muster "uberall in der Natur
finden. Zum Beispiel verhalten sich die Anzahl der links- und der
rechtsdrehenden Spiralen einer Sonnenblume zueinander wie zwei
aufeinanderfolgende\index{Fibonacci} Fibonacci-Zahlen\footnote{
Die Folge der Fibonacci-Zahlen $(a_i)_{i \in \mathbb{N}}$ ist definiert durch die \glqq rekursive'' 
Vorschrift $a_1 := a_2 := 1$ und f"ur alle Zahlen  $n=1,2,3,\cdots$ definiert man 
$a_{n+2} := a_{n+1}+a_n$.  Zu dieser historischen Folge gibt es viele interessante 
Anwendungen in der Natur (siehe z.B. \cite[S. 290 ff]{Graham1994}\index{Graham 1994} oder die Web-Seite 
von \hyperlink{knott}{Ron Knott:} \index{Knott Ron} hier dreht sich alles um Fibonacci-Zahlen). 
Die Fibonacci-Folge ist gut verstanden und wird heute als wichtiges Werkzeug in der 
Mathematik benutzt.
}, also z.B.  wie  $21 : 34$.

Au"serdem wurde sp"atestens mit den zahlentheoretischen Anwendungen der
modernen Kryptographie klar, dass eine jahrhundertelang als theoretisch
geltende Disziplin praktische Anwendung findet, nach deren Experten heute
eine hohe Nachfrage auf dem Arbeitsmarkt besteht.

Anwendungen der (Computer-)Sicherheit bedienen sich heute der Kryptographie,
weil Kryptographie als mathematische Disziplin einfach besser und
beweisbarer ist als alle im Laufe der Jahrhunderte erfundenen \glqq kreativen''
Verfahren der Substitution und besser als alle ausgefeilten physischen
Techniken wie beispielsweise beim Banknotendruck \cite[S. 4]{Beutelspacher1996}.

In diesem Artikel werden in einer leicht verst"andlichen Art die
grundlegenden Erkenntnisse der elementaren Zahlentheorie anhand vieler
Beispiele vorgestellt -- auf Beweise wird (fast) vollst"andig verzichtet
(diese finden sich in den mathematischen Lehrb"uchern).

Ziel ist nicht die umfassende Darstellung der zahlentheoretischen Erkenntnisse,
sondern das Aufzeigen der wesentlichen Vorgehensweisen. Der Umfang des Stoffes
orientiert sich daran, das RSA-Verfahren verstehen und anwenden zu k"onnen.

Dazu wird sowohl an Beispielen als auch in der Theorie erkl"art, wie man in
endlichen Mengen rechnet und wie dies in der Kryptographie Anwendung findet.
Insbesondere wird auf die klassischen Public Key-Verfahren Diffie-Hellman
(DH) und RSA eingegangen.

\vskip +40 pt

\begin{center}
\fbox{\parbox{15cm}{{\em Carl Friedrich Gauss \index{Gauss}(30.4.1777 - 23.2.1855):}\\
Die Mathematik ist die K"onigin der Wissenschaften, die Zahlentheorie aber
ist die K"onigin der Mathematik.}}
\end{center}

% ++++++++++++++++++++++++++++++++++++++++++++++++++++++++++++++++++++++++++++++++++++++++++++++
\subsection{Einf"uhrung in die Zahlentheorie}

\index{Zahlentheorie!Einf""uhrung} Die Zahlentheorie entstand aus Interesse an den positiven ganzen Zahlen $1,
2, 3, 4, \cdots ,$ die auch als die Menge der \index{Zahlen!nat""urliche} {\em nat"urlichen Zahlen} $\mathbb{N}$ bezeichnet
werden. Sie sind die ersten mathematischen Konstrukte der menschlichen
Zivilisation. Nach Kronecker hat sie der liebe Gott geschaffen, nach Dedekind der menschliche Geist.
Das ist je nach Weltanschauung ein unl"osbarer Widerspruch oder ein und dasselbe.

Im Altertum gab es keinen Unterschied zwischen Zahlentheorie und
Numerologie, die einzelnen Zahlen mystische Bedeutung zuma"s. So wie sich
w"ahrend der Renaissance (ab dem 14. Jahrhundert) die Astronomie allm"ahlich
von der Astrologie und die Chemie von der Alchemie l"oste, so lie"s auch die
Zahlentheorie die Numerologie hinter sich.

Die Zahlentheorie faszinierte schon immer Amateure wie auch professionelle
Mathematiker. Im Unterschied zu anderen Teilgebieten der Mathematik k"onnen
viele der Probleme und S"atze auch von Laien verstanden werden, andererseits
widersetzten sich die L"osungen zu den Problemen und die Beweise zu den
S"atzen oft sehr lange den Mathematikern. Es ist also leicht, gute Fragen zu
stellen, aber es ist ganz etwas anderes, die Antwort zu finden. Ein Beispiel
daf"ur ist der sogenannte letzte (oder gro"se) Satz von Fermat\footnote{
Thema der Schul-Mathematik ist der Satz von Pythagoras, wo gelehrt wird, dass
in einem rechtwinkligen Dreieck gilt: $a^2 + b^2 = c^2$, wobei $a, b$ die
Schenkell"angen sind und c die L"ange der Hypothenuse ist. Fermats ber"uhmte
Behauptung war, dass f"ur ganzzahlige Exponenten $n > 2$ immer die Ungleichheit
$a^n + b^n \not= c^n$ gilt. Leider fand \index{Fermat!letzter Satz} Fermat auf dem Brief, wo er die
Behauptung aufstellte, nicht gen"ugend Platz, um den Satz zu beweisen. Der
Satz konnte erst "uber 300 Jahre sp"ater bewiesen werden \cite[S. 433-551]{Wiles1994}. \index{Wiles}
}.

Bis zur Mitte des 20. Jahrhunderts wurde die Zahlentheorie als das reinste
Teilgebiet der Mathematik angesehen -- ohne Verwendung in der wirklichen Welt.
Mit dem Aufkommen der Computer und der digitalen Kommunikation "anderte sich
das: die Zahlentheorie konnte einige unerwartete Antworten f"ur reale
Aufgabenstellungen liefern. Gleichzeitig halfen die Fortschritte in der EDV,
dass die Zahlentheoretiker gro"se Fortschritte machten im Faktorisieren gro"ser
Zahlen, in der Bestimmung neuer Primzahlen, im Testen von (alten)
Vermutungen und beim L"osen bisher unl"osbarer numerischer Probleme.

Die moderne Zahlentheorie \index{Zahlentheorie!moderne} besteht aus Teilgebieten wie
\begin{itemize}
    \item Elementare Zahlentheorie
    \item Algebraische Zahlentheorie
    \item Analytische Zahlentheorie
    \item Geometrische Zahlentheorie
    \item Kombinatorische Zahlentheorie
    \item Numerische Zahlentheorie und
    \item Wahrscheinlichkeitstheorie.
\end{itemize}

Die verschiedenen Teilgebiete besch"aftigen sich alle mit Fragestellungen zu
den ganzen Zahlen (positive und negative ganze Zahlen und die Null), gehen
diese jedoch mit verschiedenen Methoden an.

Dieser Artikel besch"aftigt sich nur mit dem Teilgebiet der elementaren
Zahlentheorie.
\newpage

% ...................................................................................................
\subsubsection{Konvention}
Wird nichts anderes gesagt, gilt: 
\begin{itemize}
\item Die Buchstaben $a, b, c, d, e, k, n, m, p, q$ stehen f"ur ganze Zahlen.
\item Die Buchstaben $i$ und $j$ stehen f"ur nat"urliche Zahlen. 
\item Der Buchstabe $p$ steht stets f"ur eine Primzahl.
\item Die Mengen $\mathbb{N} = \{ 1, 2, 3, \cdots \}$ und $\mathbb{Z} =\{ \cdots, -3, -2, -1, 0, 1, 2, 3, \cdots \}$ 
sind die {\em nat"urlichen} und die {\em ganzen} Zahlen.
\end{itemize}

%\vskip +40 pt
\newpage

\begin{center}
\fbox{\parbox{15cm}{{\em Joanne K. Rowling\index{Rowling}\footnotemark\:}\newline Das ist nicht Zauberei, das ist Logik, ein R"atsel.
Viele von den gr"o"sten Zauberern haben keine Unze Logik im Kopf.}}
\end{center}
\addtocounter{footnote}{0}\footnotetext{Joanne K. Rowling,~\glqq Harry Potter und der Stein der Weisen'', Carlsen, (c)
1997, Kapitel ~ \glqq Durch die Fallt"ur'', S. 310, Hermine.}


% +++++++++++++++++++++++++++++++++++++++++++++++++++++++++++++++++++++++++++++++++++++++++++++++++++
\subsection{Primzahlen und der erste Hauptsatz der elementaren Zahlentheorie}
\index{Zahlentheorie!elementare} Viele der Probleme in der elementaren Zahlentheorie besch"aftigen sich mit
Primzahlen.

Jede ganze Zahl hat Teiler oder Faktoren. Die Zahl 1 hat nur einen, n"amlich
sich selbst. Die Zahl 12 hat die sechs Teiler 1, 2, 3, 4, 6 und 12\footnote{
Aufgrund der gro"sen Teilerzahl von 12 findet sich diese Zahl -- und Vielfache dieser Zahl -- oft im allt"aglichen wieder:
Die 12 Stunden-Skala der Uhr, die 60 Minuten einer Stunde, die 360 Grad-Skala der Winkelmessung, usw. Teilt man 
diese Skalen in Bruchteile auf, so ergeben sich in vielen F"allen die Br"uche als ganze Zahlen. Mit diesen kann 
man im Kopf einfacher rechnen als mit gebrochenen Zahlen.
}. Viele Zahlen sind nur teilbar durch sich selbst und durch 1. Bez"uglich der
Multiplikation sind dies die \glqq Atome'' im Bereich der Zahlen.

\index{Primzahlen}
\begin{definition}\label{def-zth-prime} 
{\bf Primzahlen} sind nat"urliche Zahlen gr"o"ser als $1$, die nur durch $1$ und sich
selbst teilbar sind.
\end{definition}

Per Definition ist $1$ keine Primzahl.

Schreibt man die Primzahlen in aufsteigender Folge (Primzahlenfolge), so
ergibt sich
$$2,~ 3,~ 5,~ 7,~ 11, ~13,~ 17,~ 19, ~23, ~29, ~31, ~37,~ 41,~ 43,~ 47,~ 53, ~59, ~61, ~67, ~71,
~73, ~79, ~83, ~89, ~97, \cdots.$$

Unter den ersten $100$ Zahlen gibt es genau $25$ Primzahlen. Danach nimmt ihr
prozentualer Anteil ab, wird aber nie Null.

Primzahlen treten als ganze Zahlen nicht selten auf. Allein im letzten
Jahrzehnt waren drei Jahre prim: $1993, 1997$ und $1999$. W"aren sie selten,
k"onnte die Kryptographie auch nicht so mit ihnen arbeiten, wie sie es tut.

Primzahlen k"onnen nur auf eine einzige (\glqq triviale'') Weise zerlegt werden:
\begin{eqnarray}
5 & = & 1 * 5 \nonumber \\
17 & =  & 1 * 17 \nonumber \\
1.013 &  = & 1 * 1.013 \nonumber \\
1.296.409 & = & 1 * 1.296.409. \nonumber
\end{eqnarray}

\index{Zahlen!zusammengesetzte}
\begin{definition}\label{def-zth-composite} 
Nat"urliche Zahlen gr"o"ser $1$, die keine Primzahlen sind, hei"sen
{\bf zusammengesetzte Zahlen}: diese haben mindestens zwei von $1$ verschiedene
Faktoren.
\end{definition}


Beispiele f"ur die Primfaktorzerlegung solcher Zahlen:
\begin{eqnarray}
4 & = & 2*2  \nonumber \\
6 & = & 2*3  \nonumber \\
91 & = & 7*13  \nonumber \\
161 & = & 7*23  \nonumber \\
767 & = & 13*59  \nonumber \\
1.029 & = & 3 * 7^3  \nonumber \\
5.324 & = & 22 * 11^3.  \nonumber 
\end{eqnarray}

\begin{satz}\label{thm-zth-cnum}
Jede zusammengesetzte Zahl $a$ besitzt einen kleinsten Teiler gr"o"ser
als $1$. Dieser Teiler ist eine Primzahl $p$ und kleiner oder gleich der Quadratwurzel
aus $a$.
\end{satz}

Aus den Primzahlen lassen sich alle ganzen Zahlen gr"o"ser als $1$
zusammensetzen -- und das sogar in einer {\em eindeutigen} Weise.

Dies besagt \index{Zahlentheorie!Hauptsatz} der 1. {\em Hauptsatz der Zahlentheorie} (= Hauptsatz der elementaren
Zahlentheorie = fundamental theorem of arithmetic = fundamental building
block of all positive integers). Er wurde das erste Mal pr"azise von Carl
Friedrich Gauss in seinen Disquisitiones Arithmeticae (1801) formuliert.  \index{Zahlentheorie!Hauptsatz}  \index{Gauss}

\begin{satz}{\bf Gauss 1801}\label{thm-zth-mthm}
Jede nat"urliche Zahl $a$ gr"o"ser als $1$ l"a"st sich als Produkt von
Primzahlen schreiben. Sind zwei solche Zerlegungen $a = p_1*p_2*\cdots*p_n = q_1*q_2*\cdots*q_m$ gegeben, dann
gilt nach eventuellem Umsortieren $n = m$ und f"ur alle $i$: $p_i = q_i$.
\end{satz}

In anderen Worten: Jede nat"urliche Zahl au"ser der $1$ l"a"st sich auf genau eine
Weise als Produkt von Primzahlen schreiben, wenn man von der Reihenfolge der
Faktoren absieht. Die Faktoren sind also eindeutig (die \glqq Expansion in
Faktoren'' ist eindeutig)!

Zum Beispiel ist $60 = 2*2*3*5 = 2^2*3*5$. Und das ist --- bis auf eine
ver"anderte Reihenfolge der Faktoren --- die einzige M"oglichkeit, die Zahl $60$
in Primfaktoren zu zerlegen.

Wenn man nicht nur Primzahlen als Faktoren zul"a"st, gibt es mehrere
M"oglichkeiten der Zerlegung in Faktoren und die {\em Eindeutigkeit} (uniqueness)
geht verloren:
$$60 = 1*60 = 2*30 = 4*15 = 5*12 = 6*10 = 2*3*10 = 2*5*6 = 3*4*5 = \cdots$$
Der 1. Hauptsatz ist nur scheinbar selbstverst"andlich. Man kann viele andere
Zahlenmengen\footnote{
Diese Mengen werden speziell aus der Menge der nat"urlichen Zahlen gebildet.
Ein Beispiel findet sich in diesem \hyperlink{uniqueness}{Skript}
auf Seite \pageref{remFundTheoOfArithm} am Ende von Kapitel 2.1.
} konstruieren, bei denen eine multiplikative Zerlegung in die Primfaktoren
dieser Mengen {\em nicht} eindeutig ist.

F"ur eine mathematische Aussage ist es deshalb nicht nur wichtig, f"ur welche
Operation sie definiert wird, sondern auch auf welcher Grundmenge diese
Operation definiert wird.

Weitere Details zu den Primzahlen (z.B. wie der \glqq Kleine Satz von Fermat'' zum
Testen sehr gro"ser Zahlen auf ihre Primzahleigenschaft benutzt werden kann)
finden sich in diesem Skript in dem Artikel "uber \hyperlink{Kapitel_2}{Primzahlen}.


% +++++++++++++++++++++++++++++++++++++++++++++++++++++++++++++++++++++++++++++++++++++++++++++++++++++++
\subsection{Teilbarkeit, Modulus und Restklassen}
\index{Modulus} \index{Teilbarkeit} Werden ganze Zahlen addiert, subtrahiert oder multipliziert, ist das
Ergebnis stets wieder eine ganze Zahl.

Die Division zweier ganzer Zahlen ergibt nicht immer eine ganze Zahl. Wenn
man z.B. $158$ durch $10$ teilt, ist das Ergebnis die Dezimalzahl $15,8$. Dies ist
keine ganze Zahl!

Teilt man $158$ dagegen durch $2$, ist das Ergebnis $79$ eine ganze Zahl.
In der Zahlentheorie sagt man, $158$ ist {\em teilbar} durch $2$, aber nicht durch $10$.
Allgemein sagt man:

\begin{definition}\label{def-zth-divisibility} \index{Teilbarkeit} \index{teilbar}
Eine ganze Zahl $n$ ist {\bf teilbar} durch eine ganze Zahl $d$, wenn der Quotient $n/d$
eine ganze Zahl $c$ ist, so dass $n = c * d$.
\end{definition}

Die Zahl $n$ wird {\em Vielfaches} von $d$ genannt; $d$ wird {\em Teiler, Divisor} \index{Teiler} \index{Divisor} oder \index{Faktor} {\em Faktor} von $n$
genannt.

Mathematisch schreibt man das: $d | n$ (gelesen: \glqq  $d$ teilt $n$'').
Die Schreibweise $d \!\!\not| n$ bedeutet, dass $d$ die Zahl $n$ nicht teilt.

Also gilt in unserem obigen Beispiel: $10\!\!\not| 158$, aber $2 | 158$.

% ........................................................................................................
\subsubsection{Die Modulo-Operation -- Rechnen mit Kongruenzen} \index{Kongruenz}

Bei Teilbarkeitsuntersuchungen kommt es nur auf die Reste der Division an:
Teilt man eine Zahl $n$ durch $m$, so benutzt man oft die folgende Schreibweise:
$$\frac{n}{m} = c + \frac{r}{m} ,$$
wobei $c$ eine ganze Zahl ist und $r$ eine Zahl mit den Werten $0,1,\cdots, m-1$.
Diese Schreibweise hei"st Division mit Rest. Dabei hei"st $c$ der ganzzahlige 
\glqq Quotient'' und $r$ der \glqq Rest'' der Division.

{\bf Beispiel:} \\
$$\frac{19}{7} = 2 + \frac{5}{7} \quad (m=7, c = 2, r = 5)$$

Was haben die Zahlen $5, 12, 19, 26, \cdots$ bei der Division durch $7$ gemeinsam?
Es ergibt sich immer der Rest $r = 5$.
Bei der Division durch $7$ sind nur die folgenden Reste m"oglich:
$$r = 0, 1, 2, \cdots, 6.$$

Wir fassen bei der Division durch $7$ die Zahlen, die den gleichen Rest $r$
ergeben, in die \glqq Restklasse $r$ modulo $7$'' zusammen. Zwei Zahlen $a$ und $b$, die
zur gleichen Restklasse modulo $7$ geh"oren, bezeichnen wir als \glqq kongruent
modulo 7''. Oder ganz allgemein:

\begin{definition}\label{def-zth-remainder} \index{Restklasse}
Als {\bf Restklasse $r$ modulo $m$} bezeichnet man alle ganzen Zahlen $a$, die bei der Division 
durch $m$ denselben Rest $r$ haben.
\end{definition}
\newpage
{\bf Beispiele:}
\begin{itemize}
\item[] Restklasse $0$ modulo $4 = \{ x | x = 4*n; \; n \in \mathbb{N} \} = \{ \dots, -16, -12, -8, -4, 0, 4, 8, 12, 16, \dots \}$
\item[] Restklasse $3$ modulo $4 = \{ x | x = 4*n + 3;\; n \in \mathbb{N} \} = \{ \dots, -13, -9, -5, -1, 3, 7, 11, 15, \dots \}$
\end{itemize}
Da modulo $m$ nur die Reste $0, 1, 2, \cdots, m-1$ m"oglich sind, rechnet die modulare Arithmetik in endlichen Mengen. 
Zu jedem Modul $m$ gibt es genau $m$ Restklassen.

\begin{definition}\label{def-zth-congruence} \index{Kongruenz}
Zwei Zahlen $a, b \in \mathbb{N}$  hei"sen \index{restgleich} \index{Kongruent}
{\bf restgleich oder kongruent bez"uglich $m \in \mathbb{N}$}  genau dann, 
wenn beim Teilen durch $m$ der gleiche Rest bleibt.
\end{definition}

Man schreibt: $a \equiv b {\rm ~(mod~} m)$. Und sagt:  {\em $a$ ist kongruent $b$ modulo $m$}. Das bedeutet, 
dass $a$ und $b$ zur gleichen Restklasse geh"oren. Der Modul ist also der Teiler. Diese Schreibweise wurde von 
Gauss eingef"uhrt. Gew"ohnlich ist der Teiler positiv, aber $a$ und $b$ k"onnen auch beliebige ganze Zahlen sein.

{\bf Beispiele:}
\begin{itemize}
   \item[] $19 \equiv 12 {\rm ~(mod~} 7)$,         
           denn die Reste sind gleich:  $19 / 7 = 2$ Rest $5$  und  $12 / 7 = 1$ Rest $5$.
   \item[] $23103 \equiv 0 {\rm ~(mod~} 453)$, denn $23103 / 453 = 51$ Rest $0$  und  $0 / 453 = 0$ Rest $0$.
\end{itemize}

\begin{satz}\label{thm-zth-div}
$a \equiv b$ (mod $m$) gilt genau dann,  wenn die Differenz $(a - b)$ durch $m$ teilbar ist, also wenn 
ein $q\in \mathbf{Z}$ existiert mit $ (a-b)=q*m.$
\end{satz}
Diese beiden Aussagen sind also "aquivalent.

Daraus ergibt sich: Wenn $m$ die Differenz teilt, gibt es eine ganze Zahl $q$, so dass gilt: $a = b + q*m$.
Alternativ zur Kongruenzschreibweise kann man auch die Teilbarkeitsschreibweise verwenden: $m | (a - b)$.

{\bf Beispiel "aquivalenter Aussagen:} \\
$35 \equiv 11$ (mod $3) \Longleftrightarrow  35 - 11 \equiv 0$ (mod $3)$, 
wobei $35 - 11 = 24$ sich ohne Rest durch $3$ teilen l"a"st, w"ahrend $35:3$ und $11:3$ beide den Rest $2$ ergeben.

{\bf Bemerkung:}\\
F"ur die Summe $(a + b)$ gilt die obige "Aquivalenz nicht!

{\bf Beispiel: }\\
$11 \equiv 2$ (mod $3$), also ist $11 - 2 \equiv 9 \equiv 0$ (mod $3$); aber $11 + 2 = 13$ ist nicht durch $3$ teilbar.
Die Aussage von Satz \ref{thm-zth-div} gilt f"ur Summen nicht einmal in eine
Richtung. Richtig ist sie bei Summen nur f"ur den Rest $0$ und nur in der
folgenden Richtung: Teilt ein Teiler beide Summanden ohne Rest, teilt er
auch die Summe ohne Rest.

Anwenden kann man die obige "Aquivalenz von Satz \ref{thm-zth-div}, wenn man schnell und
geschickt f"ur gro"se Zahlen entscheiden will, ob sie durch eine bestimmte
Zahl teilbar sind.

\newpage
{\bf Beispiel:} \\
Ist $69.993$ durch $7$ teilbar? \\
Da die Zahl in eine Differenz zerlegt werden kann, wo einfach zu ersehen
ist, dass jeder Operand durch $7$ teilbar ist, ist auch die Differenz durch $7$
teilbar: $69.993 = 70.000 - 7$.

Diese "Uberlegungen und Definitionen m"ogen recht theoretisch erscheinen, sind
uns im Alltag aber so vertraut, dass wir die formale Vorgehensweise gar nicht
mehr wahrnehmen: Bei der Uhr werden die $24$ h eines Tages durch die Zahlen $1$,
$2, \cdots, 12$ repr"asentiert. Die Stunden nach $12:00$ mittags erh"alt man als
Reste einer Division durch 12. Wir wissen sofort, dass $2$ Uhr nachmittags
dasselbe wie $14:00$ ist.

Diese \glqq modulare'', also auf die Divisionsreste bezogene Arithmetik ist die
Basis der asymmetrischen Verschl"usselungsverfahren.
Kryptographische Berechnungen spielen sich also nicht wie das Schulrechnen
unter den reellen Zahlen ab, sondern unter Zeichenketten begrenzter L"ange,
das hei"st unter positiven ganzen Zahlen, die einen gewissen Wert nicht
"uberschreiten d"urfen.
Aus diesem und anderen Gr"unden w"ahlt man sich eine gro"se Zahl $m$ und \glqq rechnet
modulo $m$'', das hei"st, man ignoriert ganzzahlige Vielfache von $m$ und rechnet
statt mit einer Zahl nur mit dem Rest bei Division dieser Zahl durch $m$.
Dadurch bleiben alle Ergebnisse im Bereich von $0$ bis $m-1$.

% ++++++++++++++++++++++++++++++++++++++++++++++++++++++++++++++++++++++++++++++++++++++++++++++++++++++++++++
\subsection{Rechnen in endlichen Mengen}

\subsubsection{Gesetze beim modularen Rechnen}

Aus S"atzen der Algebra folgt, dass wesentliche Teile der "ublichen
Rechenregeln beim "Ubergang zum modularen Rechnen "uber der Grundmenge $\mathbb{Z}$
erhalten bleiben: Die Addition ist nach wie vor kommutativ. Gleiches gilt f"ur die
Multiplikation modulo $m$. Das Ergebnis einer Division\footnote{
\label{ftn-res-divmodn}Die Division modulo $m$ ist nur f"ur Zahlen, die teilerfremd zu $m$ sind definiert. 
Vergleiche Kapitel \ref{multmodn}.
} ist kein Bruch, sondern eine ganze Zahl zwischen $0$ und $m-1$.

Es gelten die bekannten Gesetze:
\begin{itemize}
\item[\bf 1.] {\bf Assoziativgesetz:}\index{Assoziativgesetz} \\ 
    $((a+b) + c) {\rm ~(mod~ } m) \equiv  (a + (b+c)) {\rm ~(mod~ } m).$ \\
    $((a*b) * c) {\rm ~(mod~ } m) \equiv  (a * (b*c)) {\rm ~(mod~ } m).$
\item[\bf 2.] {\bf Kommutativgesetz:} \index{Kommutativgesetz}\\
    $(a+b) {\rm ~(mod~ } m) \equiv  (b+a) {\rm ~(mod~ } m).$ \\
     $(a*b) {\rm ~(mod~ } m) \equiv  (b*a) {\rm ~(mod~ } m).$
\end{itemize}
Assoziativgesetz und Kommutativgesetz gelten sowohl f"ur die Addition als auch f"ur die Multiplikation.
\begin{itemize}
\item[\bf 3.] {\bf Distributivgesetz:} \index{Distributivgesetz}\\
    $ (a * (b+c)) {\rm ~(mod~ } m) \equiv  (a*b + a*c) {\rm ~(mod~ } m).$ 
\item[\bf 4.] {\bf Reduzierbarkeit:} \index{Reduzierbarkeit} \\
    $(a+b) {\rm ~(mod~} m) \equiv  (a {\rm ~(mod~ } m) + b {\rm ~(mod~ } m)) {\rm ~(mod~} m).$ \\  
    $(a*b) {\rm ~(mod~} m) \equiv  (a {\rm ~(mod~ } m) * b {\rm ~(mod~ } m)) {\rm ~(mod~} m).$
\end{itemize}
Es ist gleichg"ultig, in welcher Reihenfolge die Modulo-Operation durchgef"uhrt wird.
\begin{itemize}
\item[\bf 5.] {\bf Existenz einer Identit"at (neutrales Element):} \index{Identit""at}\\
    $(a + 0) {\rm ~(mod~ } m) \equiv  (0 + a) {\rm ~(mod~ } m) \equiv  a {\rm ~(mod~ } m).$  \\
    $(a * 1) {\rm ~(mod~ } m) \equiv  (1 * a) {\rm ~(mod~ } m) \equiv  a {\rm ~(mod~ } m).$
\item[\bf 6.] {\bf Existenz des inversen Elements:} \\
    F"ur jedes ganzzahlige $a$ und $m$ gibt es eine ganze Zahl $-a$, so dass gilt: \\
    $(a + (-a)) {\rm ~(mod~}m) \equiv  0 {\rm ~(mod~ } m)$ \quad (additive Inverse). \index{Inverse!additive}\\
    F"ur jedes $a$ ($a \not\equiv 0 {\rm ~(mod~ } p$) ) und $p$ prim gibt es eine ganze Zahl $a^{-1}$, so dass gilt: \\
    $(a * a^{-1}) {\rm ~(mod~ } p) \equiv 1 {\rm ~(mod~}p)$ \quad (multiplikative Inverse). \index{Inverse!multiplikative}
\item[\bf 7.] \index{Abgeschlossenheit} {\bf Abgeschlossenheit:}\footnote{
\label{ftn-closed}Diese Eigenschaft wird innerhalb einer Menge immer bez"uglich einer Operation definiert. 
Siehe \hyperlink{Appendix_B}{Anhang B}.
} \\
$a, b \in G  \Longrightarrow  ( a + b ) \in G.$ \\
$a, b \in G  \Longrightarrow  ( a * b ) \in G.$
\item[\bf 8.] \index{Transitivit""at} {\bf Transitivit"at:}\\
$ [ a \equiv b {\rm ~mod~ } m, ~b \equiv c {\rm ~mod~ } m] \Longrightarrow [ a \equiv c {\rm ~mod~ } m].
$
\end{itemize}

% +++++++++++++++++++++++++++++++++++++++++++++++++++++++++++++++++++++++++++++++++++++++++++++++++++++++++++++
\subsubsection{Muster und Strukturen} \index{Struktur}
\hypertarget{Kapitel_3_5_2}{}

Generell untersuchen die Mathematiker \glqq Strukturen\grqq. Sie fragen sich z.B. bei $ a * x
\equiv b {\rm ~mod~ } m, $ welche Werte $x$ f"ur gegebene Werte $a,b,m$ annehmen kann.

Insbesondere wird dies untersucht f"ur den Fall, dass das Ergebnis $b$ der Operation 
das neutrale Element ist. Dann ist $x$ die Inverse von $a$ bez"uglich dieser Operation.

% +++++++++++++++++++++++++++++++++++++++++++++++++++++++++++++++++++++++++++++++++++++++++++++++++++++++++++++
\subsection{Beispiele f"ur modulares Rechnen}

Wir haben bisher gesehen:

F"ur zwei nat"urliche Zahlen $a$ und $m$ bezeichnet  $a$ mod $m$  den Rest, den man erh"alt, 
wenn man $a$ durch $m$ teilt. Daher ist $a {\rm ~(mod~ } m$) stets eine Zahl zwischen $0$ und $m-1$.

Zum Beispiel gilt: $1 \equiv  6  \equiv  41 {\rm ~(mod~ } 5)$, denn der Rest ist jeweils $1$.

Ein anderes Beispiel ist: $2000  \equiv  0 {\rm ~(mod~ } 4)$, denn $4$ geht in $2000$ ohne Rest auf.

In der modularen Arithmetik gibt es nur eine eingeschr"ankte Menge
nicht-negativer Zahlen. Deren Anzahl wird durch einen Modul $m$ vorgegeben.
Ist der Modul $m = 5$, werden nur die 5 Zahlen der Menge $\{ 0, 1, 2, 3, 4\}$ benutzt.

Ein Rechenergebnis gr"o"ser als $4$ wird dann \glqq modulo'' $5$ umgeformt, d.h. es ist
der Rest, der sich bei der Division des Ergebnisses durch $5$ ergibt. So ist
etwa $2*4 \equiv 8 \equiv 3 {\rm ~(mod~ } 5)$, da $3$ der Rest ist, wenn man $8$ durch $5$ teilt.

% .............................................................................................................
\subsubsection{Addition und Multiplikation} \index{Addition} \index{Multiplikation}

Im folgenden werden 
\begin{itemize}
\item die Additionstabelle\footnote{
      Bemerkung zur Subtraktion modulo 5: \\
      $2 - 4 \equiv -2 \equiv 3{\rm ~mod~}5.$\\
      Es gilt modulo $5$ also nicht, dass $-2 = 2$ (siehe auch \hyperlink{Appendix_C}{Anhang C}). 
      } f"ur ${\rm mod~ } 5$ und
\item die Multiplikationstabellen\footnote{
      Bemerkung zur Division modulo 6 :

      Bei der Division darf nicht durch die Null geteilt werden, dies liegt an der
      besonderen Rolle der $0$ als Identit"at bei der Addition:\\
      f"ur alle $a$ gilt $a*0=0, $ denn $a*0 = a*(0+0) =a*0 + a*0.$ Es ist offensischtlich, dass $0$ keine Inverse
      bzgl. der Multiplikation besitzt, denn sonst m"u"ste gelten $0 = 0 * 0^{-1} = 1.$ Vergleiche 
      Fu"snote \ref{ftn-res-divmodn}.
      } f"ur mod $5$ und f"ur mod $6$
\end{itemize}
aufgestellt.

\subsubsection*{Beispiel Additionstabelle:}
Das Ergebnis der Addition von $3$ und $4 {\rm ~(mod~ } 5)$ wird folgenderma"sen bestimmt:
berechne $3 + 4 = 7$ und ziehe solange die $5$ vom Ergebnis ab, bis sich ein
Ergebnis kleiner als der Modul ergibt: $7 - 5 = 2$. Also ist: $3 + 4 \equiv 2 {\rm ~(mod~ } 5)$.

\begin{table}[h]
\begin{center}
\begin{tabular}{r|ccccc}
Additionstabelle modulo 5: \quad + &  0 & 1 & 2 & 3 & 4  \\
\hline
0 &  0 & 1 & 2 & 3 & 4 \\  
1 & 1 &  2 & 3 & 4 & 0 \\
2 & 2 & 3 & 4 & 0 & 1 \\
3 & 3 & 4 & 0 & 1 & 2 \\
4 & 4 & 0 & 1 & 2 & 3 
\end{tabular} 
\end{center} 
\end{table}

% ..............................................................................................
\subsubsection*{Beispiel Multiplikationstabelle:}
Das Ergebnis der Multiplikation $4 * 4 {\rm ~(mod~ } 5)$ wird folgenderma"sen berechnet: berechne $ 4*4=16$ und
ziehe solange die $5$ ab, bis sich ein Ergebnis kleiner als der Modul ergibt:
$$16 - 5 = 11;~ 11 - 5 = 6;~6- 5 = 1.$$
Direkt ergibt es sich auch aus der Tabelle: $4 * 4 \equiv 1 {\rm ~(mod~} 5)$, weil $16 : 5 = 3$ Rest $1$.
Die Multiplikation wird auf der Menge $\mathbb{Z}$ ohne $0$ definiert.

        
\begin{table}[h]
\begin{center}
\begin{tabular}{r|cccc}
Multiplikationstabelle modulo 5: \quad * & 1& 2 & 3 & 4  \\
\hline 
1 & 1 &    2    &    3    & 4 \\
2 & 2 & {\bf 4} & {\bf 1} & 3 \\ 
3 & 3 & {\bf 1} & {\bf 4} & 2  \\
4 & 4 &    3    &    2    & 1 
\end{tabular}
\end{center} 
\end{table}

% ..........................................................................................
\newpage
\subsubsection{Additive und multiplikative Inversen} \label{multmodn} \index{Inverse!additive} \index{Inverse!multiplikative}

Aus den Tabellen kann man zu jeder Zahl die Inversen bez"uglich der Addition
und der Multiplikation ablesen.

Die Inverse einer Zahl ist diejenige Zahl, die bei Addition der beiden
Zahlen das Ergebnis $0$ und bei der Multiplikation das Ergebnis $1$ ergibt. So
ist die Inverse von $4$ f"ur die Addition mod $5$ die $1$ und f"ur die
Multiplikation mod $5$ die $4$ selbst, denn
\begin{alignat}{2}
4 + 1 &  =  & 5 & \equiv 0 {\rm ~(mod~ } 5); \nonumber \\
4 * 4 &  = & ~16 & \equiv 1 {\rm ~(mod~ } 5). \nonumber
\end{alignat}
Die Inverse von $1$ bei der Multiplikation mod $5$ ist $1$; die Inverse modulo $5$
von $2$ ist $3$ und weil die Multiplikation kommutativ ist, ist die Inverse von
$3$ wiederum die $2$.

Wenn man zu einer beliebigen Zahl (hier $2$) eine weitere beliebige Zahl (hier $4$) addiert bzw.
multipliziert und danach zum Zwischenergebnis ($1$ bzw. $3$)
die jeweilige Inverse der weiteren Zahl ($1$ bzw. $4$) 
addiert\footnote{
Allgemein: $x + y + (-y) \equiv x{\rm ~(mod~}m)$ [$(-y)$ = additive Inverse zu $y{\rm ~(mod~}m)$]
} bzw. multipliziert,
ist das Gesamtergebnis gleich dem Ausgangswert.

{\bf Beispiele:}
\begin{eqnarray}
2 + 4 \equiv 6 \equiv 1 {\rm ~(mod~ } 5) ; \quad 1 + 1 \equiv 2 \equiv 2 {\rm ~(mod~ } 5)  \nonumber \\
2 * 4 \equiv 8 \equiv 3 {\rm ~(mod~ } 5) ; \quad 3 * 4 \equiv 12 \equiv 2 {\rm ~(mod~ } 5) \nonumber
\end{eqnarray}

In der Menge $\mathbb{Z}_5 = \{0, 1, 2, 3, 4\}$ f"ur die Addition und in der Menge $\mathbb{Z}_5 \setminus \{ 0\}$  f"ur
die Multiplikation haben alle Zahlen eine {\bf eindeutige} Inverse
bez"uglich modulo $5$.

Bei der modularen Addition ist das f"ur jeden Modul (also nicht nur f"ur $5$)
so.

Bei der modularen Multiplikation dagegen ist das nicht so:
\begin{satz}\label{thm-zth-multinv}
F"ur eine nat"urliche Zahl $a$ aus der Menge $\{1, \cdots, m-1\}$ gibt es genau dann eine\index{ggT}
multiplikative Inverse, wenn sie mit dem Modul $m$ \index{teilerfremd}
teilerfremd\footnote{
Es gilt: Zwei ganze Zahlen $a$ und $b$ sind genau dann teilerfremd, wenn ${\rm ggT}(a, b) = 1$.\\
Desweiteren gilt: Ist $p$ prim und $a$ eine beliebige ganze Zahl, die kein
Vielfaches von $p$ ist, so sind beide Zahlen teilerfremd.\\
Weitere Bezeichnungen zum Thema Teilerfremdheit (mit $a_i \in \mathbb{Z}, i=1, \cdots, n$):
\begin{enumerate}
\item $a_1,a_2, \cdots, a_n$ hei"sen {\em relativ prim} \index{relativ prim}, wenn $ {\rm~ggT}(a_1, \cdots , a_n) =1.$
\item F"ur mehr als $2$ Zahlen  ist eine noch st"arkere Anforderung:\\
      $a_1, \cdots , a_n$ hei"sen {\em paarweise relativ prim}, wenn f"ur alle $i=1, \cdots, n$ und 
      $j=1, \cdots , n$ mit $ i \neq j $ gilt: $ {\rm~ggT} (a_i, a_j) =1. $
\end{enumerate}
Beispiel: $2,3,6 $ sind relativ prim, da $ {\rm~ggT} (2,3,6)=1.$ 
Sie sind nicht paarweise prim, da $ {\rm~ggT} (2,6)=2>1.$
} ist, d.h. wenn $a$ und $m$ keine gemeinsamen Primfaktoren haben.
\end{satz}

Da $m=5$ eine Primzahl ist, sind die Zahlen $1$ bis $4$ teilerfremd zu $5$, und es
gibt mod $5$ zu {\bf jeder} dieser Zahlen eine multiplikative Inverse.

Ein Gegenbeispiel zeigt die Multiplikationstabelle f"ur mod $6$ (da der Modul $6$
nicht prim ist, sind nicht alle Elemente aus $\mathbb{Z}_6\setminus \{0\}$ zu $6$ teilerfremd):

\begin{table}[h]
\begin{center}
\begin{tabular}{r|ccccc}
Multiplikationstabelle modulo $6$: \quad* &  1 & 2 & 3 & 4 & 5  \\
\hline 
1 &  1 & 2 & 3 & 4 & 5 \\  
2 &  2 & {\bf 4} & {\bf 0} & {\bf 2} & 4 \\
3 &  3 & {\bf 0} & {\bf 3} & {\bf 0} & 3 \\
4 &  4 & {\bf 2} & {\bf 0} & {\bf 4} & 2 \\
5 &  5 & 4 & 3 & 2 & 1 \\
\end{tabular}  
\end{center} 
\end{table}


Neben der $0$ haben hier auch die Zahlen $2, 3$ und $4$ keine eindeutige Inverse
(man sagt auch, sie haben {\bf keine} Inverse, weil es die elementare Eigenschaft
einer Inversen ist, eindeutig zu sein).

Die Zahlen $2, 3$ und $4$ haben mit dem Modul $6$ den Faktor $2$ oder $3$ gemeinsam.
Nur die zu $6$ teilerfremden Zahlen $1$ und $5$ haben multiplikative Inverse, n"amlich jeweils sich selbst.

Die Anzahl der zum Modul $m$ teilerfremden Zahlen ist auch die Anzahl
derjenigen Zahlen, die eine multiplikative Inverse haben (vgl. unten die
\hyperlink{EulerFunction}{Euler-Funktion} \index{Eulersche Phi-Funktion} $J(m)$).

F"ur die beiden in den Multiplikationstabellen verwendeten Moduli $5$ und $6$
bedeutet dies:
Der Modul $5$ ist bereits eine Primzahl. Also gibt es in mod $5$ genau $J(5) = 5 - 1 = 4$ 
mit dem Modul teilerfremde Zahlen, also alle von $1$ bis $4$.

Da $6$ keine Primzahl ist, zerlegen wir $6$ in seine Faktoren: $6 = 2 * 3$.
Daher gibt es in mod $6$ genau $J(6) = (2-1)*(3-1) = 1 * 2 = 2$ Zahlen, die eine
multiplikative Inverse haben, n"amlich die $1$ und die $5$.

F"ur gro"se Moduli scheint es nicht einfach, die Tabelle der multiplikativen
Inversen zu berechnen (das gilt nur f"ur die in den oberen Multiplikationstabellen 
fett markierten Zahlen). Mit Hilfe des kleinen Satzes von Fermat\index{Fermat!kleiner Satz} kann man daf"ur
einen einfachen Algorithmus aufstellen \cite[S. 80]{Pfleger1997}. Schnellere
Algorithmen werden z.B. in \cite{Knuth1998} \index{Euklidscher Algorithmus!erweiterter}
beschrieben\footnote{
Mit dem erweiterten Satz von Euklid \index{ggT}(erweiterter ggT) kann man die
multiplikative Inverse berechnen und die Invertierbarkeit bestimmen (Siehe \hyperlink{Appendix_A}{Anhang A}).
Alternativ kann auch die Primitivwurzel genutzt werden. 
}.

\vskip +10 pt
Kryptographisch ist nicht nur die Eindeutigkeit der Inversen, sondern auch das
Aussch"opfen des gesamten Wertebereiches\index{Wertebereich} eine wichtige Eigenschaft.

\begin{satz}\label{thm-zth-exhperm}
Sei $a,i\in \{1, \cdots , m-1\}$ mit ${\rm~ggT} (a,m)=1, $ dann nimmt f"ur eine bestimmte Zahl $a$ das Produkt 
$a*i {\rm ~mod~} m$  alle Werte aus $ \{1, \cdots ,m-1\}$ an (ersch"opfende  Permutation\index{Permutation} der L"ange 
$m-1$)\footnote{
Vergleiche auch Satz \ref{thm-zth-ordp} in \hyperlink{Kap_3_9}{Kapitel 3.9 Multiplikative Ordnung und Primitivwurzel}.
}.
\end{satz}


Die folgenden drei Beispiele\footnote{
In \hyperlink{AppArith1}{Anhang D} finden Sie den Quellcode zur Berechnung der Tabellen mit Mathematica\index{Mathematica}
oder Pari-GP.\index{Pari-GP}
} veranschaulichen Eigenschaften der multiplikativen Inversen (hier sind nicht mehr 
die vollst"andigen Multiplikationstabellen angegeben, sondern nur die Zeilen f"ur die Faktoren $5$ und $6$).

In der Multiplikationstabelle mod $17$ wurde f"ur $i = 1, 2, \cdots, 18$ berechnet:
\begin{itemize}
   \item[] $(5*i)/17 = a$ Rest $r$ und hervorgehoben $5*i \equiv 1$ (mod $17$),
   \item[] $(6*i)/17 = a$ Rest $r$ und hervorgehoben $6*i \equiv 1$ (mod $17$).
\end{itemize}
{\bf Gesucht} ist das $i$, f"ur das der Produktrest $ a*i$ modulo $17$ mit $a=5$ bzw. $a=6$
den Wert $1$ hat. 

{
\begin{table}[h] \label{SrcArith1a}
\subsubsection*{Multiplikationstabelle modulo $17$ (f"ur $a=5$ und $a=6$)}
\begin{center}
\begin{tabular}{|l||c|c|c|c|c|c|c|c|c|c|c|c|c|c|c|c||c|c|}
\hline 
i                   & 1  & 2  & 3  & 4  & 5  & 6  & 7  & 8  & 9 & 10 & 11 & 12 & 13 & 14 & 15 & 16  & 17 & 18 \\
\hline
\hline  
$5*i$                 & 5 & 10 & 15 & 20 & 25 & 30 & 35 & 40 & 45 & 50 & 55 & 60 & 65 & 70 & 75 & 80  & 85 & 90   \\
Rest                & 5 & 10 & 15  & 3  & 8 & 13  & {\bf 1}  & 6 & 11 & 16  & 4  & 9 & 14  & 2  & 7 & 12   & 0  & 5   \\
\hline
$6*i$                 & 6 & 12 & 18 & 24 & 30 & 36 & 42 & 48 & 54 & 60 & 66 & 72 & 78 & 84 & 90 & 96 & 102 & 108   \\
Rest                & 6 & 12  & {\bf 1}  & 7 & 13  & 2  & 8 & 14  & 3  & 9 & 15  & 4 & 10 & 16  & 5 & 11   & 0  & 6   \\
\hline
\end{tabular}
\end{center} 
\end{table}
}
\vskip - 12 pt
Da sowohl $5$ als auch $6$ jeweils teilerfremd\index{teilerfremd} zum Modul $m=17$ sind, kommen zwischen
$i=1, \cdots, m$ f"ur die Reste alle Werte zwischen $0, \cdots, m-1$ vor (vollst"andige $m$-Permutation\index{Permutation}).
% \enlargethispage{0.5cm}

{
    {\bf Die multiplikative Inverse von $5$ (mod $17$) ist $7$, die Inverse von $6$ (mod $17$) ist $3$.}

    \vskip +20 pt

    \begin{table}[h] \label{SrcArith1b}
    \subsubsection*{Multiplikationstabelle modulo $13$ (f"ur $a=5$ und $a=6$)}
    \begin{center}                                                                          
    \begin{tabular}{|l||c|c|c|c|c|c|c|c|c|c|c|c||c|c|c|c|c|c|}
    \hline 
    i                    & 1  & 2  & 3  & 4  & 5  & 6  & 7  & 8  & 9 & 10 & 11 & 12 & 13 & 14 & 15 & 16  & 17  & 18 \\
    \hline 
    \hline 
    $5*i$                 & 5 & 10 & 15 & 20 & 25 & 30 & 35 & 40 & 45 & 50 & 55 & 60 & 65 & 70 & 75 & 80  & 85  & 90 \\
    Rest                 & 5 & 10  & 2  & 7  & 12  & 4 & 9  & {\bf 1}  & 6  & 11 & 3  & 8  & 0 & 5  & 10  & 2   & 7   & 12 \\
    \hline 
    $6*i$                  & 6 & 12 & 18 & 24 & 30 & 36 & 42 & 48 & 54 & 60 & 66 & 72 & 78 & 84 & 90 & 96 & 102 & 108 \\
    Rest                 & 6  & 12  & 5  & 11  & 4  & 10  & 3  & 9  & 2  & 8  & {\bf 1}  & 7  & 0  & 6  & 12  & 5   & 11   & 4 \\
    \hline 
    \end{tabular}
    \end{center} 
    \end{table}
}   

\vskip -12 pt
Da sowohl $5$ als auch $6$ auch zum Modul $m=13$ jeweils teilerfremd sind, kommen
zwischen $i=1, \cdots, m$ f"ur die Reste alle Werte zwischen $0, \cdots, m-1$ vor.


{\bf Die multiplikative Inverse von $5$ (mod $13$) ist $8$, die Inverse von $6$ (mod $13$) ist $11$.}
\vskip +15 pt

Die folgende Tabelle enth"alt ein Beipiel daf"ur, wo der Modul $m$ und die Zahl $(a=6)$
{\em nicht} teilerfremd sind.
\vskip +15 pt

\newpage
\begin{table}[h]
\subsubsection*{Multiplikationstabelle modulo $12$ (f"ur $a=5$ und $a=6$)}
\begin{center}                                                                          
\begin{tabular}{|l||c|c|c|c|c|c|c|c|c|c|c||c|c|c|c|c|c|c|}
\hline 
i                    & 1  & 2  & 3  & 4  & 5  & 6  & 7  & 8  & 9 & 10 & 11 & 12 & 13 & 14 & 15 & 16  & 17  & 18 \\
\hline 
\hline 
5*i                  & 5 & 10 & 15 & 20 & 25 & 30 & 35 & 40 & 45 & 50 & 55 & 60 & 65 & 70 & 75 & 80  & 85  & 90 \\
Rest                 & 5 & 10  & 3  & 8  & {\bf 1}  & 6 & 11  & 4  & 9  & 2  & 7  & 0  & 5 & 10  & 3  & 8   & 1   & 6 \\
\hline 
6*i                  & 6 & 12 & 18 & 24 & 30 & 36 & 42 & 48 & 54 & 60 & 66 & 72 & 78 & 84 & 90 & 96 & 102 & 108 \\
Rest                 & 6  & 0  & 6  & 0  & 6  & 0  & 6  & 0  & 6  & 0  & 6  & 0  & 6  & 0  & 6  & 0   & 6   & 0 \\
\hline 
\end{tabular}
\end{center} 
\end{table}

\vskip -12 pt
Berechnet wurde $(5 * i)$ (mod $12$) und $(6*i)$ (mod $12$).
Da $6$ und der Modul $m=12$ nicht teilerfremd\index{teilerfremd} sind, kommen zwischen $i=1, \cdots, m$
nicht alle Werte zwischen $0, \cdots, m-1$ vor und $6$ hat mod $12$ auch keine
Inverse!
\index{Inverse!additive} \index{Inverse!multiplikative}

{\bf Die multiplikative Inverse von $5$ (mod $12$) ist $5$. Die Zahl $6$ hat keine Inverse (mod $12$). }



% .......................................................................................................
\subsubsection{Potenzieren}

\index{Potenzieren} Das Potenzieren ist in der modularen Arithmetik definiert als wiederholtes
Multiplizieren -- wie "ublich, nur dass jetzt Multiplizieren etwas anderes ist.
Es gelten mit kleinen Einschr"ankungen die "ublichen Rechenregeln wie
\begin{eqnarray}
a^{b+c} & = & a^b * a^c,  \nonumber \\
(a^b)^c & = & a^{b*c} = a^{c*b} = (a^c)^b. \nonumber
\end{eqnarray}


Analog der modularen Addition und der modularen Multiplikation funktioniert
das modulare Potenzieren:
$$ 3^2 \equiv 9 \equiv 4 {\rm ~(mod~} 5). $$
Auch aufeinanderfolgendes Potenzieren geht analog: 

{\bf Beispiel 1:}

\begin{quote}
$$ (4^3)^2 \equiv 64^2 \equiv 4096 \equiv 1 {\rm ~(mod~} 5). $$
(1) Reduziert man bereits {\bf Zwischenergebnisse} modulo $5$, kann man
schneller\footnote{
Die Rechenzeit der Multiplikation zweier Zahlen h"angt normalerweise von der
L"ange der Zahlen ab. Dies sieht man, wenn man nach der Schulmethode z.B.
$474*228$ berechnet: Der Aufwand steigt quadratisch, da $3*3$ Ziffern
multipliziert werden m"ussen. Durch die Reduktion der Zwischenergebnisse
werden die Zahlen deutlich kleiner.
} rechnen, mu"s aber auch aufpassen, da sich dann nicht immer alles wie in der
gew"ohnlichen Arithmetik verh"alt.
\begin{eqnarray}
(4^3)^2 & \equiv & (4^3{\rm ~(mod~}5))^2{\rm ~(mod~}5) \nonumber \\
            & \equiv & (64{\rm ~(mod~}5))^2{\rm ~(mod~}5) \nonumber \\
            & \equiv & 4^2{\rm ~(mod~}5) \nonumber \\
            & \equiv & 16 \equiv 1 {\rm ~(mod~}5). \nonumber
\end{eqnarray}

(2) Aufeinanderfolgende Potenzierungen lassen sich in der gew"ohnlichen
Arithmetik auf eine einfache Potenzierung zur"uckf"uhren, indem man die
Exponenten miteinander multipliziert:
$$ (4^3)^2 = 4^{3*2} = 4^6 = 4096. $$
In der modularen Arithmetik geht das nicht ganz so einfach, denn man erhielte:
$$ (4^3)^2 \equiv 4^{3*2{\rm ~(mod~}5)} \equiv 4^{6{\rm ~(mod~}5)} \equiv 4^1 \equiv 4{\rm ~(mod~}5). $$
Wie wir oben sahen, ist das richtige Ergebnis aber $1$ !!

(3) Deshalb ist f"ur das fortgesetzte Potenzieren in der modularen Arithmetik
die Regel etwas anders: man multipliziert die Exponenten nicht in (mod $m$),
sondern in (mod $J(m)$).

Mit $J(5) = 4$ ergibt sich:
$$
(4^3)^2 \equiv 4^{3*2{\rm ~(mod~}J(5))} \equiv 4^{6{\rm ~mod~}4} \equiv 4^2 \equiv 16 \equiv 1 {\rm ~(mod~}5).
$$
Das liefert das richtige Ergebnis.
\end{quote}
\vskip + 10 pt

\begin{satz}\label{thm-zth-pot}
$(a^b)^c \equiv a^{b*c{\rm ~(mod~}J(m))}{\rm ~(mod~}m)$.
\end{satz}

{\bf Beispiel 2:}
$$
3^{28} \equiv 3^{4*7} \equiv 3^{4*7{\rm ~(mod~}10)} \equiv 3^8 \equiv 6561 \equiv 5 {\rm ~(mod~}11).
$$
\vskip + 10 pt
\subsubsection*{3.6.3.1 ~~~Schnelles Berechnen hoher Potenzen} 
\addcontentsline{toc}{subsubsection}{~~~~~~~~~~3.6.3.1 ~Schnelles Berechnen hoher Potenzen} 
\hypertarget{hohpot}{} \index{Potenzen}
Bei RSA-Ver- und Entschl"usselungen\footnote{
Siehe \hyperlink{RSABeweis}{Kapitel 3.10 Beweis des RSA-Verfahrens mit Euler-Fermat} und 
\hyperlink{Kapitel_3_12}{Kapitel 3.13 Das RSA-Verfahren mit konkreten Zahlen}.
} m"ussen hohe Potenzen modulo $m$ berechnet werden. Schon die Berechnung $(100^5)
{\rm ~mod~}3$ sprengt den $32$ Bit langen \index{Long-Integer}
Long-Integer-Zahlenbereich, sofern man zur Berechnung von $a^n$ getreu der
Definition $a$ tats"achlich $n$-mal mit sich selbst multipliziert. Bei sehr gro"sen Zahlen w"are selbst ein
schneller Computerchip mit einer einzigen Exponentiation l"anger besch"aftigt
als das Weltall existiert. Gl"ucklicherweise gibt es f"ur die Exponentiation
(nicht aber f"ur das Logarithmieren) eine sehr wirksame Abk"urzung.

Wird der Ausdruck anhand der Rechenregeln der modularen Arithmetik anders
aufgeteilt, sprengt die Berechnung nicht einmal den $16$ Bit langen
Short-Integer-Bereich:\index{Short-Integer}
$$
(a^5) \equiv (((a^2{\rm ~(mod~}m))^2 {\rm ~(mod~}m)) * a){\rm ~(mod~}m).
$$


Dies kann man verallgemeinern, indem man den Exponenten bin"ar darstellt.
Beispielsweise w"urde die Berechnung von $a^n$ f"ur $n = 37$ auf dem naiven Wege $36$
Multiplikationen erfordern.
Schreibt man jedoch $n$ in der Bin"ardarstellung als $100101 = 1*2^5 + 1*2^2 + 1*2^0$,
so kann man umformen: $a^{37} = a^{2^5 + 2^2 + 2^0} = a^{2^5} * a^{2^2} * a^1$.

{\bf Beispiel 3:} $87^{43}{\rm ~(mod~}103)$. \\
Da $43 = 32+8+2+1$, $103$ prim , $43<J(103)$ ist und

die Quadrate (mod $103$) vorab berechnet werden k"onnen \\
\begin{eqnarray*}
87^2 & \equiv & 50 {\rm ~(mod~}103),\\
87^4 & \equiv & 50^2 \equiv 28 {\rm ~(mod~}103), \\
87^8 & \equiv & 28^2 \equiv 63 {\rm ~(mod~}103), \\
87^{16} & \equiv & 63^2 \equiv 55 {\rm ~(mod~}103),\\
87^{32} & \equiv & 55^2 \equiv 38 {\rm ~(mod~}103),
\end{eqnarray*}
gilt\footnote{
In \hyperlink{AppArith2}{Appendix D} finden Sie den Besipielcode zum Nachrechnen der Square-and-Multiply Methode
mit Mathematica und Pari-GP.
}: \label{SrcArith2}
\begin{eqnarray}
87^{43} & \equiv & 87^{32+8+2+1}{\rm ~(mod~}103) \nonumber \\
        & \equiv & 87^{32} * 87^8 * 87^2 * 87 {\rm ~(mod~}103) \nonumber \\ 
    & \equiv & 38 * 63 * 50 * 87 \equiv 85 {\rm ~(mod~}103). \nonumber
\end{eqnarray}
Die Potenzen $(a^2)^k$ sind durch fortgesetztes Quadrieren leicht zu bestimmen.
Solange sich $a$ nicht "andert, kann ein Computer sie vorab berechnen und --
bei ausreichend Speicherplatz -- abspeichern. Um dann im Einzelfall $a^n$ zu
finden, mu"s er nur noch genau diejenigen $(a^2)^k$ miteinander multiplizieren,
f"ur die an der $k$-ten Stelle der Bin"ardarstellung von n eine Eins steht. Der
typische Aufwand f"ur $n=600$ sinkt dadurch von $2^{600}$ auf $2*600$ Multiplikationen!
Dieser h"aufig verwendete Algorithmus hei"st \glqq Square and Multiply''. \index{Square and multiply}

\vskip +40 pt

% ..............................................................................................................
\subsubsection{Wurzeln und Logarithmen} \index{Wurzel}

Die Umkehrungen des Potenzierens sind ebenfalls definiert: Wurzeln und
Logarithmen sind abermals ganze Zahlen, aber im Gegensatz zur "ublichen
Situation sind sie nicht nur m"uhsam, sondern bei sehr gro"sen Zahlen in
\glqq ertr"aglicher\grqq~ Zeit "uberhaupt nicht zu berechnen.

Gegeben sei die Gleichung: $a \equiv b^c{\rm ~(mod~}m)$.

\begin{itemize}
\item [\bf a)] {\bf Logarithmieren\index{Logarithmieren} (Bestimmen von $c$) -- Diskretes Logarithmus Problem\index{Logarithmusproblem!diskret}:}

Wenn man von den drei Zahlen $a, b, c$, welche diese Gleichung erf"ullen, $a$ und
$b$ kennt, ist jede bekannte Methode, $c$ zu finden, ungef"ahr so aufwendig wie
das Durchprobieren aller $m$ denkbaren Werte f"ur $c$ -- bei einem typischen $m$ in
der Gr"o"senordnung von $10^{180}$ f"ur $600$-stellige Bin"arzahlen ein hoffnungsloses
Unterfangen. Genauer ist f"ur geeignet gro"se Zahlen $m$ der Aufwand nach heutigem
Wissensstand proportional zu ${\rm exp}\left( C*( \log m [\log \log m]^2)^{1/3}\right)$ mit einer
Konstanten $C > 1$.
\item[\bf b)] {\bf Wurzel-Berechnung (Bestimmen von $b$):}  

"Ahnliches gilt, wenn $b$ die Unbekannte ist und die Zahlen $a$ und $c$ bekannt sind. \\
Wenn die Eulerfunktion \index{Eulersche Phi-Funktion} $J(m)$ bekannt ist, findet man 
leicht\footnote{
Siehe \hyperlink{Appendix_A}{Anhang A}: der gr"o"ste gemeinsame Teiler (ggT) von ganzen Zahlen.
} $d$ mit $c*d \equiv 1 {\rm ~(mod~} J(m))$ und erh"alt mit Satz \ref{thm-zth-pot} 
$$
        a^d \equiv (b^c)^d \equiv b^{c*d} \equiv b^{c*d~(mod~J(m))} \equiv b^1 \equiv b {\rm ~(mod~} m)
$$
die {\em $c$-te Wurzel} $b$ von $a$. \par

F"ur den Fall, dass $J(m)$ in der Praxis nicht bestimmt werden 
kann\footnote{
Nach dem ersten Hauptsatz der Zahlentheorie und Satz \ref{thm-zth-phinum} kann man $J(m)$ mit Hilfe 
der Primfaktorzerlegung von $m$ bestimmen.
}, ist die Berechnung der $c$-ten Wurzel schwierig. Hierauf beruhen die Sicherheitsannahmen f"ur das 
RSA-Kryptosystem (Siehe Kapitel 4.3.1: \hyperlink{RSAVerfahren}{Das RSA-Verfahren} oder
Kapitel 3.10: \hyperlink{RSABeweis}{Beweis des RSA-Verfahrens mit Euler-Fermat}).

\end{itemize}
Dagegen ist der Aufwand f"ur die Umkehrung von Addition und Multiplikation
nur proportional zu $\log m$ beziehungsweise $(\log m)^2$.
Potenzieren (zu einer Zahl $x$ berechne $x^a$ mit festem $a$) und Exponentiation
(zu einer Zahl $x$ berechne $a^x$ mit festem $a$) sind also typische Einwegfunktionen
(siehe "Ubersicht "uber Einwegfunktionen im \hyperlink{Einwegfunktionen1}{Skript} oder in diesem 
\hyperlink{Einwegfunktionen2}{Artikel}).

% ++++++++++++++++++++++++++++++++++++++++++++++++++++++++++++++++++++++++++++++++++++++++++++
\subsection{Gruppen und modulare Arithmetik "uber $\mathbb{Z}_n $ und $\mathbb{Z}^*_n $}
\index{Gruppen}
In der Zahlentheorie und in der Kryptographie spielen mathematische 
\glqq {\em Gruppen}'' eine entscheidende Rolle. Von Gruppen spricht man nur, wenn f"ur
eine definierte Menge und eine definierte Relation (eine Operation wie
Addition oder Multiplikation) die folgenden Eigenschaften erf"ullt sind:
\begin{itemize}
\item Abgeschlossenheit \index{Abgeschlossenheit}
\item Existenz des neutralen Elements
\item Existenz des inversen Elements und
\item G"ultigkeit des Assoziativgesetzes.
\end{itemize}
Die abgek"urzte mathematische Schreibweise lautet: $(G, +)$ oder $(G,*)$.
\begin{definition}\label{def-zth-zn}
$\mathbb{Z}_n$ :\index{Z@$\mathbb{Z}_n$}
$$\mathbb{Z}_n \text{ umfa"st alle ganzen Zahlen von } 0 \text{ bis } n-1: \mathbb{Z}_n = \{0, 1, 2,\cdots, n-2, n-1\}.$$
\end{definition}
$\mathbb{Z}_n$ ist eine haufig verwendete endliche Gruppe aus den nat"urlichen Zahlen. Sie 
wird manchmal auch als Restemenge $R$ modulo $n$ bezeichnet.

Beispielsweise rechnen 32 Bit-Computer ("ubliche PCs) mit ganzen Zahlen
direkt nur in einer endlichen Menge, n"amlich in dem Wertebereich $0, 1, 2,
\cdots, 2^{32}-1$.

Dieser Zahlenbereich ist "aquivalent zur Menge $\mathbb{Z}_{2^{32}}$.

% ..............................................................................................
\subsubsection{Addition in einer Gruppe}\index{Addition} 

Definiert man auf einer solchen Menge die Operation mod+ mit
$$ a {\rm ~mod+~} b := (a + b){\rm ~(mod~}n) , $$
so ist die Menge $\mathbb{Z}_n$ zusammen mit der Relation mod+ eine Gruppe, denn es
gelten die folgenden Eigenschaften einer Gruppe f"ur alle Elemente von $\mathbb{Z}_n$:
\begin{itemize}
    \item   $ a {\rm ~mod+~} b$ ist ein Element von $\mathbb{Z}_n$  (Abgeschlossenheit),
    \item   $(a {\rm ~mod+~} b) {\rm ~mod+~} c \equiv a {\rm ~mod+~} (b {\rm ~mod+~} c)$  (mod+ ist assoziativ),
    \item   das neutrale Element ist die $0$.
  \item   jedes Element $a \in \mathbb{Z}_n$ besitzt bez"uglich dieser Operation ein Inverses, n"amlich $n-a$ \\ 
                (denn es gilt: $a {\rm ~mod+~} (n-a) \equiv a + (n-a){\rm ~(mod~}n) \equiv n \equiv 0 {\rm ~(mod~}n)$).
\end{itemize}
Da die Operation kommutativ ist, d.h. es gilt $(a {\rm ~mod+~} b) = (b {\rm ~mod+~} a)$, ist diese Struktur \index{Struktur} sogar 
eine \glqq kommutative Gruppe''.
% ..............................................................................................
\subsubsection{Multiplikation in einer Gruppe}\index{Multiplikation}

Definiert man in der Menge $\mathbb{Z}_n$ die Operation mod* mit
$$ a {\rm ~mod*~} b := (a * b){\rm ~(mod~}n), $$
so ist $\mathbb{Z}_n$ zusammen mit dieser Operation {\bf normalerweise keine} Gruppe, weil
nicht f"ur jedes $n$ alle Eigenschaften erf"ullt sind.

{\bf Beispiele:}
\begin{itemize}
\item[a)] In $\mathbb{Z}_{15}$ besitzt z.B. das Element $5$ kein Inverses.
Es gibt n"amlich kein $a$ mit \\ $5 * a \equiv 1 ({\rm ~mod~}15).$
Jedes Modulo-Produkt mit $5$ ergibt auf dieser Menge $5, 10$ oder $0$.
\item[b)] In $\mathbb{Z}_{55} \setminus \{0\}$ besitzen z.B. die Elemente $5$ und $11$ 
keine multiplikativen Inversen. Es gibt n"amlich kein $a$ aus $\mathbb{Z}_{55}$ mit 
$5 * a \equiv 1 ({\rm ~mod~}55)$ und kein $a$ mit $11 *a \equiv 1 {\rm ~mod~}55$.
Das liegt daran, dass $5$ und $11$ nicht teilerfremd zu $55$ sind.
Jedes Modulo-Produkt mit $5$ ergibt auf dieser Menge $5, 10, 15, \cdots, 50$ oder $0$.
Jedes Modulo-Produkt mit $11$ ergibt auf dieser Menge $11, 22, 33, 44$ oder $0$.
\end{itemize}
Dagegen gibt es Teilmengen von $\mathbb{Z}_n$, die bez"uglich mod* eine Gruppe bilden.
W"ahlt man s"amtliche Elemente aus $\mathbb{Z}_n$ aus, die teilerfremd zu $n$ sind, so ist
diese Menge eine Gruppe bez"uglich mod*. Diese Menge bezeichnet man mit $\mathbb{Z}_n^*$ .

\begin{definition}\label{def-zth-znmult}
{\bf $\mathbb{Z}_n^*$} : \index{Z@$\mathbb{Z}_n^*$}
$$ \mathbb{Z}_n^* := \{ a \in \mathbb{Z}_n \big| {\rm ggT}(a,n) = 1 \}.$$
\end{definition}
$\mathbb{Z}_n^*$ wird manchmal auch als \index{Restmenge!reduzierte} {\em reduzierte Restemenge} $R'$ modulo $n$ bezeichnet.

{\bf Beispiel:}
F"ur $n = 10=2*5$ gilt: \index{Restmenge!vollst""andige}
\begin{itemize}
  \item[] vollst"andige Restmenge $R = \mathbb{Z}_n = \{ 0, 1, 2, 3, 4, 5, 6, 7, 8, 9 \}.$
  \item[] reduzierte Restemenge $R' = \mathbb{Z}_n^* = \{ 1, 3, 7, 9 \} \longrightarrow J(n)=4$.
\end{itemize}

{\bf Bemerkung:}
$R'$ bzw. $\mathbb{Z}_n^*$ ist immer eine echte Teilmenge von $R$ bzw. $\mathbb{Z}_n$, da $0$ immer
Element von $\mathbb{Z}_n$, aber nie Element von $\mathbb{Z}_n^*$ ist. Da $1$ (per Definition) und $n-1$ immer teilerfremd
zu $n$ sind, sind sie stets Elemente beider Mengen.

W"ahlt man irgendein Element aus $\mathbb{Z}_n^*$ und multipliziert es mit jedem anderen
Element von $\mathbb{Z}_n^*$, so sind die Produkte\footnote{
Dies ergibt sich aus der Abgeschlossenheit von $\mathbb{Z}_n^*$ bez"uglich der Multiplikation und der 
ggT-Eigenschaft: \\
$[a, b \in \mathbb{Z}_n^* ] \Rightarrow [((a * b) {\rm ~(mod~} n)) \in \mathbb{Z}_n^*]$, genauer:\\
$[a, b \in \mathbb{Z}_n^* ] \Rightarrow  [{\rm ggT}(a, n) = 1, {\rm ggT}(b, n) = 1] 
\Rightarrow  [{\rm ggT}(a*b, n) = 1] \Rightarrow  [((a * b) {\rm ~(mod~} n)) \in \mathbb{Z}_n^*]$.
} alle wieder in $\mathbb{Z}_n^*$ und au"serdem 
sind die Ergebnisse eine eindeutige Permutation der Elemente von $\mathbb{Z}_n^*$ . Da die $1$ immer 
Element von $\mathbb{Z}_n^*$ ist, gibt es in dieser Menge eindeutig einen \glqq Partner'', so dass das 
Produkt $1$ ergibt. Mit anderen Worten:

\begin{satz}\label{thm-zth-znmult}
Jedes Element in $\mathbb{Z}_n^*$ hat eine multiplikative Inverse.
\end{satz}

Beispiel f"ur $a = 3$ modulo $10$ mit $\mathbb{Z}_n^* = \{ 1, 3, 7, 9 \}$ :
\begin{eqnarray}
3 & \equiv & 3 * 1{\rm ~(mod~}10), \nonumber \\
9 & \equiv & 3 * 3{\rm ~(mod~}10), \nonumber \\
1 & \equiv & 3 * 7{\rm ~(mod~}10), \nonumber \\
7 & \equiv & 3 * 9{\rm ~(mod~}10). \nonumber 
\end{eqnarray}
Die eindeutige Invertierung (Umkehrbarkeit)\index{Umkehrbarkeit} ist eine notwendige Bedingung f"ur die
Kryptographie (siehe Kapitel 3.10: \hyperlink{RSABeweis}{Beweis des RSA-Verfahrens mit Euler-Fermat}).

% .......................................................................................................
\subsection{Euler-Funktion, kleiner Satz von Fermat und Satz von Euler - Fermat}
\hypertarget{Kapitel_3_8}{} \index{Fermat!kleiner Satz}

\subsubsection{Muster und Strukturen}\index{Struktur}
So wie Mathematiker die Struktur $a*x \equiv b {\rm ~mod~}m $ untersuchen (s. \hyperlink{Kapitel_3_5_2}{Kapitel 3.5.2}),
so interessiert sie auch die Struktur $ x^{a}\equiv b {\rm ~mod~}m$. 

Auch hierbei ist insbesondere der Fall interessant, wenn $b=1$ ist (also den Werte der multiplikativen
Einheit annimmt) und wenn $b=x$ ist (also die Abbildung einen Fixpunkt\index{Fixpunkt} hat).

\subsubsection{Die Euler-Funktion}
Bei vorgegebenem $n$ ist die Anzahl der Zahlen aus der Menge $\{1, \cdots, n-1\}$,
die zu $n$ teilerfremd sind, gleich dem Wert der Euler\footnote{
Leonhard Euler, schweizer Mathematiker, 15.4.1707 -- 18.9.1783\index{Euler} 
}-Funktion $J(n)$.
\index{Eulersche Phi-Funktion}
\begin{definition}\label{def-zth-phiofn} \hypertarget{EulerFunction}{}
Die Euler-Funktion\footnote{
Wird oft auch als Eulersche Phi-Funktion $\Phi(n)$ geschrieben.
} $J(n)$ gibt die Anzahl der Elemente von $\mathbb{Z}_n^*$ an.
\end{definition}

$J(n)$ gibt auch an, wieviele ganze Zahlen in mod $n$ multiplikative Inverse
haben. $J(n)$ l"a"st sich berechnen, wenn man die Primfaktorzerlegung von $n$ kennt.

\begin{satz}\label{thm-zth-phiprime}
F"ur eine Primzahl gilt: $J(p) = p - 1.$
\end{satz}

\begin{satz}\label{thm-zth-phipq} \label{J_of_pq}
Ist $n$ das Produkt zweier verschiedenen Primzahlen $p$ und $q$, so gilt:
$$J(p*q) = (p - 1) * (q - 1) \quad {\rm oder} \quad J(p * q) = J(p) * J(q).$$
\end{satz}
Dieser Fall ist f"ur das RSA-Verfahren wichtig.

\begin{satz}\label{thm-zth-phimultprime} \label{J_of_p1..pk}
Ist $n = p_1 * p_2 * \cdots * p_k$, wobei $p_1$ bis $p_k$ verschiedene Primzahlen
sind (d.h. $p_i \not= p_j$ f"ur $i \not= j$), dann gilt (als Verallgemeinerung von Satz \ref{J_of_pq}):
$$J(n) = (p_1 - 1)*(p_2 - 1)* \cdots *(p_k - 1).$$
\end{satz} 
 

\begin{satz}\label{thm-zth-phinum} \label{J_of_n}
Verallgemeinert gilt f"ur jede Primzahl $p$ und jedes $n$ aus $\mathbb{N}$:
\begin{itemize}
 \item[1.] $J(p^n) = p^{n-1} * (p-1)$. 
 \item[2.] Ist $n = p_1^{e_1} * p_2^{e_2} * \cdots *p_k^{e_k}$,
           wobei $p_1$ bis $p_k$ verschiedene Primzahlen sind, dann gilt:
           $$J(n) = [(p_1^{e_1-1}) * (p_1-1)] * \cdots * [(p_k^{e_k-1})*(p_k - 1)] = n * ([(p_1-1) / p_1] * \cdots * [(p_k-1) / p_k]).$$
\end{itemize}      
\end{satz}

{\bf Beispiele:} 
\begin{itemize}
\item  $n=70=2*5*7 \Longrightarrow $ nach Satz \ref{J_of_p1..pk}: $ J(n)= 1\cdot 4 \cdot 6 =24.$ 
\item  $n=9=3^2 \Longrightarrow$ nach Satz \ref{J_of_n}: $ J(n)= 3^1\cdot 2 =6,$ weil  $\mathbb{Z}_9^* =\{ 1,2,4,5,7,8\}.$
\item $n = 2.701.125 = 3^2 * 5^3 * 7^4$ $\Longrightarrow$ nach Satz \ref{J_of_n}: 
$J(n) = [3^1 * 2] * [5^2 * 4] * [7^3 * 6] = 1.234.800.$
\end{itemize}

\subsubsection{Der Satz von Euler-Fermat}
F"ur den Beweis des RSA-Verfahrens brauchen wir den Satz von Fermat und dessen Verallgemeinerung (Satz von Euler-Fermat).
\index{Fermat!kleiner Satz}
\begin{satz}\label{thm-zth-fermat1}{\bf Kleiner Satz von Fermat}\footnote{
Pierre de Fermat, franz"osischer Mathematiker, 17.8.1601 -- 12.1.1665
\index{Fermat}
}
\hypertarget{kleiner-Fermat}Sei $p$ eine Primzahl und $a$ eine beliebige ganze Zahl, dann gilt
$$a^p \equiv a {\rm ~(mod~} p).$$
\end{satz}

Eine alternative Formulierung des kleinen Satzes von Fermat lautet:
Sei $p$ eine Primzahl und $a$ eine beliebige ganze Zahl, die teilerfremd zu $p$ ist, dann gilt:
$$a^{p-1} \equiv 1 {\rm ~(mod~} p).$$
\begin{satz}{\label{thm-zth-fermateuler}
\bf Satz von Euler-Fermat (Verallgemeinerung des kleines Satzes von Fermat)}\hypertarget{Euler-Fermnat}{}
F"ur alle Elemente $a$ aus der Gruppe $\mathbb{Z}_n^*$ gilt (d.h. $a$ und $n$ sind nat"urliche Zahlen, die
teilerfremd zueinander sind):
$$  a^{J(n)} \equiv 1  {\rm ~(mod~} n).$$
\end{satz}
\index{Euler} \index{Fermat}

Dieser Satz besagt, dass wenn man ein Gruppenelement (hier $a$) mit der Ordnung der Gruppe (hier $J(n)$) potenziert, 
ergibt sich immer das neutrale Element der Multiplikation (die Zahl 1).

Die 2. Formulierung des kleinen Satzes von Fermat ergibt sich direkt aus dem
Satz von Euler, wenn $n$ eine Primzahl ist.

Falls $n$ das Produkt zweier verschiedenen Primzahlen ist, kann man mit dem Satz von Euler
in bestimmten F"allen sehr schnell das Ergebnis einer modularen Potenz
berechnen. Es gilt: $a^{(p-1)*(q-1)} \equiv 1 {\rm ~(mod~}pq)$.
\vskip +7 pt

{\bf Beispiele zur Berechnung einer modularen Potenz:} 
\vskip -4pt
\begin{itemize}
   \item Mit $2 = 1 * 2$ und $6 = 2*3$, wobei $2$ und $3$ jeweils prim; $J(6) = 2$, da nur $1$
         und $5$ zu $6$ teilerfremd sind folgt $5^2 \equiv 5^{J(6)} \equiv 1 {\rm ~(mod~}6)$,
         ohne dass man die Potenz berechnen mu"ste.
   \item Mit $792 = 22 * 36$ und $23*37 = 851$, wobei $23$ und $37$ jeweils prim folgt \\
         $31^{792} \equiv 31^{J(23*37)} \equiv 31^{J(851)} \equiv 1 {\rm ~(mod~}851)$.
\end{itemize}

\subsubsection{Bestimmung der multiplikativen Inversen}

Eine weitere interessante Anwendung ist ein Sonderfall der Bestimmung der
multiplikativen Inverse mit Hilfe des Satzes von Euler-Fermat
(multiplikative Inverse werden ansonsten mit dem \index{Euklidscher Algorithmus!erweiterter} erweiterten Euklid'schen
Algorithmus ermittelt).

{\bf Beispiel:} \\
Finde die multiplikative Inverse von $1579$ modulo $7351$.

Nach Euler-Fermat gilt: $a^{J(n)}= 1$ mod $n$ f"ur alle $a$ aus $\mathbb{Z}_n^*$.
Teilt man beide Seiten durch $a$, ergibt sich: $a^{J(n) - 1} \equiv a^{-1}$ (mod $n$).
F"ur den Spezialfall, dass der Modul prim ist, gilt $J(n) = p - 1$.
Also gilt f"ur die modulare Inverse $ a^{-1}$ von a: 
$$a^{-1} \equiv a^{J(n) - 1} \equiv a^{(p-1)-1} \equiv a^{p-2} {\rm ~(mod~} p). $$
F"ur unser Beispiel bedeutet das:
\begin{itemize}
   \item[]  Da der Modul $7351$ prim ist, ist $p-2 = 7349$. \\
        $1579^{-1} \equiv 1579^{7349}$ (mod $p$).
\end{itemize}       
Durch geschicktes Zerlegen des Exponenten kann man diese Potenz relativ einfach berechnen 
(siehe \hyperlink{hohpot}{Kapitel 3.6.3.1 Schnelles Berechnen hoher Potenzen)}:

\begin{itemize}
   \item[] $7349 = 4096 + 2048 + 1024 + 128 + 32 + 16 + 4 + 1$ \\
       $1579^{-1} \equiv 4716$ (mod $7351$).
\end{itemize}

\subsubsection{Fixpunkte\index{Fixpunkt} modulo 26}

Laut Satz \ref{thm-zth-pot} werden die arithmetischen Operationen von modularen Ausdr"ucken in den Exponenten modulo $J(n)$ 
und nicht modulo $n$ durchgef"uhrt\footnote{
F"ur das folgende Beispiel wird der Modul wie beim RSA-Verfahren "ublich mit 
\glqq $n$'' statt mit \glqq $m$'' bezeichnet.
}.

Wenn man in $a^{e*d} \equiv a^1$ (mod $n$) die Inverse z.B. f"ur den Faktor $e$ 
im Exponenten bestimmen will, mu"s man modulo $J(n)$ rechnen.

{\bf Beispiel (mit Bezug zum RSA-Algorithmus):}\index{RSA}

Wenn man modulo $26$ rechnet, aus welcher Menge k"onnen $e$ und $d$ kommen?

L"osung: Es gilt $e*d \equiv 1$ (mod $J(26)$).
\begin{itemize}
   \item[] Die reduzierte Restemenge $R' = \mathbb{Z}_{26}^* = \{ 1, 3, 5, 7, 9, 11, 15, 17, 19, 21, 23, 25 \}$ sind die 
Elemente in $\mathbb{Z}_{26}, $ die eine multiplikative Inverse haben, also teilerfremd\index{teilerfremd} zu $26$ sind.
   \item[] Die reduzierte Restemenge $R''$ enth"alt nur die Elemente aus $R'$, die
           teilerfremd zu $J(26) = 12$ sind: $R'' = \{ 1, 5, 7, 11 \}$.
   \item[] F"ur jedes $e$ aus $R''$ gibt es ein $d$ aus $R''$, so dass $a \equiv (a^e)^d{\rm ~(mod~}n)$.
\end{itemize}
Somit gibt es also zu jedem $e$ in $R''$ genau ein (nicht unbedingt von $e$ verschiedenes) Element, so dass 
gilt: $e*d \equiv 1$ (mod $J(26)$).

F"ur alle $e$, die teilerfremd zu $J(n)$ sind, k"onnte man nach dem Satz von Euler-Fermat das $d$ 
folgenderma"sen berechnen:
\begin{eqnarray}
 d & \equiv & e^{-1}      {\rm ~(mod~} J(n)) \nonumber \\
   & \equiv & e^{J(J(n))-1}  {\rm ~(mod~} J(n)), \quad  {\rm denn~}  a^{J(n)} \equiv 1 {\rm ~(mod~} n)
      \quad {\rm ~entspricht~~} a^{J(n)-1} \equiv a^{-1} {\rm ~(mod~} n). \nonumber
\end{eqnarray}


Besteht die Zahl $n$ aus zwei verschiedenen Primfaktoren, so ist die Faktorisierung von $n$ und das Finden von $J(n)$ 
"ahnlich schwierig\footnote{
Ist die Faktorisierung von $ n=pq$  mit $p\neq q$ bekannt, so ist $ J(n) = (p-1)*(q-1)
= n -(p+q)+1.$ Ferner sind die Zahlen $p$ und $q$ L"osungen der quadratischen Gleichung
$x^2-(p+q)x+pq=0. $ Sind also nur 
$n$ und $J(n)$ bekannt, so gilt $pq=n$ und $p+q= n +1 -J(n).$  Man erh"alt somit die Faktoren $p$ und $q$ von $n$, indem man die quadratische Gleichung
$$ x^2 +(J(n)-n-1)x +n=0 $$ l"ost.
} (vergleiche Forderung 3 in \hyperlink{Kapitel_3_10_1}{Kapitel 3.10.1}).


% ++++++++++++++++++++++++++++++++++++++++++++++++++++++++++++++++++++++++++++++++++++++++++++++++++
\subsection{Multiplikative Ordnung und Primitivwurzel}\hypertarget{Kap_3_9}{} 

Mathematiker stellen sich die Frage, unter welchen Bedingungen ergibt die
wiederholte Anwendung einer Operation das neutrale Element (vergleiche Strukturen und Muster\index{Struktur}).

F"ur die $i$-fach aufeinander folgende modulare Multiplikation einer Zahl $a$ mit 
$i=1, \cdots , m-1$ ergibt sich als Produkt das neutrale Element der Multiplikation $(1)$ nur dann, wenn
$a$ und $m$ teilerfremd sind. Der Wert von $i, $ f"ur den das Produkt $a^{i}= 1$ ist, hei"st 
multiplikative Ordnung von $a.$

Die Multiplikative Ordnung (order) und die Primitivwurzel (primitive root) sind zwei n"utzliche Konstrukte 
(Konzepte) der elementaren Zahlentheorie.

\begin{definition}\label{def-zth-ordn}
Die {\bf multiplikative Ordnung} \index{Ordnung!multiplikative} $ord_m(a)$ einer ganzen Zahl $a {\rm ~(mod~} m)$ (wobei $a$ und $m$ teilerfremd sind) ist die 
kleinste ganze Zahl $e$, f"ur die gilt:  $a^e \equiv 1 {\rm ~(mod~} m)$. 
\end{definition}

Die folgende Tabelle zeigt, dass in einer multiplikativen Gruppe (hier $\mathbb{Z}_{11}^*$) nicht notwendig alle Zahlen die 
gleiche Ordnung haben: Die Ordnungen sind $1, 2, 5$ und $10$. Dabei bemerkt man:
\begin{enumerate}
  \item Die Ordnungen sind alle Teiler von 10. 
  \item Die Zahlen $a = 2, 6, 7$ und $8$ haben die Ordnung $10$. Man sagt diese Zahlen haben in $\mathbb{Z}_{11}^*$ 
        {\bf maximale Ordnung} \index{Ordnung!maximale}.
\end{enumerate}

{\bf Beispiel 1:}\\
Die folgende Tabelle\footnote{
In \hyperlink{AppArith3a}{Anhang D} finden Sie den Quelltext zur bestimmung der Tabelle mit Mathematica und
Pari-GP.
} zeigt die Werte $a^i{\rm ~mod~}11$ f"ur die Exponenten $i = 1, 2, \cdots, 10$ und f"ur 
die Basen $a = 1, 2, \cdots, 10$ sowie den sich f"ur jedes a daraus ergebenden Wert $ord_{11}(a)$:
%\newpage
{
\subsubsection*{Werte von $a^i{\rm ~mod~}11, 1\leq a,i<11$ und zugeh"orige Ordnung von $a$ modulo $m.$}
\begin{center} \label{SrcArith3a}
%\begin{table}[h]
\begin{tabular}{|l||c|c|c|c|c|c|c|c|c|c|c|c|c|c|}
\hline
              & i=1 & i=2 & i=3 & i=4 & i=5 & i=6 & i=7 & i=8 & i=9 & i=10  & $ord_{11}(a)$\\
\hline
\hline
a=1           & 1  & 1    & 1  & 1    & 1    & 1    & 1  & 1    & 1  & 1     & 1   \\
\hline
a=2           & 2  & 4    & 8  & 5   & 10    & 9    & 7  & 3    & 6  & 1    & 10  \\
\hline
a=3           & 3  & 9    & 5  & 4 & {\bf 1} & 3    & 9  & 5    & 4  & 1     & 5   \\
\hline
a=4           & 4  & 5    & 9  & 3 & {\bf 1} & 4    & 5  & 9    & 3  & 1    & 5 \\
\hline
a=5           & 5  & 3    & 4  & 9 & {\bf 1} & 5    & 3  & 4    & 9  & 1    & 5   \\
\hline
a=6           & 6  & 3    & 7  & 9   & 10    & 5    & 8  & 4    & 2  & 1    & 10  \\
\hline
a=7           & 7  & 5    & 2  & 3   & 10    & 4    & 6  & 9    & 8  & 1    & 10  \\
\hline
a=8           & 8  & 9    & 6  & 4   & 10    & 3    & 2  & 5    & 7  & 1    & 10  \\
\hline
a=9           & 9  & 4    & 3  & 5 & {\bf 1} & 9    & 4  & 3    & 5  & 1    & 5   \\
\hline
a=10         & 10  & 1   & 10  & 1   & 10    & 1   & 10  & 1   & 10  & 1    & 2   \\
\hline
\end{tabular}
%\end{table}
\end{center}
}
Aus der Tabelle kann man ersehen, dass z.B. die Ordnung von $3$ modulo $11$ den Wert $5$ hat.

\begin{definition}\label{def-zth-primitiveroot}
Sind $a$ und $m$ teilerfremd und gilt $ord_m(a) = J(m)$, (d.h. $a$ hat maximale Ordnung), dann nennt man 
$a$ eine {\bf Primitivwurzel} \index{Primitivwurzel} von $m$.
\end{definition}

Nicht zu jedem Modul $m$ gibt es  eine Zahl $a,$ die eine Primitivwurzel ist. In der obigen Tabelle ist nur $a = 2, 6, 7$
und $8$ bez"uglich mod $11$ eine Primitivwurzel ($J(11) = 10$).

Mit Hilfe der Primitivwurzel kann man die Bedingungen klar herausarbeiten,
wann Potenzen modulo $m$ {\em eindeutig invertierbar}\index{invertierbar} und die Berechnung in den
Exponenten handhabbar sind.


\newpage
Die folgenden beiden Tabellen zeigen multiplikative Ordnung und Primitivwurzel
modulo $45$ und modulo $46$.

{\bf Beispiel 2:} \\
Die folgende Tabelle\footnote{
In \hyperlink{AppArith3b}{Anhang D} Finden Sie den Quelltext zur Berechnung der Tabelle mit Mathematica und Pari-GP.
} zeigt die Werte $a^i{\rm ~mod~}45$
f"ur die Exponenten $i = 1, 2, \cdots, 12$ und f"ur die Basen $a = 1, 2, \cdots, 12$
sowie den sich f"ur jedes $a$ daraus ergebenden Wert $ord_{45}(a)$.
%\newpage
{ % \small
\subsubsection*{Werte von $a^i{\rm ~mod~}45, 1\leq a,i<13$:}
\begin{center} \label{SrcArith3b}
%\begin{table}[h]
\begin{tabular}{|l||c|c|c|c|c|c|c|c|c|c|c|c|c|c|c|c|c|c|c|c|c|c|c|c|c|}
\hline
 $a\setminus i$ & 1            & 2            & 3 & 4 & 5 & 6 & 7 & 8 & 9 & 10 & 11 & 12     & $ord_{45}(a)$       & $J(45)$ \\
\hline
\hline                                                       
1             & 1              & 1   & 1   & 1   & 1   & 1   & 1   & 1   & 1    & 1    & 1    & 1 & 1              & 24  \\
\hline
2             & 2              & 4   & 8  & 16  & 32  & 19  & 38  & 31  & 17   & 34   & 23    & 1 & 12             & 24 \\
\hline
3             & 3              & 9  & 27  & 36  & 18   & 9  & 27  & 36  & 18    & 9   & 27   & 36  & ---            & 24 \\
\hline
4             & 4             & 16  & 19  & 31  & 34   & 1   & 4  & 16  & 19   & 31   & 34    & 1  & 6              & 24 \\
\hline
5             & 5             & 25  & 35  & 40  & 20  & 10   & 5  & 25  & 35   & 40   & 20   & 10  & ---            & 24 \\
\hline
6             & 6             & 36  & 36  & 36  & 36  & 36  & 36  & 36  & 36   & 36   & 36   & 36  & ---            & 24 \\
\hline
7             & 7              & 4  & 28  & 16  & 22  & 19  & 43  & 31  & 37   & 34   & 13    & 1  & 12             & 24 \\
\hline
8             & 8             & 19  & 17   & 1   & 8  & 19  & 17   & 1   & 8   & 19   & 17    & 1  & 4              & 24 \\
\hline
9             & 9             & 36   & 9  & 36   & 9  & 36   & 9  & 36   & 9   & 36    & 9   & 36  & ---            & 24 \\
\hline
10           & 10             & 10  & 10  & 10  & 10  & 10  & 10  & 10  & 10   & 10   & 10   & 10  & ---            & 24 \\
\hline
11           & 11             & 31  & 26  & 16  & 41   & 1  & 11  & 31  & 26   & 16   & 41    & 1  & 6              & 24 \\
\hline
12           & 12              & 9  & 18  & 36  & 27   & 9  & 18  & 36  & 27    & 9   & 18   & 36  & ---            & 24 \\
\hline
\end{tabular}
%\end{table}
\end{center}
} % \small

\vskip +10 pt

$J(45)$ berechnet sich nach Satz \ref{J_of_n}: $J(45) = J(3^2*5) = [3^1*2] * [1*4] = 24$.

\vskip +20 pt

Da $45$ keine Primzahl ist, gibt es nicht f"ur alle Werte von $a$ eine
''Multiplikative Ordnung'' (z.B. f"ur die nicht zu $45$ teilerfremden Zahlen 
$3, 5, 6, 9, 10, 12, \cdots,$ da $45 = 3^2*5$).\vskip +1em

{\bf Beispiel 3:} \\
Hat $7$ eine Primitivwurzel modulo $45$?\par

Die Voraussetzung/Bedingung ggT$(7,45)=1$ ist erf"ullt.
Aus der Tabelle (Werte von $a^{i}$ mod $45$) kann man ersehen, dass die Zahl $7$ keine Primitivwurzel von $45$ ist,
denn $ord_{45}(7) = 12 \not= 24 = J(45)$.\vskip + 1em

\newpage
{\bf Beispiel 4:} \\
Die folgende Tabelle\footnote{
In \hyperlink{AppArith3c}{Anhang D} Finden Sie den Quelltext zur Berechnung der Tabelle mit Mathematica und Pari-GP.
} beantwortet die Frage, ob die Zahl $7$ eine Primitivwurzel von $46$ ist.
Die Voraussetzung/Bedingung ggT$(7,46)=1$ ist erf"ullt.
%\newpage                                                                           
{
\subsubsection*{Werte von $a^i{\rm ~mod~}46, 1\leq a,i<23$:}
\textmd \small \label{SrcArith3c}
\begin{center}
\begin{tabular}{|p{16 pt}||@{\:}r@{\:}|@{\:}r@{\:}|@{\:}r@{\:}|@{\:}r@{\:}|@{\:}r@{\:}|@{\:}r@{\:}|@{\:}r@{\:}|@{\:}r@{\:}|@{\:}r@{\:}|@{\:}r@{\:}|@{\:}r@{\:}|@{\:}r@{\:}|@{\:}r@{\:}|@{\:}r@{\:}|@{\:}r@{\:}|@{\:}r@{\:}|@{\:}r@{\:}|@{\:}r@{\:}|@{\:}r@{\:}|@{\:}r@{\:}|@{\:}r@{\:}|@{\:}r@{\:}|@{\:}r@{\:}|c|}
\hline
$a \setminus i$   & 1 & 2 & 3 & 4 & 5 & 6 & 7 & 8 & 9 & 10 & 11 & 12 & 13 & 14 & 15 & 16 & 17 & 18 & 19 & 20 & 21 & 22 & 23 & ord \\
\hline
\hline
1    & 1  & 1  & 1  & 1  & 1  & 1  & 1  & 1  & 1  & 1  & 1  & 1  & 1  & 1  & 1  & 1  & 1  & 1  & 1  & 1  & 1  & 1  & 1 & 1    \\
\hline
2 & 2  & 4  & 8 & 16 & 32 & 18 & 36 & 26  & 6 & 12 & 24  & 2  & 4  & 8 & 16 & 32 & 18 & 36 & 26  & 6 & 12 & 24  & 2 & --    \\
\hline
3 & 3  & 9 & 27 & 35 & 13 & 39 & 25 & 29 & 41 & 31  & 1  & 3  & 9 & 27 & 35 & 13 & 39 & 25 & 29 & 41 & 31  & 1  & 3 & 11   \\
\hline
4  & 4 & 16 & 18 & 26 & 12  & 2  & 8 & 32 & 36  & 6 & 24  & 4 & 16 & 18 & 26 & 12  & 2  & 8 & 32 & 36  & 6 & 24  & 4 & --  \\
\hline
5 & 5 & 25 & 33 & 27 & 43 & 31 & 17 & 39 & 11  & 9 & 45 & 41 & 21 & 13 & 19  & 3 & 15 & 29  & 7 & 35 & 37  & 1  & 5 & 22  \\
\hline
6 & 6 & 36 & 32  & 8  & 2 & 12 & 26 & 18 & 16  & 4 & 24  & 6 & 36 & 32  & 8  & 2 & 12 & 26 & 18 & 16  & 4 & 24  & 6 & -- \\
\hline
7 & 7  & 3 & 21  & 9 & 17 & 27  & 5 & 35 & 15 & 13 & 45 & 39 & 43 & 25 & 37 & 29 & 19 & 41 & 11 & 31 & 33  & 1  & 7 & 22 \\
\hline
8 & 8 & 18  & 6  & 2 & 16 & 36 & 12  & 4 & 32 & 26 & 24  & 8 & 18  & 6  & 2 & 16 & 36 & 12  & 4 & 32 & 26 & 24  & 8 & --  \\
\hline
9 & 9 & 35 & 39 & 29 & 31  & 3 & 27 & 13 & 25 & 41  & 1  & 9 & 35 & 39 & 29 & 31  & 3 & 27 & 13 & 25 & 41  & 1  & 9 & 11  \\
\hline
10 & 10  & 8 & 34 & 18 & 42  & 6 & 14  & 2 & 20 & 16 & 22 & 36 & 38 & 12 & 28  & 4 & 40 & 32 & 44 & 26 & 30 & 24 & 10 & --  \\
\hline 
11 & 11 & 29 & 43 & 13  & 5  & 9  & 7 & 31 & 19 & 25 & 45 & 35 & 17  & 3 & 33 & 41 & 37 & 39 & 15 & 27 & 21  & 1 & 11 & 22 \\
\hline
12 & 12  & 6 & 26 & 36 & 18 & 32 & 16  & 8  & 4  & 2 & 24 & 12  & 6 & 26 & 36 & 18 & 32 & 16  & 8  & 4  & 2 & 24 & 12 & -- \\
\hline
13 & 13 & 31 & 35 & 41 & 27 & 29  & 9 & 25  & 3 & 39  & 1 & 13 & 31 & 35 & 41 & 27 & 29  & 9 & 25  & 3 & 39  & 1 & 13 & 11  \\
\hline
14 & 14 & 12 & 30  & 6 & 38 & 26 & 42 & 36 & 44 & 18 & 22 & 32 & 34 & 16 & 40  & 8 & 20  & 4 & 10  & 2 & 28 & 24 & 14 & -- \\
\hline
15 & 15 & 41 & 17 & 25  & 7 & 13 & 11 & 27 & 37  & 3 & 45 & 31  & 5 & 29 & 21 & 39 & 33 & 35 & 19  & 9 & 43  & 1 & 15 & 22 \\
\hline
16 & 16 & 26  & 2 & 32  & 6  & 4 & 18 & 12  & 8 & 36 & 24 & 16 & 26  & 2 & 32  & 6  & 4 & 18 & 12  & 8 & 36 & 24 & 16 & -- \\
\hline
17 & 17 & 13 & 37 & 31 & 21 & 35 & 43 & 41  & 7 & 27 & 45 & 29 & 33  & 9 & 15 & 25 & 11  & 3  & 5 & 39 & 19  & 1 & 17 & 22 \\
\hline
18 & 18  & 2 & 36  & 4 & 26  & 8  & 6 & 16 & 12 & 32 & 24 & 18  & 2 & 36  & 4 & 26  & 8  & 6 & 16 & 12 & 32 & 24 & 18 & -- \\
\hline
19  & 19 & 39  & 5  & 3 & 11 & 25 & 15  & 9 & 33 & 29 & 45 & 27  & 7 & 41 & 43 & 35 & 21 & 31 & 37 & 13 & 17  & 1 & 19 & 22 \\
\hline
20  & 20 & 32 & 42 & 12 & 10 & 16 & 44  & 6 & 28  & 8 & 22 & 26 & 14  & 4 & 34 & 36 & 30  & 2 & 40 & 18 & 38 & 24 & 20 & --  \\
\hline
21 & 21 & 27 & 15 & 39 & 37 & 41 & 33  & 3 & 17 & 35 & 45 & 25 & 19 & 31  & 7  & 9  & 5 & 13 & 43 & 29 & 11  & 1 & 21 & 22  \\
\hline
22 & 22 & 24 & 22 & 24 & 22 & 24 & 22 & 24 & 22 & 24 & 22 & 24 & 22 & 24 & 22 & 24 & 22 & 24 & 22 & 24 & 22 & 24 & 22 & -- \\
\hline
23 & 23 & 23 & 23 & 23 & 23 & 23 & 23 & 23 & 23 & 23 & 23 & 23 & 23 & 23 & 23 & 23 & 23 & 23 & 23 & 23 & 23 & 23 & 23 & -- \\
\hline 
\end{tabular}
\end{center}
}

\vskip +10 pt
$J(46)$ berechnet sich nach Satz \ref{J_of_pq}: $J(46) = J(2*23) = 1*22 = 22$.
Die Zahl $7$ ist eine Primitivwurzel von $46$, denn $ord_{46}(7) = 22 = J(46)$.

\begin{satz}\label{thm-zth-ordp}\footnote{
Vergleiche die obige Tabelle mit den Werten von $a^i$ (mod $11$), mit $1 \le a, i < 11$.
}${}^,$ \footnote{
Das Aussch"opfen des Wertebereiches\index{Wertebereich} ist eine wichtige kryptographische Eigenschaft 
(vergleiche Satz \ref{thm-zth-exhperm}).
} F"ur einen Modul $p$ und $a < p$ nimmt $a^i, i = 1, \cdots, p-1$ alle $J(p)$ Werte 
$1, \cdots, p-1$ genau dann einmal an, wenn $ord_p(a) = J(p)$.
\end{satz}


% +++++++++++++++++++++++++++++++++++++++++++++++++++++++++++++++++++++++++++++++++++++++++++++++++++++++++++++++++++++++++++++++

\subsection{Beweis des RSA-Verfahrens mit Euler-Fermat}
\hypertarget{RSABeweis}{} Mit dem Satz von Euler-Fermat kann man in der Gruppe $\mathbb{Z}_n^*$ das 
RSA\footnote{
Das RSA-Verfahren ist das verbreitetste asymmetrische \index{Verschl""usselung!asymmetrisch} Kryptoverfahren. Es
wurde 1978 von Ronald Rivest, Adi Shamir und Leonard Adleman entwickelt und
kann sowohl zum Signieren wie zum Verschl"usseln eingesetzt werden.
Kryptographen assoziieren mit der Abk"urzung \glqq RSA'' immer dieses Verfahren -
die folgende Anmerkung soll eher humorvoll zeigen, dass man jede
Buchstabenkombination mehrfach belegen kann: in Deutschland gibt es sehr
Interessen-lastige Diskussionen im Gesundheitswesen. Dabei wird mit \glqq RSA''
der vom Gesundheitsministerium geregelte \glqq {\bf R}isiko{\bf S}truktur{\bf A}usgleich'' in der
gesetzlichen Krankenversicherung mit einem j"ahrlichen Volumen von "uber 24
Milliarden DM bezeichnet.
}-Verfahren \glqq beweisen''.


% .......................................................................................................................
\subsubsection{Grundidee der Public Key-Kryptographie}\index{Kryptographie!Public Key}
\hypertarget{Kapitel_3_10_1}{}
Die Grundidee bei der Public Key-Kryptographie besteht darin, dass alle
Teilnehmer ein unterschiedliches Paar von Schl"usseln ($P$ und $S$) besitzen und
man f"ur alle Empf"anger die "offentlichen Schl"ussel publiziert. So wie man die
Telefonnummer einer Person aus dem Telefonbuch nachschl"agt, kann man den
"offentlichen Schl"ussel $P$ (public) des Empf"angers aus einem Verzeichnis
entnehmen. Au"serdem hat jeder Empf"anger einen geheimen Schl"ussel $S$ (secret),
der zum Entschl"usseln ben"otigt wird und den niemand sonst kennt. M"ochte der
Sender eine Nachricht $M$ (message) schicken, verschl"usselt er diese Nachricht
mit dem "offentlichen Schl"ussel $P$ des Empf"angers, bevor er sie abschickt:

der Chiffretext $C$ (ciphertext) ergibt sich mit $C = E (P; M)$, wobei $E$
(encryption) die Verschl"usselungsvorschrift ist.
Der Empf"anger benutzt seinen privaten Schl"ussel $S$, um die Nachricht wieder
mit der Entschl"usselungsvorschrift $D$ (decryption) zu entschl"usseln: 
$M = D(S; C)$.

Damit dieses System mit jeder Nachricht $M$ funktioniert, m"ussen folgende 4
Forderungen erf"ullt sein:
\begin{itemize}
    \item[\bf 1.] $D ( S; E (P; M) ) = M$ f"ur jedes $M$ (Umkehrbarkeit) und $M$ nimmt
\glqq sehr viele\grqq~ verschiedene Werte an.
    \item[\bf 2.] Alle $(S, P)$-Paare aller Teilnehmer sind verschieden (d.h. es mu"s viele davon geben).
    \item[\bf 3.] Der Aufwand, $S$ aus $P$ herzuleiten, ist mindestens so hoch, wie das Entschl"usseln
                               von $M$ ohne Kenntnis von $S$.
    \item[\bf 4.] Sowohl $C$ als auch $M$ lassen sich relativ einfach berechnen.
\end{itemize}

Die 1. Forderung ist eine generelle Bedingung f"ur alle kryptographischen
Verschl"usselungsalgorithmen.

Die 2. Forderung kann leicht sichergestellt werden, weil es \glqq sehr'' viele \index{Primzahlen!Anzahl}
Primzahlen gibt\footnote{
Nach dem Primzahlsatz (prime number theorem) von Legendre und Gauss gibt es
bis zur Zahl $n$ asymptotisch $n/{\rm ln}(n)$ Primzahlen. Dies bedeutet
beispielsweise: es gibt $6,5*10^{74}$ Primzahlen unterhalb von $n=2^{256}$
$(1,1*10^{77})$ und $3,2*10^{74}$ Primzahlen unterhalb von $n=2^{255}$. Zwischen $2^{256}$
und $2^{255}$ gibt es also allein $3,3*10^{74}$ Primzahlen mit genau $256$ Bits. Diese
hohe Zahl ist auch der Grund, warum man sie nicht einfach alle abspeichern kann.
} und weil dies durch eine zentrale Stelle, die Zertifikate ausgibt,
sichergestellt werden kann.

Die letzte Forderung macht das Verfahren "uberhaupt erst anwendbar. Dies
geht, weil die modulare Potenzierung in linearer Zeit m"oglich ist (da die
Zahlen l"angenbeschr"ankt sind).

W"ahrend Whitfield Diffie und Martin Hellman schon 1976 das generelle Schema
formulierten, fanden erst Rivest, Shamir und Adleman ein konkretes
Verfahren, das alle vier Bedingungen erf"ullte.



% .........................................................................................................................
\subsubsection{Funktionsweise des RSA-Verfahrens} \hypertarget{RSA}{}
Man kann die Einzelschritte zur Durchf"uhrung \index{RSA} des RSA-Verfahren
folgenderma"sen beschreiben (s. \cite[S. 213 ff]{Eckert2001} \index{Eckert 2001} und 
\cite[S. 338 ff]{Sedgewick1990}). \index{Sedgewick 1990}
Schritt 1 bis 3 sind die Schl"usselerzeugung, Schritt 4 und 5 sind die Verschl"usselung, 6 und 7 
die Entschl"usselung:
\begin{itemize}
\item[{\bf 1.}] W"ahle zuf"allig $2$ verschiedene Primzahlen\footnote{
Compaq hatte in 2000 mit hohem Marketingaufwand das sogenannte Multiprime-Verfahren eingef"uhrt.
$n$ war dabei das Produkt von zwei gro"sen und einer relativ dazu kleinen Primzahl:
$n=o*p*q.$ Nach Satz \ref{J_of_p1..pk} ist dann $ J(n)= (o-1)*(p-1)*(q-1).$ Das
Verfahren hat sich bisher nicht durchgesetzt.\\
Mit ein Grund daf"ur d"urfte sein, dass Compaq ein Patent\index{Patent} daf"ur angemeldet hat. Generell gibt es in Europa und in der Open Source Bewegung wenig Verst"andnis
f"ur Patente auf Algorithmen. "Uberhaupt kein Verst"andnis herscht au"serhalb der USA, dass man auf den Sonderfall (3 Faktoren) eines
Algorithmus (RSA) ein Patent beantragen kann, obwohl das Patent f"ur den allgemeinen Fall schon fast abgelaufen ist.
} \footnote{
F"ur Primzahlen $p$ und $q$ mit $p=q$, und $e,d$ mit $ed\equiv 1 \mod J(n)$
gilt i.a. nicht $ (m^{e})^d \equiv m \mod n \text{ f"ur alle } m <n.$
Seien z.B. $n=5^2$ berechnet sich $J(n)$ nach Satz \ref{thm-zth-exhperm}:
$~J(n)=5*4=20,~e=3,~d=7,~ed\equiv 21\equiv 1\mod~J(n).$ Dann gilt $ (5^3)^7 \equiv 0 \mod 25.$
} $p$ und $q$ und berechne $n = p*q$\footnote{
Das BSI \index{BSI} (Bundesamt f"ur Sicherheit in der Informationstechnik) empfiehlt, die Primfaktoren $p$ 
und $q$ ungef"ahr gleich gro"s zu w"ahlen, aber nicht zu dicht beieinander, d.h. konkret etwa
$$ 0.5 < |\log_2 (p) - \log_2 (q) | <30. $$ Die Primfaktoren werden unter Beachtung der genannten 
Nebenbedingung zuf"allig und unabh"angig voneinander erzeugt 
(Siehe \href{http://www.bsi.de/aufgaben/projekte/pbdigsig/download/kryptalg.pdf}
            {\tt http://www.bsi.de/aufgaben/projekte/pbdigsig/download/kryptalg.pdf}).
}. Der Wert $n$ wird als RSA-Modul 
bezeichnet\footnote{
In CrypTool wird der RSA-Modul mit einem gro"sen \glqq N'' bezeichnet.
}. 

\item[{\bf 2.}] W"ahle zuf"allig $e \in \{2, \cdots, n-1\}$, so dass gilt: \\
                $e$ ist teilerfremd zu $J(n) = (p-1)*(q-1)$.
                Zum Beispiel kann man $e$ so w"ahlen, dass gilt:  $\max(p,q) < e < J(n) - 1$\footnote{
                Das Verfahren erlaubt es auch, $d$ frei zu w"ahlen und dann $e$ zu berechnen.
                Dies hat aber praktische Nachteile.
                Normalerweise will man \glqq schnell'' verschl"usseln k"onnen und w"ahlt deshalb
                einen "offentlichen Exponenten $e$ so, dass er m"oglichst wenige bin"are Einsen
                hat. Damit ist eine schnelle Exponentiation bei der Verschl"usselung m"oglich. Als besonders
                praktisch haben sich hierf"ur die Primzahlen $3, 17$ und $65537$ erwiesen, da
                diese im Vergleich zum Modul $n$ sehr kleine Bitl"angen haben und in ihrer Bin"ardarstellung 
                nur wenige Einsen aufweisen. H"aufig verwendet
                wird die Zahl $65537 = 2^{16}+1$, also bin"ar: $10\cdots 0\cdots 01$ (diese
                Zahl ist prim und deshalb zu sehr vielen Zahlen teilerfremd).
                }. Danach kann man $p$ und $q$ \glqq wegwerfen''.
\item[{\bf 3.}] W"ahle $d \in \{1, \cdots, n-1\}$  mit  $e*d = 1$  mod $J(n)$,
                d.h. $d$ ist die multiplikative Inverse zu $e$ modulo $J(n)$ \footnote{
                Aus Sicherheitsgr"unden darf $d$ nicht zu klein sein.
                } \footnote{
                Je nach Implementierung wird zuerst $d$ oder zuerst $e$ bestimmt.
                }. Danach kann man $J(n)$ \glqq wegwerfen''.
                \begin{itemize}
                   \item[$\rightarrow$] $(n, e)$ ist der "offentliche Schl"ussel $P$.
                   \item[$\rightarrow$] $(n, d)$ ist der geheime Schl"ussel $S$ (es ist nur $d$ geheim zu halten).
                \end{itemize}
\item[{\bf 4.}] Zum Verschl"usseln wird die als (bin"are) Zahl dargestellte Nachricht in Teile 
                aufgebrochen, so dass jede Teilzahl kleiner als $n$ ist.
\item[{\bf 5.}] Verschl"usselung des Klartextes (bzw. seiner Teilst"ucke) $M \in \{1, \cdots, n-1\}$:
                $$C = E ((n, e); M ) := M^e {\rm ~mod~} n.$$
\item[{\bf 6.}] Zum Entschl"usseln wird das bin"ar als Zahl dargestellte Chiffrat in Teile 
                aufgebrochen, so dass jede Teilzahl kleiner als $n$ ist.
\item[{\bf 7.}] Entschl"usselung des Chiffretextes (bzw. seiner Teilst"ucke) $C \in \{1, \cdots, n-1\}$:
                $$M = D ( (n, d); C ) := C^d {\rm ~mod~} n.$$
\end{itemize}
Die Zahlen $d, e, n$ sind normalerweise sehr gro"s (z.B. $d$ und $e$ $300$ Bit, $n$ $600$ Bit).

{\bf Bemerkung:}\par
Die Sicherheit des RSA-Verfahrens h"angt wie bei allen Public Key-Verfahren davon ab,
dass man den privaten Key $d$ nicht aus  dem "offentlichen Key $(n,e)$ berechnen kann.

Beim RSA-Verfahren bedeutet dies, dass
\begin{enumerate}
  \item $J(n)$ f"ur gro"se zusammengesetzte $n$ schwer zu berechnen ist, und
  \item  $n$ f"ur gro"se $n$ nur schwer in seine Primfaktoren zerlegt werden kann (Faktorisierungsproblem).
\index{Faktorisierungsproblem}
\end{enumerate}



% ......................................................................................................................
\subsubsection{Beweis der Forderung 1 (Umkehrbarkeit)} \hypertarget{RSAproof}{}\label{RSAproof}

F"ur Schl"usselpaare $(n, e)$ und $(n, d)$, die die in den Schritten 1 bis 3 des RSA-Verfahrens 
festgelegten Eigenschaften besitzen, mu"s f"ur alle $M < n$ gelten:
$$M  \equiv  (M^e)^d  {\rm ~(mod~} n) \quad {\rm wobei} \quad  (M^e)^d  =  M^{e * d}.$$
Das hei"st, der oben angegebene Dechiffrieralgorithmus arbeitet korrekt.

Zu zeigen ist also:   $M^{e * d}  = M$  (mod $n$):

Wir zeigen das in 3 Schritten (s. \cite[S. 131ff]{Beutelspacher1996}).

{\bf Schritt 1:} 

Im ersten Schritt zeigen wir: 
$$M^{e * d} \equiv M{\rm ~(mod~}p).$$
Dies ergibt sich aus den Voraussetzungen und dem Satz von Euler-Fermat (Satz \ref{thm-zth-fermateuler}).
Da $n=p*q$ und $J(p*q)=(p-1)*(q-1)$ und da $e$ und $d$ so gew"ahlt sind, dass $e*d \equiv 1 {\rm ~(mod~}J(n))$,
gibt es eine ganze Zahl $k$, so dass gilt: $e*d = 1 + k*(p-1)*(q-1)$.
\begin{eqnarray}
M^{e * d}  & \equiv & M^{1+k*J(n)} \equiv M * M^{k*J(n)} \equiv M * M^{k*(p-1)*(q-1)}{\rm ~(mod~}p) \nonumber \\
           & \equiv & M * (M^{p-1})^{k*(q-1)}{\rm ~(mod~}p) \quad {\rm ~aufgrund~des~kleinen~Fermat:~} 
                  M^{p-1} \equiv 1 {\rm~(mod~}p) \nonumber \\ 
           & \equiv & M * (1)^{k*(q-1)} {\rm~(mod~}p) \nonumber \\
       & \equiv & M {\rm ~(mod~}p). \nonumber
\end{eqnarray}
Die Voraussetzung f"ur die Anwendung des vereinfachten Satzes von Euler-Fermat 
(kleiner-Fermat Satz \ref{thm-zth-fermat1}) war, dass $M$ und $p$ teilerfremd sind.

Da das im allgemeinen nicht gilt, m"ussen wir noch betrachten, was ist, wenn
$M$ und $p$ nicht teilerfremd sind: da $p$ eine Primzahl ist, mu"s dann
notwendigerweise $p$ ein Teiler von $M$ sein. Das hei"st aber: 
$$  M \equiv 0 {\rm ~(mod~}p) $$
Wenn $p$ die Zahl $M$ teilt, so teilt $p$ erst recht $M^{e * d}$. Also ist auch:
$$M^{e * d} \equiv 0 {\rm ~(mod~}p).$$
Da $p$ sowohl $M$ als auch $M^{e * d}$ teilt, teilt er auch ihre Differenz:
$$ (M^{e * d} - M ) \equiv 0 {\rm ~(mod~}p).$$
Und damit gilt auch in diesem Spezialfall unsere zu beweisende Behauptung.

{\bf Schritt 2:} 

V"ollig analog beweist man:  $M^{e * d} \equiv M{\rm ~(mod~}q)$.

{\bf Schritt 3:} 

Nun f"uhren wir die Behauptungen aus (a) und (b) zusammen f"ur $n=p*q$, um
zu zeigen:
$$ M^{e * d} \equiv M{\rm ~(mod~}n) {\rm ~f"ur~alle~} M < n. $$
Nach (a) und (b) gilt $(M^{e * d} - M) \equiv 0 {\rm ~(mod~} p)$ und $(M^{e * d} - M) \equiv 0 {\rm ~(mod~} q)$,
also teilen $p$ und $q$ jeweils dieselbe Zahl $z = (M^{e * d} - M)$.
Da $p$ und $q$ {\bf verschiedenen} Primzahlen sind, mu"s dann auch ihr Produkt diese Zahl $z$ teilen. Also gilt:
$$
(M^{e * d} - M) \equiv 0 {\rm ~(mod~}p*q) {\rm ~~oder~~ } M^{e * d} \equiv M {\rm ~(mod~}p*q) {\rm ~~oder~~} 
 M^{e * d} \equiv M {\rm ~(mod~}n).
$$

{\bf 1. Bemerkung:} \\
Man kann die 3 Schritte auch k"urzer zusammenfassen, wenn man Satz \ref{thm-zth-fermateuler} 
(Euler-Fermat) benutzt (also nicht den vereinfachten Satz, wo $n = p$ gilt und
der dem kleinen Satz von Fermat entspricht):
$$
(M^e)^d \equiv M^{e*d} \equiv M^{(p-1)(q-1)*k + 1} \equiv 
        (\underbrace{M^{(p-1)(q-1)}}_{\equiv M^{J(n)} \equiv 1 {\rm ~(mod~}n)})^k * M
    \equiv 1^k * M \equiv M {\rm ~(mod~}n).
$$

{\bf 2. Bemerkung:} \\
Beim Signieren werden die gleichen Operationen durchgef"uhrt, aber zuerst mit
dem geheimen Schl"ussel $d$, und dann mit dem "offentlichen Schl"ussel $e$. Das
RSA-Verfahren ist auch f"ur die Erstellung von digitalen
Signaturen\index{Signatur!digitale}\index{Signatur!RSA}\index{RSA-Signatur}
einsetzbar, weil gilt:
$$
M \equiv (M^d)^e{\rm ~(mod~}n).
$$



% +++++++++++++++++++++++++++++++++++++++++++++++++++++++++++++++++++++++++++++++++++++++++++++++++++++++++++++++
\subsection{Zur Sicherheit des RSA-Verfahrens}
\label{SecurityRSA}

Gro"se Teile dieses Absatzes entstammen dem Artikel \glqq Vorz"uge und Grenzen des RSA-Verfahrens\grqq 
\linebreak[4] von F. Bourseau / D. Fox / C. Thiel, {\em in Datenschutz und Datensicherheit} (DuD), 26/2002, S. 84-89.

Die Eignung des RSA-Verfahrens f"ur digitale Signaturen und Verschl"usselung ger"at immer wieder in die Diskussion, 
z.B. im Zusammenhang mit der Ver"offentlichung aktueller Faktorisierungserfolge. Ungeachtet dessen ist das RSA-Verfahren 
seit seiner Ver"offentlichung vor mehr als 20 Jahren unangefochtener De-facto-Standard.

Die Sicherheit des RSA-Verfahrens basiert - wie die aller kryptographischen Verfahren - auf den folgenden 4 zentralen
S"aulen:
\begin{itemize}
\item der Komplexit"at des dem Problem zugrunde liegenden zahlentheoretischen Problems (hier der Faktorisierung
      \index{Faktorisierungsproblem} gro"ser Zahlen),
\item der Wahl geeigneter Sicherheitsparameter (hier der L"ange des Moduls $n$),
\item der geeigneten Anwendung des Verfahrens sowie der Schl"usselerzeugung und
\item der korrekten Implementierung des Algorithmus.
\end{itemize}
Die Anwendung und Schl"usselerzeugung wird heute gut beherrscht. 
Die Implementierung ist auf Basis einer Langzahlarithmetik sehr einfach.

Grunds"atzliche Kenngr"o"sen eines bestimmten Verfahrens sind die ersten beiden Punkte.


\subsubsection{Komplexit"at}

\begin{sloppypar}
  Ein erfolgreiches Entschl"usseln oder eine Signaturf"alschung --- ohne
  Kenntnis des geheimen Schl"us"-sels --- erfordert, die $e$-te Wurzel mod
  $n$ zu ziehen.  Der geheime Schl"ussel, n"amlich die multiplikative
  Inverse zu $e$, kann mit Hilfe der Eulerschen Funktion $J(n)$ bestimmt
  werden. $J(n)$ wiederum l"asst sich aus den Primfaktoren von $n$
  berechnen.  Das Brechen kann daher nicht schwieriger sein als das
  Faktorisieren des Moduls $n$.
\end{sloppypar}
Das beste heute bekannte Verfahren ist eine Weiterentwicklung des urspr"unglich f"ur Zahlen mit einer bestimmten 
Darstellung (z.B. Fermatzahlen) entwickelten General Number Field Sieve (GNFS). \index{General Number Field Sieve (GNFS)}
Die L"osungskomplexit"at des Faktorisierungsproblems liegt damit asymptotisch bei 
$$
O(l) = e^{c \cdot (l \cdot \ln 2)^{1/3} \cdot  (\ln(l \cdot \ln(2))^{2/3} + o(l)}
$$
Siehe: 
\begin{list}{}{\setlength{\topsep}{-7 pt}}
  \item[] A. Lenstra / H. Lenstra: \\ {\em The development of the Number Field Sieve.} 
          Lecture Notes in Mathematics 1554, Springer, New York 1993
  \item[] Robert D. Silverman: \\ {\em A Cost-Based Security Analysis of Symmetric and Asymmetric Key Lengths.} 
          In: RSA Laboratories Bulletin, No. 13, April 2000, S. 1-22
\end{list} \vskip +12 pt

Die Formel zeigt, dass das Faktorisierungsproblem zur Komplexit"atsklasse der Probleme mit subexponentieller 
Berechnungskomplexit"at geh"ort (d.h. xxxx). Diese Einordnung entspricht dem heutigen Kenntnisstand, sie bedeutet 
jedoch nicht, dass das Faktorisierungsproblem m"oglicherweise nicht auch polynominell gel"ost werden kann.

$O(l)$ gibt die Zahl der durchschnittlich erforderlichen Prozessor-Operationen abh"angig von der L"ange $l$ der zu 
faktorisierenden Zahl $n$ an. F"ur den besten allgemeinen Faktorisierungsalgorithmus ist die Konstante 
$c = (64/9)^{1/173} = 1,923$.

Die umgekehrte Aussage, dass das RSA-Verfahren ausschlie"slich durch eine Faktorisierung von N gebrochen werden kann, 
ist bis heute nicht bewiesen. Zahlentheoretiker halten das ``RSA-Problem'' und das Faktorisierungsproblem f"ur 
komplexit"atstheoretisch "aquivalent.

\noindent
Siehe:
\begin{list}{}{\setlength{\topsep}{-7 pt}}
   \item[] A.J. Menezes / van Oorschot / P.C. Vanstone: \\
           {\em Handbook of Applied Cryptography}, CRC Press, 1996
\end{list}\vskip +12 pt
Bis jetzt ist es eine unbewiesene Annahme, dass f"ur das Faktorisierungsproblem kein L"osungsalgorithmus mit 
(asymptotisch) polynominellem Aufwand besteht (s. 3.11.3).


\subsubsection{Sicherheitsparameter}

Die Komplexit"at wird im wesentlichen von der L"ange $l$ des Moduls $N$ bestimmt.

1994 wurde mit einer verteilten Implementierung des 1982 von Pomerance entwickelten Quadratic Sieve-Algorithmus (QS) 
\index{Quadratic Sieve-Algorithmus (QS)} nach knapp 8 Monaten der 1977 ver"offentlichte 129-stellige RSA-Modul 
faktorisiert.

\noindent
Siehe:
\begin{list}{}{\setlength{\topsep}{-7 pt}}
   \item[] C. Pomerance: \\ {\em The quadratic sieve factoring algorithm.}
           In: G.R. Blakley / D. Chaum (Hrsg.): Proceedings of Crypto '84, LNCS 196, Springer Berlin 1995, S. 169-182
\end{list}\vskip +12 pt
1999 wurde mit dem von Buhler, Lenstra und Pomerance entwickelten General Number Field Sieve-Algorithmus (GNFS), 
der ab ca. 116 Dezimalstellen effizienter ist als QS, nach knapp 5 Monaten ein 154-stelliger Modul (512 Bit) faktorisiert.


\noindent
Siehe:
\begin{list}{}{\setlength{\topsep}{-7 pt}}
   \item[] J.P. Buhler / H.W. Lenstra / C. Pomerance: \\
           {\em Factoring integers with the number field sieve.} 
           In: A.K. Lenstra / H.W. Lenstra (Hrsg.): The Development of the Number Field Sieve, Lecture Notes 
           in Mathematics, Vol. 1554, Springer, Heidelberg 1993, S. 50-94
\end{list}\vskip +12 pt
Damit wurde klar, dass eine Modull"ange von $512$ Bit keinen Schutz mehr vor Angreifern darstellt.

In den letzten 20 Jahren wurden also erhebliche Fortschritte gemacht. Absch"atzungen "uber die zuk"unftige Entwicklung 
der Sicherheit unterschiedlicher RSA-Modull"angen differieren und h"angen von verschiedenen Annahmen ab:
\begin{itemize}
   \item Entwicklung der Rechnergeschwindigkeit (Gesetz von Moore: Verdopplung der Rechnerleistung alle 18 Monate) 
         und des Grid-Computing. \index{Gesetz von Moore}
   \item Entwicklung neuer Algorithmen.
\end{itemize}
Selbst ohne neue Algorithmen wurden in den letzten Jahren durchschnittlich ca. 10 Bit mehr Jahr berechenbar.
Gr"o"sere Zahlen erfordern mit den heute bekannten Verfahren einen immer gr"o"seren Arbeitsspeicher f"ur die 
L"osungsmatrix. Dieser Speicherbedarf w"achst ca. quadratisch zur Modull"ange $l$.
Deshalb d"urften gr"o"sere Fortschritte auch neue Algorithmen voraussetzen.

\begin{sloppypar}
  In dem Artikel ``Vorz"uge und Grenzen des RSA-Verfahrens'' prognostiziert
  Dirk Fox einen ann"ahernd linearen Verlauf der Faktorisierungserfolge
  unter Einbeziehung aller Faktoren: Pro Jahr kommen durchschnittlich 20
  Bit dazu.
\end{sloppypar}
\noindent
Die Firma Secorvo GmbH hat eine Stellungnahme zur Schl"ussell"angenempfehlung des BSI f"ur den Bundesanzeiger abgegeben. In Kapitel 2.3.1 geht sie sehr kompetent und verst"andlich auf RSA-Sicherheit ein (existiert nur in Deutsch):

\href{http://www.secorvo.de/publikat/stellungnahme-algorithmenempfehlung-020307.pdf}
     {\texttt{http://www.secorvo.de/publikat/stellungnahme-algorithmenempfehlung-020307.pdf}}



\subsubsection{Das Papier von D.J. Bernstein und seine Auswirkungen auf die Sicherheit des RSA-Algorithmus} \label{RSABernstein} \index{Faktorisierungsproblem}

Die Sicherheit des RSA-Algorithmus basiert auf der empirischen Beobachtung, dass die Faktorisierung gro"ser ganzer 
Zahlen ein schwieriges Problem ist. Besteht wie beim RSA-Algorithmus der zugrunde liegende Modul $n$ aus dem Produkt 
zweier gro"ser Primzahlen $p, q$ (typische L"angen: $p, q$  $500-600$ bit, $n$ $1024$ bit), so l"asst sich 
$n=pq$ aus $p$ und $q$ leicht bestimmen, jedoch ist es mit den bisher bekannten Faktorisierungsalgorithmen nicht
m"oglich, $p, q$ aus $n$ zu gewinnen. Nur mit Kenntnis von $p$ und $q$ l"asst sich jedoch der private aus dem 
"offentlichen Schl"ussel ermitteln.

Die Entdeckung eines Algorithmus zur effizienten Faktorisierung von Produkten $n=pq$ gro"ser Primzahlen w"urde daher 
den RSA-Algorithmus wesentlich beeintr"achtigen. Je nach Effizienz der Faktorisierung im Vergleich zur Erzeugung von 
$p, q, n$ m"usste der verwendete Modul $n$ (z.Zt. 1024 bit) erheblich vergr"o"sert oder --- im Extremfall --- auf den 
Einsatz des RSA ganz verzichtet werden.

Die im November 2001 ver"offentlichte Arbeit ``Circuits for integer factorization: a proposal''
(siehe \href{http://cr.yp.to/djb.html}{\texttt{http://cr.yp.to/djb.html}} von D.J. Bernstein behandelt 
das Problem der Faktorisierung gro"ser Zahlen. 
Die Kernaussage des Papers besteht darin, einen Faktorisierungsalgorithmus anzugeben, der mit gleichem Aufwand wie die 
besten bisher bekannten Algorithmen Zahlen n mit 3-mal gr"o"serer Stellenzahl (Bit-L"ange) faktorisieren kann.

Wesentlich bei der Interpretation des Resultats ist die Definition des Aufwandes: Als Aufwand wird das Produkt von 
ben"otigter Rechenzeit und Kosten der Maschine (insbesondere des verwendeten Speicherplatzes) angesetzt. Zentral f"ur 
das Ergebnis des Papiers ist die Beobachtung, dass ein wesentlicher Teil der Faktorisierung auf Sortierung zur"uckgef"uhrt 
werden kann und mit dem  Schimmlerschen Sortierschema ein Algorithmus zur Verf"ugung steht, der sich besonders gut f"ur 
den Einsatz von Parallelrechnern eignet. Am Ende des Abschnittes 3 gibt Bernstein konkret an, dass die Verwendung von 
$m^2$ Parallelrechnern mit jeweils gleicher Menge an Speicherplatz mit Kosten in der Gr"o"senordnung von $m^2$ einhergeht 
--- genau so wie ein einzelner Rechner mit $m^2$ Speicherzellen. Der Parallelrechner bew"altigt die Sortierung jedoch 
(unter Verwendung der o.g. Sortierverfahrens) in linearer Zeit proportional zu m, wohingegen der Rechner mit dem gro"sen 
Speicherplatz Zeit proportional $m^2$ ben"otigt. Verringert man daher den verwendeten Speicherplatz und erh"oht --- bei 
insgesamt gleich bleibenden Kosten --- die Anzahl der Prozessoren entsprechend, verringert sich die ben"otigte Zeit um 
die Gr"o"senordnung $1/m$. In Abschnitt 5 wird ferner angef"uhrt, dass der massive Einsatz der parallelisierten Elliptic 
Curve-Methode von Lenstra die Kosten der Faktorisierung ebenfalls um eine Gr"o"senordnung verringert (ein Suchalgorithmus 
hat dann quadratische statt kubische Kosten).
Alle Ergebnisse von Bernstein gelten nur asymptotisch f"ur gro"se Zahlen $n$. Leider liegen keine Absch"atzungen "uber 
den Fehlerterm, d.h. die Abweichung der tats"achlichen Zeit von dem asymptotischen Wert, vor --- ein Mangel, den auch 
Bernstein in seinem Papier erw"ahnt. Daher kann zur Zeit keine Aussage getroffen werden, ob die Kosten (im Sinne der 
Bernsteinschen Definition) bei der Faktorisierung z.Zt. verwendeter RSA-Zahlen (1024-2048 bit) bereits signifikant 
sinken w"urden.

Zusammenfassend l"asst sich sagen, dass der Ansatz von Bernstein durchaus innovativ ist. Da die Verbesserung der 
Rechenzeit unter gleichbleibenden Kosten durch einen massiven Einsatz von Parallelrechnern erkauft wird, stellt sich die 
Frage nach der praktischen Relevanz. Auch wenn formal der Einsatz von einem Rechner "uber 1 sec  und  1.000.000 Rechnern 
f"ur je 1/1.000.000 sec dieselben Kosten erzeugen mag, ist die Parallelschaltung von 1.000.000 Rechnern praktisch nicht 
(oder nur unter immensen Fixkosten, insbesondere f"ur die Vernetzung der Prozessoren) zu realisieren. Solche Fixkosten 
werden aber nicht in Ansatz gebracht.
Die Verwendung von distributed computing "uber ein gro"ses Netzwerk k"onnte einen Ausweg bieten. Auch hier m"ussten  
Zeiten und Kosten f"ur Daten"ubertragung einkalkuliert werden.

Solange noch keine (kosteng"unstige) Hardware oder verteilte Ans"atze (distributed computing) entwickelt wurden, die auf 
dem Bernsteinschen Prinzip basieren, besteht noch keine akute Gef"ahrdung des RSA. Es bleibt zu kl"aren, ab welchen 
Gr"o"senordnungen von n die Asymptotik greift. 

Zum Zeitpunkt der Erstellung dieses Kapitels (Mai 2002) war mir nichts dar"uber bekannt,
inwieweit die im Bernstein-Papier vorgeschlagenen theoretischen Ansatzes realisiert wurden oder
wieweit die Finanzierung seines Forschungsprojektes ist.

Verweise:
\begin{itemize}
  \item[] \href{http://cr.yp.to/djb.html}
               {\texttt{http://cr.yp.to/djb.html}}
  \item[] \href{http://www.counterpane.com/crypto-gram-0203.html\#6}
               {\texttt{http://www.counterpane.com/crypto-gram-0203.html\#6}}
  \item[] \href{http://www.math.uic.edu}
               {\texttt{http://www.math.uic.edu}}
\end{itemize}

\subsubsection{Anmerkungen zur Faktorisierung von Zahlen} \label{NoteFactorisation}

Eine hervorragende "Ubersicht "uber die Rekorde im Faktorisieren zusammengesetzter Zahlen \index{Faktorisierungsrekorde}
mit unterschiedlichen Methoden findet sich auf der Webseite 
\href{http://www.crypto-world.com}{\texttt{http://www.crypto-world.com}}. Der aktuelle Rekord
(Stand Juni 2000) mit der GNFS-Methode (General Number Field Sieve) \index{General Number Field Sieve (GNFS)}
liegt in der Zerlegung einer 158-stelligen Zahl in ihre beiden Primfaktoren (diese haben 73 und 86 Dezimalstellen).

Dieser Rekord, aufgestellt am 18. Januar 2002 von Forschern der Universit"at Bonn, fand deutlich 
weniger Aufmerksamkeit in der Presse als die L"osung der RSA-Challenge vom 22. Oktober 1999, als niederl"andische
Forscher die 155-stellige Zahl in ihre beiden 78--stellige Primfaktoren zerlegten.

Die Aufgabe der Bonner Wissenschaftler entsprang auch nicht einer Challenge, sondern die Aufgabe war, die letzten
Primfaktoren der Zahl $2^{953}+1$ zu finden. Als letzter Faktor blieb der sogenannte Teiler  ''C158'',
von dem man bis dahin wusste, dass er zusammengesetzt ist, aber man kannte seine Primfaktoren nicht:
$$
\begin{array}{c}
39505874583265144526419767800614481996020776460304936 \\
45413937605157935562652945068360972784246821953509354 \\
4305870490251995655335710209799226484977949442955603
\end{array}
$$
Die Faktorisierung ergab die beiden Primfaktoren:
$$
\begin{array}{c}
3388495837466721394368393204672181522 \\
815830368604993048084925840555281177
\end{array}
$$
und
$$
\begin{array}{c}
1165882340667125990314837655838327081813101 \\
2258146392600439520994131344334162924536139.
\end{array}
$$

% ++++++++++++++++++++++++++++++++++++++++++++++++++++++++++++++++++++++++++++++++++++++++++++++++++++++++++++
\newpage
\begin{center}
\fbox{\parbox{15cm}{{\em Joanne K.\index{Rowling} Rowling\footnotemark\:}\newline
Viel mehr als unsere F"ahigkeiten sind es 
unsere Entscheidungen ..., die zeigen, wer wir wirklich sind.}}
\end{center}

\addtocounter{footnote}{0}\footnotetext{Joanne K. Rowling, \glqq Harry Potter und die Kammer des Schreckens'', Carlsen,
(c) 1998, letztes Kapitel \glqq Dobbys Belohnung'', S. 343, Dumbledore.}
% +++++++++++++++++++++++++++++++++++++++++++++++++++++++++++++++++++++++++++++++++++++++++++++++++++++++++++++++
\subsection{Weitere zahlentheoretische Anwendungen in der Kryptographie}

In der modernen Kryptographie \index{Kryptographie!moderne} werden die Ergebnisse der modularen Arithmetik
extensiv angewandt. Hier werden exemplarisch einige wenige Beispiele aus der
Kryptographie mit kleinen\footnote{
\glqq Klein'' bedeutet beim RSA-Verfahren, dass die 
Bitl"angen der Zahlen sehr viel kleiner sind als $1024$ Bit (das sind $308$ Dezimalstellen). 
$1024$ Bit gilt im Moment in der Praxis als Mindestl"ange f"ur einen sicheren Certification
Authority-RSA-Modul.
} Zahlen vorgestellt.

Die Chiffrierung eines Textes besteht darin, dass man aus einer Zeichenkette
(Zahl) durch Anwenden einer Funktion (mathematische Operationen) eine andere
Zahl erzeugt. Dechiffrieren hei"st, diese Funktion umzukehren: aus dem
Zerrbild, das die Funktion aus dem Klartext gemacht hat, das Urbild
wiederherzustellen. Beispielsweise k"onnte der Absender einer vertraulichen
Nachricht zum Klartext $M$ eine geheimzuhaltende Zahl, den Schl"ussel $S$,
addieren und dadurch den Chiffretext $C$ erhalten:
$$ C = M + S. $$
Durch Umkehren dieser Operation, das hei"st durch Subtrahieren von $S$, kann
der Empf"anger den Klartext rekonstruieren:
$$ M = C - S. $$
Das Addieren von $S$ macht den Klartext zuverl"assig unkenntlich. Gleichwohl
ist diese Verschl"usselung sehr schwach; denn wenn ein Abh"orer auch nur ein
zusammengeh"origes Paar von Klar- und Chiffretext in die H"ande bekommt, kann
er den Schl"ussel berechnen
$$ S = C - M, $$
und alle folgenden mit $S$ verschl"usselten Nachrichten mitlesen.

% .......................................................................................
\subsubsection{Einwegfunktionen}
\hypertarget{Einwegfunktionen2}{} Der wesentliche Grund ist, dass Subtrahieren eine ebenso einfache Operation
ist wie Addieren. Wenn der Schl"ussel auch bei gleichzeitiger Kenntnis von
Klar- und Chiffretext nicht ermittelbar sein soll, braucht man eine
Funktion, die einerseits relativ einfach berechenbar ist - man will ja
chiffrieren k"onnen. Andererseits soll ihre Umkehrung zwar existieren (sonst
w"urde beim Chiffrieren Information verlorengehen), aber de facto
unberechenbar sein.

Was sind denkbare Kandidaten f"ur eine solche \index{Einwegfunktion} {\bf Einwegfunktion}? Man k"onnte an
die Stelle der Addition die Multiplikation setzen; aber schon Grundsch"uler
wissen, dass deren Umkehrung, die Division, nur geringf"ugig m"uhsamer ist als
die Multiplikation selbst. Man mu"s noch eine Stufe h"oher in der Hierarchie
der Rechenarten gehen: Potenzieren ist immer noch eine relativ einfache
Operation; aber ihre beiden Umkehrungen {\em Wurzelziehen} (finde $b$ in der
Gleichung $a = b^c$ , wenn $a$ und $c$ bekannt sind) und {\em Logarithmieren} (in
derselben Gleichung finde $c$, wenn $a$ und $b$ bekannt sind) sind so kompliziert,
dass ihre Ausf"uhrung in der Schule normalerweise nicht mehr gelehrt wird.

W"ahrend bei Addition und Multiplikation noch eine gewisse Struktur
wiedererkennbar ist, wirbeln Potenzierung und Exponentiation alle Zahlen
wild durcheinander: Wenn man einige wenige Funktionswerte kennt, wei"s man
(anders als bei Addition und Multiplikation) noch kaum etwas "uber die
Funktion im ganzen.

% ......................................................................................
\subsubsection{Das Diffie-Hellman Schl"usselaustausch-Protokoll (Key Exchange-Protokoll)} \index{Diffie-Hellman Key Exchange}
\index{Diffie Whitfield} \index{Hellman Martin} Das DH-Schl"usselaustauschprotokoll wurde 1976 in Stanford von Whitfield
Diffie, Martin E. Hellman und Ralph Merkle erdacht. \index{Schl""usselaustausch!Diffie-Hellman}

Eine Einwegfunktion dient Alice und Bob\footnote{
Alice und Bob werden standardm"a"sig als die beiden berechtigten Teilnehmer
eines Protokolls bezeichnet (siehe \cite[Seite 23]{Schneier1996}).
} dazu, sich einen Schl"ussel $S$, den Sessionkey, f"ur die nachfolgende
Verst"andigung zu verschaffen. Dieser ist dann ein Geheimnis, das nur diesen
beiden bekannt ist. Alice w"ahlt sich eine Zufallszahl $a$ und h"alt sie geheim.
Aus $a$ berechnet sie mit der Einwegfunktion die Zahl $A = g^a$ und schickt sie
an Bob. Der verf"ahrt ebenso, indem er eine geheime Zufallszahl $b$ w"ahlt,
daraus $B = g^b$ berechnet und an Alice schickt. Die Zahl $g$ ist beliebig und
darf "offentlich bekannt sein. Alice wendet die Einwegfunktion mit ihrer
Geheimzahl $a$ auf $B$ an, Bob tut gleiches mit seiner Geheimzahl $b$ und der
empfangenen Zahl $A$.

Das Ergebnis $S$ ist in beiden F"allen dasselbe, weil die Einwegfunktion
kommutativ ist: $g^{a*b} = g^{b*a}$. Aber selbst Bob kann Alices Geheimnis $a$ nicht aus
den ihm vorliegenden Daten rekonstruieren, Alice wiederum Bobs Geheimnis $b$
nicht ermitteln, und ein Lauscher, der $g$ kennt und sowohl $A$ als auch $B$
mitgelesen hat, vermag daraus weder $a$ noch $b$ noch $S$ zu berechnen.

\vskip +10 pt
\input{figures/DH-de.latex}
\vskip +10 pt

{\bf Ablauf:}\par
Alice und Bob wollen also einen geheimen Sessionkey $S$ "uber einen
abh"orbaren Kanal aushandeln.
\begin{itemize}
   \item[\bf 1.] Sie w"ahlen eine Primzahl $p$ und eine Zufallszahl $g$, und tauschen diese
                 Information offen aus.
   \item[\bf 2.] Alice w"ahlt nun $a$, eine Zufallszahl kleiner $p$ und h"alt diese geheim.

                 Bob w"ahlt ebenso $b$, eine Zufallszahl kleiner $p$ und h"alt diese geheim.
   \item[\bf 3.] Alice berechnet nun $A \equiv g^a {\rm ~(mod~} p)$. \\
                 Bob berechnet $B \equiv g^b {\rm ~(mod~} p)$.
   \item[\bf 4.] Alice sendet das Ergebnis $A$ an Bob. \\
                 Bob sendet das Ergebnis $B$ an Alice.
   \item[\bf 5.] Um den nun gemeinsam zu benutzenden Sessionkey zu bestimmen, potenzieren sie
                 beide jeweils f"ur sich das jeweils empfangene Ergebnis mit ihrer geheimen
                 Zufallszahl modulo $p$. Das hei"st:
         \begin{itemize}
                    \item[-] Alice berechnet $S \equiv B^a {\rm ~(mod~} p)$, und 
                    \item[-] Bob berechnet   $S \equiv A^b {\rm ~(mod~} p)$.
         \end{itemize}
\end{itemize}       
Auch wenn ein Spion $g, p$, und die Zwischenergebnisse $A$ und $B$ abh"ort, kann er den 
schlie"slich bestimmten Sessionkey, wegen der Schwierigkeit, den diskreten Logarithmus zu \index{Logarithmusproblem!diskret}
bestimmen, nicht berechnen.

\vskip +10 pt
Das ganze soll an einem Beispiel mit (unrealistisch) kleinen Zahlen gezeigt
werden.\vskip +1em

{\bf Beispiel in Zahlen:} 
\begin{itemize}
   \item[\bf 1.] Alice und Bob w"ahlen $g = 11$, $p = 347$.
   \item[\bf 2.] Alice w"ahlt $a = 240$, Bob w"ahlt $b = 39$ und behalten $a$ und $b$ geheim.
   \item[\bf 3.] Alice berechnet $A \equiv g^a \equiv 11^{240} \equiv 49 {\rm ~(mod~} 347).$ \\
                 Bob berechnet $B \equiv g^b \equiv 11^{39} \equiv 285 {\rm ~(mod~} 347).$
   \item[\bf 4.] Alice sendet Bob: $A = 49$, \\
                 Bob sendet Alice: $B = 285$.
   \item[\bf 5.] Alice berechnet $B^a \equiv 285^{240} \equiv 268 {\rm ~(mod~}347),$ \\
                 Bob berechnet $A^b \equiv 49^{39} \equiv 268 {\rm ~(mod~}347)$.
\end{itemize}
Nun k"onnen Alice und Bob mit Hilfe ihres gemeinsamen Sessionkeys sicher
kommunizieren. Auch wenn ein Spion alles, was "uber die Leitung ging, abh"orte:
$g = 11, p = 347, A = 49$ und $B = 285$, den geheimen Schl"ussel kann er nicht
berechnen.

{\bf Bemerkung:} \\
In diesem Beispiel mit den kleinen Zahlen ist das Diskrete Logarithmus-Problem\index{Logarithmusproblem!diskret} l"osbar,
aber mit gro"sen Zahlen ist es kaum zu l"osen\footnote{
Versucht man mit Mathematica, den diskreten Logarithmus\index{Logarithmusproblem!diskret} $x$, der die Gleichung
$11^x \equiv 49 {\rm ~(mod~}347)$ l"ost, mit Hilfe von Solve zu bestimmen, erh"alt man die
{\em tdep}-Message ''The equations appear to involve the variables to be solved for
in an essentially non-algebraic way''. Mathematica meint also, keine
algebraisch direkten Verfahren zu kennen, um die Gleichung zu l"osen.
Doch mit der verallgemeinerten Funktion f"ur die multiplikative Ordnung kann
es Mathematica (hier f"ur Alice): 
{\texttt{MultiplicativeOrder[11, 347, 49]}} liefert den Wert $67$. \\
In Pari-GP lautet die Syntax: {\texttt{znlog(Mod(49,347),Mod(11,347))}}
\\
Auch mit anderen Tools wie dem \index{LiDIA} LiDIA- oder BC-Paket (siehe Anhang URLs)
k"onnen solche zahlentheoretischen Aufgaben gel"ost werden.
Die Funktion {\tt dl} im Userinterface {\bf LC} zu LiDIA liefert mit {\tt dl(11,49,347)}
ebenfalls den Wert $67$.
}${}^,$ \footnote{
Warum haben die Funktionen bei dem DL-Problem f"ur Alice den Wert $67$ und nicht den Wert 240 geliefert?
Der diskrete Logarithmus ist der kleinste nat"urliche Exponent, der die
Gleichung $11^x \equiv 49{\rm ~(mod~}347)$ l"ost. Sowohl $x=67$ als auch $x=240$ (die im Beispiel
gew"ahlte Zahl) erf"ullen die Gleichung und k"onnen damit zur Berechnung des
Sessionkeys benutzt werden: $285^{240}  \equiv 285^{67} \equiv 268 {\rm ~(mod~}347)$.
H"atten Alice und Bob als Basis $g$ eine Primitivwurzel modulo $p$ gew"ahlt, dann
gibt es f"ur jeden Rest aus der Menge
$\{1, 2, \cdots, p-1\}$ genau einen Exponenten aus der Menge $\{0, 1, \cdots, p-2\}.$ \\
Info: Zum Modul $347$ gibt es $172$ verschiedene Primitvwurzeln, davon sind $32$
prim (ist nicht notwendig).
Da die im Beispiel f"ur $g$ gew"ahlte Zahl $11$ keine Primitivwurzel von $347$ ist,
nehmen die Reste nicht alle Werte aus der Menge $\{1, 2, \cdots, 346\}$ an. Somit
kann es f"ur einen bestimmten Rest mehr als einen oder auch gar keinen
Exponenten aus der Menge $\{0, 1, \cdots, 345\}$ geben, der die Gleichung erf"ullt. \\
{\tt 
PrimeQ[347] = True; EulerPhi[347] = 346; GCD[11, 347] = 1;
MultiplicativeOrder[11, 347] = 173} \\
In Pari-GP lautet die Syntax:\\
{\tt isprime(347); eulerphi(347); gcd(11,347); znorder(Mod(11,347))}
\vskip +10 pt
\begin{tabular}{|c|c|l|}
\hline
i  & $11^i{\rm ~mod~}347$ & \\
\hline
      0  &          1   &  \\
      1  &         11   &  \\                                     
      2  &        121   &  \\                                     
      3  &        290   &  \\                                     
     67  &         49   & gesuchter Exponent  \\                    
    172  &        284   &  \\                                                  
    173  &          1   &= Multiplikative Ordnung von $11 {\rm  ~(mod~} 347)$ \\ 
    174  &         11   &  \\                                                     
    175  &        121   &  \\                                     
    176  &        290   &  \\                                     
    240  &         49   & gesuchter Exponent  \\                     
\hline
\end{tabular}
\vskip +6 pt
}.

Zu berechnen ist hier: \\
Von Alice: $11 ^ x \equiv 49 {\rm ~(mod~}347)$, also $\log_{11}(49) {\rm ~(mod~}347).$\\
Von Bob: $11 ^ y \equiv 285 {\rm ~(mod~}347)$, also $\log_{11}(285){\rm ~(mod~}347)$.



% +++++++++++++++++++++++++++++++++++++++++++++++++++++++++++++++++++++++++++++++++++++++++++++++++++++++++++++++++
\subsection{Das RSA-Verfahren mit konkreten Zahlen} \hypertarget{RSAKonkret}{}
\index{RSA}
\hypertarget{Kapitel_3_12}{}
Nachdem oben \hyperlink{RSA}{die Funktionsweise des RSA-Verfahrens} beschrieben wurde, sollen diese 
Schritte hier mit konkreten, aber kleinen Zahlen durchgef"uhrt werden.

\subsubsection{RSA mit kleinen Primzahlen und mit einer Zahl als Nachricht}
Bevor wir RSA auf einen Text anwenden, wollen wir es erst direkt mit einer
Zahl zeigen\footnote{
Mit CrypTool v1.3 k"onnen Sie dies per \glqq Einzelverfahren/RSA-Demo/RSA-Kryptosystem'' l"osen.
}.
\begin{itemize}
  \item[\bf 1.] Die gew"ahlten Primzahlen seien $p=5$ und $q=11$. \\
                Also ist $n=55$ und $J(n) = (p-1)*(q-1)=40$.
  \item[\bf 2.] $e = 7$ (sollte zwischen $11$ und $40$ liegen und mu"s teilerfremd zu $40$ sein).
  \item[\bf 3.] $d = 23$ (da $23*7 \equiv 161 \equiv 1{\rm ~(mod~}40)$)
  \begin{itemize} 
     \item[] $\rightarrow$ Public Key des Senders: $(55, 7),$
     \item[] $\rightarrow$ Private Key des Empf"angers: $(55, 23).$ 
  \end{itemize}
  \item[\bf 4.] Nachricht sei nur die Zahl $M = 2$ (also ist kein Aufbrechen in Bl"ocke n"otig).
  \item[\bf 5.] Verschl"usseln: $C \equiv 2^7 \equiv 18 {\rm ~(mod~}55).$
  \item[\bf 6.] Chiffrat ist nur die Zahl $C = 18$ (also kein Aufbrechen in Bl"ocke n"otig).
  \item[\bf 7.] Entschl"usseln: $M \equiv 18^{23} \equiv 18^{(1+2+4+16)} \equiv 18*49*36*26 \equiv 2 {\rm ~(mod~}55).$
\end{itemize}
Nun wollen wir RSA auf einen Text anwenden: zuerst mit dem
Gro"sbuchstabenalphabet (26 Zeichen), dann mit dem gesamten
ASCII-Zeichensatz als Bausteine f"ur die Nachrichten.

% ..............................................................................................................
\subsubsection{RSA mit etwas gr"o"seren Primzahlen und einem Text aus Gro"sbuchstaben}
Gegeben ist der Text \glqq ATTACK AT DAWN'' und die Zeichen werden auf folgende
einfache Weise \index{Blockl""ange} codiert\footnote{
Mit CrypTool v1.3 k"onnen Sie dies per 
\glqq Einzelverfahren/RSA-Demo/RSA-Kryptosystem'' l"osen. Dies ist auch im Tutorial/Szenario der Online-Hilfe 
zu CrypTool beschrieben [Optionen, Alphabet vorgeben, Basissystem, Blockl"ange 2 und Dezimaldarstellung].
}:
\vskip +5 pt
\begin{tabular}{|c|l||c|l|}
\hline
Zeichen & Zahlenwert & Zeichen & Zahlenwert \\
\hline
\hline
Blank    & 0   & M  & 13 \\
A        & 1   & N    & 14 \\ 
B        & 2   & O    & 15 \\ 
C        & 3   & P    & 16 \\  
D        & 4   & Q    & 17 \\ 
E        & 5   & R    & 18 \\ 
F        & 6   & S    & 19 \\  
G        & 7   & T    & 20 \\  
H        & 8   & U    & 21 \\ 
I        & 9   & V    & 22 \\   
J       & 10   & W    & 23 \\  
K       & 11   & X    & 24 \\ 
L       & 12   & Y    & 25 \\
&              & Z    & 26 \\
\hline
\end{tabular}  
\hypertarget{Grossbuchstaben-Alphabet}{}

{\bf Tabelle Gro"sbuchstabenalphabet}\index{Gro""sbuchstabenalphabet}

\vskip +20 pt

{\bf Schl"usselerzeugung (Schritt 1 bis 3):} \\
%\vskip 5 pt
{\bf 1.} $p=47, q=79$ $( n= 3713;~ J(n) = (p-1)*(q-1)=3.588).$ \\
{\bf 2.} $e=37$ (sollte zwischen 79 und 3588 liegen und mu"s teilerfremd zu $3588$ sein). \\
{\bf 3.} $d=97$ (denn $e*d=1{\rm ~mod~}J(n); 37*97 \equiv 3.589 
\equiv 1{\rm ~(mod~}3588) \;$)\footnote{
Wie man $d = 97$ mit Hilfe des erweiterten ggT berechnet wird im \hyperlink{Appendix_A}{Anhang A} gezeigt.
}.

{\bf 4. Verschl"usselung:} \\ 
{\tt
\begin{tabular}{rcccccccccccccccccccc}
{\rm Text:} & A & T & T & A & C & K & & A & T &  & D & A & W & N \\
{\rm Zahl:} & 01 & 20 & 20 & 01 & 03 & 11 & 00 & 01 & 20 & 00 & 04 & 01 & 23 & 14
\end{tabular}
}
\index{Blockl""ange}

Aufteilung dieser 28-stelligen Zahl in 4-stellige Teile (denn $2626$ ist noch kleiner als $n=3713$), d.h. dass
die Blockl"ange 2 betr"agt.\\
{\tt 0120 2001 0311 0001 2000 0401 2314}

\label{SrcArith4a}
Verschl"usselung aller 7 Teile jeweils per: $C \equiv M^{37}{\rm ~(mod~}$3713)\footnote{
In \hyperlink{AppArith4a}{Anhang D} finden Sie den Beispiel-Quelltext zur RSA-Verschl"usselung mit
Mathematica und Pari-GP. \\
Mit CrypTool v1.3 k"onnen Sie dies per \glqq Einzelverfahren/RSA-Demo/RSA-Kryptosystem'' l"osen. 
}: \\
{\tt 1404 2932 3536 0001 3284 2280 2235}


{\bf 5. Entschl"usselung:} \\ 
Chiffrat: {\tt 1404 2932 3536 0001 3284 2280 2235 }

Aufteilung dieser 28-stelligen Zahl in 4-stellige Teile.\\
Entschl"usselung aller $7$ Teile jeweils per: $M \equiv C^{97}{\rm ~(mod~}3.713)$: \\
{\tt 0120 2001 0311 0001 2000 0401 2314}

Umwandeln von 2-stelligen Zahlen in Gro"sbuchstaben und Blanks.\\
Bei den gew"ahlten Werten ist es f"ur einen Kryptoanalytiker einfach, aus den
"offentlichen Parametern $n=3713$ und $e=37$ die geheimen Werte zu finden, indem
er offenlegt, dass $3713 = 47 * 79$.

Wenn $n$ eine $768$-Bit-Zahl ist, bestehen daf"ur -- nach heutigen Kenntnissen -- wenig Chancen.

% ......................................................................................................
\subsubsection{RSA mit noch etwas gr"o"seren Primzahlen und mit einem Text aus ASCII-Zeichen}

Real wird das ASCII-Alphabet benutzt, um die Einzelzeichen der Nachricht in
8-Bit lange Zahlen zu codieren. 

Diese Aufgabe\footnote{
Mit CrypTool v1.3 k"onnen Sie dies per \glqq Einzelverfahren/RSA-Demo/RSA-Kryptosystem'' l"osen.
} ist angeregt durch das Beispiel aus \cite[S. 215]{Eckert2001}. \index{Eckert 2001}

Der Text \glqq RSA works!'' bedeutet in Dezimalschreibweise codiert: 

{\tt
\begin{tabular}{rcccccccccccccccccccc}
{\rm Text:} & R & S & A &   & w & o & r & k & s & ! \\
{\rm Zahl:} & 82 & 83 & 65 & 32 & 119 & 111 & 114 & 107 & 115 & 33 
\end{tabular} } % \tt

Das Beispiel wird in 2 Varianten durchgespielt. Gemeinsam f"ur beide sind die Schritte $1$ bis $3$.
\vskip +10pt
{\bf Schl"usselerzeugung (Schritt 1 bis 3):} \\ \label{SrcArith4b}
{\bf 1.} $p=509, q=503 \quad (n= 256.027; \; ~J(n)=(p-1)*(q-1)=255.016=2^3*127*251)$\footnote{
In \hyperlink{AppArith4b}{Anhang D} finden Sie den Quelltext zur Faktorisierung von $J(n)$ mit Mathematica und Pari-GP. \\
Mit CrypTool v1.3 k"onnen Sie dies per ``Einzelverfahren/RSA-Demo/Faktorisieren'' l"osen.
}. \\
{\bf 2.} $e=65.537$ (soll zwischen $509$ und $255.016$ liegen u. mu"s teilerfremd
zu $255.016$ sein)\footnote{
$e$ soll also nicht $2, 127$ oder $251$ sein.\\
Real wird $J(n)$ nicht faktorisiert, sondern f"ur das gew"ahlte $e$ wird mit dem Euklidschen Algorithmus
sichergestellt, dass ggT$(d,J(n))=1$.
}.\\
{\bf 3.} $d=231.953$ (denn $e \equiv d^{-1}{\rm ~mod~}J(n); ~65.537*231.953 \equiv 15.201.503.761 \equiv 1
{\rm ~(mod~}255.016))$\footnote{
Andere m"ogliche Kombinationen von (e,d) sind z.B.: (3, 170.011), (5, 204.013), (7, 36.431).
}.


\subsubsection*{Variante 1:\\ Alle ASCII-Zeichen werden einzeln ver- und entschl"usselt (keine Blockbildung).} 

{\bf 4. Verschl"usselung:} \\ 
{\tt
\begin{tabular}{rcccccccccccccccccccc}
{\rm Text:} & R & S & A &   & w & o & r & k & s & ! \\
{\rm Zahl:} & 82 & 83 & 65 & 32 & 119 & 111 & 114 & 107 & 115 & 33 
\end{tabular} } % \tt

Keine Zusammenfassung der Buchstaben\footnote{
F"ur sichere Verfahren braucht man gro"se Zahlen, die m"oglichst alle Werte bis
$n-1$ annehmen. Wenn die m"ogliche Wertemenge der Zahlen in der Nachricht zu
klein ist, nutzen auch gro"se Primzahlen nichts f"ur die Sicherheit.
Ein ASCII-Zeichen ist durch $8$ Bits repr"asentiert. Will man gr"o"sere Werte,
mu"s man mehrere Zeichen zusammenfassen. Zwei Zeichen ben"otigen 16 Bit, womit
maximal der Wert 65.536 darstellbar ist; dann mu"s der Modul $n$ gr"o"ser sein als
$2^{16} = 65.536$. Dies wird in Variante 2 angewandt.
Beim Zusammenfassen bleiben in der Bin"ar-Schreibweise die f"uhrenden Nullen
erhalten (genauso wie wenn man oben in der Dezimalschreibweise alle Zahlen
3-stellig schreiben w"urde und dann die Folge {\tt 082 083, 065 032,
119 111, 114 107, 115 033} h"atte). 
}!

\label{SrcArith4c}
Verschl"usselung pro Zeichen per: $C \equiv M^{65.537}{\rm ~(mod~}256.027)$\footnote{
In \hyperlink{AppArith4c}{Anhang D} finden Sie den Quelltext zur RSA-Exponentiation mit Mathematica und Pari-GP.
}:

{\tt
\begin{tabular}{lllll}
212984 & 025546 & 104529 & 031692 & 248407 \\
100412 & 054196 & 100184 & 058179 & 227433\\
\end{tabular} 
}

{\bf 5. Entschl"usselung:}\\
Chiffrat: \\ 
{\tt
\begin{tabular}{lllll}
212984 & 025546 & 104529 & 031692 & 248407 \\
100412 & 054196 & 100184 & 058179 & 227433\\
\end{tabular} }

Entschl"usselung pro Zeichen per: $M \equiv C^{231.953}{\rm ~mod~}256.027$: \\
{\tt 82 83 65 32 119 111 114 107 115 33}

\subsubsection*{Variante 2: Jeweils zwei ASCII-Zeichen werden als Block ver- und entschl"usselt.} 

Bei der Variante 2 wird die Blockbildung in zwei verschiedenen Untervarianten 4./5. und 4'./5'. dargestellt.

{\tt
\begin{tabular}{rcccccccccccccccccccc}
{\rm Text:} & R & S & A &   & w & o & r & k & s & ! \\
{\rm Zahl:} & 82 & 83 & 65 & 32 & 119 & 111 & 114 & 107 & 115 & 33 
\end{tabular} } % \tt

{\bf 4. Verschl"usselung:} \\
Blockbildung\footnote{
\\ \tt \begin{tabular}{lll}
& Bin"ardarstellung  &Dezimaldarstellung\\
01010010, 82 & 01010010 01010011 & =21075 \\
01010011, 83 & \\
01000001, 65 & 01000001 00100000 & =16672  \\
00100000, 32  \\
01110111, 119 & 01110111 01101111 & =30575 \\
01101111, 111 \\ 
01110010, 114 & 01110010 01101011 & =29291 \\
01101011, 107 \\
01110011, 115 & 01110011 00100001 & =29473 \\
00100001, 33: 
\end{tabular}
} (die ASCII-Zeichen werden als 8-stellige Bin"arzahlen hintereinander geschrieben):\\
{\tt 21075 16672 30575 29291 29473}\footnote{
Mit CrypTool v1.3 k"onnen Sie dies per \glqq Einzelverfahren/RSA-Demo/RSA-Kryptosystem'' mit den 
folgenden Optionen l"osen: alle $256$ Zeichen, b-adisch, Blockl"ange 2, dezimale Darstellung.
}

\label{SrcArith4d}
Verschl"usselung pro Block per: $C \equiv M^{65.537}{\rm ~(mod~}256027)$\footnote{
In \hyperlink{AppArith4d}{Anhang D} finden Sie den Quelltext zur RSA-Exponentiation mit Mathematica und Pari-GP.
}: \\
{\tt 158721 137346 37358 240130 112898}

{\bf 5. Entschl"usselung:} \\
Chiffrat: \\
{\tt 158721 137346 37358 240130 112898}

Entschl"usselung pro Block per: $M \equiv C^{231.953}{\rm ~(mod~}256.027)$: \\
{\tt 21075 16672 30575 29291 29473}

\vskip +10 pt
%{\bf Umwandeln:} jeden Block bin"ar in 2 Zahlen; danach jede Zahl in ASCII-Zeichen.

{\bf 4`. Verschl"usselung:} 

Blockbildung (die ASCII-Zeichen werden als 3-stellige Dezimalzahlen hintereinander geschrieben): \\
{\tt 82083 65032 119111 114107 115033}\footnote{
Die RSA-Verschl"usselung mit dem Modul $n=256.027$ ist bei dieser Einstellung korrekt,
da die ASCII-Bl"ocke in Zahlen kleiner oder gleich $255.255$ kodiert werden.
} 

\label{SrcArith4e}
Verschl"usselung pro Block per: $C \equiv M^{65537}{\rm ~(mod~}256.027)$ \footnote{
In \hyperlink{AppArith4e}{Anhang D} finden Sie den Quelltext zur RSA-Exponentiation mit Mathematica und Pari-GP.
}: \\ 
{\tt 198967 051405 254571 115318 014251}

{\bf 5'. Entschl"usselung:} \\
Chiffrat: \\
{\tt 198967 051405 254571 115318 014251}

Entschl"usselung pro Block per: $M \equiv C^{231.953}{\rm ~(mod~}256.027)$: \\
{\tt 82083 65032 119111 114107 115033} 



% ..........................................................................................................................
\subsubsection{Eine kleine RSA-Cipher-Challenge (1)}
\index{RSA!Cipher-Challenge}
Die Aufgabe stammt aus \cite[Exercise 4.6]{3Stinson1995}: \index{Stinson 1995}
Die pure L"osung hat Prof. Stinson unter

\href{http://www.cacr.math.uwaterloo.ca/~dstinson/solns.html}
     {\texttt{http://www.cacr.math.uwaterloo.ca/\~{}dstinson/solns.html}}\footnote{
oder
\href{http://bibd.unl/~stinson/solns.html}
     {\texttt{http://bibd.unl/\~{}stinson/solns.html}}.
}

ver"offentlicht. Es geht aber nicht nur um das Ergebnis, sondern vor allem um die Einzelschritte der L"osung, 
also um die Darlegung der Kryptoanalyse\footnote{
Im Szenario der Online-Hilfe zu CrypTool und in der Pr"asentation auf der
Web-Seite wird der L"osungsweg skizziert. Wenn uns jemand einen gut
aufbereiteten konkreten L"osungsweg schickt, nehmen wir ihn gerne in die
Dokumentation auf.
}.

Two samples of RSA ciphertext are presented in Tables 4.1 and 4.2. Your task
is to decrypt them. The public parameters of the system are 

$n = 18923$ and $e = 1261$ (for Table 4.1) and \\
$n = 31313$ and $e = 4913$ (for Table 4.2).

This can be accomplished as follows. First, factor $n$ (which is easy because
it is so small). Then compute the exponent $d$ from $J(n)$, and, finally,
decrypt the ciphertext. Use the square-and-multiply \index{Square and multiply} algorithm to
exponentiate modulo $n$.

In order to translate the plaintext back into ordinary English text, you
need to know how alphabetic characters are ''encoded'' as elements in $\mathbb{Z}_n$. Each
element of $\mathbb{Z}_n$ represents three alphabetic characters as in the following
examples: 

{\tt \begin{tabular}{lll}
DOG & $\mapsto$ & $3 * 26^2 + 14 * 26 + 6= 2398$ \\
CAT & $\mapsto$ & $2 * 26^2 + 0 * 26 + 19 = 1371$ \\
ZZZ & $\mapsto$ & $25 * 26^2 + 25 * 26 + 25 = 17575$. 
\end{tabular} }

You will have to invert this process as the final step in your program.

The first plaintext was taken from ''The Diary of Samuel Marchbanks'', by
Robertson Davies, 1947, and the second was taken from ''Lake Wobegon Days'',
by Garrison Keillor, 1985.

\newpage
TABLE 4.1\footnote{
Die Zahlen dieser Tabelle k"onnen Sie mit Copy und Paste weiter bearbeiten.
}: RSA Ciphertext

{\tt 
\begin{tabular}{llllllll}
12423 & 11524  & 7243  & 7459 & 14303  & 6127 & 10964 & 16399 \\
 9792 & 13629 & 14407 & 18817 & 18830 & 13556  & 3159 & 16647 \\
 5300 & 13951    & 81  & 8986  & 8007 & 13167 & 10022 & 17213 \\
 2264   & 961 & 17459  & 4101  & 2999 & 14569 & 17183 & 15827 \\
12693  & 9553 & 18194  & 3830  & 2664 & 13998 & 12501 & 18873 \\
12161 & 13071 & 16900  & 7233  & 8270 & 17086  & 9792 & 14266 \\
13236  & 5300 & 13951  & 8850 & 12129  & 6091 & 18110  & 3332 \\
15061 & 12347  & 7817  & 7946 & 11675 & 13924 & 13892 & 18031 \\
 2620  & 6276  & 8500   & 201  & 8850 & 11178 & 16477 & 10161 \\
 3533 & 13842  & 7537 & 12259 & 18110    & 44  & 2364 & 15570 \\
 3460  & 9886  & 8687  & 4481 & 11231  & 7547 & 11383 & 17910 \\
12867 & 13203  & 5102  & 4742  & 5053 & 15407  & 2976  & 9330 \\
12192    & 56  & 2471 & 15334   & 841 & 13995 & 17592 & 13297 \\
 2430  & 9741 & 11675   & 424  & 6686   & 738 & 13874  & 8168 \\
 7913  & 6246 & 14301  & 1144  & 9056 & 15967  & 7328 & 13203 \\
  796   & 195  & 9872 & 16979 & 15404 & 14130  & 9105  & 2001 \\
 9792 & 14251  & 1498 & 11296  & 1105  & 4502 & 16979  & 1105 \\
   56  & 4118 & 11302  & 5988  & 3363 & 15827  & 6928  & 4191 \\
 4277 & 10617   & 874 & 13211 & 11821  & 3090 & 18110    & 44 \\
 2364 & 15570  & 3460  & 9886  & 9988  & 3798  & 1158  & 9872 \\
16979 & 15404  & 6127  & 9872  & 3652 & 14838  & 7437  & 2540 \\
 1367  & 2512 & 14407  & 5053  & 1521   & 297 & 10935 & 17137 \\
 2186  & 9433 & 13293  & 7555 & 13618 & 13000  & 6490  & 5310 \\
18676  & 4782 & 11374   & 446  & 4165 & 11634  & 3846 & 14611 \\
 2364  & 6789 & 11634  & 4493  & 4063  & 4576 & 17955  & 7965 \\
11748 & 14616 & 11453 & 17666   & 925    & 56  & 4118 & 18031 \\
 9522 & 14838  & 7437  & 3880 & 11476  & 8305  & 5102  & 2999 \\
18628 & 14326  & 9175  & 9061   & 650 & 18110  & 8720 & 15404 \\
 2951   & 722 & 15334   & 841 & 15610  & 2443 & 11056  & 2186 
\end{tabular} } % tt

\newpage
TABLE 4.2\footnote{
Die Zahlen dieser Tabelle befinden sich auch in der Online-Hilfe \glqq Szenario f"ur die
RSA-Demonstration\grqq~ der CrypTool-Auslieferung.
}: RSA Ciphertext

{\tt 
\begin{tabular}{llllllll}
 6340  & 8309 & 14010  & 8936 & 27358 & 25023 & 16481 & 25809 \\
23614  & 7135 & 24996 & 30590 & 27570 & 26486 & 30388  & 9395 \\
27584 & 14999  & 4517 & 12146 & 29421 & 26439  & 1606 & 17881 \\
25774  & 7647 & 23901  & 7372 & 25774 & 18436 & 12056 & 13547 \\
 7908  & 8635  & 2149  & 1908 & 22076  & 7372  & 8686  & 1304 \\
 4082 & 11803  & 5314   & 107  & 7359 & 22470  & 7372 & 22827 \\
15698 & 30317  & 4685 & 14696 & 30388  & 8671 & 29956 & 15705 \\
 1417 & 26905 & 25809 & 28347 & 26277  & 7897 & 20240 & 21519 \\
12437  & 1108 & 27106 & 18743 & 24144 & 10685 & 25234 & 30155 \\
23005  & 8267  & 9917  & 7994  & 9694  & 2149 & 10042 & 27705 \\
15930 & 29748  & 8635 & 23645 & 11738 & 24591 & 20240 & 27212 \\
27486  & 9741  & 2149 & 29329  & 2149  & 5501 & 14015 & 30155 \\
18154 & 22319 & 27705 & 20321 & 23254 & 13624  & 3249  & 5443 \\
 2149 & 16975 & 16087 & 14600 & 27705 & 19386  & 7325 & 26277 \\
19554 & 23614  & 7553  & 4734  & 8091 & 23973 & 14015   & 107 \\
 3183 & 17347 & 25234  & 4595 & 21498  & 6360 & 19837  & 8463 \\
 6000 & 31280 & 29413  & 2066   & 369 & 23204  & 8425  & 7792 \\
25973  & 4477 & 30989                               
\end{tabular} } % tt

% .....................................................................................
\subsubsection{Eine kleine RSA-Cipher-Challenge (2)}
\index{RSA!Cipher-Challenge}

Die folgende Aufgabe ist eine korrigierte Variante aus dem sehr guten Buch
von Prof.~Yan \cite[Example 3.3.7, S. 318]{Yan2000}. \index{Yan 2000}
Es geht aber nicht nur um das Ergebnis, sondern vor allem um die
Einzelschritte der L"osung, also um die Darlegung der Kryptoanalyse\footnote{
Im Szenario der Online-Hilfe zu CrypTool und in der Pr"asentation auf der
Web-Seite wird der L"osungsweg skizziert. Wenn uns jemand einen gut
aufbereiteten konkreten L"osungsweg schickt, nehmen wir ihn gerne in die
Dokumentation auf.
}.

Man kann sich drei v"ollig unterschiedlich schwierige Aufgaben vorstellen:
Gegeben ist jeweils der Ciphertext und der "offentliche Schl"ussel $(e,n)$:
\begin{itemize}
\item \index{Angriff!Known-Plaintext} Known-Plaintext: finde den geheimen Schl"ussel $d$ unter Benutzung der
zus"atzlich bekannten Ursprungsnachricht.
\item \index{Angriff!Ciphertext-only} Ciphertext-only: finde $d$ und den Cleartext.
\item \index{Faktorisierungsproblem} RSA-Modul knacken, d.h. faktorisieren (ohne Kenntnis der Messages).
\end{itemize}
\newpage
$n = 63978486879527143858831415041, ~e = 17579$

Message\footnote{
Die Zahlen dieser Tabelle befinden sich auch in der Online-Hilfe \glqq Szenario
f"ur die RSA-Demonstration\grqq~ der CrypTool-Auslieferung.
}:

{\tt
\begin{tabular}{l}
1401202118011200, \\
1421130205181900, \\
0118050013010405, \\
0002250007150400 
\end{tabular} } % tt

Cipher:

{\tt
\begin{tabular}{l}
45411667895024938209259253423, \\
16597091621432020076311552201, \\
46468979279750354732637631044, \\
32870167545903741339819671379
\end{tabular} } % tt
\vskip +8pt

{\bf Bemerkung:}\\
Die urspr"ungliche Nachricht bestand aus einem Satz mit 31 Zeichen (codiert
mit dem \hyperlink{Grossbuchstaben-Alphabet}{Gro"sbuchstabenalphabet}\index{Gro""sbuchstabenalphabet} aus Abschnitt 3.12.2). Dann wurden je 16 Dezimalziffern zu
einer Zahl zusammengefa"st (die letzte Zahl wurde mit Nullen aufgef"ullt).
Diese Zahlen wurden mit $e$ potenziert.

Beim Entschl"usseln ist darauf zu achten, dass die berechneten Zahlen vorne
mit Nullen aufzuf"ullen sind, um den Klartext zu erhalten.

Wir betonen das, weil in der Implementierung und Standardisierung die Art
des Padding sehr wichtig ist f"ur interoperable Algorithmen.



\newpage
%\addcontentsline{toc}{subsection}{Literaturverzeichnis}
\begin{thebibliography}{99999}
\addcontentsline{toc}{subsection}{Literaturverzeichnis}
\bibitem[Bartholome1996]{Bartholome1996}  
    A. Bartholom\'e, J. Rung, H. Kern, \\
    {\em Zahlentheorie f"ur Einsteiger}, Vieweg 1995, 2. Auflage 1996.

\bibitem[Bauer1995]{Bauer1995} \index{Bauer 1995}
    Friedrich L. Bauer, \\
    {\em Entzifferte Geheimnisse}, Springer, 1995.

\bibitem[Bauer2000]{Bauer2000} \index{Bauer 2000}
    Friedrich L. Bauer, \\
    {\em Decrypted Secrets}, Springer, 2nd edition, 2000.

\bibitem[Bourseau2002]{Bourseau2002}
    F. Bourseau / D. Fox / C. Thiel, \\
    {\em Vorz"uge und Grenzen des RSA-Verfahrens}, in Datenschutz und Datensicherheit (DuD), 26/2002, S. 84-89.
\bibitem[Beutelspacher1996]{Beutelspacher1996} \index{Beutelspacher 1996}
    Albrecht Beutelspacher, \\
    {\em Kryptologie}, Vieweg, 5. Auflage 1996.

\bibitem[Buchmann1999]{Buchmann1999} \index{Buchmann 1999}
    Johannes Buchmann, \\
    {\em Einf"uhrung in die Kryptographie}, Springer, 1999.

\bibitem[Eckert2001]{Eckert2001} \index{Eckert 2001} 
    Claudia Eckert, \\
    {\em IT-Sicherheit: Konzepte-Verfahren-Protokolle}, Oldenbourg, 2001.

\bibitem[Graham1994]{Graham1994} \index{Graham 1994}
     Graham, Knuth, Patashnik, \\
     {\em Concrete Mathemathics, a Foundation of Computer Science}, 2nd edition, Addison Wesley, 1994.

\bibitem[Knuth1998]{Knuth1998} \index{Knuth 1998}
    Donald E. Knuth, \\
    {\em The Art of Computer Programming, vol 2: 
    Seminumerical Algorithms}, Addison-Wesley, 2nd edition, 1998.

\bibitem[Pfleger1997]{Pfleger1997} \index{Pfleger 1997}
    Charles P. Pfleger, \\
    {\em Security in Computing}, Prentice-Hall, 2nd edition, 1997.

\bibitem[Schneier1996]{Schneier1996} \index{Schneier 1996} 
    Bruce Schneier, \\
    {\em Applied Cryptography, Protocols, Algorithms,
    and Source Code in C}, Wiley, 2nd edition, 1996.

\bibitem[Sedgewick1990]{Sedgewick1990} \index{Sedgewick 1990}
    Robert Sedgewick,\\
    {\em Algorithms in C}, Addison-Wesley, 1990.

\bibitem[Stinson1995]{3Stinson1995} \index{Stinson 1995}
    Douglas R. Stinson,\\
    {\em Cryptography - Theory and Practice}, CRC Press, 1995.

\bibitem[Wiles1994]{Wiles1994}\index{Wiles}
    Wiles, Andrew, \\
    {\em Modular elliptic curves and Fermat's Last Theorem}, in Annals of
    Mathematics 141 (1995).

\bibitem[Wolffenstetter1998]{Wolffenstetter1998} \index{Wolffenstetter 1998}
    Albrecht Beutelspacher, J"org Schwenk, Klaus-Dieter Wolfenstetter, \\
    {\em Moderne Verfahren in der Kryptographie}, Vieweg, 2. Auflage, 1998.

\bibitem[Yan2000]{Yan2000} \index{Yan 2000} 
    Song Y. Yan, \\
    {\em Number Theory for Computing}, Springer, 2000.
\end{thebibliography}

\newpage
\subsection*{URLs / Links (Auswahl)}\addcontentsline{toc}{subsection}{URLs}

\begin{enumerate}
   \item \hypertarget{knott}{} \index{Knott Ron}
          Fibonacci\index{Fibonacci} Seite von Ron Knott, \\
      Hier dreht sich alles um Fibonacci-Zahlen. \\
          \href{http://www.mcs.surrey.ac.uk/personal/R.Knott/Fibonacci/fib.html}{\texttt{http://www.mcs.surrey.ac.uk/personal/R.Knott/Fibonacci/fib.html}}
   \item CrypTool Version 1.3, 2002, \\
          Freeware zur Veranschaulichung von Kryptographie, \\
          \href{http://www.cryptool.de}{\texttt{http://www.cryptool.de}}, \\
          \href{http://www.cryptool.org}{\texttt{http://www.cryptool.org}},\\ 
          \href{http://www.cryptool.com}{\texttt{http://www.cryptool.com}}
   \item Mathematica, \index{Mathematica}\\
          Kommerzielles Mathematik-Paket \\
          \href{http://www.wolfram.com}{\texttt{http://www.wolfram.com }}
   \item  LiDIA, \index{LiDIA}\\ 
          Umfangreiche Bibliothek mit zahlentheoretischen Funktionen und dem
          Interpreter LC \\
          \href{http://www.informatik.tu-darmstadt.de/TI/LiDIA}{\texttt{http://www.informatik.tu-darmstadt.de/TI/LiDIA}}
   \item BC, \\
         Interpreter mit zahlentheoretischen Funktionen \\
         \href{http://www.maths.uq.edu.au/~krm/gnubc.html}{\texttt{http://www.maths.uq.edu.au/\~{}}krm/gnubc.html}
   \item Pari-GP, \index{Pari-GP}\\
         Hervorragender, schneller Interpreter mit zahlentheoretischen Funktionen \\
         \href{http://www.parigp-home.de}
              {\texttt{http://www.parigp-home.de}} und 
         \href{http://www.parigp-home.com}
              {\texttt{http://www.parigp-home.com}}
   \item Erst nach Vollendung dieses Artikels wurde mir eine Web-Seite bekannt, die
         interaktiv und didaktisch sehr ausgereift die grundlegenden Denkweisen der
         Mathematik anhand der elementaren Zahlentheorie nahebringt. Sie entstand f"ur
         ein Unterrichtsprojekt der 11. Klasse des Technischen Gymnasium (ist leider
         nur in Deutsch verf"ugbar): \\
         \href{http://www.hydrargyrum.de/kryptographie}
              {\texttt{http://www.hydrargyrum.de/kryptographie}}\index{Munchenbach@M""unchenbach}
   \item Ebenfalls erst nach Vollendung dieses Artikels wurde mir die Seite des Beauftragten f"ur die
         Lehrplanentwicklung des Fachs Informatik an der Oberstufe des Landes Saarland bekannt.
         Hier befindet sich eine Sammlung von Texten und Programmen (in Java), die aus didaktischen 
         "Uberlegungen entstand (ist leider nur in deutsch verf"ugbar). \\
         \href{http://www.hom.saar.de/~awa/kryptolo.htm}
              {\texttt{http://www.hom.saar.de/\~{}awa/kryptolo.htm}}
   \item BSI, \index{BSI}\\ 
         Bundesamt f"ur Sicherheit in der Informationstechnik \\
         \href{http://www.bsi.bund.de}{\texttt{http://www.bsi.bund.de}}
   \item Bernsteins RSA-Aufsatz\\
         \href{http://cr.yp.to/papers/nfscircuit.ps}{\texttt{http://cr.yp.to/papers/nfscircuit.ps}}
\end{enumerate}


\subsection*{Dank} \addcontentsline{toc}{subsection}{Dank}
F"ur das anregende und konstruktive Korrekturlesen und viele Verbesserungen dieses Artikels: Hr.
Henrik Koy.


\newpage
\subsection*{Anhang A: Der gr"o"ste gemeinsame Teiler (ggT) von ganzen Zahlen \index{ggT}
und die beiden Algorithmen von Euklid} \hypertarget{Appendix_A}{}
\addcontentsline{toc}{subsection}{Anhang A: Der gr"o"ste gemeinsame Teiler (ggT) von ganzen Zahlen \index{ggT}
und die beiden Algorithmen von Euklid}
\begin{enumerate}
\item Der gr"o"ste gemeinsame Teiler zweier nat"urlicher Zahlen $a$ und $b$ ist eine wichtige und 
sehr schnell zu berechnende Gr"o"se. Wenn eine Zahl $c$ die Zahlen $a$ und $b$ 
teilt (d.h. es gibt ein $a'$ und ein $b'$, so dass $a = a'*c$ und $b = b'*c$), dann teilt $c$ auch den
Rest $r$ der Division von $a$ durch $b$. Wir schreiben in aller K"urze: Aus $c$ teilt $a$ und $b$ folgt: $c$ teilt  
$r = a - \lfloor a/b \rfloor * b$\footnote{
Die Gau"sklammer \index{Gau""sklammer} $\lfloor x \rfloor $ der reellwertigen Zahl $x$ ist definiert als: 
$\lfloor x \rfloor $ ist die gr"o"ste ganze Zahl kleiner oder gleich $x$.
}.

Weil die obige Aussage f"ur alle gemeinsamen Teiler $c$ von $a$ und $b$ gilt, folgt f"ur den gr"o"sten gemeinsamen Teiler
von $a$ und $b$ (ggT($a,b$)) die Aussage
$$ {\rm ggT}(a,b) = {\rm ggT}(a - \lfloor a/b \rfloor * b, b). $$
Mit dieser Erkenntnis l"a"st sich der Algorithmus zum Berechnen des ggT zweier Zahlen wie folgt (in Pseudocode) 
beschreiben:
\begin{verbatim}
INPUT: a,b != 0
1. if ( a < b ) then  x = a; a = b; b = x; // Vertausche a und b (a > b)
2. a = a - int(a/b) * b        // a wird kleiner b, der ggT(a, b) 
                               // bleibt unveraendert 
3. if ( a != 0 ) then goto 1.  // nach jedem Schritt faellt a, und 
                               // der Algorithmus endet, wenn a == 0.
OUTPUT "ggT(a,b) = " b         // b ist der ggT vom urspr"unglichen a und b
\end{verbatim}
\item Aus dem ggT lassen sich aber noch weitere Zusammenh"ange bestimmen:
Dazu betrachtet man f"ur $a$ und $b$ das Gleichungssystem:
\begin{eqnarray}
 a & = & 1*a + 0*b \nonumber \\
 b & = & 0*a + 1*b, \nonumber
\end{eqnarray}
bzw. in Matrix-Schreibweise
$$ \left(\begin{array}{c}a \\ b\end{array}\right) = 
   \left(\begin{array}{cc} 1 & 0 \\ 0 & 1 \end{array}\right) *
   \left(\begin{array}{c} a \\ b \end{array} \right).$$
Wir fassen diese Informationen in der erweiterten Matrix 
$$\left(\begin{array}{cccc} a & | & 1 & 0 \\ b & | & 0 & 1 \end{array} \right)$$
zusammen. Wendet man den ggT-Algorithmus auf diese Matrix an, so erh"alt man den 
{\em erweiterten ggT Algorithmus}, \index{Euklidscher Algorithmus!erweiterter} mit dem die multiplikative Inverse bestimmt wird. 

\newpage
{\tt INPUT:} $a,b \not= 0$
\begin{itemize}
  \item[\tt 0.] $x_{1,1} := 1, x_{1,2} := 0, x_{2,1} := 0, x_{2,2} := 1.$
  \item[\tt 1.] $ \left(\begin{array}{cccc} a & | & x_{1,1} & x_{1,2} \\ b & | & x_{2,1} & x_{2,2} \end{array} \right) := 
                   \left(\begin{array}{cc} 0 & 1  \\ 1 & - \lfloor a/b \rfloor * b \end{array} \right)*
           \left(\begin{array}{cccc} a & | & x_{1,1} & x_{1,2} \\ b & | & x_{2,1} & x_{2,2} \end{array} \right).$
  \item[\tt 2.] {\tt if (b != 0) then goto 1.}
\end{itemize}
{\tt OUTPUT:} ''ggT$(a,b) = a*x +b*y$: '', ''ggT$(a,b) =$ '' $b$, ''$x = $'' $x_{2,1}$, ''$y = $'' $x_{2,2}.$

Da dieser Algorithmus nur lineare Transformationen durchf"uhrt, gelten immer die Gleichungen
\begin{eqnarray}
 a & = & x_{1,1}*a + x_{1,2}*b, \nonumber \\
 b & = & x_{2,1}*a + x_{2,2}*b, \nonumber
\end{eqnarray}
und es folgt am Ende des Algorithmus\footnote{
Wenn der ggT-Algorithmus endet, steht in den Programmvariablen $a$ und $b$:
$a = 0$ und $b=ggT(a,b)$. Bitte beachten Sie, dass die Programmvariablen zu den Zahlen $a$ und $b$ verschieden sind
und ihre G"ultigkeit nur im Rahmen des Algorithmus haben.
} die erweiterte ggT-Gleichung:
$${\rm ggT}(a,b) = a*x_{2,1} + b*x_{2,2}.$$

\vskip +8 pt

{\bf Beispiel:}\\
Mit dem erweiterten ggT l"a"st sich f"ur $e = 37$ die modulo $3588$ multiplikativ inverse Zahl $d$ bestimmen (d.h 
$37*d \equiv 1 {\rm ~(mod~} 3588$)): 

{\tt 0.}
 $ \left(\begin{array}{cccc} 3588 & | & 1 & 0 \\ 37 & | & 0 & 1 \end{array} \right)$ 
 
{\tt 1.}
 $ \left(\begin{array}{cccc} 37 & | & 1 & 0 \\ 36 & | & 0 & -96 \end{array} \right) = 
   \left(\begin{array}{cc} 0 & 1  \\ 1 & - (\lfloor 3588/36 \rfloor = 96) * 37 \end{array} \right)*
   \left(\begin{array}{cccc} 3588 & | & 1 & 0 \\ 37 & | & 0 & 1 \end{array} \right).$
   
{\tt 2.}
 $ \left(\begin{array}{cccc} 36 & | & 1 & -96 \\ 1 & | & -1 & 97 \end{array} \right) = 
   \left(\begin{array}{cc} 0 & 1  \\ 1 & - (\lfloor 37/36 \rfloor = 1) * 36 \end{array} \right)*
   \left(\begin{array}{cccc} 37 & | & 1 & 0 \\ 36 & | & 0 & -96 \end{array} \right).$
   
{\tt 3.}
 $ \left(\begin{array}{cccc} {\bf 1} & | & {\bf -1} & {\bf 97} \\ 0 & | & 37 & -3588 \end{array} \right) = 
   \left(\begin{array}{cc} 0 & 1  \\ 1 & - (\lfloor 36/1 \rfloor = 36) * 1 \end{array} \right)*
   \left(\begin{array}{cccc} 36 & | & 1 & -96 \\ 1 & | & -1 & 97 \end{array} \right).$

\vskip + 8pt
{\tt OUTPUT:} \\
ggT($37,3588) = a*x + b*y$: ggT($37,3588$) = 1, $x = -1$, $y=97$.
\vskip +8pt

Es folgt 
\begin{itemize}
   \item[\bf 1.] $37$ und $3588$ sind teilerfremd ($37$ ist invertierbar modulo $3588$).
   \item[\bf 2.] $37*97 = (1*3588)+1$ mit anderen Worten $37*97 \equiv 1 {\rm ~(mod~} 3588)$ \\
                 und damit ist modulo $3588$ die Zahl $97$ multiplikativ invers zu $37$.
\end{itemize}
\end{enumerate}
\newpage
\subsection*{Anhang B: Abschlussbildung} \hypertarget{Appendix_B}{}
\addcontentsline{toc}{subsection}{Anhang B: Abschlussbildung}
Die Eigenschaft der Abgeschlossenheit\index{Abgeschlossenheit} wird innerhalb einer Menge immer
bez"uglich einer Operation definiert.
Im folgenden wird gezeigt, wie man f"ur eine gegebene Ausgangsmenge $G_0$ die
abgeschlossene Menge G bez"uglich der Operation �$+ {\rm ~(mod~} 8)$ konstruiert
(\glqq Abschlussbildung''):
\begin{eqnarray*}
G_0 & = & \{ 2, 3 \} {\rm ~die~Addition~der~Zahlen~in~} G_0 {\rm ~bestimmt~weitere~Zahlen:} \nonumber \\
    & &    2 + 3 \equiv 5{\rm ~(mod~}8) = 5 \nonumber \\
    & &    2 + 2 \equiv 4{\rm ~(mod~}8) = 4 \nonumber \\
    & &    3 + 3 \equiv 6{\rm ~(mod~}8) = 6 \nonumber \\ 
G_1 & = & \{ 2, 3, 4, 5, 6 \} {\rm ~die~Addition~der~Zahlen~in~} G_1 {\rm ~bestimmt:}\nonumber \\
    & &    3 + 4 \equiv 7{\rm ~(mod~}8) = 7 \nonumber \\
    & &    3 + 5 \equiv 8{\rm ~(mod~}8) = 0 \nonumber \\
    & &    3 + 6 \equiv 9{\rm ~(mod~}8) = 1 \nonumber \\ 
G_2 & = & \{ 0, 1, 2, 3, 4, 5, 6, 7 \} {\rm ~~die~Addition~der~Zahlen~in~} G_2 {~erweitert~die~Menge~nicht!} \nonumber \\
G_3 & = & G_2 {\rm ~~man~sagt:~} G_2 {\rm~ist~abgeschlossen~bez"uglich~der~Addition~~(mod~}8). \nonumber \\
\end{eqnarray*}
Ende der Abschlussbildung.
\vskip +40pt

\subsection*{Anhang C: Bemerkungen zur modulo Subtraktion} \hypertarget{Appendix_C}{}
\addcontentsline{toc}{subsection}{Anhang C: Bemerkungen zur modulo Subtraktion}
Beispielweise gilt f"ur die Subtraktion modulo 5: $2 - 4 \equiv -2 \equiv 3{\rm ~mod~}2$.
Es gilt modulo $5$ also nicht, dass $-2 = 2$ ! Dies gleichzusetzen ist ein h"aufig
gemachter Fehler. Dies kann man sich gut verdeutlichen, wenn man die Permutation $(0, 1, 2, 3,
4)$ aus $\mathbb{Z}_5$ von z.B. $-11$ bis $+11$ wiederholt "uber den Zahlenstrahl aus $\mathbb{Z}$ legt.

\vskip +10 pt
\input{figures/line-de.latex}

\newpage
\subsection*{Anhang D: Beispiele mit Mathematica und Pari-GP} \hypertarget{AppendixD}{}
\addcontentsline{toc}{subsection}{Anhang D: Beispiele mit Mathematica und Pari-GP}
\index{Mathematica}\index{Pari-GP}

In diesem Anhang finden sie den Quellcode der berechneten Tabellen und Beispiele mit Mathematica 
und dem freien Programm Pari-GP.


\subsubsection*{Multiplikationstabellen modulo $m$} \hypertarget{AppArith1}{}

Die auf Seite \pageref{SrcArith1a} aufgef"uhrten Tabelle der Multiplikationstabelle modulo $m = 17$
f"ur $a=5$ und $a=6$ wird mit Mathematica durch die folgenden Befehle bestimmt: 
\begin{verbatim}
m = 17; iBreite = 18; iFaktor1 = 5; iFaktor2 = 6; 
Print[ ''i '', Table[ i, {i, 1, iBreite} ] ]; 
Print[ iFaktor1, ''*i '', Table[ iFaktor1*i, {i, 1, iBreite } ] ]; 
Print[ ''Rest '', Table[ Mod[iFaktor1*i, m], {i, 1, iBreite } ] ]; 
Print[ iFaktor2, ''*i '', Table[ iFaktor2*i, {i, 1, iBreite } ] ]; 
Print[ ''Rest '', Table[ Mod[iFaktor2*i, m], {i, 1, iBreite } ] ]; 
\end{verbatim}
Mit Pari-GP k"onnen Sie die Tabellenwerte folgenderma"sen berechnen:

{\tt m=17; iBreite=18; iFaktor1=5; iFaktor2=6;}

{\tt matrix(1,iBreite, x,y, iFaktor1*y)} ergibt \\
{\tt [5 10 15 20 25 30 35 40 45 50 55 60 65 70 75 80 85 90]}

{\tt matrix(1,iBreite, x,y, (iFaktor1*y)\%m )} ergibt \\
{\tt [5 10 15 3 8 13 1 6 11 16 4 9 14 2 7 12 0 5]}

Beachte: Pari-GP erzeugt bei Verwendung von {\tt Mod} das Objekt {\tt Mod} und gibt es auch aus:
\begin{verbatim}
matrix(1,iBreite, x,y, Mod(iFaktor1*y, m))
[Mod(5, 17) Mod(10, 17) Mod(15, 17) Mod(3, 17) Mod(8, 17) Mod(13, 17) Mod(1, 17)
 Mod(6, 17) Mod(11, 17) Mod(16, 17) Mod(4, 17) Mod(9, 17) Mod(14, 17) Mod(2, 17)
 Mod(7, 17) Mod(12, 17) Mod(0, 17) Mod(5, 17)]
\end{verbatim}

Die weiteren Beispiele der Multiplikationstabelle modulo $13$ und modulo $12$ auf Seite 
\pageref{SrcArith1b} bestimmen Sie, indem Sie im Quelltext jeweils {\tt m=17} durch den 
entsprechenden Zahlenwert ({\tt m=13} bzw. {\tt m=12}) ersetzen.


\subsubsection*{Schnelles Berechnen hoher Potenzen} \hypertarget{AppArith2}{}

Das schnelle Potenzieren modulo $m$ geh"ort zu den Grundfunktionen von Mathematica und 
Pari-GP. Sie k"onnen mit Hilfe dieser Programme nat"urlich auch die Idee der
Square-and-Multiply-Methode nachvollziehen. In Mathematica bestimmen sie Sie die Einzelexponentiation des Beispiels auf 
Seite \pageref{SrcArith2} mit
\begin{verbatim}
Mod[{87^43, 87^2, 87^4, 87^8, 87^16, 87^32}, 103] = {85, 50, 28, 63, 55, 38}.
\end{verbatim}
und in Pari-GP lautet die Syntax:
\begin{verbatim}
Mod([87^43,87^2,87^4,87^8,87^16,87^32],103)
\end{verbatim}



\subsubsection*{Multiplikative Ordnung und Primitivwurzel}

\hypertarget{AppArith3a}{}
Die Ordnung $ord_m(a)$ einer Zahl in der Multiplikativen Gruppe $Z_m^*$ modulo $m$ ist die 
kleinste ganze Zahl $i \ge 1$ f"ur die gilt $a^i \equiv 1$ mod $m$. F"ur das Beispiel auf
Seite \pageref{SrcArith3a} k"onnen Sie mit Mathematica alle Exponentiationen $a^i$ mod $11$
durch die folgende Programmzeile ausgeben lassen. 
\begin{verbatim}
m=11;  Table[ Mod[a^i, m], {a, 1, m-1}, {i, 1, m-1} ]
\end{verbatim}
In Pari-GP lautet die Syntax:
\begin{verbatim}
m=11; matrix(10,10, x,y, (x^y)%m )
\end{verbatim}


\vskip +12 pt \hypertarget{AppArith3b}{}
Im Beispiel auf Seite \pageref{SrcArith3b} wird die Ordnung $ord_{45}(a)$ 
sowie die Eulersche Zahl $J(45)$ ausgegeben.
Mit Mathematica bestimmen Sie diese Tabelle durch die folgende Programmschleife (innerhalb von
Print kann man keine Do-Schleife verwenden und jedes Print gibt am Ende ein Newline aus). 
\begin{verbatim}
m = 45; 
Do[ Print[ Table[ Mod[a^i, m], {i, 1, 12} ],
'', '', MultiplicativeOrder[a, m, 1],
'', '', EulerPhi[m] ], 
{a, 1, 12} ]; 
\end{verbatim}
In Pari-GP lautet die Syntax:
\begin{verbatim}
m=45; 
matrix(12,14, x,y, 
       if( y<=12, (x^y)%m, 
       if( y==13, if( gcd(x,m)==1, znorder(Mod(x,m)), "--"), 
       eulerphi(m))))
\end{verbatim}

{\tt znorder(Mod(x,m))} kann nur berechnet werden, wenn $x$ teilerfremd zu $m$ ist, was mit {\tt gcd(x,m)} "uberpr"uft wird.
Bessere Performance\index{Performance} erreicht man mit {\tt Mod(x,m)${}^\wedge$y} statt mit {\tt (x${}^\wedge$y)\%m}.

Auch in Pari-GP kann man Schleifen erzeugen. Verzichtet man auf die Tabellenform, sieht das so aus:
\begin{verbatim}
for( x=1,12, 
     for(y=1,12, print(Mod(x^y,m))); 
     if(gcd(x,m)==1,print(znorder(Mod(x,m))),print("--"));
     print(eulerphi(m)))
\end{verbatim}


\vskip +12 pt \hypertarget{AppArith3c}{}
Im dritten Beispiel auf Seite \pageref{SrcArith3c} werden die Exponentiationen $a^i$ mod $46$
sowie die Ordnungen $ord_{46}(a)$ ausgegeben.
Mit Mathematica bestimmen Sie diese Tabelle durch die folgende Programmschleife.
\begin{verbatim}
m = 46; Do[ Print[ Table[ Mod[a^i, m], {i, 1, 23} ], 
'', '', MultiplicativeOrder[a, m, 1]
{a, 1, 23} ];
\end{verbatim}
In Pari-GP lautet die Syntax:
\begin{verbatim}
m=46; 
matrix(23,24, x,y, 
       if( y<=23, (x^y)%m,
       if( y==24, if( gcd(x,m)==1, znorder(Mod(x,m)), "--"))))
\end{verbatim}



\subsubsection*{RSA Beispiele}

In diesem Abschnitt sind die Quelltexte f"ur die RSA-Beispiele im Abschnitt
\hyperlink{RSAKonkret}{Das RSA-Verfahren mit konkreten Zahlen} in Mathematica und Pari-GP angegeben.   


\vskip +10 pt \hypertarget{AppArith4a}
{{\bf Beispiel auf Seite}} \pageref{SrcArith4a}. \\
Die RSA-Exponentiation $M^{37}$ mod $3713$ f"ur die Nachricht $M = 120$ kann mit Mathematica per 
{\tt PowerMod[120, 37, 3713]} berechnet werden.\\
In Pari-GP lautet die Syntax  \\
{\tt Mod(120,3713)\^{}37 oder Mod(120\^{}37,3713)}.

\vskip +10 pt \hypertarget{AppArith4b}
{{\bf Beispiel auf Seite}} \pageref{SrcArith4b}. \\
Die Faktorisierung von $J(256.027) = 255.016 = 2^3 * 127 * 251$ bestimmt Mathematica mit 
{\tt FactorInteger[255.016]= \{\{2,3\}, \{127,1\}, \{251,1\}\}}. \\
In Pari-GP lautet die Syntax: \\
{\tt factor(255016)}.

\vskip +10 pt \hypertarget{AppArith4c}{}
{{\bf Beispiel auf Seite}} \pageref{SrcArith4c}. \\ 
Mit Mathematica kann die RSA-Verschl"usselung durch den Befehl \\
{\tt PowerMod[\{82, 83, 65, 32, 119, 111, 114, 107, 115, 33\}, 65537, 256.027]\}}  \\
berechnet werden.  \\ 
In Pari-GP lautet die Syntax: \\
{\tt vecextract( [Mod(82,256027)\^{}65537, Mod(83,256027)\^{}65537, Mod(65,256027)\^{}65537, \\
                 Mod(32,256027)\^{}65537, Mod(119,256027)\^{}65537], 31)
}

{\bf Bemerkung zur Verwendung von {\tt Mod} in Pari-GP:} \\
{\tt Mod(82,256027)\^{}65537} ist wesentlich schneller als \\
--  {\tt Mod(82\^{}65537, 256027)} oder \\
--  {\tt (82\^{}65537) \% 256027}.

\vskip +10 pt \hypertarget{AppArith4d}{}
{{\bf Beispiel auf Seite}} \pageref{SrcArith4d}.  \\
Die RSA-Verschl"usselung kann mit Mathematica per \\
{\tt PowerMod[\{21075, 16672, 30575, 29291, 29473\}, 65537, 256027]}  \\
berechnet werden.\\
In Pari-GP lautet die Syntax: \\
{\tt vecextract( [Mod(21075,256027)\^{}65537, Mod(16672,256027)\^{}65537, \\
             Mod(30575,256027)\^{}65537, Mod(29291,256027)\^{}65537, \\
             Mod(29473,256027)\^{}65537], 31) 
}

\vskip +10 pt \hypertarget{AppArith4e}{}
{{\bf Beispiel auf Seite}} \pageref{SrcArith4e}. \\
Die RSA-Verschl"usselung kann mit Mathematica per \\
{\tt PowerMod[\{82083, 65032, 119111, 114107, 115033\}, 65537, 256027]} \\
berechnet werden. \\
In Pari-GP lautet die Syntax: \\
{\tt vecextract( [Mod(82083,256027)\^{}65537, Mod(65032,256027)\^{}65537, \\
             Mod(119111,256027)\^{}65537, Mod(114107,256027)\^{}65537, \\
             Mod(115033,256027)\^{}65537], 31)
}


% -----------------------------------------------------------------------------------------------------------------

\newpage
\subsection*{Anhang E: Liste der hier formulierten Definitionen und S"atze} \hypertarget{AppendixE}{}
\addcontentsline{toc}{subsection}{Anhang E: Liste der hier formulierten Definitionen und S"atze}
\vskip +8 pt
\begin{center}
\begin{tabular}{|l|l|l|}\hline
 & Kurzbeschreibung~~ & Seite \\ \hline
Definition \ref{def-zth-prime} & Primzahlen &  \pageref{def-zth-prime} \\
Definition \ref{def-zth-composite} & Zusammengesetzte Zahlen & \pageref{def-zth-composite}  \\ \hline
Satz \ref{thm-zth-cnum} & Teiler von zusammengesetzten Zahlen~~~~~~~ & \pageref{thm-zth-cnum}\\
Satz \ref{thm-zth-mthm} &  Erster Hauptsatz der elementaren Zahlentheorie&  \pageref{thm-zth-mthm} \\  \hline
Definition \ref{def-zth-divisibility} & Teilbarkeit &  \pageref{def-zth-divisibility} \\
Definition \ref{def-zth-remainder} & Restklasse $r$ modulo $m$ &  \pageref{def-zth-remainder} \\
Definition \ref{def-zth-congruence} & restgleich oder kongruent & \pageref{def-zth-congruence}  \\ \hline
Satz \ref{thm-zth-div} & Kongruenz mittels Differenz  &  \pageref{thm-zth-div} \\
Satz \ref{thm-zth-multinv} & Multiplikative Inverse (Existenz) & \pageref{thm-zth-multinv}   \\
Satz \ref{thm-zth-exhperm} & Ersch"opfende Permutation &  \pageref{thm-zth-exhperm} \\
Satz \ref{thm-zth-pot} & Gestaffelte Exponentiation mod $m$ & \pageref{thm-zth-pot}  \\ \hline
Definition \ref{def-zth-zn} & $\mathbb{Z}_n$  & \pageref{def-zth-zn} \\
Definition \ref{def-zth-znmult} &   $\mathbb{Z}_n^*$ & \pageref{def-zth-znmult}  \\ \hline
Satz \ref{thm-zth-znmult} & Multiplikative Inverse in $\mathbb{Z}_n^*$& \pageref{thm-zth-znmult}  \\ \hline
Definition \ref{def-zth-phiofn} & Euler-Funktion $J(n)$ & \pageref{def-zth-phiofn}  \\
Satz \ref{thm-zth-phiprime} & $J(p)$ &  \pageref{thm-zth-phiprime} \\
Satz \ref{thm-zth-phipq} & $J(p*q)$ &  \pageref{thm-zth-phipq} \\
Satz \ref{thm-zth-phimultprime} & $J(p_1 * \cdots *p_k)$ & \pageref{thm-zth-phimultprime}  \\
Satz \ref{thm-zth-phinum} & $J(p_1^{e_1} * \cdots *p_k^{e_k})$ & \pageref{thm-zth-phinum}  \\
Satz \ref{thm-zth-fermat1} & Kleiner Satz von Fermat  &  \pageref{thm-zth-fermat1} \\
Satz \ref{thm-zth-fermateuler} & Satz von Euler-Fermat & \pageref{thm-zth-fermateuler}  \\ \hline
Definition \ref{def-zth-ordn} & Multiplikative Ordnung $ {\rm ord}_{m} (a)$ &  \pageref{def-zth-ordn} \\
Definition \ref{def-zth-primitiveroot} & Primitivwurzel von $m$ &  \pageref{def-zth-primitiveroot} \\
Satz \ref{thm-zth-ordp} & Aussch"opfung des Wertebereiches & \pageref{thm-zth-ordp}  \\ \hline
\end{tabular}
\end{center}
\vskip +6 pt
