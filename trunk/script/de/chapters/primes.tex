% $Id$
% ...........................................................................
%       K A P I T E L  3 :     P R I M Z A H L E N
% /~~~~~~~~~~~~~~~~~~~~~~~~~~~~~~~~~~~~~~~~~~~~~~~~~~~~~~~~~~~~~~~~~~~~~~~~~~~

\newpage
\hypertarget{Kapitel_2}{}
\section{Primzahlen}
\label{Label_Kapitel_2}  %Die Kapitelnr. von \label wird per per \ref geholt.
(Bernhard Esslinger, Mai 1999; Updates: Nov. 2000, Dez. 2001, Juni 2003, Mai 2005, M"arz 2006)

\begin{center}
\fbox{\parbox{15cm}{
    \emph{Albert Einstein\footnotemark:}\\
    Der Fortschritt lebt vom Austausch des Wissens.
}}
\end{center}
\addtocounter{footnote}{0}\footnotetext{%
  Albert Einstein, deutscher Physiker und Nobelpreistr"ager, 
  14.03.1879$-$14.04.1955.
}

% --------------------------------------------------------------------------
\subsection{Was sind Primzahlen?}
\index{Primzahl} \index{Zahlen!Primzahl} Primzahlen sind ganze,
positive Zahlen gr"o"ser gleich $2$, die man nur durch 1 und durch
sich selbst teilen kann. Alle anderen nat"urlichen Zahlen gr"o"ser
gleich $2$ lassen sich durch Multiplikation von Primzahlen
bilden.

Somit bestehen die {\em nat"urlichen} \index{Zahlen} Zahlen $ \mathbb{N} = \{1, 2,
3, 4,\cdots \} $ aus
\begin{itemize}
   \item der Zahl $1$ (dem Einheitswert)
   \item den Primzahlen (primes) und
   \item den zusammengesetzten Zahlen (composite numbers).
\end{itemize}

Primzahlen haben aus 3 Gr"unden besondere Bedeutung erlangt:
\begin{itemize}
  \item Sie werden in der Zahlentheorie als die Grundbausteine der
        nat"urlichen Zahlen betrachtet, anhand derer eine Menge genialer
        mathematischer "Uberlegungen gef"uhrt wurden.
  \item Sie haben in der modernen \index{Kryptographie!moderne} Kryptographie
        (Public Key \index{Kryptographie!Public Key} Kryptographie) gro"se
        praktische Bedeutung erlangt. Das verbreitetste Public Key Verfahren
        ist die Ende der siebziger Jahre erfundene \index{RSA} 
        RSA-Verschl"usselung. Nur die Verwendung (gro"ser) Primzahlen f"ur
        bestimmte Parameter garantiert die Sicherheit des Algorithmus sowohl
        beim RSA-Verfahren als auch bei noch moderneren Verfahren (digitale
        \index{Signatur!digitale} Signatur, Elliptische Kurven).
  \item Die Suche nach den gr"o"sten bekannten Primzahlen hat wohl bisher
        keine praktische Verwendung, erfordert aber die besten Rechner, gilt
        als hervorragender Benchmark (M"oglichkeit zur Leistungsbestimmung von
        Computern) und f"uhrt zu neuen Formen der Berechnungen auf mehreren
        Computern \\
        (siehe auch: \href{http://www.mersenne.org/prime.htm}
                      {\tt http://www.mersenne.org/prime.htm}).
\end{itemize}
Von Primzahlen lie"sen sich im Laufe der letzten zwei Jahrtausende sehr viele
Menschen faszinieren.
Der Ehrgeiz, zu neuen Erkenntnissen "uber Primzahlen zu gelangen, f"uhrte
dabei oft zu genialen Ideen und Schlussfolgerungen.
Im folgenden wird in einer leicht verst"andlichen Art in die mathematischen
Grundlagen der Primzahlen eingef"uhrt. Dabei kl"aren wir auch, was "uber die
Verteilung (Dichte, Anzahl von Primzahl in einem bestimmten Intervall) der
Primzahlen bekannt ist oder wie Primzahltests funktionieren.


% --------------------------------------------------------------------------
\subsection{Primzahlen in der Mathematik}\label{primesinmath}

Jede ganze Zahl hat Teiler. Die Zahl 1 hat nur einen, n"amlich
sich selbst. Die Zahl $12$ hat die sechs Teiler $1, 2, 3, 4, 6,
12$. Viele Zahlen sind nur durch sich selbst und durch $1$ teilbar. 
Bez"uglich der Multiplikation sind dies die \glqq Atome\grqq
~ im Bereich der Zahlen. Diese Zahlen nennt man Primzahlen.

In der Mathematik ist eine etwas andere (aber "aquivalente) Definition "ublich.

\begin{definition}\label{def-pz-prime}
Eine ganze Zahl $p \in {\bf N}$ hei"st Primzahl \index{Zahlen!Primzahl}, wenn $p > 1$ und $p$ nur die trivialen
Teiler $\pm 1$ und $\pm p$ besitzt.
\end{definition}


Per definitionem ist die Zahl $1$ keine Primzahl. Im weiteren bezeichnet der Buchstabe $p$ stets eine Primzahl.

Die Primzahlenfolge startet mit
$$ 2,~ 3,~ 5,~ 7, ~ 11, ~ 13, ~ 17, ~ 19, ~ 23, ~ 29, ~ 31, ~ 37, ~ 41, ~ 43, ~ 47, ~ 53, ~ 59, ~ 61, ~ 67, ~ 71, ~ 73, ~ 79, ~ 83, ~ 89, ~ 97, \cdots .$$
Unter den ersten 100 Zahlen gibt es genau 25 Primzahlen. Danach nimmt ihr prozentualer Anteil stets
ab. Primzahlen k"onnen nur auf eine einzige {\em triviale} Weise zerlegt werden:
$$5 = 1 \cdot 5,\quad  17 = 1 \cdot 17, \quad 1013 = 1 \cdot 1.013,  \quad 1.296.409 = 1 \cdot 1.296.409.$$
Alle Zahlen, die $2$ und mehr von 1 verschiedene Faktoren haben, nennt man \index{Zahlen!zusammengesetzte} {\em zusammengesetzte} Zahlen. Dazu geh"oren
$$ 4 = 2 \cdot 2, \quad 6 = 2\cdot 3 $$
aber auch Zahlen, die {\em wie Primzahlen aussehen}, aber doch keine sind:
$$ 91 = 7 \cdot 13, \quad 161=7 \cdot 23, \quad 767 =13 \cdot 59. $$

\begin{satz}\label{thm-pz-sqr}
Jede ganze Zahl $m$ gr"o"ser als $1$ besitzt einen kleinsten Teiler gr"o"ser als $1$.
Dieser ist eine Primzahl $p$. Sofern $m$ nicht selbst eine Primzahl ist, gilt:
$p$ ist kleiner oder gleich der Quadratwurzel aus $m$.
\end{satz}

Aus den Primzahlen lassen sich alle ganzen Zahlen gr"o"ser als $1$ zusammensetzen --- und das sogar in
einer eindeutigen Weise. Dies besagt der 1. Hauptsatz der Zahlentheorie (= Hauptsatz der elementaren Zahlentheorie =
fundamental theorem of arithmetic = fundamental building block of all positive integers).\index{Zahlentheorie!Hauptsatz}

\begin{satz}\label{thm-pz-prod}
Jedes Element $n$ gr"o"ser als $1$ der nat"urlichen Zahlen l"asst sich als Produkt
$n = p_1 \cdot p_2 \cdot \dots \cdot p_m$ von Primzahlen schreiben.
Sind zwei solche Zerlegungen
$$n =  p_1 \cdot p_2 \cdots p_m = p'_1 \cdot p'_2 \cdots p'_{m'}$$
gegeben, dann gilt nach eventuellem Umsortieren $\;m = m'\;$ und  f"ur alle $i$:  $\;p_i = p'_i$. \\
($p_1, p_2, \dots, p_m$ nennt man die Primfaktoren\index{Primfaktor} von n).
\end{satz}

In anderen Worten: Jede nat"urliche Zahl au"ser der $1$ l"asst sich auf genau eine Weise als Produkt von
Primzahlen schreiben, wenn man von der Reihenfolge der Faktoren absieht. Die Faktoren sind also
eindeutig (die {\em Expansion in Faktoren} ist eindeutig)!
Zum Beispiel ist
$$ 60 = 2 \cdot 2 \cdot 3 \cdot 5 = 2^2\cdot 3^1 \cdot 5^1 $$
Und das ist --- bis auf eine ver"anderte Reihenfolge der Faktoren
--- die einzige M"oglichkeit, die Zahl $60$ in Primfaktoren zu
zerlegen. Wenn man nicht nur Primzahlen als Faktoren zul"asst,
gibt es mehrere M"oglichkeiten der Zerlegung in Faktoren und die
Eindeutigkeit (\hypertarget{uniqueness}{uniqueness}) geht verloren:
$$ 60 = 1 \cdot 60 = 2 \cdot 30 = 4 \cdot 15 = 5 \cdot 12 =6 \cdot 10 = 2 \cdot 3 \cdot 10 =
        2 \cdot 5 \cdot 6 = 3 \cdot 4 \cdot 5 = \cdots . $$

Der folgende Absatz wendet sich eher an die mit der mathematischen Logik vertrauteren Menschen:
Der 1. Hauptsatz ist nur scheinbar selbstverst"andlich\label{remFundTheoOfArithm}. Man kann viele andere Zahlenmengen
(ungleich der positiven ganzen Zahlen gr"o"ser als 1) konstruieren, bei denen selbst eine Zerlegung in
die Primfaktoren dieser Mengen nicht eindeutig ist:
In der Menge $M = \{1, 5, 10, 15, 20, \cdots\}$ gibt es unter der Multiplikation kein Analogon zum Hauptsatz.
Die ersten f"unf Primzahlen dieser Folge sind $5, 10, 15, 20, 30$ (beachte: $10$ ist prim, da innerhalb
dieser Menge $5$ kein Teiler von $10$ ist --- das Ergebnis $2$ ist kein Element der gegebenen Grundmenge
$M$). Da in $M$ gilt:
$$ 100 = 5 \cdot 20 = 10 \cdot 10 $$
und sowohl $5, 10, 20$ Primzahlen dieser Menge sind, ist hier die Zerlegung in Primfaktoren nicht
eindeutig.

% --------------------------------------------------------------------------
\subsection{Wie viele Primzahlen gibt es?}

F"ur die nat"urlichen Zahlen sind die Primzahlen vergleichbar mit den 
Elementen in der Chemie oder den Elementarteilchen in der Physik 
(vgl. \cite[S. 22]{Blum1999}).

W"ahrend es nur $92$ nat"urliche chemische Elemente gibt, ist die Anzahl 
der Primzahlen unbegrenzt.
Das wusste schon der Grieche \index{Euklid} Euklid\footnote{%
Euklid, griechischer Mathematiker des 4./3. Jahrhunderts vor Christus.
Wirkte an der Akademie in Alexandria und verfasste mit den
\glqq Elementen\grqq~ das bekannteste systematische Lehrbuch
der griechischen Mathematik.}
im dritten vorchristlichen Jahrhundert.
\begin{satz}[Euklid\footnote{Die "ublich gewordene Benennung bedeutet nicht
unbedingt, dass Euklid der Entdecker des Satzes ist, da dieser
nicht bekannt ist. Der Satz wird bereits in Euklids \glqq Elementen\grqq ~(Buch IX, Satz 20) 
formuliert und bewiesen. Die dortige Formulierung ist insofern bemerkenswert,
als sie das Wort \glqq unendlich\grqq~ nicht verwendet; sie lautet
$$
O\acute{\iota}~\pi\varrho\tilde{\omega}\tau o \iota~\grave{\alpha}\varrho\iota\vartheta\mu o\grave{\iota}~
\pi\lambda\varepsilon\acute{\iota}o \upsilon\varsigma~\varepsilon\grave{\iota}\sigma\grave{\iota}~
\pi\alpha\nu\tau\grave{o}\varsigma~\tau o \tilde{\upsilon}~
\pi\varrho o \tau\varepsilon\vartheta\acute{\varepsilon}\nu\tau o \varsigma~
\pi\lambda\acute{\eta}\vartheta\ o \upsilon\varsigma~
\pi\varrho\acute{\omega}\tau\omega\nu~
\grave{\alpha}\varrho\iota\vartheta\mu\tilde{\omega}\nu,
$$
zu deutsch: Die Primzahlen sind mehr als jede vorgegebene Menge von Primzahlen.
}]\label{thm-pz-euklid}\hypertarget{thm-pz-euklid}{} % Ende der Fu"snote
Die Folge der Primzahlen bricht nicht ab, es gibt also unendlich 
viele Primzahlen.
\end{satz}

Sein Beweis, dass es unendlich viele Primzahlen gibt, gilt bis heute als 
ein Glanzst"uck mathematischer "Uberlegung und Schlussfolgerung 
(Widerspruchsbeweis\index{Widerspruchsbeweis}). 
Er nahm an, es gebe nur endlich viele Primzahlen und damit eine gr"o"ste
Primzahl. Daraus zog er solange logische Schl"usse, bis er auf einen
offensichtlichen Widerspruch stie"s. Damit musste etwas falsch sein. Da 
sich in die Schlusskette kein Lapsus eingeschlichen hatte, konnte es nur 
die Annahme sein. Demnach musste es unendlich viele Primzahlen geben!

\hypertarget{euklid}{}
\paragraph{Euklid's Widerspruchsbeweis}
\index{Euklid's Widerspruchsbeweis}\index{Widerspruchsbeweis}
f"uhrt die Argumentation wie folgt:

\begin{Beweis}{}
{\bf Annahme:} \quad Es gibt {\em endlich} viele Primzahlen.
\\*[4pt] {\bf Schluss:} \quad Dann lassen sie sich auflisten $p_1
< p_2 < p_3 < \dots < p_n$, wobei $n$ f"ur die (endliche) Anzahl
der Primzahlen steht. $p_n$ w"are also die gr"o"ste Primzahl. Nun
betrachtet Euklid die Zahl $a = p_1 \cdot p_2 \cdots p_n +1$.
Diese Zahl kann keine Primzahl sein, da sie in unserer
Primzahlenliste nicht auftaucht. Also muss sie durch eine Primzahl
teilbar sein. D.h. es gibt eine nat"urliche Zahl $i$ zwischen $1$
und $n$, so dass $p_i$ die Zahl $a$ teilt. Nat"urlich teilt $p_i$
auch das Produkt $a-1 = p_1 \cdot p_2 \cdots p_n$, da $p_i$ ja ein
Faktor von $a-1$ ist. Da $ p_i $ die Zahlen $ a $ und $ a-1 $
teilt, teilt sie auch die Differenz dieser Zahlen. Daraus folgt:
$p_i$ teilt  $a - (a-1) = 1$. $p_i$ m"usste also $1$ teilen und
das ist unm"oglich.\\*[4pt] 
{\bf Widerspruch}: \quad Unsere Annahme war falsch.\par
Also gibt es {\em unendlich} viele Primzahlen
(siehe \hyperlink{primhfk}{"Ubersicht unter \ref{s:primhfk}} "uber die
Anzahl von Primzahlen in verschiedenen Intervallen).
\end{Beweis} 

\par \vskip + 10pt

Wir erw"ahnen hier auch noch eine andere, auf den ersten Blick 
"uberraschende Tatsache, dass n"amlich in der Folge aller 
Primzahlen $p_1, p_2, \cdots $ L"ucken von beliebig gro"ser 
L"ange $n$ auftreten. Unter den $n$ aufeinanderfolgenden 
nat"urlichen Zahlen
$$ 
    (n+1)!+2, \cdots, (n+1)!+(n+1),
$$
ist keine eine Primzahl, da ja in ihnen der Reihe nach die 
Zahlen $2,\cdots, n+1$ als echte Teiler enthalten sind
(Dabei bedeutet $n!$ das Produkt der ersten $n$ nat"urlichen 
Zahlen, also $n!=n*(n-1)* \cdots *3*2*1$). 


% --------------------------------------------------------------------------
% --------------------------------------------------------------------------
\vskip + 20pt
\subsection{Die Suche nach sehr gro"sen Primzahlen}
\label{search_for_very_big_primes}

Die gr"o"sten heute bekannten Primzahlen haben mehrere
Millionen Stellen. Das ist unvorstellbar gro"s. Die Anzahl
der Elementarteilchen im Universum wird auf eine \glqq nur\grqq\
$80$-stellige Zahl gesch"atzt \hyperlink{grosord}{(siehe
"Ubersicht unter \ref{s:grosord} "uber verschiedene Gr"o"senordnungen /
Dimensionen)}.


% --------------------------------------------------------------------------
\hypertarget{RecordPrimes}{}
\subsubsection{Die 11 gr"o"sten bekannten Primzahlen (Stand M"arz 2006)} 
\index{Primzahlrekorde}
\label{RecordPrimes}

In der folgenden Tabelle sind die gr"o"sten bekanntesten Primzahlen und
eine Beschreibung des jeweiligen Zahlentyps aufgef"uhrt\footnote{%
Eine jeweils aktuelle Fassung findet sich im Internet unter 
     \href{http://primes.utm.edu/largest.html}
  {\texttt{http://primes.utm.edu/largest.html}}.
}:


%	10 & $1.372.930^{131.072}+1$ &   804.474 & 2003 & Verallg. Fermat\footnote{%
% $ 1.372.930^{131.072} + 1 = 1.372.930^{(2^{17})}+1 $ } \\

\begin{table}[h]
\begin{center}
\begin{tabular}{|c|cccc|}
\hline
 	& {\bf Definition} & {\bf Dezimalstellen} & {\bf Wann} & {\bf Beschreibung} \\
\hline    % Eyecatcher_neue-Mersenne
	1  & $2^{30.402.457}-1$ & 9.152.052 & 2005 & Mersenne, 43. bekannte \\
	2  & $2^{25.964.951}-1$ & 7.816.230 & 2005 & Mersenne, 42. bekannte \\
	3  & $2^{24.036.583}-1$ & 7.235.733 & 2004 & Mersenne, 41. bekannte \\
	4  & $2^{20.996.011}-1$ & 6.320.430 & 2003 & Mersenne, 40. bekannte \\
	5  & $2^{13.466.917}-1$ & 4.053.946 & 2001 & Mersenne, M-39 \\
	6  & $28.433 \cdot 2^{7.830.457}+1$ & 2.357.207 & 2004 & Verallgem. Mersenne \\
	7  & $2^{ 6.972.593}-1$ & 2.098.960 & 1999 & Mersenne, M-38 \\
	8  & $5.359 \cdot 2^{5.054.502}+1$ & 1.521.561 & 2003 & Verallgem. Mersenne \\
	9  & $2^{ 3.021.377}-1$ &   909.526 & 2001 & Mersenne, M-37 \\
	10  & $2^{ 2.976.221}-1$ &   895.932 & 2001 & Mersenne, M-36 \\
	11 & $1.372.930^{131.072}+1$ &   804.474 & 2003 & Verallgem. Fermat\footnotemark \\
\hline
\end{tabular}
\caption{Die 11 gr"o"sten Primzahlen und ihr jeweiliger Zahlentyp
         (Stand M"arz 2006)}
\end{center}
\end{table} 
\footnotetext{%
$ 1.372.930^{131.072} + 1 = 1.372.930^{(2^{17})}+1 $ 
} 
%be_2005: footnetmark und footnotetext statt nur footnote (sonst sieht man nur die
%         Fussnoten-Nr., aber nicht den Text!
%be_2005: Erzwingen, dass die Abb. noch in diesem Kapitel !

Die gr"o"ste bekannte Primzahl ist eine Mersenne-Primzahl.
Diese wurde vom \hyperlink{GIMPS-project}{GIMPS-Projekt}
(Kapitel~\ref{zahlentyp_mersenne}) gefunden.

Unter den gr"o"sten bekannten Primzahlen befinden sich au"serdem Zahlen
vom Typ \hyperlink{generalizedMersennenumbers}{verallgemeinerte Mersennezahl} 
(Kapitel~\ref{generalized-mersenne-no1})
und 
vom Typ \hyperlink{generalizedFermatprimes}{verallgemeinerte Fermatzahl} 
(Kapitel~\ref{generalized-fermat}).


% --------------------------------------------------------------------------
\hypertarget{MersenneNumbers01}{}
\subsubsection{Spezielle Zahlentypen -- Mersennezahlen und Mersenne-Primzahlen}
\label{zahlentyp_mersenne}
\index{Mersenne!Mersennezahl}

Nahezu alle bekannten riesig gro"sen Primzahlen sind spezielle 
Kandidaten, sogenannte {\em Mersennezahlen}\footnote{%
Marin Mersenne, franz"osischer Priester und Mathematiker,
08.09.1588$-$01.09.1648.
\index{Mersenne, Marin}
}
der Form $2^p -1,$ wobei $p$ eine Primzahl ist.
Nicht alle Mersennezahlen sind prim:
$$
\begin{array}{cl}
2^2 - 1 = 3 & \Rightarrow {\rm prim} \\
2^3 - 1 = 7 & \Rightarrow {\rm prim} \\
2^5 - 1 = 31    & \Rightarrow {\rm prim} \\
2^7 - 1 = 127    & \Rightarrow {\rm prim} \\
2^{11} - 1 = 2.047 = 23 \cdot 89    & \Rightarrow  {\rm NICHT~prim} !
\end{array}
$$

\index{Zahlen!Mersennezahl}
\index{Mersenne!Mersennezahl} 
\index{Mersenne!Satz} 

Dass Mersennezahlen nicht immer Primzahlen (Mersenne-Primzahlen) sind, 
wusste auch schon Mersenne (siehe Exponent $p = 11$).
Eine Mersennezahl, die prim ist, wird Mersenne-Primzahl
\index{Mersenne!Mersenne-Primzahl} genannt.  \\

Dennoch ist ihm der interessante Zusammenhang zu verdanken, dass eine 
Zahl der Form $2^n-1$ keine Primzahl sein kann, wenn $n$ eine 
zusammengesetzte Zahl ist:

\begin{satz}[Mersenne]\label{thm-pz-mersenne}
  Wenn $2^n - 1$ eine Primzahl ist, dann folgt, $n$ ist ebenfalls eine 
  Primzahl (oder anders formuliert: $2^n - 1$ ist nur dann prim, 
  wenn $n$ prim ist).
\end{satz}

\begin{Beweis}{}
Der Beweis des Satzes von Mersenne kann durch Widerspruch
\index{Widerspruchsbeweis}
durchgef"uhrt werden. Wir nehmen also an, dass es eine
zusammengesetzte nat"urliche Zahl $ n $ mit echter Zerlegung
$\; n=n_1 \cdot n_2 $
gibt, mit der Eigenschaft, dass $ 2^n -1 $ eine
Primzahl ist.

Wegen
\begin{eqnarray*}
(x^r-1)((x^r)^{s-1} + (x^r)^{s-2} + \cdots + x^r +1) & = &  ((x^r)^s + (x^r)^{s-1} + (x^r)^{s-2} + \cdots + x^r) \\
&  & -((x^r)^{s-1} + (x^r)^{s-2} + \cdots + x^r +1)  \\
& = & (x^r)^s -1 = x^{rs } -1,
\end{eqnarray*}
folgt
\[ 2^{n_1 n_2} - 1 = (2^{n_1} -1)((2^{n_1})^{n_2 -1} + (2^{n_1})^{n_2 -2} + \cdots + 2^{n_1} + 1). \]
Da $ 2^n - 1 $ eine Primzahl ist, muss einer der obigen beiden
Faktoren auf der rechte Seite gleich 1 sein. Dies kann nur dann
der Fall sein, wenn $ n_1 =1 $ oder $ n_2 =1$ ist. Dies ist aber
ein Widerspruch zu unserer Annahme. Deshalb ist unsere Annahme
falsch. Also gibt es keine zusammengesetzte Zahl $ n, $ so dass $
2^n -1 $ eine Primzahl ist.
\end{Beweis} 

\vskip + 5pt
\hypertarget{Mer-nums-not-always-prim}{}
Leider gilt dieser Satz nur in einer Richtung (die Umkehrung gilt
nicht, keine "Aquivalenz): das hei"st, dass es prime Exponenten gibt,
f"ur die die zugeh"orige Mersennezahl {\bf nicht} prim ist (siehe das obige 
Beispiel $2^{11}-1, $ wo $11$ prim ist, aber $2^{11}-1$ nicht).

Mersenne behauptete, dass $2^{67}-1$ eine Primzahl ist. Auch zu
dieser Behauptung gibt es eine interessante mathematische Historie:
Zuerst dauerte es "uber 200 Jahre, bis \index{Lucas, Edouard}
Edouard Lucas (1842-1891) bewies, dass diese Zahl zusammengesetzt
ist. Er argumentierte aber indirekt und kannte keinen der
Faktoren. Dann zeigte Frank Nelson Cole\index{Cole, Frank Nelson}\footnote{%
Frank Nelson Cole, amerikanischer Mathematiker, 20.09.1861$-$26.05.1926.} 1903,
aus welchen Faktoren diese Primzahl besteht:
$$ 2^{67} -1 =147. 573. 952. 589. 676. 412. 927 = 193. 707. 721 \cdot 761. 838. 257. 287. $$
Er gestand, 20 Jahre an der Faktorisierung \index{Faktorisierung} (Zerlegung in Faktoren)%
\footnote{%
  Mit CrypTool\index{CrypTool} k"onnen Sie Zahlen auf folgende Weise 
  faktorisieren: Men"u {\bf Einzelverfahren \textbackslash{} RSA-Kryptosystem 
  \textbackslash{} Faktorisieren einer Zahl}. \\
  In sinnvoller Zeit zerlegt CrypTool Zahlen bis 250 Bit L"ange.
  Zahlen gr"o"ser als 1024 Bit werden zur Zeit von CrypTool nicht angenommen. \\
  Die aktuellen Faktorisierungsrekorde finden Sie in Kapitel \ref{NoteFactorisation}.
  \index{Faktorisierung!Faktorisierungsrekorde}
}
dieser 21-stelligen Dezimalzahl gearbeitet zu haben!

Dadurch, dass man bei den Exponenten der Mersennezahlen nicht alle 
nat"urlichen Zahlen verwendet, sondern nur die Primzahlen, engt man
den {\em Versuchsraum} deutlich ein. Die derzeit bekannten 
Mersenne-Primzahlen \index{Mersenne!Mersenne-Primzahl} gibt es 
f"ur die Exponenten\footnote{%
Auf der folgenden Seite von Landon Curt Noll\index{Noll, Landon Curt} 
werden in einer Tabelle alle Mersenne-Primzahlen samt Entdeckungsdatum und 
Wert in Zahlen- und Wortform aufgelistet:
      \href{http://www.isthe.com/chongo/tech/math/prime/mersenne.html}
   {\texttt{http://www.isthe.com/chongo/tech/math/prime/mersenne.html}}. \\
Siehe auch:
      \href{http://www.utm.edu/}
   {\texttt{http://www.utm.edu/}}.
                             }
$$
\begin{array}{c}
2, ~ 3, ~ 5, ~ 7, ~ 13, ~ 17, ~ 19, ~ 31, ~ 61, ~ 89, ~ 107, ~ 127, ~ 521, ~ 607, ~ 1.279, ~ 2.203, ~ 2.281, ~ 3.217, ~ 4.253, \\
4.423, ~9.689, ~ 9.941, ~ 11.213, ~ 19.937, ~ 21.701, ~ 23.207, ~ 44.497, ~ 86.243, ~ 110.503, ~ 132.049,\\
216.091, ~ 756.839, ~ 859.433, ~ 1.257.787, ~ 1.398.269, ~ 2.976.221, ~ 3.021.377, ~ 6.972.593,\\
 ~ 13.466.917,  ~ 20.996.011,  ~ 24.036.583,  ~ 25.964.951,  ~ 30.402.457.
% be_2005_UPDATEN_if-new-mersenne-prime-appears          ~ xxx.xxx.xxx.
\end{array}
$$
Damit sind heute    % Eyecatcher_neue-Mersenne
$43$                  % be_2005_UPDATEN_if-new-mersenne-prime-appears
Mersenne-Primzahlen\index{Primzahl!Mersenne}\index{Mersenne!Mersenne-Primzahl} 
bekannt. F"ur die ersten 39              % be_2005_UPDATEN_if-new-mersenne-prime-appears    % Eyecatcher_neue-Mersenne
Mersenne-Primzahlen wei"s man inzwischen, dass diese Liste vollst"andig ist. 
Die Exponenten bis zur 40.  % be_2005_UPDATEN_if-new-mersenne-prime-appears    % Eyecatcher_neue-Mersenne
bekannten Mersenne-Primzahl sind noch nicht vollst"andig gepr"uft\footnote{%
Den aktuellen Status der Pr"ufung findet man auf der Seite:
      \href{http://www.mersenne.org/status.htm}
   {\texttt{http://www.mersenne.org/status.htm}}.\\
Hinweise, wie man Zahlen auf ihre Primalit"at pr"ufen kann, finden sich
in Kapitel \ref{primality_tests}, Primzahltests\index{Primzahltest}.
                                                                          }. 

Die $19$. Zahl
mit dem Exponenten $4.253$ war die erste mit mindestens $1.000$
Stellen im Zehnersystem (der Mathematiker Samual \index{Yates,
Samual} Yates pr"agte daf"ur den Ausdruck {\em titanische}
\index{Primzahl!titanische} Primzahl; sie wurde 1961 von Hurwitz
gefunden); die $27$. Zahl mit dem Exponenten $44.497$ war die
erste mit mindestens $10.000$ Stellen im Zehnersystem (Yates
pr"agte daf"ur den Ausdruck \index{Primzahl!gigantische}  {\em
gigantische} Primzahl. Diese Bezeichnungen sind heute l"angst
veraltet).

% \vskip - 4pt \noindent
% \begin{quote}
% (Der Supercomputerhersteller SGI Cray Research besch"aftigte nicht
% nur hervorragende Mathematiker, sondern benutzte die Primzahltests
% auch als Benchmarks f"ur seine Maschinen.)
% \end{quote}

% \vspace{-10pt}
% \begin{itemize}
%  \item[] \href{http://reality.sgi.com/chongo/prime/prime_press.html}
%               {\texttt{http://reality.sgi.com/chongo/prime/prime\_press.html}}
% \end{itemize}


\vskip +25 pt
\paragraph{M-37 -- Januar 1998} \index{Mersenne!Mersenne-Primzahl!M-37}\mbox{}

Die 37. Zahl Mersenne-Primzahl, $$2^{3.021.377} - 1 $$
wurde im Januar 1998 gefunden und hat
909.526 Stellen im Zehnersystem, was 33 Seiten in der FAZ entspricht!


\vskip +25 pt
\paragraph{M-38 -- Juni 1999}\index{Mersenne!Mersenne-Primzahl!M-38}\mbox{}

Die 38. Mersenne-Primzahl, genannt M-38, $$ 2^{6.972.593} - 1 $$
wurde im Juni 1999 gefunden und hat $2.098.960$ Stellen im
Zehnersystem (das entspricht rund 77 Seiten in der FAZ).


\vskip +25 pt
\hypertarget{M-39}{}
\paragraph{M-39 -- Dezember 2001}
\index{Mersenne!Mersenne-Primzahl!M-39}\mbox{}

Die 39. Mersenne-Primzahl, genannt M-39, $$2^{13.466.917}-1$$ 
wurde am 6.12.2001 bekanntgegeben: genau genommen war am 6.12.2001 
die Verifikation der am 14.11.2001 von dem kanadischen Studenten 
Michael Cameron gefundenen Primzahl abgeschlossen.
Diese Zahl hat rund 4 Millionen Stellen (genau 4.053.946 Stellen). 
Allein zu ihrer Darstellung
$$
924947738006701322247758 \; \cdots  \; 1130073855470256259071
$$
br"auchte man in der FAZ knapp 200 Seiten.

Inzwischen (Stand Mai 2005) wurden alle primen Exponenten kleiner als
$ 13.466.917 $ getestet und nochmal gepr"uft (siehe die Homepage des 
GIMPS-Projekts: {\href{http://www.mersenne.org} {\tt http://www.mersenne.org}}):
somit k"onnen wir sicher sein, dass dies wirklich 
die 39. Mersenne-Primzahl ist und dass keine kleineren unentdeckten
Mersenne-Primzahlen existieren (es ist "ublich, die Bezeichnung M-nn erst dann
zu verwenden, wenn die nn. bekannte Mersenne-Primzahl auch bewiesenerma"sen die
nn. Mersenne-Primzahl ist).

%\vskip +25 pt
%\paragraph{Mxxxxxxxxx -- Juni 2003 -- M-40 ?}
%\index{Mersenne!Mersenne-Primzahl!M-40}\mbox{}

%Als 40. Mersenne-Primzahl wurde die folgende Zahl gefunden (und schon als M-40
%bezeichnet, obwohl noch nicht sicher ist, ob es zwischen M-39 und
%Mxxxxxxxxxx nicht noch weitere Mersenne-Primzahlen gibt), $$2^{xx.xxx.xxx}-1$$ 
%Bekanntgegeben wurde sie am xx.06.2003: genau genommen war am xx.06.2003 
%die Verifikation der am 02.06.2003 von xxxxxxxxxxxxxxxx 
%gefundenen Primzahl abgeschlossen. 
%Initiator und Projektleiter George Woltman gibt eine gefundene Mersenne-Zahl
%erst bekannt, wenn eine Kontrollrechnung best"atigt, dass sie prim ist.
%Diese Zahl hat rund xxx Millionen Stellen (genau xx.xxx.xxx Dezimalstellen).
%Die erste Meldung in Deutsch dazu erschien 
%m.W. im Heise-Ticker\index{Heise-Ticker}:
%{\href{http://www.heise.de/newsticker/data/as-02.06.03-000/} 
%  {\tt http://www.heise.de/newsticker/data/as-02.06.03-000/} }.




\vskip +25 pt
\paragraph{GIMPS}\index{GIMPS}\mbox{}
\hypertarget{GIMPS-project}{}

Das GIMPS-Projekt\index{GIMPS} (Great Internet Mer"-senne-Prime Search)
wurde 1996 von George Woltman\index{Woltman, George} gegr"undet, um neue 
gr"o"ste Mersenne-Primzahlen zu finden 
({\href{http://www.mersenne.org} {\tt http://www.mersenne.org}}).
Genauere Erl"auterungen zu diesem Zahlentyp finden sich unter
\hyperlink{MersenneNumbers02}{Mersennezahlen} und 
\hyperlink{MersenneNumbers01}{Mersenne-Primzahlen}.

Bisher hat das GIMPS-Projekt nine
   % be_2005_UPDATEN_if-new-mersenne-prime-appears    % Eyecatcher_neue-Mersenne
gr"o"ste Mersenne-Primzahlen entdeckt, inclusive der gr"o"sten bekannten
Primzahl "uberhaupt. 

Die folgende Tabelle enth"alt diese Mersenne Rekord-Primzahlen\footnote{%
Eine up-to-date gehaltene Version dieser Tabelle steht im Internet unter
     \href{http://www.mersenne.org/history.htm}
  {\texttt{http://www.mersenne.org/history.htm}}.
}$^,$\footnote{%
Bei jedem neuen Rekord, der gemeldet wird, beginnen in den einschl"agigen
Foren die immer die gleichen, oft ironischen Diskussion: Hat diese Forschung
einen tieferen Sinn? Lassen sich diese Ergebnisse f"ur irgendwas verwenden?
Die Antwort ist, dass das noch unklar ist. Bei Grundlagenforschung sieht man
oft gleich, wie es die Menschheit voranbringt.
}:

   % be_2005_UPDATEN_if-new-mersenne-prime-appears    % Eyecatcher_neue-Mersenne
\begin{table}[h]
\begin{center}
\begin{tabular}{|cccc|}
\hline
	{\bf Definition} & {\bf Dezimalstellen} & {\bf Wann} & {\bf Wer} \\
\hline
	$2^{30.402.457}-1$ & 9.152.052 & 15. Dezember 2005 & Curtis Cooper/Steven Boone \\
	$2^{25.964.951}-1$ & 7.816.230 & 18. Februar 2005  & Martin Nowak \\
	$2^{24.036.583}-1$ & 7.235.733 & 15. Mai 2004      & Josh Findley     \\
	$2^{20.996.011}-1$ & 6.320.430 & 17. November 2003 & Michael Shafer   \\
	$2^{13.466.917}-1$ & 4.053.946 & 14. November 2001 & Michael Cameron  \\
	$2^{ 6.972.593}-1$ & 2.098.960 & 1. Juni 1999      & Nayan Hajratwala \\
	$2^{ 3.021.377}-1$ &   909.526 & 27. Januar 1998   & Roland Clarkson  \\
	$2^{ 2.976.221}-1$ &   895.932 & 24. August 1997   & Gordon Spence    \\
	$2^{ 1.398.269}-1$ &   420.921 & November 1996     & Joel Armengaud   \\
\hline
\end{tabular}
\caption{Die gr"o"sten vom GIMPS-Projekt gefundenen Primzahlen (Stand M"arz 2006)}
\end{center}
\end{table} 

%be_2005: Erzwingen, dass die Abb. noch in diesem Kapitel !

Dr. Richard Crandall\index{Crandall, Richard} erfand den Transformations-Algorithmus,
der im GIMPS-Programm benutzt wird. George Woltman implementierte 
Crandall's Algorithmus in Maschinensprache, wodurch das Primzahlenprogramm 
eine vorher nicht da gewesene Effizienz erhielt. 
Diese Arbeit f"uhrte zum GIMPS-Projekt.

Am 1. Juni 2003 wurde dem GIMPS-Server eine Zahl gemeldet, die evtl. die 40. 
Mersenne-Primzahl sein konnte. Diese wurde dann wie "ublich "uberpr"uft,
bevor sie ver"offentlich werden sollte. 
Leider musste der Initiator und GIMPS-Projektleiter George Woltman Mitte Juni
melden, dass diese Zahl zusammengesetzt war (dies war die erste falsche
positive R"uckmeldung eines Clients an den Server in 7 Jahren).

Am GIMPS-Projekt beteiligen sich z.Zt. rund 130.000 
freiwillige Amateure und Experten, die ihre Rechner in das von der 
Firma entropia organisierte \glqq primenet\grqq~ einbinden.






% --------------------------------------------------------------------------
\vskip +25 pt
\subsubsection{Wettbewerb der Electronic Frontier Foundation (EFF)}\index{EFF}
Angefacht wird diese Suche noch zus"atzlich durch einen Wettbewerb, den die
Non\-profit-Orga"-nisa"-tion EFF (Electronic Frontier Foundation) mit den
Mitteln eines unbekannten Spenders gestartet hat. Den Teilnehmern winken 
Gewinne im Gesamtwert von 500.000 USD, wenn sie die l"angste Primzahl
finden. Dabei sucht der unbekannte Spender nicht nach dem schnellsten Rechner,
sondern er will auf die M"oglichkeiten des {\em cooperative networking} 
aufmerksam machen: \\
{\href{http://www.eff.org/coopawards/prime-release1.html}{\tt http://www.eff.org/coopawards/prime-release1.html}}

Der Entdecker von M-38 erhielt f"ur die Entdeckung der ersten Primzahl
mit "uber 1 Million Dezimalstellen von der EFF eine Pr"amie von 50.000 USD. 

% be_2005_UPDATEN_if-new-mersenne-prime-appears
Die n"achste Pr"amie von 100.000 USD gibt es von der EFF f"ur eine Primzahl
mit mehr als 10 Millionen Dezimalstellen.

Nach den Preisregeln der EFF sind dann als n"achste Stufe 150.000 US-Dollar
f"ur eine Primzahl mit mehr als 100 Millionen Stellen ausgelobt.                         


Edouard Lucas\index{Lucas, Edouard} (1842-1891) hielt "uber 70 Jahre den 
Rekord der gr"o"sten bekannten Primzahl, indem er nachwies, dass 
$2^{127}-1$ prim ist. Solange wird wohl kein neuer Rekord mehr Bestand haben.


% --------------------------------------------------------------------------
\subsection{Primzahltests}
\label{primality_tests}   % chap. 3.5
\index{Primzahltest}

F"ur die Anwendung sicherer Verschl"usselungsverfahren braucht man sehr gro"se
Primzahlen (im Bereich von $2^{2.048}$, das sind Zahlen im Zehnersystem mit
"uber $600$ Stellen!).

Sucht man nach den Primfaktoren, um zu entscheiden, ob eine Zahl prim ist,
dauert die Suche zu lange, wenn auch der kleinste Primfaktor riesig ist.
Die Zerlegung in Faktoren mittels rechnerischer systematischer Teilung oder
mit dem \hyperlink{SieveEratosthenes01}{Sieb des Eratosthenes} 
\index{Eratosthenes!Sieb} ist mit heutigen Computern anwendbar 
f"ur Zahlen mit bis zu circa $20$ Stellen im Zehnersystem.
Die gr"o"ste Zahl, die bisher in ihre beiden ann"ahernd gleich gro"sen
Primfaktoren zerlegt werden konnte, hatte 200 Stellen 
(vgl. \hyperlink{RSA-200-chap3}{RSA-200} in Kapitel~\ref{NoteFactorisation}).
% be_2005 Das ist eine gute Art zu referenzieren !!!!!!!!!!!!!!!

% be_2005_UPDATEN_if-new-factorization-record-appears

Ist aber etwas "uber die {\em Bauart} (spezielle Struktur) der fraglichen 
Zahl bekannt, gibt es sehr hochentwickelte Verfahren, die deutlich schneller
sind. Diese Verfahren beantworten nur die Primalit"atseigenschaft einer Zahl,
k"onnen aber nicht die Primfaktoren sehr gro"ser zusammengesetzter Zahlen
bestimmen.

\hypertarget{FermatNumbers01}{}\label{FermatNumbers01}%
Fermat\footnote{%
Pierre de Fermat, franz"osischer Mathematiker, 17.8.1601 -- 12.1.1665.
\index{Fermat, Pierre}
}
\index{Fermat, Pierre} hatte im 17. Jahrhundert an Mersenne
\index{Mersenne, Marin} geschrieben, dass er vermute, 
dass alle Zahlen der Form $$ f(n) = 2^{2^n} + 1 $$ 
f"ur alle ganzen Zahlen $ n \geq 0 $ prim seien 
(\hyperlink{FermatNumbers02}{siehe unten}, Kapitel~\ref{L-FermatNumbers02}).
\index{Zahlen!Fermatzahl}\index{Fermat!Fermatzahl}

Schon im 19. Jahrhundert wusste man, dass die $29$-stellige Zahl
$$ f(7) = 2^{2^7} + 1 $$
keine Primzahl ist. Aber erst 1970 fanden Morrison/Billhart ihre Zerlegung.
\begin{eqnarray*}\label{F7Morrison}
f(7) & = & 340.282.366.920.938.463.463.374.607.431.768.211.457 \\
& = & 59. 649. 589. 127. 497. 217 \cdot  5.704.689.200.685.129.054.721
\end{eqnarray*}

Auch wenn sich Fermat bei seiner Vermutung irrte, so stammt in diesem
Zusammenhang von ihm doch ein sehr wichtiger Satz: Der (kleine)
Fermatsche Satz, den Fermat im Jahr 1640 aufstellte, ist der 
Ausgangspunkt vieler schneller Primzahltests 
(\hyperlink{KleinerSatzFermat-chap3}{siehe Kap.
\ref{Label_KleinerSatzFermat-chap3}}).

\hypertarget{KleinerSatzFermat-chap2}{}
\index{Fermat!kleiner Satz}
\begin{satz}[\glqq kleiner\grqq\ Fermat]\label{thm-pz-fermat1}
Sei $p$ eine Primzahl und $a$ eine beliebige ganze Zahl, dann gilt f"ur
alle $a$ $$a^p \equiv a \; {\rm mod} \; p.$$
Eine alternative Formulierung lautet: \\
Sei $p$ eine Primzahl und $a$ eine beliebige ganze Zahl, die kein
Vielfaches von $p$ ist (also $a \not\equiv 0 \; {\rm mod} \; p$),
dann gilt $a^{p-1} \equiv 1 \; {\rm mod} \; p$.
\end{satz}

Wer mit dem Rechnen mit Resten (Modulo-Rechnung) nicht so vertraut ist, 
m"oge den Satz einfach so hinnehmen oder erst \hyperlink{Chapter_ElementaryNT}
{Kapitel \ref{Chapter_ElementaryNT} \glqq Einf"uhrung
in die elementare Zahlentheorie mit Beispielen\grqq} lesen.
Wichtig ist, dass aus diesem Satz folgt, dass wenn
diese Gleichheit f"ur irgendein ganzes $a$ nicht erf"ullt ist, dann ist
$p$ keine Primzahl! Die Tests lassen sich (zum Beispiel f"ur die erste
Formulierung) leicht mit der {\em Testbasis} $a = 2$ durchf"uhren.

Damit hat man ein Kriterium f"ur Nicht-Primzahlen, also einen negativen
Test, aber noch keinen Beweis, dass eine Zahl $a$ prim ist.
Leider gilt die Umkehrung zum Fermatschen Satz nicht, sonst h"atten
wir einen einfachen Beweis f"ur die Primzahleigenschaft (man sagt auch,
man h"atte dann ein einfaches Primzahlkriterium).


\vskip +25 pt
\paragraph{Pseudoprimzahlen}%
\index{Primzahl!Pseudoprimzahl}\index{Zahlen!Pseudoprimzahl}%
\hypertarget{HT--Pseudoprimenumber01}{}\label{L-Pseudoprimenumber01}%
\mbox{}
\vskip +10 pt
Zahlen n, die die Eigenschaft
$$ 2^n \equiv 2 \;{\rm mod}\; n $$
erf"ullen, aber nicht prim sind, bezeichnet man als {\em Pseudoprimzahlen} 
(der Exponent n ist also keine Primzahl).
Die erste Pseudoprimzahl ist $$ 341 = 11 \cdot 31 .$$


\vskip +25 pt
\paragraph{Carmichaelzahlen}%
\index{Zahlen!Carmichaelzahl}%
\hypertarget{HT-Carmichael-number01}{}\label{L-Carmichael-number01}%
\mbox{}
\vskip +10 pt
Es gibt Pseudoprimzahlen n, die den Fermat-Test 
$$ a^{n-1} \equiv 1 \;{\rm mod}\; n $$
mit allen Basen a, die teilerfremd zu n sind [$ gcd (a,n) = 1 $], bestehen,
obwohl die zu testenden Zahlen n nicht prim sind: Diese Zahlen hei"sen
{\em Carmichaelzahlen}. Die erste ist
$$ 561 = 3 \cdot 11 \cdot 17 .$$

Beispiel: Die zu testende Zahl sei 561.
Da $561 = 3 \cdot 11 \cdot 17$ ist, ergibt sich: \\
Die Testbedingung $a^{560} \;{\rm mod}\; 561 = 1$ \\
ist erf"ullt f"ur $a = 2, 4, 5, 7, \cdots $, \\
aber nicht f"ur $a = 3, 6, 9, 11, 12, 15, 17, 18, 21, 22, \cdots$.\\
D.h. die Testbedingung muss nicht erf"ullt sein, wenn die Basis ein
Vielfaches von 3, von 11 oder von 17 ist.\\
Der Test angewandt auf $a=3$ ergibt: $3^{560} \;{\rm mod}\; 561 = 375$. \\
Der Test angewandt auf $a=5$ ergibt: $5^{560} \;{\rm mod}\; 561 = 1$.


\vskip +25 pt
\paragraph{Starke Pseudoprimzahlen}%
\index{Primzahl!starke Pseudoprimzahlen} \index{Zahlen!starke Pseudoprimzahl}%
\hypertarget{HT-Strongpseudoprimenumber01}{}\label{L-Strongpseudoprimenumber01}%
\mbox{}
\vskip +10 pt
Ein st"arkerer Test stammt von\index{Miller Gary L.}\index{Rabin, Michael O.}
Miller/Rabin\footnote{
1976 ver"offentlichte Prof. Rabin einen effizienten probabilistischen
Primzahltest, der auf einem zahlentheoretischen Ergebnis von Prof. Miller aus
der Jahr davor basierte. \\
Prof. Miller arbeitete an der Carnegie-Mellon Universit"at, School of Computer
Science. Prof. Rabin, geboren 1931, arbeitete an der Harvard und Hebrew
Universit"at.
}%
: er wird nur von sogenannten {\em starken Pseudoprimzahlen} bestanden. 
Wiederum gibt es starke Pseudoprimzahlen, die keine Primzahlen sind, aber
das passiert deutlich seltener als bei den (einfachen) Pseudoprimzahlen oder
bei den Carmichaelzahlen. Die kleinste starke Pseudoprimzahl zur Basis $2$ ist
$$ 15.841 = 7 \cdot 31 \cdot 73. $$
Testet man alle 4 Basen $2, 3, 5$ und $7$, so findet man bis $25 \cdot 10^9$
nur eine starke Pseudoprimzahl, also eine Zahl, die den Test besteht und doch keine Primzahl ist.

Weiterf"uhrende Mathematik hinter dem Rabin-Test gibt dann die Wahrscheinlichkeit an, mit der die
untersuchte Zahl prim ist (solche Wahrscheinlichkeiten liegen heutzutage bei circa  $10^{-60}$).

Ausf"uhrliche Beschreibungen zu Tests, um herauszufinden, ob eine Zahl
prim ist, finden sich zum Beispiel unter:
\vspace{-10pt}
\begin{itemize}
  \item[] \href{http://www.utm.edu/research/primes/mersenne.shtml}
               {\texttt{http://www.utm.edu/research/primes/mersenne.shtml}} \\
          \href{http://www.utm.edu/research/primes/prove/index.html}
               {\texttt{http://www.utm.edu/research/primes/prove/index.html}} 
\end{itemize}



% --------------------------------------------------------------------------
\vskip +30 pt
\subsection{"Ubersicht Spezial-Zahlentypen und die Suche nach einer Formel f"ur
Primzahlen}\label{spezialzahlentypen}
\index{Primzahl!Formel}
Derzeit sind keine brauchbaren, offenen (also nicht rekursiven) Formeln bekannt, 
die nur Primzahlen liefern (rekursiv bedeutet, dass zur Berechnung der Funktion
auf dieselbe Funktion in Abh"angigkeit einer kleineren Variablen zugegriffen wird).
Die Mathematiker w"aren schon zufrieden, wenn sie eine Formel f"anden, die wohl 
L"ucken l"asst (also nicht alle Primzahlen liefert), aber sonst keine
zusammengesetzten Zahlen (Nicht-Primzahlen) liefert.

Optimal w"are, man w"urde f"ur das Argument $n$ sofort die $n$-te Primzahl 
bekommen, also f"ur
$f(8) = 19\,$ oder f"ur  $f(52) = 239$.

Ideen dazu finden sich in
\vspace{-10pt}
\begin{itemize}
  \item[] {\href{http://www.utm.edu/research/primes/notes/faq/p_n.html}
          {\tt http://www.utm.edu/research/primes/notes/faq/p\_n.html}}.
\end{itemize}


\hyperlink{ntePrimzahl}{Die Tabelle unter \ref{s:ntePrimzahl}} enth"alt 
die exakten Werte f"ur die $n$-ten Primzahlen f"ur ausgew"ahlte $ n.$
\\

%  \glqq Primzahlformeln''
%  \glqq Primzahlformeln \grqq  (das Blank vor UND nach dem Text ist n�tig!
%        be_20050527: Aber dann war das Blank innerhalb der Hochkommata -> auch nicht gut
%  Fehlt es danach, wird der folgende Text direkt ohne Blank dahintergestellt!
F"ur \glqq Primzahlformeln'' werden meist ganz spezielle Zahlentypen
benutzt. Die folgende Aufz"ahlung enth"alt die verbreitetsten Ans"atze 
f"ur \glqq Primzahlformeln'', und welche Kenntnisse wir "uber
sehr gro"se Folgeglieder haben: Konnte die Primalit"at bewiesen werden?
Wenn es zusammengesetzte Zahlen sind, konnten die Primfaktoren bestimmt werden?


% --------------------------------------------------------------------------
\vskip +10 pt
\hypertarget{MersenneNumbers02}{}
\subsubsection{Mersennezahlen $f(n) = 2^n - 1$ f"ur $ n $ prim}
    \index{Primzahl!Mersenne} \index{Mersenne!Mersenne-Primzahl}
    Wie \hyperlink{MersenneNumbers01}{oben} gesehen, liefert diese 
    Formel wohl relativ viele gro"se Primzahlen, aber es kommt -- wie 
    f"ur $n=11$ [$f(n)=2.047$] -- immer wieder vor, dass das Ergebnis 
    auch bei primen Exponenten \hyperlink{Mer-nums-not-always-prim}{nicht}
    prim ist. \\
    Heute kennt man alle Mersenne-Primzahlen mit bis zu ca. 4.000.000 
    Dezimalstellen (\hyperlink{M-39}{M-39}
    \index{Mersenne!Mersenne-Primzahl!M-39}): \\
\vspace{-10pt}
\begin{itemize}
  \item[] {\href{http://perso.wanadoo.fr/yves.gallot/primes/index.html}
           {\tt http://perso.wanadoo.fr/yves.gallot/primes/index.html}}
\end{itemize}     % be_2005_UPDATEN_if-new-mersenne-prime-appears

% --------------------------------------------------------------------------
\vskip +10 pt
\subsubsection
    [Verallgemeinerte Mersennezahlen $f(k,n) = k \cdot 2^n \pm 1$]
    {Verallgemeinerte Mersennezahlen 
    $f(k,n) = k \cdot 2^n \pm 1 $ f"ur $ n $ prim und $ k $ kleine Primzahlen}
% \quad f"ur
    \label{generalized-mersenne-no1}
    F"ur diese 1. Verallgemeinerung der 
    Mersennezahlen\index{Mersenne!Mersennezahl!verallgemeinerte} 
    gibt es (f"ur kleine $k$) 
    ebenfalls sehr schnelle Primzahltests (vgl. \cite{Knuth1981}).
    Praktisch ausf"uhren l"asst sich das zum Beispiel mit der Software 
    Proths von Yves Gallot\index{Gallot, Yves} 
    ({\href{http://www.prothsearch.net/index.html}
           {\tt http://www.prothsearch.net/index.html}}).


% --------------------------------------------------------------------------
\vskip +10 pt
\hypertarget{generalizedMersennenumbers}{}
\subsubsection
    [Verallgemeinerte Mersennezahlen $ f(b,n) = b^n \pm 1$ / Cunningham-Projekt]
    {Verallgemeinerte Mersennezahlen
    $ f(b,n) = b^n \pm 1$ / Cunningham-Projekt}

Dies ist eine 2. m"ogliche Verallgemeinerung der 
    Mersennezahlen\index{Mersenne!Mersennezahl!verallgemeinerte}.
    Im \index{Cunningham-Projekt} \textbf{Cunningham-Projekt} werden die
    Faktoren aller zusammengesetzten Zahlen bestimmt, die sich in folgender
    Weise bilden:
    $$ f(b,n) = b^n \pm 1  \quad {\rm f"ur~} b = 2, 3, 5, 6, 7, 10, 11, 12 $$
    ($b$ ist ungleich der Vielfachen von schon benutzten Basen wie $4, 8, 9$).

    Details hierzu finden sich unter:
\vspace{-10pt}
\begin{itemize}
  \item[] \href{http://www.cerias.purdue.edu/homes/ssw/cun}
               {\tt http://www.cerias.purdue.edu/homes/ssw/cun}
\end{itemize}


% --------------------------------------------------------------------------
\vskip +10 pt
\hypertarget{FermatNumbers02}{}
\subsubsection[Fermatzahlen $f(n) = 2^{2^n} + 1$]
              {Fermatzahlen\footnotemark~$f(n) = 2^{2^n} + 1$}
    \footnotetext{%
       Die Fermatschen Primzahlen spielen unter
       anderem eine wichtige Rolle in der Kreisteilung. 
       Wie Gauss \index{Gauss, Carl Friedrich}
       bewiesen hat, ist das regul"are $p$-Eck f"ur eine Primzahl $p>2$ dann
       und nur dann mit Zirkel und Lineal konstruierbar, wenn $p$ eine 
       Fermatsche Primzahl ist.
    }
    \label{L-FermatNumbers02}
    \index{Zahlen!Fermatzahl}\index{Fermat!Fermatzahl}
    \index{Primzahl!Fermat} \index{Fermat!Fermat-Primzahl}
    Wie \hyperlink{FermatNumbers01}{oben} in Kapitel~\ref{FermatNumbers01}
    erw"ahnt, schrieb Fermat an 
    Mersenne, dass er vermutet, dass alle Zahlen dieser Form prim seien. 
    Diese Vermutung wurde jedoch von Euler (1732) widerlegt. Es gilt $641 | f(5)$\footnote{%
    Erstaunlicherweise kann man mit Hilfe des Satzes von Fermat diese Zahl leicht finden (siehe z.B. 
    \cite[S. 176]{Scheid1994})
    }. 
    
	       
%    Erstaunlicherweise w"are es f"ur ihn m"oglich gewesen, mit dem auf 
%    seinem kleinen Satz beruhenden negativen Primzahltest f"ur $n=5$ ein 
%    positives Ergebnis zu erhalten.
$$
\begin{array}{lll}
f(0) = 2^{2^0} + 1  = 2^1 + 1 & = 3 &   \mapsto {\rm ~prim}  \\
f(1) = 2^{2^1} + 1  = 2^2 + 1 & = 5 &   \mapsto {\rm ~prim}  \\
f(2) = 2^{2^2} + 1  = 2^4 + 1 & = 17 &  \mapsto {\rm ~prim}  \\
f(3) = 2^{2^3} + 1  = 2^8 + 1 & = 257 & \mapsto {\rm ~prim}  \\
f(4) = 2^{2^4} + 1  = 2^{16} + 1 &  = 65.537 &  \mapsto {\rm ~prim}  \\
f(5) = 2^{2^5} + 1  = 2^{32} + 1 &  = 4.294.967.297 = 641 \cdot
6.700.417 &  \mapsto {\rm ~NICHT~prim} !\\
f(6) = 2^{2^6} + 1  = 2^{64} + 1 &  = 18.446.744.073.709.551.617 \\
                                 &  = 274.177 \cdot 67.280.421.310.721  
																 & \mapsto {\rm ~NICHT~prim} ! \\
f(7) = 2^{2^7} + 1  = 2^{128} + 1 & = \mbox{(siehe Seite~\pageref{F7Morrison})}  
																 & \mapsto {\rm ~NICHT~prim} !
															
\end{array}
$$

    Innerhalb des Projektes ``Distributed Search for Fermat Number Dividers'',
    das von Leonid Durman angeboten wird, gibt es ebenfalls Fortschritte 
    beim Finden von neuen Primzahl-Riesen
    ({\href{http://www.fermatsearch.org/}
       {\tt http://www.fermatsearch.org/}}  --  diese Webseite hat 
    Verkn"upfungen zu Seiten in russisch, italienisch und deutsch).
    \begin{sloppypar}
    Die entdeckten Faktoren k"onnen sowohl zusammengesetzte nat"ur"-liche als
    auch prime nat"urliche Zahlen sein.
    \end{sloppypar}
     
    Am 22. Februar 2003 entdeckte John Cosgrave
    \begin{itemize} 
     \item die gr"o"ste bis dahin bekannte zusammengesetzte Fermatzahl und
     \item die gr"o"ste bis dahin bekannte prime nicht-einfache Mersennezahl
           mit 645.817 Dezimalstellen.
    \end{itemize}


    Die Fermatzahl
    $$ f(2.145.351) = 2^{(2^{2.145.351})} + 1 $$ 
    ist teilbar durch die Primzahl
    $$ p = 3*2^{2.145.353} + 1 $$ \\
    Diese Primzahl p war damals die gr"o"ste bis dahin bekannte prime 
    verallgemeinerte Mersennezahl\index{Mersenne!Mersennezahl!verallgemeinerte}
    und die 5.-gr"o"ste damals bekannte Primzahl "uberhaupt. 

    Zu diesem Erfolg trugen bei: NewPGen von Paul Jobling's, PRP von 
    George Woltman's, Proth von Yves Gallot's Programm\index{Gallot, Yves} und
    die Proth-Gallot-Gruppe am St. Patrick's College, Dublin.

    Weitere Details finden sich unter
    \vspace{-10pt}
    \begin{itemize}
      \item[] \href{http://www.fermatsearch.org/history/cosgrave_record.htm/}
          {\texttt{http://www.fermatsearch.org/history/cosgrave\_record.htm/}}
    \end{itemize}


%    ({\href{http://perso.wanadoo.fr/yves.gallot/primes/index.html}
%           {\tt http://perso.wanadoo.fr/yves.gallot/primes/index.html}}).
% M.E. ist die Zahl auf seiner Webseite zu gro"s, da M-39 "nur" 4 Mio. Stellen hat !
%    Heute kennt man alle Fermatschen Primzahlen mit bis zu 2.000.000.000 ???
%    Dezimalstellen. \\




% --------------------------------------------------------------------------
\vskip +10 pt
\hypertarget{generalizedFermatprimes}{}    %be_2005 Eigentlich sind es "Zahlen", nicht nur "Primes" !
\subsubsection
    [Verallgemeinerte Fermatzahlen $f(b,n) = b^{2^n} + 1$]
    {Verallgemeinerte Fermatzahlen\footnotemark~$f(b,n) = b^{2^n} + 1$}
    \footnotetext{%
      Hier ist die Basis b nicht notwendigerweise 2 . \\
      Noch allgemeiner w"are:  $f(b,c,n) = b^{c^n} \pm 1$
    }
\label{generalized-fermat}
\index{Fermat!Fermatzahl!verallgemeinerte}
    Verallgemeinerte Fermatzahlen kommen h"aufiger vor als Mersennezahlen
    gleicher Gr"o"se, so dass wahrscheinlich noch viele gefunden werden 
    k"onnen, die die gro"sen L"ucken zwischen den Mersenne-Primzahlen 
    verkleinern.
    Fortschritte in der Zahlentheorie haben es erm"og"-licht, dass Zahlen,
    deren Repr"asentation nicht auf eine Basis von 2 beschr"ankt ist, 
    nun mit fast der gleichen Geschwindigkeit wie Mersennezahlen getestet
    werden k"onnen.
    
    Yves Gallot\index{Gallot, Yves} schrieb das Programm Proth.exe zur 
    Untersuchung verallgemeinerter Fermatzahlen.
    
    Mit diesem Programm fand Michael Angel am 16. Februar 2003 eine
    prime verallgemeinerte Fermatzahl mit 628.808 Dezimalstellen, 
    die zum damaligen Zeitpunkt zur 5.-gr"o"sten bis dahin bekannten
    Primzahl wurde:
    $$ b^{2^{17}} + 1  =  62.722^{131.072} + 1. $$ 

    Weitere Details finden sich unter
    \vspace{-10pt}
    \begin{itemize}
      \item[] \href{http://primes.utm.edu/top20/page.php?id=12}
              {\texttt{http://primes.utm.edu/top20/page.php?id=12}} 
    \end{itemize}




% --------------------------------------------------------------------------
\vskip +10 pt
\subsubsection{Carmichaelzahlen\index{Zahlen!Carmichaelzahl}}
Wie \hyperlink{HT-Carmichael-number01}{oben} in
Kapitel~\ref{L-Carmichael-number01} erw"ahnt,
sind nicht alle Carmichaelzahlen prim.


% --------------------------------------------------------------------------
\vskip +10 pt
\subsubsection{Pseudoprimzahlen%
\index{Primzahl!Pseudoprimzahl}\index{Zahlen!Pseudoprimzahl}%
}
Siehe \hyperlink{HT-Pseudoprimenumber01}{oben} in
Kapitel~\ref{L-Pseudoprimenumber01}.


% --------------------------------------------------------------------------
\vskip +10 pt
\subsubsection{Starke Pseudoprimzahlen%
\index{Primzahl!starke Pseudoprimzahlen} \index{Zahlen!starke Pseudoprimzahl}%
}
Siehe \hyperlink{HT-Strongpseudoprimenumber01}{oben} in
Kapitel~\ref{L-Strongpseudoprimenumber01}.



% --------------------------------------------------------------------------
\vskip +10 pt
\subsubsection
    [Idee aufgrund von Euklids Beweis $p_1 \cdot p_2 \cdots p_n +1$]
    {Idee aufgrund von Euklids Beweis $p_1 \cdot p_2 \cdots p_n +1$}
Diese Idee entstammt \hyperlink{thm-pz-euklid}{Euklids Beweis}, dass es
unendlich viele Primzahlen gibt.
$$
\begin{array}{lll}
2{\cdot}3 +1 &      = 7 &          \mapsto {\rm ~prim} \\
2{\cdot}3{\cdot}5 +1 &      = 31    &      \mapsto {\rm ~prim} \\
2{\cdot}3{\cdot}5{\cdot}7 +1 &      = 211   &      \mapsto {\rm ~prim} \\
2{\cdot}3{\cdots}11 +1 &        = 2311  &      \mapsto {\rm ~prim} \\
2\cdot3 \cdots 13 +1 &  = 59 \cdot 509 &    \mapsto {\rm ~NICHT~prim} ! \\
2\cdot3 \cdots 17 +1 &  = 19 \cdot 97 \cdot 277 &   \mapsto {\rm ~NICHT~prim} ! \\
\end{array} 
$$


% --------------------------------------------------------------------------
\vskip +10 pt
\subsubsection{Wie zuvor, nur $-1$ statt $+1$: $p_1 \cdot p_2 \cdots p_n -1$}
$$
 \begin{array}{lll}
2\cdot 3 -1     &   = 5 &   \mapsto {\rm ~prim} \\
2\cdot 3 \cdot  5  -1   &   = 29 &  \mapsto {\rm ~prim} \\
2\cdot 3 \cdots 7  -1   &   = 11 \cdot 19 & \mapsto {\rm ~NICHT~prim} ! \\
2\cdot 3 \cdots 11 -1   &   = 2309 &    \mapsto {\rm ~prim} \\
2\cdot 3 \cdots 13 -1  &    = 30029 &   \mapsto {\rm ~prim} \\
2\cdot 3 \cdots 17 -1    &  = 61 \cdot 8369 &   \mapsto {\rm ~NICHT~prim!}
\end{array}
$$


% --------------------------------------------------------------------------
\vskip +10 pt
\subsubsection[Euklidzahlen $e_n = e_0 \cdot e_1 \cdots e_{n-1} + 1$]
              {Euklidzahlen $e_n = e_0 \cdot e_1 \cdots e_{n-1} + 1$  
	       mit $ n \geq 1 $ und $ e_0 := 1 $}
%	       mit $n$ gr"o"ser oder gleich $1$ und $e_0 := 1$}
    \index{Euklidzahlen}
    $e_{n-1}$ ist nicht die $(n-1)$-te Primzahl, sondern die zuvor hier 
    gefundene Zahl.
    Diese Formel ist leider nicht offen, sondern rekursiv.
    Die Folge startet mit
$$
\begin{array}{lll}
e_1 = 1 + 1 &   = 2 &   \mapsto {\rm ~prim} \\
e_2 = e_1 + 1   &   = 3 &   \mapsto {\rm ~prim} \\
e_3 = e_1 \cdot e_2 + 1 &   = 7 &   \mapsto {\rm ~prim} \\
e_4 = e_1 \cdot e_2 \cdot e_3 + 1 & = 43 &  \mapsto {\rm ~prim} \\
e_5 = e_1 \cdot e_2 \cdots e_4 + 1 &    = 13 \cdot 139 &    \mapsto {\rm ~NICHT~prim} ! \\
e_6 = e_1 \cdot e_2 \cdots e_5 + 1 &    = 3.263.443 &   \mapsto {\rm ~prim} \\
e_7 = e_1 \cdot e_2 \cdots e_6 + 1 &    = 547 \cdot 607 \cdot 1.033 \cdot 31.051 & \mapsto {\rm ~NICHT~prim} ! \\
e_8 = e_1 \cdot e_2 \cdots e_7 + 1 &    = 29.881\cdot 67.003 \cdot 9.119.521 \cdot 6.212.157.481 & \mapsto {\rm ~NICHT~prim} !
\end{array}
$$

Auch $e_9, \cdots, e_{17}$ sind zusammengesetzt, so dass dies auch
keine brauchbare Primzahlformel ist.

Bemerkung: Das Besondere an diesen Zahlen ist, dass sie jeweils paarweise
keinen gemeinsamen Teiler au"ser $1$ haben\footnote{%
Dies kann leicht mit Hilfe der {\em gr"o"ste gemeinsame Teiler} ($ggT$)-Rechenregel $ggT(a,b) = ggT(b-\lfloor b/a\rfloor,a)$ (siehe Seite \pageref{Appendix_A}) 
gezeigt werden: Es gilt f"ur $i<j$: \\
$ggT(e_i,e_j) \le ggT(e_1 \cdots e_i \cdots e_{j-1}, e_j) = ggT(e_j - e_1 \cdots e_i \cdots e_{j-1}, e_1 \cdots e_i \cdots e_{j-1}) 
= ggT(1, e_1 \cdots e_i \cdots e_{j-1}) = 1$.
}, sie sind also%
\index{Primzahl!relative}\index{relativ prim} {\em relativ zueinander prim}.


% --------------------------------------------------------------------------
% \pagebreak %%%%%%%%%%%%%%%%%%%%%%%%%%%%%%%%%%%%%%%% be_24.503 raus in D
% f(40) hier und f(80) im folgenden Kapitel nach Hinweis von Erik Berger korr. (2005-05)
\vskip +10 pt
\subsubsection{$f(n) = n^2 + n + 41$}
\label{L-Polynomfunktion01-41}
   Diese Folge hat einen sehr {\em erfolgversprechenden} Anfang,
   aber das ist noch lange kein Beweis.
$$
\begin{array}{lll}
f(0) = 41 & &  \mapsto {\rm ~prim} \\
f(1) = 43 & &  \mapsto {\rm ~prim} \\
f(2) = 47 & &  \mapsto {\rm ~prim} \\
f(3) = 53 & &  \mapsto {\rm ~prim} \\
f(4) = 61 & &  \mapsto {\rm ~prim} \\
f(5) = 71 & &  \mapsto {\rm ~prim} \\
f(6) = 83 & &  \mapsto {\rm ~prim} \\
f(7) = 97 & &  \mapsto {\rm ~prim} \\
\vdots \\
f(33) = 1.163 & & \mapsto {\rm ~prim} \\
f(34) = 1.231 & & \mapsto {\rm ~prim} \\
f(35) = 1.301 & & \mapsto {\rm ~prim} \\
f(36) = 1.373 & & \mapsto {\rm ~prim} \\
f(37) = 1.447 & & \mapsto {\rm ~prim} \\
f(38) = 1.523 & & \mapsto {\rm ~prim} \\
f(39) = 1.601 & & \mapsto {\rm ~prim} \\
f(40) = 1681 & = 41 \cdot 41 &  \mapsto {\rm ~NICHT~prim}! \\
f(41) = 1763 & = 41 \cdot 43 &   \mapsto {\rm ~NICHT~prim}! \\
\end{array}
$$
        Die ersten $40$ Werte sind Primzahlen (diese haben die auffallende
        Regelm"a"sigkeit, dass ihr Abstand beginnend mit dem Abstand $2$ 
        jeweils um $2$ w"achst), aber der $41$. und der $42$. Wert
        sind keine Primzahlen.
        Dass $f(41)$ keine Primzahl sein kann, l"asst sich leicht "uberlegen:
        $f(41) = 41^2 + 41 + 41 = 41 (41 + 1 + 1) = 41 \cdot 43$.



% --------------------------------------------------------------------------
% \pagebreak %%%%%%%%%%%%%%%%%%%%%%%%%%%%%%%%%%%%%%%%   be_2005
\vskip +10 pt
\subsubsection{$f(n) = n^2 - 79 \cdot n + 1.601$}
\label{L-Polynomfunktion02-1601}
   Diese Funktion liefert f"ur die Werte $n=0$ bis $n=79$ stets Primzahlwerte.
   Leider ergibt $f(80) = 1.681 = 11 \cdot 151$ keine Primzahl. Bis heute
   kennt man keine Funktion, die mehr aufeinanderfolgende Primzahlen annimmt.
   Andererseits kommt jede Primzahl doppelt vor (erst in der absteigenden,
   dann in der aufsteigenden Folge), so dass sie insgesamt genau 40 
   verschiedene Primzahlwerte liefert (dieselben wie die, die die Funktion
   aus Kapitel~\ref{L-Polynomfunktion01-41} liefert).
$$
\begin{array}{|ll||ll|}
\hline
f(0) = 1.601    & \mapsto {\rm ~prim} &  f(28) = 173    & \mapsto {\rm ~prim} \\
f(1) = 1.523    & \mapsto {\rm ~prim} &  f(29) = 151    & \mapsto {\rm ~prim} \\
f(2) = 1.447    & \mapsto {\rm ~prim} &  f(30) = 131 & \mapsto {\rm ~prim} \\
f(3) = 1.373    & \mapsto {\rm ~prim} &  f(31) = 113 & \mapsto {\rm ~prim} \\
f(4) = 1.301    & \mapsto {\rm ~prim} &  f(32) = 97 & \mapsto {\rm ~prim} \\
f(5) = 1.231    & \mapsto {\rm ~prim} &  f(33) = 83 & \mapsto {\rm ~prim} \\
f(6) = 1.163    & \mapsto {\rm ~prim} &  f(34) = 71 & \mapsto {\rm ~prim} \\
f(7) = 1.097    & \mapsto {\rm ~prim} &  f(35) = 61 & \mapsto {\rm ~prim} \\
f(8) = 1.033    & \mapsto {\rm ~prim} &  f(36) = 53 & \mapsto {\rm ~prim} \\
f(9) = 971  & \mapsto {\rm ~prim} &  f(37) = 47 & \mapsto {\rm ~prim} \\
f(10) = 911 & \mapsto {\rm ~prim} &  f(38) = 43 & \mapsto {\rm ~prim} \\
f(11) = 853 & \mapsto {\rm ~prim} &  f(39) = 41 & \mapsto {\rm ~prim} \\
f(12) = 797 & \mapsto {\rm ~prim} &  f(40) = 41 & \mapsto {\rm ~prim} \\                 
f(13) = 743 & \mapsto {\rm ~prim} &  f(41) = 43 & \mapsto {\rm ~prim} \\                 
f(14) = 691 & \mapsto {\rm ~prim} &  f(42) = 47 & \mapsto {\rm ~prim} \\                 
f(15) = 641 & \mapsto {\rm ~prim} &  f(43) = 53 & \mapsto {\rm ~prim} \\                 
f(16) = 593 & \mapsto {\rm ~prim} &  \cdots  &  \\                                       
f(17) = 547 & \mapsto {\rm ~prim} &  f(77) = 1.447  & \mapsto {\rm ~prim} \\             
f(18) = 503 & \mapsto {\rm ~prim} &  f(78) = 1.523  & \mapsto {\rm ~prim} \\             
f(19) = 461 & \mapsto {\rm ~prim} &  f(79) = 1.601  & \mapsto {\rm ~prim} \\             
f(20) = 421 & \mapsto {\rm ~prim} &  f(80) = 41 \cdot 41 & \mapsto {\rm ~NICHT~prim!} \\
f(21) = 383 & \mapsto {\rm ~prim} &  f(81) = 41 \cdot 43 & \mapsto {\rm ~NICHT~prim!} \\ 
f(22) = 347 & \mapsto {\rm ~prim} &  f(82) = 1.847  & \mapsto {\rm ~prim} \\             
f(21) = 383 & \mapsto {\rm ~prim} &  f(83) = 1.933  & \mapsto {\rm ~prim} \\             
f(22) = 347 & \mapsto {\rm ~prim} &  f(84) = 43 \cdot 47 &  \mapsto {\rm ~NICHT~prim!} \\
f(23) = 313 & \mapsto {\rm ~prim} & & \\ 
f(24) = 281 & \mapsto {\rm ~prim} & & \\
f(25) = 251 & \mapsto {\rm ~prim} & & \\
f(26) = 223 & \mapsto {\rm ~prim} & & \\
f(27) = 197 & \mapsto {\rm ~prim} & & \\
\hline
\end{array}
$$



% --------------------------------------------------------------------------
\vskip +10 pt
\subsubsection[Polynomfunktionen
    $f(x) = a_n x^n + a_{n-1}x^{n-1} + \cdots + a_1 x^1 + a_0$]
    {Polynomfunktionen
    $f(x) = a_n x^n + a_{n-1}x^{n-1} + \cdots + a_1 x^1 + a_0$  
    ($a_i$ aus ${\mathbb Z}$, $n \geq 1$)}\index{Polynom}
%\item Polynomfunktionen} $f(x) = a_n x^n + a_{n-1}x^{n-1} + 
%\cdots + a_1 x^1 + a_0$  ($a_i$ aus ${\mathbb Z}$, $n \geq 1$):

    Es existiert kein solches Polynom, das f"ur alle $x$ aus ${\mathbb Z}$ 
    ausschlie"slich Primzahlwerte annimmt.
    Zum Beweis sei auf \cite[S. 83 f.]{Padberg1996}, verwiesen, wo sich auch
    weitere Details zu Primzahlformeln finden.

    Damit ist es hoffnungslos, weiter nach Formeln (Funktionen) wie in
    Kapitel~\ref{L-Polynomfunktion01-41} oder 
    Kapitel~\ref{L-Polynomfunktion02-1601} zu suchen.

    

% --------------------------------------------------------------------------
\vskip +10 pt
\subsubsection[Vermutung von Catalan]{Vermutung von Catalan\footnotemark}
    \footnotetext{%
Eugene Charles Catalan, belgischer Mathematiker, 30.5.1814$-$14.2.1894.\\
Nach ihm sind auch die sogenannten {\em Catalanzahlen}
$A(n) = (1 / (n+1) ) * (2n)! / (n!)^2$ \\
$=  1, 2, 5, 14, 42, 132, 429, 1.430, 
4.862, 16.796, 58.786, 208.012, 742.900, 2.674.440, 9.694.845, ... $
benannt.
    }
Catalan\index{Catalan, Eugene}\index{Zahlen!Catalanzahl}
"au"serte die Vermutung, dass $ C_4 \;$ eine Primzahl ist:
$$
\begin{array}{l}
C_0 = 2, \\
C_1 = 2^{C_0} - 1,  \\
C_2 = 2^{C_1} - 1,  \\
C_3 = 2^{C_2} - 1, \\
C_4 = 2^{C_3} - 1, \cdots \\
\end{array}
$$
\begin{sloppypar}
    (siehe
        {\href{http://www.utm.edu/research/primes/mersenne.shtml}{\tt http://www.utm.edu/research/primes/mersenne.shtml}}
        unter Conjectures and Unsolved Problems).
\end{sloppypar}

    Diese Folge ist ebenfalls rekursiv definiert und w"achst sehr schnell. 
    Besteht sie nur aus Primzahlen?
$$
\begin{array}{lll}
C(0) = 2 & & \mapsto {\rm ~prim}\\
C(1) = 2^2 - 1 &    = 3 & \mapsto {\rm ~prim}\\
C(2) = 2^3 - 1 &    = 7 & \mapsto {\rm ~prim} \\
C(3) = 2^7 - 1 &    = 127& \mapsto {\rm ~prim} \\
C(4) = 2^{127} - 1 &      = 170. 141. 183. 460. 469. 231. 731. 687. 303. 715. 884. 105. 727 & \mapsto {\rm ~prim} \\
\end{array}
$$
Ob $C_5$ bzw.\ alle h"oheren Elemente prim sind, ist (noch) nicht bekannt,
aber auch nicht wahrscheinlich.
Bewiesen ist jedenfalls nicht, dass diese Formel nur Primzahlen liefert.




% --------------------------------------------------------------------------
\vskip +10 pt
\subsection{Dichte und Verteilung der Primzahlen}

Wie Euklid herausfand, gibt es unendlich viele Primzahlen. Einige unendliche Mengen sind aber
{\em dichter} \index{Primzahl!Dichte} als andere. Innerhalb der nat"urlichen Zahlen gibt es unendlich viele gerade, ungerade und
quadratische Zahlen.

Nach folgenden Gesichtspunkten gibt es mehr gerade Zahlen als quadratische:
\begin{itemize}
  \item die Gr"o"se des $n$-ten Elements: \\
    Das $n$-te Element der geraden Zahlen ist $2n$; das $n$-te Element der Quadratzahlen ist $n^2$. Weil f"ur
    alle $n>2$ gilt: $2n < n^2$, kommt die $n$-te gerade Zahl viel fr"uher als die $n$-te quadratische Zahl.
    Daher sind die geraden Zahlen dichter verteilt, und wir k"onnen sagen, es gibt mehr gerade als
    quadratische Zahlen.
  \item die Anzahl der Werte, die kleiner oder gleich einem bestimmten {\em Dachwert} $x$ aus ${\mathbb R}$ ist: \\
    Es gibt $[x/2]$ solcher gerader Zahlen und $[\sqrt{x}]$ Quadratzahlen. Da f"ur gro"se $x$ der
    Wert $x/2$ viel gr"o"ser ist als die Quadratwurzel aus $2$, k"onnen wir wiederum sagen, es gibt mehr
    gerade Zahlen.
\end{itemize}


\vskip +15 pt
\paragraph{Der Wert der $n$-ten Primzahl $P(n)$}%
\index{P(n)}%
\mbox{}
\vskip +10 pt
\begin{satz}\label{thm-pz-density}
F"ur gro"se $n$ gilt: Der Wert der $n$-ten Primzahl $P(n)$ ist
asymptotisch zu $n \cdot ln(n)$, d.h. der Grenzwert des
Verh"altnisses  $P(n)/(n\cdot \ln n)$ ist gleich $1$, wenn $n$
gegen unendlich geht.
\end{satz}

Es gilt f"ur $n \ge 5$, dass  $P(n)$  zwischen  $2n$  und  $n^2$  liegt.
Es gibt also weniger Primzahlen als gerade nat"urliche Zahlen, aber es gibt mehr Primzahlen als Quadratzahlen\footnote{%
Vergleiche auch \hyperlink{ntePrimzahl}{Tabelle \ref{s:ntePrimzahl}}.
}.


\vskip +15 pt
\paragraph{Die Anzahl der Primzahlen $PI(x)$}%
\index{PI(x)}%
\mbox{}
\vskip +10 pt
"Ahnlich wird die Anzahl der Primzahlen $PI(x)$ definiert, die den
Dachwert $x$ nicht "ubersteigen:

\begin{satz}\label{thm-pz-pi-x}
$PI(x)$  ist asymptotisch zu  $x / ln(x)$.
\end{satz}


Dies ist der \index{Primzahlsatz} \textbf{Primzahlsatz} (prime
number theorem). Er wurde von Legendre\footnote{%
  Adrien-Marie Legendre\index{Legendre, Adrien-Marie}, 
  franz"osischer Mathematiker, 18.9.1752$-$10.1.1833.
}
und Gauss\footnote{%
  Carl Friedrich Gauss\index{Gauss, Carl Friedrich}, 
  deutscher Mathematiker und Astronom, 30.4.1777$-$23.2.1855.
}
aufgestellt und erst "uber 100 Jahre sp"ater bewiesen.

\hyperlink{primhfk}{Die "Ubersicht unter \ref{s:primhfk} zeigt die
Anzahl von Primzahlen in verschiedenen Intervallen.}


Diese Formeln, die nur f"ur $n$ gegen unendlich gelten, k"onnen durch 
pr"azisere Formeln ersetzt werden.
F"ur $x \geq 67$ gilt:
$$ ln(x) - 1,5 < x / PI(x) < ln(x) - 0,5 $$
Im Bewusstsein, dass $PI(x)  =  x / \ln x$ nur f"ur sehr gro"se $x$ ($x$ gegen unendlich) gilt, kann man
folgende "Ubersicht erstellen:
$$
\begin{array}{ccccc}
x     &  ln(x)  &  x / ln(x) & PI(x)(gez"ahlt) &       PI(x) / (x/ln(x)) \\
10^3  &  6,908  &   144      &  168        &       1,160 \\
10^6  &  13,816 &   72.386    &  78.498          &       1,085 \\
10^9  & 20,723  &   48.254.942 &  50.847.534       &       1,054
\end{array}
$$

F"ur eine Bin"arzahl\footnote{%
Eine Zahl im Zweiersystem besteht nur aus den Ziffern 0 und 1.
} 
$ x $ der L"ange $250$ Bit 
($2^{250}$ ist ungef"ahr = $1,809 251 * 10^{75}$)  
gilt:
$$ PI(x) = 2^{250} / (250 \cdot \ln 2) \; \;  {\rm ist~ungef"ahr} 
\; = 2^{250} / 173,28677 = 1,045 810 \cdot 10^{73}. $$
Es ist also zu erwarten, dass sich innerhalb der Zahlen der
Bitl"ange kleiner als 250 ungef"ahr $10^{73}$ Primzahlen
befinden (ein beruhigendes Ergebnis?!).

Man kann das auch so formulieren: Betrachtet man eine {\em zuf"allige} nat"urliche Zahl $n$, so sind die
Chancen, dass diese Zahl prim ist, circa $1 / \ln(n)$. Nehmen wir zum Beispiel Zahlen in der Gegend von
$10^{16}$, so m"ussen wir ungef"ahr (durchschnittlich) $ 16 \cdot \ln 10 = 36,8 $
Zahlen betrachten, bis wir eine Primzahl finden.
Ein genaue Untersuchung zeigt: Zwischen $10^{16}-370$ und $10^{16}-1$ gibt es $10$
Primzahlen.

Unter der "Uberschrift {\em How Many Primes Are There} finden sich unter
\vspace{-10pt}
\begin{itemize}
  \item[] \href{http://www.utm.edu/research/primes/howmany.shtml}
               {\tt http://www.utm.edu/research/primes/howmany.shtml}
\end{itemize}
\vspace{-10pt}
viele weitere Details.

$PI(x)$ l"asst sich leicht per 
\vspace{-10pt}
\begin{itemize}
  \item[] \href{http://www.math.Princeton.EDU/~arbooker/nthprime.html}
               {\tt http://www.math.Princeton.EDU/\~{}arbooker/nthprime.html}
\end{itemize}
\vspace{-10pt}
bestimmen.


Die \textbf{Verteilung} der Primzahlen weist viele
Unregelm"a"sigkeiten auf, f"ur die bis heute kein \glqq System''
gefunden wurde: einerseits liegen viele eng benachbart wie $2$ und
$3,$ $ 11$ und $13, $ $ 809$ und $811$, andererseits tauchen auch
l"angere Primzahll"ucken auf. So liegen zum Beispiel zwischen
$113$ und $127, $ $ 293$ und $307, $ $317$ und $331, $ $ 523$ und
$541, $ $ 773$ und $787, $ $ 839$ und $853$ sowie zwischen $887$
und $907$ keine Primzahlen. \\
Details siehe: 
\vspace{-10pt}
\begin{itemize}
  \item[] \href{http://www.utm.edu/research/primes/notes/gaps.html}
               {\tt http://www.utm.edu/research/primes/notes/gaps.html}
\end{itemize}
Gerade dies macht einen Teil des Ehrgeizes der Mathematiker aus, 
ihre Geheimnisse herauszufinden.


%----------------------------------------
\index{Eratosthenes!sieve}
\paragraph{Sieb des Eratosthenes\index{Eratosthenes!Sieb}}\mbox{}
\hypertarget{SieveEratosthenes01}{}

Ein einfacher Weg, alle $PI(x)$ Primzahlen kleiner oder gleich $x$
zu berechnen, ist das Sieb des Erathostenes. Er fand schon im 3.
Jahrhundert vor Christus einen sehr einfach automatisierbaren Weg,
das herauszufinden. Zuerst werden ab 2 alle Zahlen bis $x$
aufgeschrieben, die 2 umkreist und dann streicht man alle
Vielfachen von 2. Anschlie"send umkreist man die kleinste noch
nicht umkreiste oder gestrichene Zahl (3), streicht wieder alle
ihre Vielfachen, usw. Durchzuf"uhren braucht man das nur bis zu
der gr"o"sten Zahl, deren Quadrat kleiner oder gleich $x$ ist.

Abgesehen von 2 sind Primzahlen nie gerade. Abgesehen von 2 und 5 haben Primzahlen nie die
Endziffern 2, 5 oder 0. Also braucht man sowieso nur Zahlen mit den Endziffern 1, 3, 7, 9 zu
betrachten (es gibt unendlich viele Primzahlen mit jeder dieser letzten Ziffern; vergleiche \cite[Bd. 1,
S. 137]{Tietze1973}).

Inzwischen findet man im Internet auch viele fertige Programme,
oft mit komplettem Quellcode, so dass man auch selbst mit gro"sen
Zahlen experimentieren kann (vergleiche Kap. \ref{spezialzahlentypen}). 
Ebenfalls zug"anglich sind gro"se Datenbanken, die entweder viele 
Primzahlen oder die Zerlegung in Primfaktoren vieler zusammengesetzter
Zahlen enthalten.



% --------------------------------------------------------------------------
\pagebreak
\subsection{Anmerkungen zu Primzahlen}

\paragraph{Weitere interessante Themen rund um Primzahlen} \mbox{}\\
In diesem Kapitel \ref{Label_Kapitel_2} wurden weitere, eher zahlentheoretische 
Themen wie Teilbarkeitsregeln, Modulo-Rechnung, modulare Inverse, modulare Potenzen
und Wurzeln, chinesischer Restesatz, Eulersche Phi-Funktion und perfekte Zahlen
nicht betrachtet. Auf einige dieser Themen geht das \hyperlink{Chapter_ElementaryNT}
{{\bf n"achste Kapitel}} (Kapitel \ref{Chapter_ElementaryNT}) ein.

Die folgenden Anmerkungen listen einzelne interessante S"atze, Vermutungen und 
Fragestellungen zu Primzahlen auf, aber auch Kurioses und "Ubersichten.


% --------------------------------------------------------------------------
\vskip +20 pt
\subsubsection{Bewiesene Aussagen / S"atze zu Primzahlen}
\begin{itemize}

  \item Zu jeder Zahl $n$ aus ${\bf N}$ gibt es $n$ aufeinanderfolgende 
     nat"urliche Zahlen, die keine Primzahlen sind.
     Ein Beweis findet sich in \cite[S. 79]{Padberg1996}.


  \item Paul Erd"os\footnote{%
        Paul Erd"os\index{Erd""os, Paul}, ungarischer Mathematiker,
        26.03.1913$-$20.09.1996. }
     bewies:
     Zwischen jeder beliebigen Zahl ungleich $1$ und ihrem Doppelten gibt
     es mindestens eine Primzahl. Er bewies das Theorem nicht als erster,
     aber auf einfachere Weise als andere vor ihm.

      
  \item \hypertarget{link-Primzahlfunktion-base-a-Proof}{}
     Es existiert eine reelle Zahl a, so dass die Funktion
     $f: {\bf N} \rightarrow {\mathbb Z}$ mit $n \mapsto a^{3^n}$
     f"ur alle $n$ nur Primzahlenwerte annimmt 
     (siehe \cite[S. 82]{Padberg1996}).
     Leider macht die Bestimmung von $a$ Probleme 
     (siehe \hyperlink{link-Primzahlfunktion-base-a-Offen}{unten}).


  \vskip +6 pt      
  \item \hypertarget{link-Arithmetic-sequence-of-primes}{}
     Es gibt arithmetische Primzahlfolgen \index{Primzahlfolge!arithmetische}
     beliebig gro"ser L"ange\footnote{%
     Quellen: \\
     - \href{http://primes.utm.edu/glossary/page.php?sort=ArithmeticSequence}
     {\texttt{http://primes.utm.edu/glossary/page.php?sort=ArithmeticSequence}}
         Original-Quelle\\
     - GEO 10 / 2004: \glqq Experiment mit Folgen\grqq\\
     - \href{http://www.faz.net}
       {\texttt{http://www.faz.net}} \glqq Hardys Vermutung -- Primzahlen ohne 
                                  Ende\grqq~von Heinrich Hemme (06. Juli 2004)
     }.

     Die 1923 von dem ber"uhmten englischen Mathematiker Godfrey Harold 
     Hardy\footnote{%
     Godfrey Harold Hardy\index{Hardy, Godfrey Harold}, britischer
     Mathematiker, 7.2.1877$-$1.12.1947.}
     aufgestellte Vermutung, dass es arithmetische Folgen beliebiger L"ange
     gibt, die nur aus Primzahlen bestehen, wurde 2004 von zwei jungen
     amerikanischen Mathematikern bewiesen.
  
     Jedes Schulkind lernt im Mathematikunterricht irgendwann einmal die 
     arithmetischen Zahlenfolgen kennen. Das sind Aneinanderreihungen von
     Zahlen, bei denen die Abst"ande zwischen je zwei aufeinander folgenden
     Gliedern gleich sind - etwa bei der Folge 5, 8, 11, 14, 17, 20. Der
     Abstand der Glieder betr"agt hierbei jeweils 3 und die Folge hat 6
     Folgenglieder. Eine arithmetische Folge muss mindestens 3 Folgenglieder
     haben, kann aber auch unendlich viele haben.

     Arithmetische Folgen sind seit Jahrtausenden bekannt und bergen eigentlich 
     keine Geheimnisse mehr.
     Spannend wird es erst wieder, wenn die Glieder einer arithmetischen Folge
     noch zus"atzliche Eigenschaften haben sollen, wie das bei Primzahlen der
     Fall ist. 

     Primzahlen sind ganze Zahlen, die gr"o"ser als 1 und nur durch 1 und sich
     selbst ohne Rest teilbar sind. Die zehn kleinsten Primzahlen sind
     2, 3, 5, 7, 11, 13,  17, 19, 23 und 29. 

     Eine arithmetische Primzahlfolge mit f"unf Gliedern ist beispielsweise 
     5, 17, 29, 41, 53. Der Abstand der Zahlen betr"agt jeweils 12. 

     Diese Folge l"asst sich nicht verl"angern, ohne ihre Eigenschaft
     einzub"u"sen, denn das n"achste Glied m"usste 65 sein, und diese Zahl ist
     das Produkt aus 5 und 13 und somit keine Primzahl.

     Wie viele Glieder kann eine arithmetische Primzahlfolge haben? Mit dieser
     Frage haben sich schon um 1770 der Franzose Joseph-Louis Lagrange und der
     Engl"ander Edward Waring besch"aftigt. Im Jahre 1923 vermuteten der
     ber"uhmte britische Mathematiker Godfrey Harold Hardy und sein Kollege
     John Littlewood, dass es keine Obergrenze f"ur die Zahl der Glieder gebe.
     Doch es gelang ihnen nicht, das zu beweisen. Im Jahr 1939 gab es jedoch
     einen anderen Fortschritt. Der holl"andische Mathematiker Johannes van der
     Corput konnte nachweisen, dass es unendlich viele arithmetische
     Primzahlfolgen mit genau drei Gliedern gibt. Zwei Beispiele hierf"ur sind
     3, 5, 7  und  47, 53, 59.

     Die l"angste Primzahlfolge, die man bisher kennt, hat 23 Glieder.

     \begin{table}[h]
     \begin{center}
     \begin{tabular}{|r|r|r|r|r|}
     \hline
     Elementanzahl &Startelement & Abstand & Wann & Entdecker \\ \hline
     22 &  11.410.337.850.553 &   4.609.098.694.200 & 1993 & Paul A. Pritchard,\\
        &                     &                     &      & Andrew Moran,\\
        &                     &                     &      & Anthony Thyssen\\
        &                     &                     &      &\\

     22 & 376.859.931.192.959 &  18.549.279.769.020 & 2003 & Markus Frind\\
        &                     &                     &      &\\

     23 &  56.211.383.760.397 &  44.546.738.095.860 & 2004 & Markus Frind,\\
        &                     &                     &      & Paul Jobling,\\
        &                     &                     &      & Paul Underwood \\
     \hline
     \end{tabular}
     \caption{Die l"angsten bekannten arithmetischen Primzahlfolgen (Stand Mai 2005)}
     \end{center}
     \end{table}

     Den beiden jungen\footnote{%
     In seinen Memoiren hat Hardy\index{Hardy, Godfrey Harold} 1940 geschrieben, 
     dass die Mathematik mehr als alle anderen Wissenschaften und K"unste 
     ein Spiel f"ur junge Leute sei. \\
     Der damals 27 Jahre alte Ben Green von der University of British Columbia
     in Vancouver und der 29 Jahre alte Terence Tao von der University of
     California in Los Angeles scheinen ihm recht zu geben.
     }
     Mathematikern  Ben Green and Terence Tao ist es im Jahre 2004 gelungen,
     die mehr als achtzig Jahre alte Hardysche Vermutung zu beweisen: 
     Es gibt arithmetische Primzahlfolgen beliebiger L"ange. Au"serdem bewiesen
     sie, dass es zu jeder vorgegebenen L"ange unendlich viele verschiedene
     solcher Folgen gibt.

     Eigentlich hatten Green und Tao nur beweisen wollen, dass es unendlich
     viele arithmetische Primzahlfolgen mit vier Gliedern gibt. Dazu
     betrachteten sie Mengen, die neben Primzahlen auch 
     Beinaheprimzahlen\index{Beinaheprimzahl}\index{Primzahl!Beinaheprimzahl} 
     enthielten. Das sind Zahlen, die nur wenige Teiler haben - beispielsweise
     die Halbprimzahlen\index{Halbprimzahl}\index{Primzahl!Halbprimzahl}
     \index{Zahlen!semiprime}, die Produkte aus genau zwei Primzahlen sind.
     Dadurch konnten die beiden Mathematiker ihre Arbeit wesentlich
     erleichtern, denn "uber Beinaheprimzahlen gab es schon zahlreiche
     n"utzliche Theoreme. Schlie"slich erkannten sie, dass ihr Verfahren viel
     m"achtiger ist, als sie selbst angenommen hatten, und sie bewiesen damit
     die Hardysche Vermutung.
 
     Der Beweis von Green und Tao umfasst immerhin 49 Seiten. Tats"achlich
     beliebig lange arithmetische Primzahlfolgen kann man damit aber nicht
     finden. Der Beweis ist nicht konstruktiv, sondern ein so genannter 
     Existenzbeweis\index{Existenzbeweis}. 
     Das hei"st, die beiden Mathematiker haben \glqq nur\grqq~gezeigt, dass 
     beliebig lange Folgen existieren, aber nicht, wie man sie findet.

     Das hei"st, in der Menge der nat"urlichen Primzahlen gibt es zum Beispiel
     eine Folge von einer Milliarde Primzahlen, die alle den gleichen Abstand 
     haben; und davon gibt es unendlich viele. Dies Folgen liegen aber sehr 
     \glqq weit drau"sen\grqq.   %\glqq D\grqq~

     Wer solche Folgen entdecken m"ochte, sollte folgendes ber"ucksichtigen.
     Die L"ange der Folge bestimmt den Mindestabstand zwischen den einzelnen
     Primzahlen. Bei einer Folge mit 6 Gliedern muss der Abstand 30 oder ein
     Vielfaches davon betragen. Die Zahl 30 ergibt sich als das Produkt aller
     Primzahlen, die kleiner als die Folgenl"ange, also kleiner als 6, sind:
     $ 2 * 3 * 5 = 30 $. Sucht man Folgen mit der L"ange 15, so muss der 
     Abstand mindestens $ 2 * 3 * 5 * 7 * 11 * 13 = 30.030 $ betragen.

     Daraus ergibt sich, dass die Folgenl"ange beliebig gro"s sein kann, aber
     der Abstand kann nicht nicht jede beliebige Zahl annehmen: es kann keine
     arithmetische Primzahlfolge mit dem Abstand $100$ geben, denn 100 ist
     nicht durch die Zahl 3 teilbar.

     Zerlegt man die Abst"ande der obigen Folgen (mit den L"angen 22 und 23) in 
     ihre Teiler, ergibt sich:
     $$~4.609.098.694.200 = 2^3 * 3 * 5^2 * 7 * 11 * 13 * 17 * 19 * 23 * 1033$$
     $$18.549.279.769.020 = 2^2 * 3 * 5 * 7^2 * 11 * 13 * 17 * 19 * 23 * 5939$$
     $$44.546.738.095.860 = 2^2 * 3 * 5 * 7 * 11 * 13 * 17 * 19 * 23 * 99.839$$ 

     {\bf Weitere Restriktion:} Sucht man arithmetische Primzahlfolgen, 
     die die {\em zus"atzliche} Bedingung erf"ullen, dass alle Primzahlen auch 
     direkt hintereinander liegen, wird es noch etwas schwieriger. Auf der 
     Webseite von Chris Caldwell\footnote{%
     \href{http://primes.utm.edu/glossary/page.php?sort=ArithmeticSequence}
     {\texttt{http://primes.utm.edu/glossary/page.php?sort=ArithmeticSequence}}
     } 
     finden Sie einen Hinweis darauf: Die l"angste bekannte arithmetische
     Folge, die nur aus direkt hintereinander liegenden Primzahlen besteht,
     hat die L"ange $10$ und der Abstand betr"agt $210 = 2 * 3 * 5 * 7$. \\

\end{itemize}


% --------------------------------------------------------------------------
\vskip +10 pt
\subsubsection{Unbewiesene Aussagen / Vermutungen zu Primzahlen}
\begin{itemize}
  \item Christian Goldbach\footnote{%
     Christian Goldbach\index{Goldbach, Christian}, deutscher Mathematiker,
     18.03.1690$-$20.11.1764. }
     vermutete:
     Jede gerade nat"ur"-liche Zahl gr"o"ser $2$ l"asst sich als die Summe
     zweier Primzahlen darstellen.
     Mit Computern ist die Goldbachsche Vermutung f"ur alle geraden Zahlen
     bis $4*10^{14}$ verifiziert\footnote{%
     Dass die Goldbachsche Vermutung wahr ist, d.h. f"ur alle geraden 
     nat"urlichen Zahlen gr"o"ser als $2$ gilt, wird heute allgemein nicht 
     mehr angezweifelt. Der Mathematiker J"org Richstein\index{Richstein 1999}
     vom Institut f"ur Informatik der Universit"at Gie"sen hat 1999 die geraden
     Zahlen bis 400 Billionen untersucht und kein Gegenbeispiel gefunden.
     Inzwischen wurden noch umfangreichere "Uberpr"ungen vorgenommem:
     (siehe 
     \href{http://www.mscs.dal.ca/\~{}joerg/res/g-de.html}
      {\tt http://www.mscs.dal.ca/\~{}joerg/res/g-de.html},\\
     \href{http://de.wikipedia.org/wiki/Goldbachsche\_Vermutung}
      {\tt http://de.wikipedia.org/wiki/Goldbachsche\_Vermutung},\\
     \href{http://primes.utm.edu/glossary/page.php/GoldbachConjecture.html}
      {\tt http://primes.utm.edu/glossary/page.php/GoldbachConjecture.html}
     ).\\
% http://www.mscs.dal.ca/\~{}joerg/res/g-en.html
     Trotzdem ist das kein allgemeiner Beweis.\\ 
     Dass die Goldbach'sche Vermutung trotz aller Anstrengungen bis heute 
     nicht bewiesen wurde, f"ordert allerdings einen Verdacht:
     Seit den bahnbrechenden Arbeiten des "osterreichischen Mathematikers 
     Kurt G"odel\index{G\""odel, Kurt} ist bekannt, dass nicht jeder wahre
     Satz in der Mathematik auch beweisbar ist (siehe 
     \href{http://www.mathematik.ch/mathematiker/goedel.html}
          {\tt http://www.mathematik.ch/mathematiker/goedel.html}).
     M"oglicherweise hat Goldbach also Recht, und trotzdem wird nie ein 
     Beweis gefunden werden. Das wiederum l"asst sich aber vermutlich auch
     nicht beweisen.
     },   % footnote
     aber allgemein noch nicht bewiesen\footnote{%
     Der englische Verlag {\em Faber} und die amerikanische
     Verlagsgesellschaft {\em Bloomsbury} publizierten 2000 das 1992 erstmals
     ver"offentlichte Buch \glqq Onkel Petros und die Goldbachsche 
     Vermutung\grqq~ von Apostolos Doxiadis (deutsch bei
     L"ubbe 2000 und bei BLT als Taschenbuch 2001). Es ist 
     die Geschichte eines Mathematikprofessors, der daran scheitert, ein 
     mehr als 250 Jahre altes R"atsel zu l"osen.\\
     Um die Verkaufszahlen zu f"ordern, schrieben die beiden Verlage einen 
     Preis von 1 Million USD aus, wenn jemand die Vermutung beweist -- 
     ver"offentlicht in einer angesehenen mathematischen Fachzeitschrift 
     bis Ende 2004 (siehe 
     \href{http://www.mscs.dal.ca/~dilcher/Goldbach/index.html}
          {\tt http://www.mscs.dal.ca/\~{}dilcher/Goldbach/index.html}).\\
     Erstaunlicherweise durften nur englische und amerikanische Mathematiker
     daran teilnehmen.\\
     Die Aussage, die der Goldbach-Vermutung bisher am n"achsten kommt, wurde
     1966 von Chen Jing-Run bewiesen -- in einer schwer nachvollziehbaren Art
     und Weise: Jede gerade Zahl gr"o"ser 2 ist die Summe einer Primzahl und
     des Produkts zweier Primzahlen. Z.B. $20=5+3*5.$\\ 
     Die wichtigsten Forschungsergebnisse zur Goldbachschen Vermutung sind
     zusammengefasst in dem von Wang Yuan herausgegebenen Band: 
     \glqq Goldbach Conjecture\grqq, 1984, World Scientific Series in Pure
     Maths, Vol. 4.\\
     Gerade diese Vermutung legt nahe, dass wir auch heute noch nicht in 
     aller Tiefe den Zusammenhang zwischen der Addition und der Multiplikation
     der nat"urlichen Zahlen verstanden haben.
     }.   % footnote
    
\item Bernhard Riemann\footnote{%
        Bernhard Riemann\index{Riemann, Bernhard}, deutscher Mathematiker,
        17.9.1826$-$20.7.1866. }
    stellte eine bisher unbewiesene, aber auch nicht widerlegte (?) Formel
    f"ur die Verteilung von Primzahlen auf, die die Absch"atzung 
    weiter verbessern w"urde.
\end{itemize}


% --------------------------------------------------------------------------
\vskip +10 pt
\subsubsection{Offene Fragestellungen}
Primzahlzwillinge sind Primzahlen, die den Abstand 2 voneinander haben, 
zum Beispiel 5 und 7 oder 101 und 103. 
Primzahldrillinge gibt es dagegen nur eines: 3, 5, 7.  Bei allen anderen 
Dreierpacks aufeinanderfolgender ungerader Zahlen ist immer eine durch 3 
teilbar und somit keine Primzahl.
\begin{itemize}

\item Offen ist die Anzahl der Primzahlzwillinge: unendlich viele oder eine
      begrenzte Anzahl?
      Das bis heute bekannte gr"o"ste Primzahlzwillingspaar ist
      $1.693.965 \cdot 2^{66.443} \pm 1.$

\item Gibt es eine Formel f"ur die Anzahl der Primzahlzwillinge pro Intervall?

\item \hypertarget{link-Primzahlfunktion-base-a-Offen}{}
    Der Beweis \hyperlink{link-Primzahlfunktion-base-a-Proof}{oben} zu der
    Funktion  $f: \: N \rightarrow Z$ mit $n \mapsto a^{3^n}$
    garantiert nur die Existenz einer
    solchen Zahl $a$.  Wie kann diese Zahl $a$ bestimmt werden, und wird sie
    einen Wert haben, so dass die Funktion auch von praktischem Interesse ist?

\item Gibt es unendlich viele Mersenne-Primzahlen?
   \index{Primzahl!Mersenne}\index{Mersenne!Mersenne-Primzahl}

\item Gibt es unendlich viele Fermatsche Primzahlen?

\item Gibt es einen Polynomialzeit-Algorithmus\index{Polynom} zur 
   Zerlegung einer Zahl in ihre Primfaktoren (vgl. \cite[S. 167]{Klee1997})?
   Diese Frage kann man auf die folgenden drei Fragestellungen aufsplitten:
   \begin{itemize}
        \item Gibt es einen Polynomialzeit-Algorithmus, der entscheidet, ob 
	   eine Zahl prim ist? \\
	   Diese Frage wurde durch den AKS-Algorithmus\index{AKS} 
	   beantwortet (vgl.  Kapitel \ref{PrimesinP}, \glqq PRIMES in P\grqq:
	   Testen auf Primalit"at ist polynominal).
        \item Gibt es einen Polynomialzeit-Algorithmus, mit dem man 
           feststellen kann, aus wievielen Primfaktoren eine zusammengesetzte
           Zahl besteht (ohne diese Primfaktoren zu berechnen)?
        \item Gibt es einen Polynomialzeit-Algorithmus, mit dem sich f"ur 
	   eine zusammengesetzte Zahl $n$ ein nicht-trivialer (d.h. 
	   von $1$ und von $n$ verschiedener) Teiler von $n$ berechnen 
	   l"asst?\footnote{Vergleiche auch Kapitel \ref{RSABernstein} 
	   und Kapitel \ref{NoteFactorisation}.}
   \end{itemize}
\end{itemize}

Am Ende von Kapitel \ref{NoteFactorisation}, 
Abschnitt \hyperlink{RSA-200-chap3}{RSA-200}
k"onnen Sie die Gr"o"senordnungen ersehen, f"ur die heutige Algorithmen
Ergebnisse bei Primzahltests\index{Primzahltest} und bei der
Faktorisierung\index{Faktorisierung!Faktorisierungsrekorde} liefern.



% --------------------------------------------------------------------------
\vskip +25 pt
% Die eckigen Klammern sind n�tig, will man \footnotmark innerhalb der {}
% verwenden.
% Man muss den Titel sowohl in die [] als auch in die {} stellen:
% - Fehlt er innerhalb {}, ist die �berschrift innerhalb des Textes leer.
% - Fehlt er innerhalb [], fehlt die �berschrift links im PDF-Rahmen und
%   im Inhaltsverzeichnis.
\subsubsection[Kurioses und Interessantes zu Primzahlen]
              {Kurioses und Interessantes zu Primzahlen\footnotemark}
    \footnotetext{%
Weitere kuriose und seltsame Dinge zu Primzahlen finden sich auch unter:\\
- \href{http://primes.utm.edu/curios/home.php}
   {\tt http://primes.utm.edu/curios/home.php}\\
- \href{http://www.primzahlen.de/files/theorie/index.htm}
   {\tt http://www.primzahlen.de/files/theorie/index.htm}.
}
\label{HT-Quaint-curious-Primes-usage}

Primzahlen sind nicht nur ein sehr aktives und ernstes mathematisches
Forschungsgebiet. Mit ihnen besch"aftigen sich Menschen auch hobbym"a"sig und
au"serhalb der wissenschaftlichen Forschung.


\vskip +25 pt
\hypertarget{HT-GoogleRecruitment2004}{}
%\subsubsubsection{xxx} 3*sub kannte er nicht, deshalb paragraph !
\paragraph{Mitarbeiterwerbung bei Google im Jahre 2004}%
\index{Google!Mitarbeiterwerbung}%
\label{HT-GoogleRecruitment2004}%
\mbox{}
\vskip +10 pt
Im Sommer 2004 benutzte die Firma Google die Zahl e\footnote{Die Basis des
nat"urlichen Logarithmus e ist ungef"ahr 2,718 281 828 459. Dies ist eine der
bedeutendsten Zahlen in der Mathematik. Sie wird gebraucht f"ur komplexe Analysis,
Finanzmathematik, Physik und Geometrie. Nun wurde sie -- meines Wissens -- das
erste Mal f"ur Marketing oder Personalbeschaffung verwendet.},
um Bewerber zu gewinnen\footnote{Die meisten Informationen f"ur diesen Paragraphen
stammen aus dem Artikel \glqq e-number crunching\grqq~  von John Allen Paulos
in TheGuardian vom 30.09.2004
und aus dem Internet:\\
- \href{http://www.mkaz.com/math/google/}
   {\tt http://www.mkaz.com/math/google/}\\
- \href{http://epramono.blogspot.com/2004/10/7427466391.html}
   {\tt http://epramono.blogspot.com/2004/10/7427466391.html}\\
- \href{http://mathworld.wolfram.com/news/2004-10-13/google/}
   {\tt http://mathworld.wolfram.com/news/2004-10-13/google/}\\
- \href{http://www.math.temple.edu/~paulos/}
   {\tt http://www.math.temple.edu/\~{}paulos/}.
}.

Auf einer hervorstechenden Reklamewand im kalifornischen Silicon Valley erschien am
12. Juli das folgende geheimnisvolle R"atsel:
\begin{center}
{\em  (first 10-digit prime found in consecutive digits of e).com  }
\end{center}
In der Dezimaldarstellung von e sollte man also die erste 10-stellige Primzahl
finden, die sich in den Nachkommastellen befindet. Mit verschiedenen Software-Tools
kann man die Antwort finden:
$$ 7.427.466.391 $$

Wenn man dann die Webseite $www.7427466391.com$ besuchte, bekam man ein noch
schwierigeres R"atsel gezeigt. Wenn man auch dieses zweite R"atsel l"oste, kam
man zu einer weiteren Webseite, die darum bat, den Lebenslauf an Google zu senden.
Diese Werbekampagne erzielte eine hohe Aufmerksamkeit.

Wahrscheinlich nahm Google an, dass man gut genug ist, f"ur sie zu arbeiten, wenn
man diese R"atsel l"osen kann. Aber bald konnte jeder mit Hilfe der Google-Suche
ohne Anstrengung die Antworten finden, weil viele die L"osung der R"atsel ins
Netz gestellt hatten.\footnote{%
Im zweiten Level der R"atsel musste man das 5. Glied einer gegebenen
Zahlenfolge finden: Dies hatte nichts mehr mit Primzahlen zu tun.}



\vskip +25 pt
\hypertarget{HT-Movie-Contact01}{}
\paragraph{Contact [Regie Robert Zemeckis, 1997] -- Primzahlen zur Kontaktaufnahme}
\index{Filme}\index{Zemeckis 1997}
\label{HT-Movie-Contact01}%
\mbox{}
\vskip +10 pt

Der Film entstand nach dem gleichnamigen Buch von Carl Sagan.

Die Astronomin Dr. Ellie Arroway (Jodie Foster) entdeckt nach jahrelanger vergeblicher
Suche Signale vom 26 Lichtjahre entfernten Sonnensystem Wega. 
In diesen Signalen haben Au"serirdische die Primzahlen l"uckenlos in der richtigen 
Reihenfolge verschl"usselt. Daran erkennt die Heldin, dass diese Nachricht anders
ist als die ohnehin st"andig auf der Erde eintreffenden Radiowellen, die 
kosmischer und zuf"alliger Natur sind (von Radiogalaxien, Pulsaren oder Quasaren). 
In einer entlarvenden Szene fragt sie daraufhin ein Politiker, warum diese 
intelligenten Wesen nicht gleich Englisch sprechen ...

Eine Kommunikation mit absolut fremden und unbekannten Wesen aus dem All ist 
aus 2 Gr"unden sehr schwierig: 
Zun"achst kann man sich bei der hier angenommenen Entfernung und dem damit 
verbundenen langen "Ubertragungsweg in einem durchschnittlichen Menschenleben 
nur einmal in jeder Richtungen nacheinander austauschen.
Zum Zweiten muss man f"ur den Erstkontakt darauf achten, dass eine m"oglichst
hohe Chance besteht, dass der Empf"anger der Radiowellen die Botschaft "uberhaupt
bemerkt und dass er sie als Nachricht von intelligenten Wesen einstuft. 
Deshalb senden die Au"serirdischen am Anfang ihrer Nachricht Zahlen, die als
der einfachste Teil jeder h"oheren Sprache angesehen werden k"onnen  und die nicht
ganz trivial sind: die Folge der Primzahlen. Diese speziellen Zahlen spielen in der
Mathematik eine so fundamentale Rolle, dass man annehmen kann, dass sie jeder
Spezies vertraut sind, die das technische Know-how hat, Radiowellen zu 
empfangen.

Die Aliens schicken danach den Plan zum Bau einer mysteri"osen Maschine ...



% --------------------------------------------------------------------------
\newpage
\hypertarget{primhfk}{}
\subsubsection{Anzahl von Primzahlen in verschiedenen Intervallen}
\label{s:primhfk}
\vskip +10 pt

\begin{table}[h]
\begin{center}
\begin{tabular}{|l|l||l|l||l|l|}\hline
\multicolumn{2}{|l||}{Zehnerintervall} & \multicolumn{2}{l||}{Hunderterintervall} & \multicolumn{2}{l|}{Tausenderintervall} \\ \hline
Intervall  &     Anzahl &    Intervall  &  Anzahl &  Intervall  &    Anzahl\\ \hline \hline
1-10     &       4     &     1-100   &     25  &     1-1.000     &    168 \\
11-20    &       4     &     101-200 &     21  &     1.001-2.000  &    135 \\
21-30    &       2     &     201-300 &     16  &     2.001-3.000  &    127  \\
31-40    &       2     &     301-400 &     16  &     3.001-4.000  &    120 \\
41-50    &       3     &     401-500 &     17  &     4.001-5.000  &    119 \\
51-60    &       2     &     501-600 &     14  &     5.001-6.000  &    114 \\
61-70    &       2     &     601-700 &     16  &     6.001-7.000  &    117 \\
71-80    &       3     &     701-800 &     14  &     7.001-8.000  &    107 \\
81-90    &       2     &     801-900 &     15  &     8.001-9.000  &    110 \\
91-100   &       1     &     901-1.000 &     14 &     9.001-10.000 &    112 \\ \hline
\end{tabular}
\caption{Wieviele Primzahlen gibt es innerhalb der ersten Zehnerintervalle?}
\end{center}
\end{table}
\vskip +20 pt


\begin{table}[h]
\begin{center}
\begin{tabular}{|l|r|r|}\hline
Intervall & Anzahl & Durchschnittl. Anzahl pro 1000 \\ \hline
1 - 10.000          &    1.229        &    122,900 \\
1 - 100.000         &    9.592        &    95,920 \\
1 - 1.000.000       &    78.498      &    78,498 \\
1 - 10.000.000       &   664.579     &    66,458 \\
1 - 100.000.000      &   5.761.455   &    57,615 \\
1 - 1.000.000.000    &   50.847.534  &    50,848 \\
1 - 10.000.000.000   &   455.052.512 &    45,505 \\ \hline
\end{tabular}
\caption{Wieviele Primzahlen gibt es innerhalb der ersten Dimensionsintervalle?}
\end{center}
\end{table}
\vskip +6 pt



% --------------------------------------------------------------------------
\newpage
\hypertarget{ntePrimzahl}{}
\subsubsection{Indizierung von Primzahlen ($n$-te Primzahl)}
\label{s:ntePrimzahl}
\vskip +10 pt

\begin{table}[h]
\begin{center}
\begin{tabular}{|l|l|l|l|}\hline
Index   &   Genauer Wert  &       Gerundeter Wert &    Bemerkung \\
\hline \hline
1       &   2             &     2  & \\
2       &   3             &     3  &  \\
3       &   5             &     5  & \\
4       &   7             &     7  & \\
5       &   11            &     11 & \\
6       &   13            &     13 & \\
7       &   17            &     17 & \\
8       &   19            &     19 & \\
9       &   23            &     23 & \\
10      &   29            &     29 & \\
100     &   541           &     541 & \\
1000    &   7917          &     7917 & \\
664.559  &  9.999.991     &     9,99999E+06 & Alle Primzahlen bis zu 1E+07 waren am\\
         &                &                 & Beginn des 20. Jahrhunderts bekannt.\\
1E+06  &    15.485.863   &      1,54859E+07 & \\
6E+06  &    104.395.301    &    1,04395E+08  & Diese Primzahl wurde 1959 entdeckt.\\
1E+07  &    179.424.673     &    1,79425E+08 & \\
1E+09  &    22.801.763.489  &    2,28018E+10 & \\
1E+12  &    29.996.224.275.833 & 2,99962E+13 & \\ \hline
\end{tabular}
\caption{Liste selektierter $n$-ter Primzahlen}
\end{center}
\end{table}


\vskip +10pt 
Bemerkung: Mit L"ucke wurden fr"uh sehr gro"se Primzahlen entdeckt. \\


\vskip +12pt 
Web-Links (URLs):\\
{\href{http://www.math.Princeton.EDU/~arbooker/nthprime.html}
      {http://www.math.Princeton.EDU/\~{}arbooker/nthprime.html}.} \\
{\href{http://www.utm.edu/research/primes/notes/by_year.html}
      {\tt http://www.utm.edu/research/primes/notes/by\_year.html}.}



% --------------------------------------------------------------------------
\newpage
\subsubsection{Gr"o"senordnungen / Dimensionen in der Realit"at}
\label{s:grosord}
Bei der Beschreibung kryptographischer Protokolle und Algorithmen
treten Zahlen auf, die so gro"s bzw. so klein sind, dass sie einem
intuitiven Verst"andnis nicht zug"anglich sind. Es kann daher
n"utzlich sein, Vergleichszahlen aus der uns umgebenden realen
Welt bereitzustellen, so dass man ein Gef"uhl f"ur die Sicherheit
kryptographischer Algorithmen entwickeln kann. Die angegebenen
Werte stammen gr"o"stenteils aus \cite{Schwenk1996} und
\cite[S.18]{Schneier1996p}.  \hypertarget{grosord}{}

% be_2005: noch mit tabbing:
% Wahrscheinlichkeit, dass Sie auf ihrem n"achsten Flug entf"uhrt werden:	&~~ \= abcdefhijk \= \kill
% Wahrscheinlichkeit, dass Sie auf ihrem n"achsten Flug entf"uhrt werden \> $ 5,5 \cdot 10^{-6} $\> \\

\vskip +20pt 
\begin{table}[h]
\begin{center}
\begin{tabular}{|l|l|}\hline
Wahrscheinlichkeit, dass Sie auf ihrem n"achsten Flug entf"uhrt werden	&  $ 5,5 \cdot 10^{-6} $  \\
J"ahrliche Wahrscheinlichkeit, von einem Blitz getroffen zu werden	&  $ 10^{-7} $            \\
Wahrscheinlichkeit f"ur 6 Richtige im Lotto				&  $ 7,1 \cdot 10^{-8} $  \\
Risiko, von einem Meteoriten erschlagen zu werden			&  $ 1,6 \cdot 10^{-12} $ \\
\hline
Zeit bis zur n"achsten Eiszeit (in Jahren)	&  $14.000 $   =  $(2^{14})$ \\
Zeit bis die Sonne vergl"uht (in Jahren)	&  $10^{9} $   =  $(2^{30})$ \\
Alter der Erde (in Jahren)			&  $ 10^9 $    =  $(2^{30})$ \\
Alter des Universums (in Jahren)		&  $ 10^{10} $ =  $(2^{34})$ \\
Anzahl der Atome der Erde 			&  $10^{51} $  =  $(2^{170})$ \\
Anzahl der Atome der Sonne			&  $10^{57}$   =  $(2^{190})$ \\
Anzahl der Atome im Universum (ohne dunkle Materie)	&  $10^{77}$  = $(2^{265})$ \\
Volumen des Universums (in $cm^3$)		&  $10^{84}$   = $(2^{280})$ \\ \hline
\end{tabular}
\caption{Wahrscheinlichkeiten aus Physik und Alltag (Gr"o"senordnungen / Dimensionen)}
\end{center}
\end{table}


% --------------------------------------------------------------------------
\newpage
\subsubsection{Spezielle Werte des Zweier- und Zehnersystems}

% be_2005 vorher nur getappt, nun als Tabelle mit Rahmen
% \begin{tabbing}
% DualSystem~~~ \= \kill
% Dualsystem \> Zehnersystem \\*[4pt]
% $2^{10}$ \> $1024$ \\
% ...
% $2^{2048}$ \>   $3,23170\cdot 10^{616}$ \\
% \end{tabbing}

\begin{table}[h]
\begin{center}
\begin{tabular}{|l|l|}\hline
Dualsystem   &   Zehnersystem \\
\hline \hline
$2^{10}$ 	&   $1024$ \\
$2^{40}$ 	&   $1,09951\cdot 10^{12}$ \\
$2^{56}$ 	&   $7,20576\cdot 10^{16}$ \\
$2^{64}$ 	&   $1,84467\cdot 10^{19}$ \\
$2^{80}$ 	&   $1,20893\cdot 10^{24}$ \\
$2^{90}$ 	&   $1,23794\cdot 10^{27}$ \\
$2^{112}$ 	&   $5,19230\cdot 10^{33}$ \\
$2^{128}$ 	&   $3,40282\cdot 10^{38}$ \\
$2^{150}$ 	&   $1,42725\cdot 10^{45}$ \\
$2^{160}$ 	&   $1,46150\cdot 10^{48}$ \\
$2^{250}$ 	&   $1,80925\cdot 10^{75}$ \\
$2^{256}$ 	&   $1,15792\cdot 10^{77}$ \\
$2^{320}$ 	&   $2,13599\cdot 10^{96}$ \\
$2^{512}$ 	&   $1,34078\cdot 10^{154}$ \\
$2^{768}$ 	&   $1,55252\cdot 10^{231}$ \\
$2^{1024}$ 	&   $1,79769\cdot 10^{308}$ \\
$2^{2048}$ 	&   $3,23170\cdot 10^{616}$ \\ \hline
\end{tabular}
\caption{Spezielle Werte des Zweier- und Zehnersystems}
\end{center}
\end{table}

Berechnung zum Beispiel per GMP:
{\href{http://www.gnu.ai.mit.edu}{\tt http://www.gnu.ai.mit.edu}}.



% --------------------------------------------------------------------------
\newpage
\begin{thebibliography}{99999}
\addcontentsline{toc}{subsection}{Literaturverzeichnis}


\bibitem[Aaronson2003]{Aaronson2003} \index{Aaronson 2003}
    Scott Aaronson, \\
    {\em The Prime Facts: From Euclid to AKS}, \\
    \href{http://www.cs.berkeley.edu/~aaronson/prime.ps}
         {\texttt{http://www.cs.berkeley.edu/\~{}aaronson/prime.ps}}. \\
    Dieses sch"one Paper wurde mir erst nach Vollendung dieses Artikels
    bekannt: es bietet ebenfalls einen einfachen, didaktisch 
    gut aufbereiteten und trotzdem nicht oberfl"achlichen Einstieg 
    in dieses Thema (ist leider nur in Englisch verf"ugbar).


%already defined in elementaryNumberTheory.inc -> 2 davor
\bibitem[Bartholome1996]{2Bartholome1996}  \index{Bartholome 1996}
    A. Bartholom\'e, J. Rung, H. Kern, \\
    {\em Zahlentheorie f"ur Einsteiger}, Vieweg 1995, 2. Auflage 1996.

\bibitem[Blum1999]{Blum1999} \index{Blum 1999}
    W. Blum, \\
    {\em Die Grammatik der Logik}, dtv, 1999.

\bibitem[Bundschuh1998]{Bundschuh1998} \index{Bundschuh 1998}
    Peter Bundschuh, \\
    {\em Einf"uhrung in die Zahlentheorie}, Springer 1988, 4. Auflage 1998.

\bibitem[Doxiadis2000]{Dioxadis2000}
    Apostolos Doxiadis, \\
    {\em Onkel Petros und die Goldbachsche Vermutung}, \\
    Bloomsbury 2000 (deutsch bei L"ubbe 2000 und bei BLT als Taschenbuch 2001).

\bibitem[Graham1989]{Graham1989} \index{Graham 1989}
    R.E. Graham, D.E. Knuth, O. Patashnik, \\
    {\em Concrete Mathematics}, Addison-Wesley, 1989.

\bibitem[Klee1997]{Klee1997} \index{Klee 1997}
    V. Klee, S. Wagon, \\
    {\em Ungel"oste Probleme in der Zahlentheorie und der
    Geometrie der Ebene}, \\ Birkh"auser Verlag, 1997.

\bibitem[Knuth1981]{Knuth1981} \index{Knuth 1981}
    Donald E. Knuth, \\
    {\em The Art of Computer Programming, vol 2: Seminumerical
    Algorithms}, \\ Addison-Wesley, 1969, 2nd edition 1981.

\bibitem[Lorenz1993]{Lorenz1993} \index{Lorenz 1993}
    F. Lorenz, \\
    {\em Algebraische Zahlentheorie}, BI Wissenschaftsverlag, 1993.

\bibitem[Padberg1996]{Padberg1996} \index{Padberg 1996}
    F. Padberg, \\
    {\em Elementare Zahlentheorie}, 
    Spektrum Akademischer Verlag 1988, 2. Auflage 1996.

\bibitem[Pieper1983]{Pieper1983} \index{Pieper 1983}
    H. Pieper, \\
    {\em Zahlen aus Primzahlen}, 
    Verlag Harri Deutsch 1974, 3. Auflage 1983.

\bibitem[Richstein1999]{Richstein1999} \index{Richstein 1999}
    J. Richstein, \\
    {\em Verifying the Goldbach Conjecture up to $4*10^{14},$} 
    Mathematics of Computation 70, 2001, S. 1745-1749. 

\bibitem[Scheid1994]{Scheid1994} \index{Scheid 1994}
    Harald Scheid, \\ 
    {\em Zahlentheorie}, BI Wissenschaftsverlag, 2. Auflage, 1994.

%already defined in elementaryNumberTheory.inc -> 2 davor
\bibitem[Schneier1996]{Schneier1996p} \index{Schneier 1996}
    Bruce Schneier, \\
    {\em Applied Cryptography, Protocols, Algorithms, and Source Code in C},\\
    Wiley and Sons, 2nd edition 1996.

\bibitem[Schroeder1999]{Schroeder1999} \index{Schroeder 1999}
    M.R. Schroeder, \\
    {\em Number Theory in Science and Communication}, \\
    Springer 1984, 3rd edition 1997, Corrected Printing 1999.

\bibitem[Schwenk1996]{Schwenk1996} \index{Schwenk 1996}
    J. Schwenk \\
    {\em Conditional Access},
    in taschenbuch der telekom praxis 1996, \\
    Hrgb. B. Seiler, Verlag Schiele und Sch"on, Berlin.

\bibitem[Tietze1973]{Tietze1973} \index{Tietze 1973}
    H. Tietze, \\
    {\em Gel"oste und ungel"oste mathematische Probleme}, 
    Verlag C.H. Beck 1959, 6. Auflage 1973.

\end{thebibliography}


% --------------------------------------------------------------------------
\newpage
\section*{Web-Links}\addcontentsline{toc}{subsection}{Web-Links}

\begin{enumerate}
   \item GIMPS (Great Internet Mersenne-Prime Search) 
      \index{Mersenne!Mersenne-Primzahl} \\
         www.mersenne.org ist die Homepage des GIMPS-Projekts, \index{GIMPS} \\
      \href{http://www.mersenne.org/prime.htm}
       {\tt http://www.mersenne.org/prime.htm }

\item Die Proth Search Page mit dem Windows-Programm von Yves Gallot \\
      \href{http://www.utm.edu/research/primes/programs/gallot/index.html}
       {\tt http://www.utm.edu/research/primes/programs/gallot/index.html}

\item Verallgemeinerte Fermat-Primzahlen-Suche \\
      \href{http://primes.utm.edu/top20/page.php?id=12}
       {\tt http://primes.utm.edu/top20/page.php?id=12}

\item Verteilte Suche nach Teilern von Fermatzahlen \\
      \href{http://www.fermatsearch.org/}{\tt http://www.fermatsearch.org/}

\item An der Universit"at von Tennessee findet man umfangreiche 
      Forschungsergebnisse "uber Primzahlen. \\
      \href{http://www.utm.edu/}{\tt http://www.utm.edu/ }

\item Den besten "Uberblick zu Primzahlen (weil sehr aktuell und vollst"andig)
      bieten m.E.  ~\glqq The Prime Pages\grqq~ von Professor Chris Caldwell.
      \index{Caldwell, Chris} \\
      \href{http://www.utm.edu/research/primes}
       {\tt http://www.utm.edu/research/primes }

\item Beschreibungen u.a. zu Primzahltests \\
      \href{http://www.utm.edu/research/primes/mersenne.shtml}
           {\texttt{http://www.utm.edu/research/primes/mersenne.shtml}} \\
      \href{http://www.utm.edu/research/primes/prove/index.html}
           {\texttt{http://www.utm.edu/research/primes/prove/index.html}} 

\item Ausgabe der $n$-ten Primzahl \\
      \href{http://www.utm.edu/research/primes/notes/by_year.html}
      {\tt http://www.utm.edu/research/primes/notes/by\_year.html }

\item Der Supercomputerhersteller SGI Cray Research besch"aftigte nicht
      nur hervorragende Mathematiker, sondern benutzte die Primzahltests
      auch als Benchmarks f"ur seine Maschinen. \\
      \href{http://www.isthe.com/chongo/tech/math/prime/prime_press.html}
       {\tt http://www.isthe.com/chongo/tech/math/prime/prime\_press.html }

\item Das Cunningham-Projekt, \index{Cunningham-Projekt}\\ 
      \href{http://www.cerias.purdue.edu/homes/ssw/cun/}
      {\texttt{http://www.cerias.purdue.edu/homes/ssw/cun/}}

\item \href{http://www.eff.org/coop-awards/prime-release1.html}
       {\tt http://www.eff.org/coop-awards/prime-release1.html }


\item \href{http://www.informatik.tu-darmstadt.de/TI/LiDIA/}
              {\tt http://www.informatik.tu-darmstadt.de/TI/LiDIA/ }


\item \href{http://www.math.Princeton.EDU/~arbooker/nthprime.html}{\tt
http://www.math.Princeton.EDU/\~{}arbooker/nthprime.html }

\item \href{http://www.cerias.purdue.edu/homes/ssw/cun}{\tt
http://www.cerias.purdue.edu/homes/ssw/cun }

\item \href{http://www.informatik.uni-giessen.de/staff/richstein/de/Goldbach.html}{\tt http://www.informatik.uni-giessen.de/staff/richstein/de/Goldbach.html}

\item \href{http://www.mathematik.ch/mathematiker/goedel.html}
       {\tt http://www.mathematik.ch/mathematiker/goedel.html}

\item \href{http://www.mscs.dal.ca/~dilcher/golbach/index.html}
       {\tt http://www.mscs.dal.ca/\~{}dilcher/goldbach/index.html}

\end{enumerate}


\vskip +10 pt
% --------------------------------------------------------------------------(5\cdot2^7 - 1)(5^2\cdot2^14 + 1)
\subsection*{Dank} \addcontentsline{toc}{subsection}{Dank}
F"ur das sehr konstruktive Korrekturlesen dieses Artikels: Hr.
Henrik Koy und Hr. Roger Oyono.

% Local Variables:
% TeX-master: "../script-de.tex"
% End:(5\cdot2^7 - 1)(5^2\cdot2^14 + 1)(5\cdot2^7 - 1)(5^2\cdot2^14 + 1)
